\documentclass[a4paper,11pt, oneside,italian]{article}

\usepackage[utf8]{inputenc}
\usepackage{hyperref}
\usepackage{framed}
\usepackage{xcolor}
\usepackage{amsmath}
\usepackage{enumitem}
\usepackage{relsize}
\usepackage{microtype}
\usepackage{typearea}

\hypersetup{
  pdftitle = {Algoritmi per la trasformata di Burrows--Wheeler posizionale con
    compressione run-length}, 
  pdfauthor = {Davide Cozzi},
  pdfsubject = {Riassunto della tesi},
  pdfpagemode = UseNone
}


\usepackage{fancyhdr}
% \pagestyle{fancy}
% \fancyhead[LE,RO]{\slshape \rightmark}
% \fancyhead[LO,RE]{\slshape \leftmark}
\fancyfoot[C]{\thepage}

\title{Algoritmi per la trasformata di Burrows--Wheeler
  posizionale con compressione run-length} 
\author{Davide Cozzi\\\smaller matr.~829827}
\date{}
\makeatletter
\renewcommand{\paragraph}{%
  \@startsection{paragraph}{4}%
  {\z@}{0.75ex \@plus 1ex \@minus .2ex}{-1em}%
  {\normalfont\normalsize\bfseries}%
}
\makeatother

\def\SLP{\mbox{\rm {\sf SLP}}}
\def\rank{\mbox{\rm {\sf rank}}}
\def\lcs{\mbox{\rm {\sf lcs}}}
\def\lcp{\mbox{\rm {\sf lcp}}}
\def\lce{\mbox{\rm {\sf lce}}}
\def\LCE{\mbox{\rm {\sf LCE}}}
\def\ROT{\mbox{\rm {\sf ROT}}}
\def\select{\mbox{\rm {\sf select}}}
\def\col{\mbox{\rm {\sf col}}}
\def\NULL{\mbox{\rm {\sf null}}}
\def\len{\mbox{\rm {\sf len}}}
\def\pos{\mbox{\rm {\sf pos}}}
\def\row{\mbox{\rm {\sf row}}}
\def\LF{\mbox{\rm {\sf LF}}}
\def\FL{\mbox{\rm {\sf FL}}}
\def\W{\mbox{\rm {\sf w}}}
\def\A{\mbox{\rm {\sf A}}}
\def\A{\mbox{\rm {\sf A}}}
\def\SA{\mbox{\rm {\sf SA}}}
\def\LCP{\mbox{\rm {\sf LCP}}}
\def\ISA{\mbox{\rm {\sf ISA}}}
\def\PLCP{\mbox{\rm {\sf PLCP}}}
\def\RLCP{\mbox{\rm {\sf RLCP}}}
\def\RLBWT{\mbox{\rm {\sf RLBWT}}}
\def\MEM{\mbox{\rm {\sf MEM}}}
\def\KMEM{\mbox{\rm {\sf K-MEM}}}
\def\KSMEM{\mbox{\rm {\sf K-SMEM}}}
\def\MS{\mbox{\rm {\sf MS}}}
\def\LCP{\mbox{\rm {\sf LCP}}}
\def\NSV{\mbox{\rm {\sf NSV}}}
\def\PSV{\mbox{\rm {\sf PSV}}}
\def\RMQ{\mbox{\rm {\sf RMQ}}}
\def\BWT{\mbox{\rm {\sf BWT}}}
\def\BWM{\mbox{\rm {\sf BWM}}}
\def\ITR{\mbox{\rm {\sf index\_to\_run}}}
\def\GS{\mbox{\rm {\sf get\_symbol}}}
\def\PBWT{\mbox{\rm {\sf PBWT}}}
\def\MPBWT{\mbox{\rm {\sf mPBWT}}}
\def\MRLPBWT{\mbox{\rm {\sf mRLPBWT}}}
\def\DPBWT{\mbox{\rm {\sf dPBWT}}}
\def\RLPBWT{\mbox{\rm {\sf RLPBWT}}}
\def\SMEM{\mbox{\rm {\sf SMEM}}}
\def\M{\mbox{\rm {\sf M}}}
\def\C{\mbox{\rm {\sf C}}}
\def\Occ{\mbox{\rm {\sf Occ}}}
\def\L{\mbox{\rm {\sf L}}}
\def\F{\mbox{\rm {\sf F}}}
\def\DA{\mbox{\rm {\sf DA}}}
\def\DM{\mbox{\rm {\sf DM}}}
\def\PA{\mbox{\rm {\sf PA}}}
\def\LCE{\mbox{\rm {\sf LCE}}}
\def\UP{\mbox{\rm {\sf update}}}
\def\lceb{\mbox{\rm {\sf lce\_bounded}}}
\def\RM{\mbox{\rm {\sf reverse\_map}}}


\begin{document}
\maketitle
\setlist{leftmargin = 2cm}
\noindent
%\subsection*{Introduzione}
Negli ultimi anni si è assistito a un cambio di paradigma nel campo della
bioinformatica, ovvero il passaggio dallo studio della sequenza lineare di un
singolo genoma a quello di un insieme di genomi, provenienti da un gran numero
di individui, al fine di poter considerare anche le varianti
  geniche. Questo nuovo concetto è stato introdotto da
Tettelin, nel 2005, con il termine di \textit{pangenoma}. Grazie ai risultati
ottenuti in pangenomica, ci sono stati miglioramenti sia nel 
campo della biologia che in quello della medicina personalizzata, grazie al
fatto che, con il pangenoma, si migliora la precisione della rappresentazione di
multipli genomi e delle loro differenze. 
Il genoma umano di riferimento (GRCh38.p14), è composto da circa
3.1 miliardi di basi, con più di 88 milioni 
varianti tra i genomi sequenziati, secondo i risultati ottenuti nel 1000 Genome
Project. Considerando che, grazie al
miglioramento delle tecnologie di sequenziamento, la quantità dei dati di
sequenziamento sia destinata 
ad aumentare esponenzialmente nei prossimi anni, risulta necessaria la
costruzione di algoritmi e   
strutture dati efficienti per gestire una tale mole di dati.
A questo scopo, uno degli approcci più usati per rappresentare il pangenoma è
attraverso un 
pannello di aplotipi, ovvero, da un punto di vista computazionale, una matrice
di $M$ 
righe, corrispondenti agli individui, e $N$ colonne, corrispondenti ai siti con
le varianti. Si specifica che, con il termine
aplotipo, si intende l'insieme di alleli, ovvero di varianti che, a meno di
mutazioni, un organismo eredita da ogni genitore.

In questo contesto trova spazio uno dei problemi fondamentali della
bioinformatica, ovvero quello del pattern matching. Inizialmente tale problema
era relativo alla ricerca di una stringa (pattern) all'interno di un testo di
grandi dimensioni, cioè il genoma di riferimento.
Ora, con l'introduzione del pangenoma, il problema deve essere risolto sulle
nuove strutture 
di rappresentazione del pangenoma.

Lo scopo di questa tesi è progettare strutture dati e algoritmi efficienti per
risolvere il problema del pattern 
matching, inteso come ricerca dei set-maximal exact match ($\SMEM$) tra un
aplotipo 
esterno e un pannello di aplotipi, in una delle 
strutture dati più utilizzata per la rappresentazione del pangenoma: la
\textit{trasformata di Burrows--Wheeler 
  Posizionale} ($\PBWT$). Il progetto di tesi, svolto in collaborazione con
Prof.~Gagie (Dalhousie University) e Prof.~Boucher 
(University of Florida), ha quindi permesso lo sviluppo di una variante
\textbf{run-length encoded} della $\PBWT$, detta $\RLPBWT$, che
risolve questo problema riducendo la memoria impiegata rispetto
alla soluzione attualmente allo stato dell'arte.

\subsection*{Stato dell'arte}
Gli algoritmi più efficienti per risolvere il problema del pattern matching tra
un pattern e un testo 
sono attualmente basati sulla \emph{Trasformata di Burrows--Wheeler} ($\BWT$), a
sua 
volta strettamente legata ad altre due strutture dati: il Suffix Array ($\SA$) e
il Longest Common Prefix ($\LCP$). 
% Infatti, è stato dimostrato che la BWT è fortemente legata a due strutture dati
% fondamentali per il pattern matching: 
% il Suffix Array (SA), cioè la lista delle posizioni di partenza di
% tutti i suffissi di un testo presi in ordine lessicografico, e il Longest Common
% Prefix (LCP), cioè la lista delle lunghezze del più lungo prefisso comune tra
% ogni coppia di suffissi consecutivi indicizzati nel SA. 

Invece, allo scopo di risolvere il problema del calcolo degli $\SMEM$ tra un
pannello di aplotipi e un aplotipo esterno,
Durbin, nel 2014, propose una struttura dati ispirata alla $\BWT$, detta
\textit{Trasformata di Burrows--Wheeler Posizionale} ($\PBWT$). Tale trasformata
viene costruita a partire da un pannello di aplotipi, rappresentato, riferendosi
al solo caso bi-allelico, tramite una matrice binaria.
La $\PBWT$ permette di calcolare gli $\SMEM$ tra
un aplotipo esterno e il pannello in tempo $Avg.\,\,\,\mathcal{O}(N+c)$ (dove 
$c$ è il numero complessivo di $\SMEM$), mentre 
una soluzione semplice impiegherebbe $\mathcal{O}(N^2M)$.
% Gli SMEM sono tali per
% cui gli indici di inizio 
% e fine del match sono i medesimi sia sul pannello, in termini di indici di
% colonna, che sull'aplotipo query.
La motivazione essenziale della $\PBWT$ è considerare match, e quindi anche
$\SMEM$, 
dove anche le posizioni di inizio e fine sono rispettate.
Tale vincolo, da cui deriva il termine
``posizionale'', non è soddisfacibile dalla $\BWT$ ed è dovuto al fatto che ogni 
colonna (o indice della query) rappresenta un preciso sito per una specifica
variante genica. Il tradeoff di questo algoritmo è la richiesta in termini di
spazio ($13NM$ 
byte). Superare questo limite è l'obiettivo principale di
questo progetto di tesi.
Il funzionamento della $\PBWT$ prevede la costruzione di due insiemi di
array, tramite 
l'\textit{ordinamento dei prefissi inversi} a ogni colonna del pannello, detti
\textit{insieme dei prefix array} e \textit{insieme dei divergence array}.
% Per
% ogni colonna, secondo l'ordinamento dettato dalla stessa, i due array presentano
% un comportamento analogo a quanto visto per il suffix array e l'array LCP nella
% BWT.
Il pannello, permutato tramite l'insieme dei prefix array, è
detto matrice $\PBWT$.

%  Il primo, denotato $a$, contiene, per ogni colonna e ogni
% posizione, l'indice dell'aplotipo nel pannello originale riordinato secondo
% l'ordinamento inverso alla data colonna. Quindi, si ha, in ogni colonna, la
% permutazione degli indici di riga secondo l'ordinamento inverso alla colonna
% $k$-esima. 
% Il secondo insieme, invece, indica l'indice della colonna iniziale del suffisso
% comune più lungo, che termina nella colonna nella colonna $k$, tra una riga e la
% sua precedente secondo l'ordinamento inverso ottenuto per la $k$-esima colonna.
% Il pannello ottenuto con le permutazioni dettate dal prefix array, viene
% chiamato matrice PBWT.
Un altro obiettivo raggiunto negli ultimi anni è stato quello di studiare la
costruzione di un'unica $\BWT$ partendo da multipli genomi.
A tal fine si è sfruttata una caratteristica della $\BWT$, ovvero la
tendenza a produrre una sequenza formata da run, ovvero caratteri uguali in
posizioni consecutive. La causa di tale fenomeno si ritrova nelle ripetizioni di
determinate sottostringhe nel testo, ripetizioni che sono frequenti se si
studiano diverse sequenze genomiche concatenate.
% Partendo da molte sequenze genomiche concatenate si
% hanno, ovviamente, molte regioni ripetute che comportano tali sequenze massimali
% di caratteri uguali nella BWT, dette run.
% Questo fenomeno è dovuto alle ripetizioni di determinate
% sottostringhe nel testow, avendo che tali ripetizioni, per motivazioni
% biologiche, sono frequenti in ambito genomico.
%
In letteratura è stata quindi proposta la \textit{trasformata di
  Burrows--Wheeler 
  Run-Length encoded} ($\RLBWT$) dove ogni run viene memorizzata in modo
efficiente 
come coppia \texttt{(carattere, lunghezza della run)}. Ad esempio, la stringa
\texttt{aaaaaa} sarebbe memorizzata come \texttt{(a,6)}.
Recentemente, per questa struttura dati compressa, è stato proposto un nuovo
tipo di indicizzazione: il cosiddetto \textit{r-index}. Tale indice riduce lo
spazio di memoria richiesto perché non è lineare sulla lunghezza del testo ma
sul numero di run della sua $\BWT$.
L'r-index include la
$\RLBWT$ e un suffix array sample, ovvero i valori del SA all’inizio e alla fine
di ogni run. 
Tali articoli hanno portato alla produzione di due tool, \textit{MONI} e 
\textit{PHONI}, alle cui tecniche è fortemente ispirata questa tesi. Queste due
soluzioni hanno entrambe l'obiettivo di 
calcolare i maximal exact match ($\MEM$) tra un testo e un pattern, ovvero
sottostringhe del pattern che 
sono anche sottostringhe del testo, avendo che tale match non può essere esteso,
in entrambe le direzioni, senza introdurre un mismatch. 

Il calcolo dei $\MEM$ è direttamente
correlato alla costruzione dell'array delle \textit{Matching Statistics}
($\MS$). Tale array, lungo quanto il pattern e formato da coppie (posizione
$\pos$, lunghezza $\len$), annota la lunghezza
$\len$ di un match, anche non massimale, con una sottostringa nel testo avente
indice iniziale $\pos$, per ogni posizione del pattern.
Estraendo da tali match i $\MEM$ si possono poi ottenere
tutte le altre occorrenze del medesimo $\MEM$ nel testo. 
In MONI, per il calcolo delle $\MS$, è stato
usato anche il concetto di \textit{threshold}, definito come
il minimo valore dell'array $\LCP$ tra due run consecutive dello stesso
carattere. Un ulteriore miglioramento si è registrato in PHONI con
l'uso delle $\LCE$ query per il calcolo dell'array $\MS$. Infatti, tale
struttura, dati due indici del testo $i$ e $j$, restituisce il più lungo
prefisso comune tra i suffissi $i$-esimo e $j$-esimo del testo.

% Si noti, infine, un collegamento naturale che si ha tra la PBWT
% e la BWT, avendo, ad esempio, che il prefix array, in una certa colonna, della
% prima corrisponde al suffix array della seconda mentre il divergence array è una
% diversa rappresentazione dell'array LCP. Anche grazie a queste premesse, si è
% basato il progetto di tesi sulla costruzione RLPBWT a partire dalle tecniche
% teorizzate per la RLBWT.

% Al fine di raggiungere tale obiettivo sono state usate altre nozioni. In primis,
% si sono sfruttate le cosiddette \textbf{strutture dati succinte}, ovvero
% strutture per le quali, assumendo che $\mathcal{X}$ sia il numero di bit
% ottimale per memorizzare dei dati, si richiede uno spazio in memoria pari a
% $\mathcal{X}+o(\mathcal{X})$. Nel dettaglio si sono usati i cosiddetti
% \textbf{bitvector sparsi}, strutture che richiedono in memoria $\approx
% m\left(2+\log\frac{n}{m}\right)$ bit, con $n$ lunghezza del bitvector e $m$
% numero di simboli $\sigma=1$ in esso. L'efficienza delle operazioni che si
% possono fare con tali strutture dati, incredibilmente efficienti dal punto di
% vista dello spazio occupato, sono uno dei punti cruciali del funzionamento della
% \textit{RLPBWT}.\\
% Infine, per memorizzare il pannello in modo compatto, si è usata una
% struttura dati, detta \textbf{Straight-Line Programs (\textit{SLP})}. Tale
% struttura è una \textbf{grammatica context-free}, che
% genera una e una sola parola, la quale permette sia di effettuare \textit{random
% access} al testo, non in tempo costante, che di calcolare le \textit{LCE
% query} in modo efficiente.

\subsection*{Contributo}
% L'idea con la quale si è sviluppata questa tesi vede, come punto di partenza,
% alcuni primi risultati teorici, presenti in letteratura, che caratterizzavano
% una variante run-length encoded della PBWT. Tuttavia, tale proposta era
% inefficiente in termini di memoria e su di essa non era possibile il calcolo
% degli SMEM.
% Infatti, un adattamento dell'algoritmo per il calcolo degli SMEM proposto da
% Durbin all'uso con una struttura 
% run-length encoded non sembra essere possibile mantenendo lo spazio impiegato
% per memorizzare i dati del divergence array proporzionale al numero di 
% run. Inoltre non sembra possibile utilizzare i sample del prefix array ad inizio
% e fine di ogni run.

% Il primo approccio aveva come obiettivo l'ottenere un riadattamento ``diretto''
% dell'\textit{algoritmo 5 di Durbin}, pur tenendo in memoria informazioni legate
% principalmente alle run della \textit{matrice PBWT}. Per quanto si siano trovate
% soluzioni interessanti per gestire il \textit{mapping} tra una colonna e la sua
% successiva (ma anche tra una colonna e la sua precedente), per poter
% ``seguire'' una riga del pannello originale nella 
% \textit{matrice PBWT}, si dovute memorizzare anche quantità non relative alle
% run.  Infatti, è risultato necessario memorizzare l'intero
% \textit{divergence array} (in forma di \textit{LCP array} quindi
% memorizzando la lunghezza del prefisso comune in ordine inverso e non la colonna
% d'inizio dello stesso), al fine di poter computare i \textit{MEM} con
% un aplotipo esterno. Inoltre, l'assenza delle informazioni relative al
% \textit{prefix array} ha impedito di poter annotare quali righe del pannello
% presentassero un match massimale esatto, terminante in una certa colonna, con
% l'aplotipo query, avendo solo 
% informazioni relative alla cardinalità di tale sottoinsieme di righe. Le
% informazioni in memoria si è 
% stimato non fossero ottimali, non solo per il \textit{divergence array}, ma anche
% perché non veniva utilizzato alcun approccio tramite \textit{strutture dati
%   succinte}.

%Nel lavoro di tesi si è quindi deciso di cambiare approccio al problema,
%ispirandosi ai risultati già ottenuti per la RLBWT.
Questo lavoro di tesi è stato incentrato sullo sviluppo di approcci run-length
encoded per la $\PBWT$, ispirandosi ai risultati già ottenuti per la $\RLBWT$.
Essendoci diverse soluzioni
possibili, si è deciso di suddividere il lavoro in varie componenti che
permettessero l'assemblaggio di strutture dati composte atte al calcolo degli
SMEM.  

Alcuni di questi approcci si basano sull'uso dei bitvector 
sparsi, una struttura dati succinta che permette di memorizzare e interrogare
vettori binari efficientemente. Infatti, tali strutture permettono due
operazioni, dette $\rank$ e $\select$, in tempo costante. La prima conteggia il 
numero di simboli $\sigma=1$ fino ad una certa posizione mentre la seconda
restituisce l'indice sul bitvector dell'$i$-esimo simbolo $\sigma=1$. 
Quindi, le strutture necessarie all'indicizzazione delle run e al cosiddetto
mapping, ovvero il ``seguire'' una riga dalla sua posizione permutata in una
certa colonna alla posizione permutata nella colonna successiva,
sono state implementate, anche, tramite bitvector sparsi (componente
\texttt{MAP-BV}). Tali bitvector presentano un numero di simboli 
$\sigma=1$ proporzionale al numero delle run. È stata anche proposta una
struttura per il mapping basata sull'uso di vettori di interi 
compressi, di lunghezza proporzionale al numero di run (componente
\texttt{MAP-INT}).  

Al fine di avvicinarsi alle idee proposte in MONI e PHONI, è
stato necessario uno studio teorico preliminare per ridefinire: matching
statistics (e conseguente calcolo degli $\SMEM$), threshold e $\LCE$ query. In
particolare, grazie agli  
ultimi due concetti, si è potuto evitare di memorizzare il divergence array.
Le matching statistics sono definite tramite un array,
lungo quanto il pattern e formato da coppie (riga $\row$, lunghezza $\len$),
che, per ogni colonna, tiene traccia del suffisso comune che non sia estendibile
a sinistra, terminante in quella colonna e lungo $\len$, tra la riga $\row$ del
pannello e l'aplotipo query. Una volta calcolato tale
array è possibile estrarre uno $\SMEM$ tra la query e
la riga $\row$, terminante in colonna $k$. Infine, è possibile estendere,
tramite una struttura dati a 
supporto (componente \texttt{PHI}), tale risultato 
a tutte le altre righe del pannello che presentano il medesimo $\SMEM$.
Il primo metodo di calcolo di tale array è stato basato sull'utilizzo delle
threshold. Ragionando sulla matrice $\PBWT$, una threshold è definita
come l'indice del primo massimo valore del divergence array all'interno di una
run (comprendendo anche la testa, qualora esistente, della run successiva,
essendo 
il suo valore del divergence array calcolato sfruttando anche la coda della run
corrente). L'insieme delle threshold per una certa colonna è stato memorizzato
come bitvector sparso (componente \texttt{THR-BV}) oppure, come per il mapping,
tramite vettore di interi compresso (componente \texttt{THR-INT}). Inoltre, si è
provveduto a tenere in memoria i  
sample di prefix array a inizio e fine di ogni run (componente \texttt{PERM}). 

Grazie all'uso delle threshold, si è potuto sviluppare un algoritmo
efficiente, dal punto di vista della memoria richiesta, per il calcolo delle
matching statistics in due ``sweep'' sul pattern, calcolandone prima le
posizioni e poi, tramite random access al pannello, anche le
lunghezze. Tale soluzione richiede in memoria l'intero pannello, per il quale
si è scelto di usare uno Straight-Line Program ($\SLP$), ovvero una grammatica
context-free che genera una e una sola parola, molto compatta in memoria, sulla
quale è possibile effettuare random access (in tempo, da un punto di vista
prettamente teorico, $\mathcal{O}(\log (NM))$, avendo un pannello con $M$
aplotipi e $N$ siti).  
Ad esempio, un pannello  $4.908 \times 6.196.151$ richiede, in forma non
compressa, circa 28GB di memoria, mentre l'$\,\SLP$ 
richiede 
solamente in 0,2GB (garantendo random access). Per riferimento, comprimere il
pannello con una tecnica 
standard (GZip) richiede 0,5GB. Oltre all'uso dell'$\,\SLP$ (componente
\texttt{RA-SLP}), si è anche proposto l'uso di un insieme di bitvector per la
memorizzazione 
del pannello (componente \texttt{RA-BV}), avendo maggior spazio in memoria
(3,6GB) ma minor tempo d'interrogazione, garantendo random access in tempo
costante.  

Inoltre, è possibile risparmiare ulteriore spazio eliminando
l'uso delle threshold e usando, per il calcolo delle matching
statistics, le $\LCE$ query (componente \texttt{LCE}). Esse sono
ridefinite, per la $\RLPBWT$, come il suffisso comune più lungo tra due righe
del pannello, eventualmente 
fissando il termine a una colonna precisata. Il loro calcolo è permesso
in modo efficiente dall'uso dell'$\,\SLP$, avendo la complessità in tempo
limitata superiormente da quella del random access (ovvero $\mathcal{O}(\log
(NM))$). Calcolando la 
lunghezza di tale suffisso comune, è possibile computare le
lunghezze delle matching statistics contemporaneamente al calcolo delle
righe. Quindi, si è ottenuto il calcolo completo di tale array in
un singolo ``sweep'' sul pattern e, avendo che il calcolo degli $\SMEM$ viene
fatto 
contemporaneamente al calcolo delle lunghezze, si è ridotto lo spazio
necessario. Infatti, sono sufficienti la 
coppia (riga $\row$, lunghezza $\len$) corrente e quella precedente dell'array
delle matching statistics per calcolare gli $\SMEM$.

\subsection*{Risultati sperimentali}
L'obiettivo della fase sperimentale è stato quello di confrontarsi con
l'algoritmo di Durbin per il calcolo degli $\SMEM$ sulla $\PBWT$ al fine 
(1) di verificare se le soluzioni per la $\RLPBWT$ riducessero la memoria
necessaria al calcolo degli 
$\SMEM$ e (2) di quantificare l'aumento dei tempi di calcolo.
Infatti, per quanto riguarda (1), la caratterizzazione asintotica teorica non è
sufficientemente stringente da garantire una riduzione pratica, in quanto
fortemente dipendente dalle caratteristiche dei dati impiegati.
Per quanto riguarda (2), un aumento dei tempi è atteso poiché la $\RLPBWT$,
a differenza dell'algoritmo di Durbin, utilizza strutture dati succinte e/o
compresse che comportano tempi di calcolo superiori per le operazioni di accesso
e/o interrogazione. 
I test sono stati effettuati su 5 pannelli della phase 3 del 1000 Genome
Project, uno dei più importanti progetti di catalogazione delle varianti
geniche 
umane. Tali pannelli sono relativi ai cromosomi 22, 20, 18, 16 e 1
(in ordine crescente di siti). Tutti i pannelli presentano 5.008 sample.
Per questa sintesi, si è scelto di riportare alcuni risultati relativi al
pannello del cromosoma 1, uno dei più grandi avendo dimensioni $4.908 \times
6.196.151$, che è stato 
interrogato con $100$ query (estratte dal pannello originale), al 
fine di ridurre la fluttuazione statistica dei risultati.
I risultati qui riportati, relativi alla $\PBWT$, sono stati ottenuti usando
l'implementazione ufficiale della stessa. Per la $\RLPBWT$ si riportano solo i
risultati dell'implementazione con $\LCE$ query e mapping tramite vettori di
interi compressi, essendo la soluzione con minor richiesta di memoria. 

In merito ai tempi di calcolo, si è ottenuto il risultato atteso: con
la $\PBWT$ si ha il calcolo degli $\SMEM$ in 1.026s
mentre con la $\RLPBWT$ in 1.142s, avendo una differenza, per quanto attesa,
molto piccola.
Si è passato, quindi, ad analizzare i risultati in termini di memoria. Secondo
le stime di Durbin, gli array necessari al funzionamento del suo algoritmo
richiederebbero $13NM$ byte, implicando, per il pannello in analisi, circa
368GB di memoria. 
Sperimentalmente, si sono registrati
picchi di memoria prossimi ai 369GB, confermando le stime. Invece,
per la variante della $\RLPBWT$ presa in esame, il picco
registrato è stato di appena 4GB, 
ottenendo una riduzione di memoria di quasi il 99\%. Tale riduzione renderebbe
possibile l'esecuzione del calcolo anche su un comune portatile, assumendo di
aver a disposizione l'$\,\SLP$ . 
\subsection*{Conclusioni e sviluppi futuri}
In conclusione, le strutture dati composte per la $\RLPBWT$, basate sul calcolo
dell'array delle matching statistics, sono state tutte in
grado, al variare della soluzione, di ridurre sensibilmente la 
quantità di memoria necessaria a risolvere il problema del calcolo degli $\SMEM$
tra un pannello di aplotipi e un aplotipo esterno.  

Seppur questa tesi abbia raggiunto l'obiettivo prefissato, il lavoro può essere
ulteriormente esteso in diverse direzioni, come ad esempio sviluppando
un'ulteriore ottimizzazione delle strutture, sia in termini 
di gestione del mapping che, eventualmente, di gestione di più query
contemporaneamente. In merito a quest'ultimo sviluppo, si segnala che Durbin
ha proposto anche un algoritmo per il calcolo degli $\SMEM$ tra un pannello di
aplotipi e uno di query. Questo algoritmo è basato sulla creazione ``virtuale''
di un unico pannello, su cui 
costruire la $\PBWT$, e sul calcolo dei match interni al pannello stesso. Tale
implementazione, nel setup di test 
descritto nella sezione precedente (che comporta il perfetto input per
tale soluzione), risulta essere molto performante sia in
termini di tempo (93s) che di memoria utilizzata (0,1GB) durante
l'esecuzione. Si segnala che, per quanto riguarda i tempi di calcolo, tale
algoritmo diventa inefficiente, rispetto alle altre soluzioni proposte, al
diminuire del numero di query. 

Inoltre, sono possibili diverse generalizzazioni rispetto alle caratteristiche
del pannello, come ad esempio pannelli multiallelici o con dati mancanti, che
sono comuni in dati reali. In particolare, la gestione 
di dati 
mancati, molto rilevante a causa della non perfezione delle tecnologie di
sequenziamento, risulta essere un problema aperto in bioinformatica. Quindi,
sarà richiesto lo sviluppo di nuove metodologie, basate, ad esempio,
su algoritmi parametrici o algoritmi approssimati, per la risoluzione del
problema del calcolo degli $\SMEM$ su pannelli di questo tipo.
% Per quanto i pannelli di aplotipi prodotti dal sequencing del
% genoma umano raramente presentino siti multi-allelici, si ha una 
% presenza stimata, al momento, di circa il 2\% di siti tri-allelici, avendo che
% tale percentuale risulti fortemente sottostimata. Una prima generalizzazione
% quindi sarebbe quella di studiare pannelli multi-allelici, riformulando la
% RLPBWT al fine di poter funzionare anche con pannelli costruiti su un
% alfabeto arbitrario. Un'altra generalizzazione interessante riguarda
% l'ammissione di 
% dati mancanti all'interno del pannello stesso. La maggior parte delle soluzioni
% attualmente sviluppate sono basate su una forte assunzione: non si hanno dati
% mancanti. Una variante della RLPBWT che sia quindi in grado di
% lavorare, eventualmente con algoritmi parametrici o algoritmi
%   approssimati, 
% su pannelli in cui sono presenti wildcard per rappresentare tali dati,
% permetterebbe di fare studi più completi su dati reali, migliorandone gli usi,
% ad esempio, nel campo della medicina personalizzata.\\

Le potenzialità di tale struttura dati sono quindi molteplici e, grazie al
ridotto consumo di memoria, si hanno le giuste premesse perché venga utilizzata
per gestire e interrogare grandi moli di dati reali, incrementando le capacità
di studio, previsione e inferenza che si possono avere per mezzo dello studio
del pangenoma.
\end{document}


% LocalWords:  pangenoma naive sottostringa BWT sottostringhe Durbin prefix MEM
% LocalWords:  array matching threshold divergence PBWT SMEM query statistics
% LocalWords:  RLBWT aplotipi aplotipo Burrows Wheeler RLPBWT maximal exact LCP
% LocalWords:  LCE
