\documentclass[a4paper,11pt, oneside]{article}
\usepackage[left=15mm, right=15mm, top=15mm, bottom=15mm]{geometry}
\usepackage[utf8]{inputenc}
\usepackage{hyperref}
\usepackage{framed}
\usepackage{xcolor}
\usepackage{enumitem}
\usepackage[tickmarkheight=0.7ex]{todonotes}
\newcommand{\gdv}[1]{\todo[backgroundcolor=green]{\textbf{GDV} #1}}
\newcommand{\pb}[1]{\todo[backgroundcolor=red]{\textbf{PB} #1}}
\newcommand{\dc}[1]{\todo[backgroundcolor=yellow]{\textbf{DC} #1}}
\usepackage[style=numeric, sorting=none, doi=false, isbn=false, url=false,
eprint=false,maxcitenames=2, mincitenames=1, maxbibnames=2, minbibnames=2,
giveninits=true]{biblatex} 
\addbibresource{main.bib}


\usepackage{fancyhdr}
% \pagestyle{fancy}
% \fancyhead[LE,RO]{\slshape \rightmark}
% \fancyhead[LO,RE]{\slshape \leftmark}
\fancyfoot[C]{\thepage}

\title{\vspace{-1.75cm}Ph.D. Proposal}
\author{Davide Cozzi, 829827,
  \href{mailto:d.cozzi@campus.unimib.it}{d.cozzi@campus.unimib.it}} 
\date{}
\makeatletter
\renewcommand{\paragraph}{%
  \@startsection{paragraph}{4}%
  {\z@}{0.75ex \@plus 1ex \@minus .2ex}{-1em}%
  {\normalfont\normalsize\bfseries}%
}
\makeatother

\begin{document}
\maketitle
\setlist{leftmargin = 2cm}
\subsubsection*{Introduction}
\textit{Computational pangenomics} is a new emerging research field in
computational biology based on the idea that we need to move
from the traditional view of a reference genome as a \textit{linear sequence} to
the one where we consider the \textit{genetic variations} or \textit{variants},
in a large collection of individual genomes. This new reference is called a
\emph{pangenome}.  
Pangenomics is becoming increasingly essential in the
field of biology and personalized medicine, mainly thanks to 
\textbf{genome-wide association studies (\textit{GWAS})}, which need
to analyze a large collection of individual genomes.  Moreover, 
these studies also require such data  to be indexed, queried and analyzed. Just
to give some examples, from a Computer Science point of view, the 
\textit{homo sapiens reference genome (GRCh38.p14)} consists of $\sim 3.1$
characters \textit{Gb} and contains $\sim 59,265$ genes, as 
reported by NCBI. Biological studies based on the \textit{1000 Genome Project}
have pointed out that there are over 88 
million variants between those human genomes. Among these variants, 84.7 million
are \textbf{Single Nucleotide Polymorphisms (\textit{SNPs})}, 3.6 million are
\textbf{short insertions/deletions (\textit{indel})} and 60000 are
\textbf{structural variants}, involving more than 50 nucleotides. Moreover, it
is necessary to consider that the final goal is the sequencing of at least 100
thousand individuals, in the next few years, also thanks to latest developments
in sequencing technologies, (\textbf{Next Generation Sequencing
  (\textit{NGS})} and \textbf{third-generation sequencing}).
\dc{NGS etc citati nell'introduzione ma potrebbero essere superflui}
Therefore, it is
clear that dealing with this amount of data is challenging for the current
state-of-the-art algorithms. The problem of \textit{pattern matching} is one of
the most fundamental topics in 
the field of algorithmics and bioinformatics, where the input is a long text (a
genome) and a short pattern (a read) and the output is the list of all
occurrences of the pattern as a substring of the text.  
We can notice that such problem
has practical and efficient linear-time solutions, if there is a proper indexing
of the text. The interest in such 
problems is due to the need to align sequences or to search for specific
patterns within a database of \textit{DNA} sequences. 
In this context, a large number of data structures and algorithms have been
designed. Among these, one of the most used is the \textbf{Burrows-Wheeler
  transform (\textit{BWT})} on which the \textit{FM-index} is based. 
%Thanks
%to these algorithms and to the fact that it is returned an interval over the
%\textit{BWT}, the problem of pattern matching, over an indexed text, has become 
%linear on the length of the pattern. 
The use of such 
algorithms is essential in speeding up alignment algorithms such as
\textbf{BLAST}, one of the most widely used aligners and implementation of one
of the most cited theoretical studies in the field of bioinformatics.
% \gdv{spiegare la rilevanza di blast} 
%\dc{BLAST ha moltissime citazioni ed è uno dei top 4 aligner}. 
This aligner based on the \textit{seed-and-extend paradigm}, where pattern
matching occurrences, the so-called \textit{seed}, are chosen to be the starting
point of 
the alignment. Then, from the seed, the algorithm proceeds to extend the match
to compute the alignment, increasing the efficiency of the
algorithm compared, for example, to those based on dynamic programming.
%\dc{confronto con DP etc}
The key point, talking about \textit{BWT}, is the ability to compress large
texts and query the compressed texts themselves, allowing alignment even 
in very large databases such as those used by \textit{BLAST}.\\
%\dc{Essenziale compressione dei db su cui lavora blast}.\\  
In recent years, there has been an important development
in the field of bioinformatics. Now the 
researchers are focusing on the study of \textbf{pangenomes}, which
term was introduced
% by Tettelin
in 2005 for dealing with the comparison of genes in
bacterial species. Briefly, the \textit{pangenome} is a compact 
representation of multiple genomes, encoding the variations of a multitude of
individual genomes from the same species. From a computational point of view,
\textit{pangenome} is often represented as a \textit{pangenome graph} but also
as a \textit{collection of haplotypes} \cite{pancon}, for which space complexity
can be optimised thanks to \textit{succinct data structures}. By
\textit{pangenome}, it has been possible to take into account the high
variability in population genomes, in 
addition to the specificity of single genome, allowing even more the use of
computational techniques for a \textit{personalized approach to medicine},
approach that will become increasingly important even in near future.\\ 
%In fact, the need to take into 
%account the high variability in population genomes as well as the specificity of
%an individual genome in a personalized approach to medicine is rapidly pushing
%towards leaving the traditional paradigm based on a single reference genome
%\dc{paragrafo confuso aggiungere miglioramenti tramite ds succinte e
%collegamento poi con PBWT} 
The main focus of this proposal is improving the current state of the art
of pattern matching algorithms and data structures to represent and query
biological sequences, with a focus on space 
complexity. In fact, the design of data structures and algorithms to handle
large collections of data will fill the current gap that exists between the
exponential growth of genomic data and the available state of the art algorithms
to manage  
such data. These research field will have an impact on the ability of
molecular biologists to analyze the current amount of data to extract important
genomic information from pangenome graphs. 
%\pb{the main focus of this project proposal is...}
%\pb{IO menzionerei il tempo di calcolo del pattern matching del pattern ad un
%testo che è lineare nel pattern se vi è una adeguata rappresentazione o
%indicizzazione del testo}
%\pb{il tempo che si vuole avere per ricercare tutte le occorrenze del pattern
%con la BWT è lineare nel pattern anche se il numero di occorrenze potrebbe
%essere più alto perchè si restituisce un intervallo...in altre parole tieni
%presente che dovresti dare motivazioni all'uso dell'FM-index, tenere compresso
%il testo e nel contempo avere tutte le occorrenze come intervallo di
%BWT... dire qualcosa tipo che la dimensione dei dati richieste tenere il dato
%in forma compressa e quindi poter fare le query direttamente sul dato
%compresso...puoi anche dire che lo sviluppo e il disegno di strutture succinte
%per trattare grandi collezioni di dati e la loro implementazione %applicazione
%a sequenze reali (esempio aplotipi) è uno dei temi focus del tuo progetto...} 
%\pb{tenuto conto che il focus è filling  the current gap between the genomic
%data production and the ability of computer scintests to manage such data by
%developing new  data structure and algorithms that have to deal with these
%data. This improvement will have an impact on the ability of molecolar
%biologists to analyze the current amount of data represented by pangenome
%graphs as I will discuss in the specific goals of my project...}
%\subsubsection*{Preliminaries}
%\dc{Spostare su definizioni di MS, LCE, MEM,; etc...}
%\dc{Spiegare meglio pannello di aplotipi come un pangenoma con tutti genomi di
%stessa lunghezza} 
%\dc{Per risparmiare spazio non so se l'introduzione nei preliminari della RLBWT
%  com'è ora poss andare bene} 
% The latest developments in sequencing technologies, with \textbf{Next
%   Generation Sequencing (\textit{NGS})} and \textbf{third-generation
%   sequencing}, have greatly reduced the cost of sequencing data and
% improved the quality of the data produced, the
% researchers were able to theorize the \textbf{pangenome graph}. Furthermore, the
% new amount of sequences has led to new algorithms for \textit{pattern matching}
% for collections of sequences. 
% \gdv{Questo paragrafo mi sembra una ripetizione di quanto detto nell'introduzione.
% Direi di toglierlo}
% A simple but effective strategy to compress texts is based on encoding a
% \textbf{run} of consecutive identical characters as the pair (characters, length
% of the run). For example, the string \texttt{AAAAAA} is encoded as (\texttt{A},
% 6). 
% Clearly, such an encoding achieves a better compression ratio the more
% consecutive character we can find. 
% This fact is coupled with the use of BWT, which is designed to obtain long run
% on repetitive texts, such as a pangenome, to obtain the \textbf{run-length
%   encoded BWT (\textit{RLBWT})}. 
% In fact, some recent papers --- \textit{MONI}~\cite{moni} and
% \textit{PHONI}~\cite{phoni} --- present two practical indexing tools that are
% tailored for pangenomes. 
%%In 2021, Rossi et al. proposed  as a
%data structure to handle a \textbf{run-length encoded version of BWT
%(\textit{RLBWT})},
%with the ultimate intention of indexing and using multiple
%genomes as a reference \cite{moni}, by run-length compression. 
%With this type of encoding, it is possible
%to further improve the compression of the text, having that consecutive
%identical characters, the so-called \textbf{runs}, are compactly stored, making
%it possible to study pangenomic sequences. 
%A recent
%improvement has been made through the implementation of \textit{PHONI}
%\cite{phoni}. 
%\gdv{Si riesce a dare un indizio del legame fra MEM, MS, LCE?}
%According to Rossi et al.,
%\textit{BWA-MEM}, one of the most used read aligners, uses between 1.1 and 3.8
%times more memory than \textit{MONI}. 
%Together with this data structure, the
%authors used the concept of \textbf{matching statistics (\textit{MS})}
%to efficiently compute the matches between a pattern and a text. A recent
%improvement has been made through the implementation of \textit{PHONI}
%\cite{phoni}, where the \textbf{longest-common-extension (\textit{LCE})
%  queries}, achieving the same results as \textit{MONI} with only a single sweep
%over \textit{MS} array instead of two as in \textit{MONI} (details in the next
%section). 
% On the other hand, several computational problems arising from computational
% biology have led to the development and implementation of algorithms and data
% structures for analyzing specific sequence data such as \textit{haplotypes} and
% \textit{genotypes} in the \textbf{genotyping variants problem}.
% Briefly, we
% could define \textit{haplotypes} as a combination of allelic variants, each one
% inherited from a parent. Instead, the \textit{genotype} is combined information
% of the haplotypes. For example, humans have two haplotypes, being diploid, and
% the genotype is their combined information. So, since 2005, publication of the
% \textbf{GWAS} has begun. The goal of this type of studies is to screen the
% \textit{pangenome} looking for associations between genetic variants, for
% example to study the outcomes of diseases. In this particular historical
% period, it is impossible not to mention viruses. In fact, by nature, viruses
% replicate a lot but often in a not perfect way, during infections. This produces
% many inexact clones, referred as \textbf{viral haplotypes}, which, taken
% together, form the \textbf{viral pangenome}. The identification of all these
% haplotypes is crucial both to the study of the spread of viruses and to the
% production of efficient drugs, in a context of high pharmacological resistance.
% One of the most important data structures, developed to handle the
% study of haplotypes sequences, is the \textbf{positional Burrows-Wheeler
%   transform (\textit{PBWT})}, proposed by Durbin in 2014
% \cite{pbwt_durbin}. \textit{PBWT} 
% aims to represent efficiently an haplotype panel with bi-allelic sites, storing
% them in a queryable 
% compressed way (details will be described later).
% According to Durbin, the used
% memory by \textit{PBWT} is nearly six times smaller than the raw data.
%\pb{qua dovresti essere più specifico sull'obiettivo di ridurre la memoria
%  per tenere gli aplotipi e fare analisi come l'estrazione di pattern
%  comuni...non hai menzionato le questioni della memoria...la compressione
%  serve a quello}
% From a computational point of view, he aim is to study panels built on an
% alphabet $\Sigma\in\{0,1\}$, for example, to extract common patterns in a set of
% haplotypes, computing the so-called \textit{blocks} (maximal intervals of
% columns, not extendable in any direction, such that there is a substet of
% identical rows).
%\dc{Ricordando quanto detto dalla prof Bonizzoni ho citato molto brevemente il
%concetto di blocco} 
% As introduced, haplotypes panel can be thought of as a particular case of 
% \textit{pangenome}, with fixed-length genomes, so we can understand the
% importance of such compressed and efficient data structure in computational
% pangenomics. 
%\dc{Quest'ultima parte devo sistemarla meglio}
% In addition, recent papers, such as \textit{IMPUTE5} \cite{impute5}, are based
% on the use of \textit{PBWT} for inference of unobserved genotypes in a sample of
% individuals. 
% In addition, the use of the \textit{PBWT} is found in many 
% software for the study of haplotypes and in various genotype imputation
% methods that infer unobserved genotypes in a sample of individuals
% \cite{impute5}. 
% \pb{va spegato meglio}
%\pb{idem qua ...se citi anche il risparmio di memoria in sintesi può essere utile}
% During the development of my master thesis, I worked to create a run-length
% encoded variant of the \textit{PBWT}, the \textbf{RLPBWT}, using and adapting
% the various theories developed for the \textit{RLBWT}, in collaboration with
% the authors of \textit{MONI} and \textit{PHONI}.
% \pb{di in sintesi cosa è l'idea del run-length encoding e sopra direi cha la
% BWT ha rivoluzionato gli algoritmi e i tools usati in Bioinformatica perchè è
% possibile indicizzare il genoma e i genomi in poco spazio ...dire come si
% riduce lo spazio ad esempio se si passa a MONI}
% \dc{Aggiunta breve spiegazione al RL quando introdotto}
% \dc{Aggiunto risparmio di memoria di MONI rispetto a BWA-MEM}
% In this context, my Ph.D. is going to be focused on the development of new
% algorithms in 
% various topics related to open problems in the study of the
% \textit{haplotyping/genotyping}, of the \textit{genome 
%   variants} and of the imputation issues related. My intention is also to
%   deepen 
% experimental themes  
% of \textit{pattern matching} and of the \textit{pangenome graph}, in detail
% the 
% new developments on \textit{BWT} and indexing structures, as well as
% \textit{succinct data structures}, with particular attention to the use of
% \textit{bitvectors}. PARTE MESSA A FINE INTRODUZIONE
%\pb{se rimandi a dopo allo stato dell'arte queste cose la mia paura è che non
%capisco questa parte...devi a mio parere antcipare l'indispensabile prima
%dicendo che i dettagli li dai nella sezione dello stato dell'arte} 
%\pb{lo stato dell'arte serve a dire cosa si è fatto prima e cosa tu pensi di
%fare per avanzare nello stato dell'arte...di solito serve ad evidenziare i
%problemi aperti}

\subsubsection*{State of the art}
% \gdv{Questa sezione deve essere fusa con preliminaries, in modo da evitare
%   ripetizioni. Al momento in entrambe le sezioni si parla di RLBWT, MONI.}
Now I present a brief overview of the main algorithms, data structures and tools
that are the state of the art of computational pangenomics.  
% \paragraph*{BWT}
% The \textbf{Burrows-Wheeler Transform (\textit{BWT})} was introduced
% in 1994 in order to compress texts but it has been used widely
% in bioinformatics, above all thanks to the already cited
% \textit{FM-index}. Given a text $T$, \$-terminated, such that $|T|=n$, we can
% define, denoting with $SA_T$ the \textit{suffix array} of $T$, the $BWT_T$ as:
% $BWT_T[i] = T[SA_T[i]-1]$, if $SA_T[i]>0$, and $BWT_T[i] = \$ $ otherwise. Less
% formally, we can say that $BWT_T[i]$ is the character that precedes the $i$-th
% suffix in the lexicographical order. It is important to note that this
% transform is reversible, so we can reconstruct the text $T$ from its transform
% $BWT_T$ using the \textbf{LF-mapping}. Given $BWT_T$ and an array,
% called $F_T$, with all the characters of $T$ in the lexicographical order, we
% can say that, thanks to the \textit{LF-mapping}, the $j$-th occurrence of a
% certain character in $BWT_T$ corresponds to the $j$-th occurrence of the same
% character in $F_T$. Thank to this we can reconstruct $T$ starting from its last
% character 
% \$. With the use of the \textit{LF-mapping}, we can perform the
% \textit{backward-search} in order to use the $BWT_T$ to look for a pattern $P$
% within $T$. This can be done efficiently thank to the \textit{FM-index} which
% consists of two functions. The first one is $C$ function, such that, given an
% alphabet $\Sigma$ (that includes the ending character \$),
% $C:\Sigma\to[1,n]$. This function, given a character $\sigma\in \Sigma$, returns
% the number of occurrences of characters lexicographically minor in $T$ than the
% one given as argument. The second one is the $Occ$ function,
% $Occ:\Sigma\times[1,n]\to[1,n]$, which has a character $\sigma\in\Sigma$ 
% and an index $i$ of $BWT_T$ as arguments and returns the count of occurrences of
% $\sigma$ in $BWT_T[1,i]$.\\
% The use of \textit{BWT} has allowed the construction of efficient algorithms
% both in the field of pattern matching and sequence alignment. 
%\pb{questa parte io la imposterei con titoli diversi che evidenziano da quali
%%problemi aperti dello stato dell'arte parti per poi sviluppare gli obiettivi
%del progetto...ad esempio sui Bitvectors ...se metti invece come titolo
%qualcosa tipo: 
%Efficient and compact boolean data structures and applications...
%Inizierei dicendo che una struttura dati estremamente utile per algoritmi di
%analisi sequenze è quello dei bitvectors utilizzati tra l'altro anche per
%implementare Bloom filters...sono compatti etec... magari anche dire chi ha
%fatto SDSL in che anno ...} 
% \paragraph*{Bitvectors}
\paragraph*{Efficient and compact boolean data structures.}
\dc{Non so se ho lo spazio per introdurre i bloom filters}
If $\mathcal{X}$ is the optimal number of bits needed to store some data, a
representation of this information is defined \textit{succint} if it takes
$\mathcal{X} +o(\mathcal{X})$ bits of space. 
\textbf{Bitvectors} are one of the most important \textit{succint data
  structures}.  
A \textit{bitvector} is an array on $n$ bits that allows two particular
operations, called \textbf{rank} and \textbf{select}, in addition to the classic
operations on boolean arrays, such as \textit{random access} in
\textit{constant time}. More in detail, the \textit{rank
function} allows calculating how many occurrences of one are up to a certain
index. Instead, the \textit{select function} allows obtaining the index of every
one present in the \textit{bitvector}. Formally, given a bitvector $B$, such
that $|B|=n$, and given an index $i$, such that $0\leq i<n$, we can define
$rank_B(i)=\sum_{k=0}^{k<i} B[k]$. Instead, about the select function, given an
integer $i$, such that $0<i\leq rank_B(n)$, we can define
$select(i)=\min\{j \,| \,\, rank_B(j+1)=i\}$.
From a theoretical point of view, these two operations can be supported in
\textit{constant time}, with the additional cost of $o(n)$ bits in
memory. There are several implementations of the 
same (for example \textit{plain bitvector, interleaved
  bitvector, sparse bivector}, etc$\ldots$) within \textbf{SDSL (\textit{Succint
    Data Structures Library})}, one of the most important C++ library used in
bioinformatics proposed by Gog et al. in 2014. Thanks to the various
implementations, both the computational time of the two main 
operations and the additional bits needed vary, allowing a
better choice of the best variant possible depending on the use case. Due to
their compactness in memory, \textit{bitvectors} are widely used in algorithms
for the analysis of biological sequences. An example of the use is 
tracking the runs in the run-length encoded implementations of \textit{BWT} and
\textit{PBWT}, where we put one at each head of every run, allowing fast
operations for indexing and mapping. 
%\paragraph*{RLBWT}
\paragraph*{Run-length encoding and succinct data structures.}
%\pb{Run-length encoding and succinct data structures... potrebbe essere il
%titolo generale qua...tieni presente che il run lengh encoding è una vecchia
%idea usata nella compressione dei dati alternativa a LZ.. Lempel Ziv...ed è
%stata estesa alla BWT...}
\dc{Non so se parlare di LZ}
The \textbf{Burrows-Wheeler Transform (\textit{BWT})} was introduced
in 1994 to compress texts but it has been used widely
in bioinformatics, above all thanks to the already cited
\textit{FM-index}.
Speaking of \textit{pangenome}, linear indexing via \textit{FM-index} is no
longer the 
best solution as it does not handle the large repetitions there are in this new
type of sequences and a simple but effective strategy to compress texts is based
on encoding a \textbf{run} of consecutive identical characters as the pair
(characters, length of the run). For example, the string \texttt{AAAAAA} is
encoded as (\texttt{A}, 6). 
Clearly, such an encoding achieves a better compression ratio the more
consecutive character we can find. In 2005 M\"{a}niken and Navarro defined 
the \textbf{Run-Length encoded Burrows–Wheeler Transform
  (\textit{RLBWT})}. Given a text 
$T$, $RLBWT_T$ is a representation of $BWT_T$ with a compact storage of
consecutive equal characters, the so-called \textit{runs}. With this new
perspective, some recent papers --- \textit{MONI}~\cite{moni} and
\textit{PHONI}~\cite{phoni} --- present two practical indexing tools that are
tailored for pangenomes. In these works, algorithms have changed from being
linear over the length of the text, $n$, to be linear over the number of runs,
$r$, so sub-linear over the length of the text.
The new indexing method, introduced by Gagie et al., was called \textbf{r-index} 
and it corresponds to the \textit{RLBWT} plus the \textit{suffix array sampling}
at the beginning and the end of every run. \\
To better describe how \textit{pattern matching} works in these papers, we need
to introduce some definitons. First of all, the main purpose of these algorithms
is to compute \textbf{Maximal Exact Matches (\textit{MEMs})} of a pattern in a
text. Given a text $T$ and a pattern $P$, a substring of the pattern $P[i :
i+l-1]$, of length $l$, is a \textit{MEM} of $P$ in $T$ if $P[i:i+l-1]$ is a
substring of $T$ but neither $P[i-1:i+l-1]$ nor $P[i:i+l]$ are, that is the
match cannot be extended in any direction. The authors state that computing
\textit{MEMs} is equivalent to computing \textbf{matching statistics
  (\textit{MS})}. Formally, the \textit{MS} of a pattern $P$ in respect to a
text $T$ is an array $M$ of pairs position/length ($pos$/$len$), $|M|=|P|$, such
that $T[M[i].pos:M[i].pos+M[i].len-1]=P[i:i+M[i].len]-1]$ and $P[i:i+M[i].len]$
does not occur in $T$. To compute \textit{MS}, \textit{random-access} over the
text is required  but it is also required that the text is stored in a
compressed way. For this purpose, the authors get a compressed 
representation of the text, via a \textbf{straight-line program
(\textit{SLP})}, that is based on a \textit{grammar-compression} algorithm 
(by a \textit{context-free grammar}). To compute \textit{MS}, in \textit{MONI},
the authors used the so-called \textbf{thresholds}, defined as the minimum
\textbf{Longest Common Prefix (\textit{LCP})} value between two 
consecutive runs of the same character This solution requires two sweep over the
\textit{MS array} and \textit{random access} over the \textit{SLP}. The
improvement in \textit{PHONI} is due to the fact that it was decided to use
\textbf{longest common extension (\textit{LCE}) queries}, to compute
\textit{MS}, by a single sweep. Formally, given two positions $i$ and $j$ in a
text $T$, such that 
$|T|=n$, an \textit{LCE query} compute the length of the longest common prefix
between $T[i:n-1]$ and $T[j:n-1]$. So, \textit{LCE} is the right equal common
extension between two positions in the text. Summarizing, after computing the
\textit{MS array}, as stated by Bannai et al., it is possible to obtain
\textit{MEMs}. Furthermore, using a particular
function called $\varphi$ (and its inverse $\varphi^{-1}$), based on the use
of the \textbf{inverse suffix array (\textit{ISA})}, it has been possible to
find all starting positions of all copies of $P$ in $T$ from starting position
of the match extracted by \textit{MS}, quickly calculating, given a position $p$
in $SA$, the previous and next index stored in the suffix array. \\
Thanks to these works, it is
possible to perform \textit{pattern matching} 
efficiently even on long sequences of nucleotides, such as those studied in
a pangenomic context. In fact, for example, according to Rossi et al.,
\textit{BWA-MEM}, one of the most used \textit{read aligners}, uses between 1.1
and 3.8 times more memory than \textit{MONI}, having further improved results in
\textit{PHONI}.  
% Briefly, thanks to \textit{LCE
%   queries}, we can compute efficiently \textit{MS array}, calculating both
% position and length in a single sweep. After computing \textit{MS} we have a
% kind of representation of the \textit{MEMs} themselves.\\
%  The algorithm for querying through
% the \textit{RLBWT} takes advantage of other methods, such as the use of
% \textbf{thresholds}, defined as the minimum \textbf{Longest Common Prefix
%   (\textit{LCP})} value between two 
% consecutive runs of the same character, in \textit{MONI}, and \textit{LCE
% queries} in \textit{PHONI}, achieving the same results with only a single sweep
% over \textit{MS} array instead of two as using the \textit{thresholds}. 
%Instead, in
%\textit{PHONI}, the authors use the \textbf{longest common extension
%  (\textit{LCE}) query}, to compute the \textit{MS}. Formally, an \textit{LCE
%  query}, given two positions $i$ and $j$ in a text $T$, such that $|T|=n$,
%compute the length of the longest common prefix between $T[i:n-1]$ and
%$T[j:n-1]$, so an \textit{LCE} is the right equal common extension 
%between two positions in the text. 
%Both solutions operate on a compressed representation of the text, via a
%\textbf{straight-line program (\textit{SLP})}. Briefly, an \textit{SLP} is based
%on a \textit{grammar-compression} algorithm, by a \textit{context-free grammar},
%and here it is used for \textit{random access} and \textit{LCE queries}. 
%Given a pattern $P$ and a text $T$, the \textit{MS}
%of $P$ in respect to $T$ is an array $M$ of pairs position/length %($pos$/$len$),
%$|M|=|P|$, such that $T[M[i].pos:M[i].pos+M[i].len-1]=P[i:i+M[i].len]-1]$ and
%$P[i:i+M[i].len]$ does not occur in $T$.
%Given a
%text $T$ and a pattern $P$, a substring of the pattern $P[i : i+l-1]$, of length
%$l$, is a \textit{MEM} of $P$ in $T$ if $P[i:i+l-1]$ is a substring of $T$ but
%neither $P[i-1:i+l-1]$ nor $P[i:i+l]$ are, so if the substring cannot be
%extended either to the right or to the left. 

% More formally, given a starting position
% $p$, $\varphi(p)=SA[ISA[p]-1]$, $NULL$ if $ISA[p]=0$, and
% $\varphi^{-1}(p)=SA[ISA[p]+1]$, $NULL$ if $ISA[p]=|T|-1$, where $ISA[i]=j$ iff
% $SA[j]=i$.\\
\paragraph*{Genotyping variants problems and GWAS.}
% \paragraph*{PBWT.}
Several computational problems, arising from computational
biology, have led to the development and of algorithms and data
structures for analyzing specific sequence data such as \textit{haplotypes} and
\textit{genotypes} in the \textbf{genotyping variants problem}. Briefly, we
could define \textit{haplotypes} as a combination of allelic variants, each one
inherited from a parent. Instead, the \textit{genotype} is combined information
of the haplotypes. For example, humans have two haplotypes, being diploid, and
the genotype is their combined information. So, since 2005, publication of the
\textbf{GWAS} has begun. The goal of this type of studies is to screen the
\textit{pangenome} looking for associations between genetic variants, for
example to study the outcomes of diseases. In this particular historical
period, it is impossible not to mention viruses. In fact, by nature, viruses
replicate a lot but often in a not perfect way, during infections. This produces
many inexact clones, referred as \textbf{viral haplotypes}, which, taken
together, form the \textbf{viral pangenome}. The identification of all these
haplotypes is crucial both to the study of the spread of viruses and to the
production of efficient drugs, in a context of high pharmacological
resistance.\\
In 2014, Durbin proposed the
\textbf{positional Burrows–Wheeler transform (\textit{PBWT})} to
solve the problem of \textit{pattern matching} on panels of haplotypes,
with efficient compression. According to the author, the used
memory by \textit{PBWT} is nearly six times smaller than the raw data.
In detail, he analyzed a biallelic-sites panel $X$, built on an
alphabet $\Sigma\in\{0,1\}$, with $M$ haplotypes/individuals and $N$ sites. This
data structure is based on a \textit{reversed-prefix ordering} at 
each column $k$ that produces two different multidimensional arrays. The first
one is the set of the \textbf{prefix arrays}, denoted by $a$, which contains the
index of the haplotype $m$ in the original panel, for each column $k$ and
each position $i$ of $a_k$. More formally, we can say that $a_k[i]=m$ iff $X_m$
is the $i$-th haplotype in the reversed-prefix ordering at column $k$. We can
note $x_m$, such that $a_k[i]=m$, could be denoted by $y_i^k$. The second
bidimensional array is the set of the \textbf{divergence arrays}, denoted by
$d$, which indicates the index of the starting column of the longest common
suffix, ending in column $k$, between a row and its previous one, at
reversed-prefix ordering at column $k$. More formally, we can define $d_k[i]=h$
iff $h$ is the smallest column index such that $y_i^k[h,k)=y_{i-1}^k[h,k)$. 
Thanks to these two bidimensional arrays, it is possible to compute all matches
within $X$ longer than a minimum length $L$, all set-maximal matches within $X$
in linear time and all set-maximal matches, maximal in the width of the match,
from a new sequence $z$ to $X$ in $\mathcal{O}(M^2N)$. Another interesting use
of the \textit{PBWT} is to extract common patterns in a set of
haplotypes, computing the so-called \textit{blocks} (maximal intervals of
columns, not extendable in any direction, such that there is a substet of
identical rows).
Since its development, there has been a growth in
research based on it, both in terms of variant design, such as the
\textbf{multiallelic PBWT} or the \textbf{dynamic PBWT}, where the static PBWT
is generalized to a dynamic data 
structure via linked lists, and in terms of its use to study haplotype panels.
A first example could be found in the paper of Alanko et al., published in
2021, where the authors used the \textit{PBWT} to look for \textit{maximal
  perfect haplotype block}, within a haplotypes panel. These types of
algorithms are essential for the identification of genomic regions that show
signatures of natural selection.
Another interesting paper is the one by Williams and Mumey, published in 2020,
who also studied \textit{maximal perfect haplotype blocks}, but with the
addition of a \textit{missing data} management attempt, handled with the help of
wildcards, using the \textit{PBWT}. 
A very intriguing tool, talking about \textit{GWAS}, is
\textbf{IMPUTE5} \cite{impute5}, proposed by Rubinacci et al. in 2020. The main
purpose of this 
software is making \textit{genotype imputation} to predict unobserved
genotypes from a panel with millions of haplotypes, so it was necessary to use
\textit{PBWT}. Improving this data structure, it will be possible to help the
\textit{GWAS} on tumors and other disorders.\\
To concludes, as introduced,
haplotypes panel can be thought of as a particular case of  
\textit{pangenome}, with fixed-length genomes, so we can understand the
importance of such compressed and efficient data structure in
\textit{computational pangenomics}. 
% Durbin himself, the author of the \textit{PBWT}, with Shchur et al. in 2019,
% proposed a use of his data structure in \textit{GWAS} context. In fact, they
% studied the associations between genetic variants in order to identify signals
% of natural selection and to build the so-called \textbf{ancestral recombination
%   graph}, that contains complete information about history of samples.
% For the sake of completeness, an example where \textit{PBWT} is not used is
% \textbf{Ranbow}, proposed by Moeinzadeh et al. in 2020. The 
% aim of this project was the haplotype reconstruction of polyploid genome from
% short read sequencing data, studying also the multi-allelic case.
% togliere ranbow e paper durbin
\subsubsection*{Research goals}
For my master's thesis, I had to adapt the concept described for
\textit{RLBWT} in \textit{MONI} and \textit{PHONI}. 
% More formally, given a panel $X$ (where every row is 
% defined by $X_j$) and a pattern $P$, $|P|=X_{width}=n$, we define \textit{MS} as
% an array of length $n$, such that, for each position $1\leq i\leq n$,
% $X_{MS[i].row}[i-MS[i].len+1: i]=P[i-MS[i].len+1:i]$ and $P[i-MS[i].len:i]$ is
% not a suffix of $X_1[1:i],\ldots,X_{X_{height}}[1:i]$.
% Thresholds are defined
% by the index of the minimum \textit{LCP value} inside a run, considering also
% the \textit{LCP value} of the head of the next run, if it exists. We stretch the
% reverse panel (from the right to the left) to get the \textit{SLP}, in order to
% make \textit{LCE queries} possible. This has to be done because \textit{LCE
%   queries} are computed, between two rows, from the right to the left. Instead,
% regarding the $\varphi$  function and its inverse, I have implemented a new
% simple data structure to identify which prefix array values are above and below
% a given index in a column of the \textit{RLPBWT matrix}, that is the panel
% permuted via the prefix array and run-length encoded.
Obviously, there are some limitations, as in Durbin's \textit{PBWT}, such as
studying only biallelic panels and ignoring the management of missing data,
which are very frequent in real cases. \\
During my master’s degree, I have deepened some theoretical
issues related to algorithms, especially in bioinformatics, and
some modeling and inference topics, related to computational 
systems biology. So, it is my interest to continue my studies with the Ph.D.
to be able to deepen the computer science potential in the biological
context.
Summarizing, the most important research goal of my Ph.D. is designing and
validating algorithms and data structures to deal with open problems in
pangenomics, such as: 
\vspace{-1.25mm}
\begin{enumerate}[leftmargin=.2in]
  \setlength\itemsep{-0.2em}
  \item \textbf{Multiallelic data}. Thanks to the growth in the
  production of genotypic data, the number of multiallelic sites is expected to
  grow, as well as the number of individuals (even if there is only a 2\%
  presence of triallelic sites, actually known in the human genome)
  %\pb{spiegare in sintesi cosa è il sample che viene menzionato anche sotto}.
  Moreover, not only 
  such sites could be more than expected but they are 
  usually not considered by the majority of tools. The first step in this
  direction was made 
  by Naseri with the already cited \textit{m-PBWT}.
  Talking about space complexity, the run-length encoded version of the stored
  arrays for the \textit{FM-index} could allow the management of increasingly
  large data for the imputation phases.   
  In this context, the aim is to implement a new 
  version of the \textit{RLPBWT} that can manage multiallelic data, adapting the
  current use of bitvectors to handle efficiently also the \textit{LF-mapping}
  with more than two alleles. 
  \item \textbf{Missing data}. A second extension, that would be more
  complicated to 
  design, is a \textit{RLPBWT} version that admits the presence of
  missing data. Most algorithms and data structures mentioned above assume that
  they work on exact data. In reality, real data can contain errors or even
  gaps, both mainly due to the imperfections of sequencing technologies,
  although now they have high correctness rates. Unfortunately, this is
  still an open problem, even if most of the errors are corrected in a
  preliminary step (mostly by heuristic methods).
  Handling missing data is known to be \textit{hard} because every gap could
  assume any possible value. Having said this, I should  
  probably deal with \textit{parametric} or \textit{approximate algorithms},
  based on research already developed in \textit{BIAS}.
  \item \textbf{Data structures and algorithms for the representation and
    indexing  of pangenome graphs}. The above mentioned  research 
  goals are relevant in the context of a more general goal which is to deepen 
  the study of data structures for  representing the information of
  \textit{pangenome graphs}. In fact, it is possible to interpret 
  haplotype sequences as a 
  pangenome graphs, in a compact way via the \textbf{Graph BWT (\textit{GBWT})},
  even if it is more a \textit{multi-string BWT} than a classical graph. A first
  improvement would be space optimization. There are a lot of open problems 
  regarding the use of \textit{GBWT}, such as handling missing/erroneous data
  and representing complex and nested 
  variants. Once I have tried to solve the problem of missing data for the
  \textit{PBWT/RLPBWT}, I could think of an adaptation 
  for the \textit{GBWT}, because it would be possible to
  analyze more precisely real data. I would also point out my interest in
  deepening all the other issues related to 
  pattern matching, indexing structures and pangenome graphs, according to the
  \textit{BIAS} laboratory's topics of interest. So far there has been talk of
  \textit{bitvector-based} and \textit{BWT-based indexing}. There are other data
  structures that would be interesting to study and extend, such as
  \textit{wavelet tree}, another \textit{succint data structure} used to
  compress texts. The design of new indexing structures, both for strings and
  graphs, could lead to very important results in the field of computational
  pangenomics, for example regarding the \textit{string-graph alignment problem}.
  \dc{Eventualmente penso si possa aggiungere qualcosa sull'allineamento tra
    stringhe e il grafo del pangenoma... ma non so quanto si finisca "off-topic"
    rispetto a quanto discusso nei research goals} 
  % chiarire questa parte
  %\item \textbf{Imputation}.
  % As introduced, I have
  % developed a wide-range interest in the application of computer science to
  % biology. So, I do 
  % not intend to forget the actual use of the data structures that I will study
  % and develop during my Ph.D.. The management of missing and
  % multiallelic data can further improve the state of the art of \textit{GWAS}
  % and of genotype imputation, allowing even more precise inference of unobserved
  % genotypes. All this must be read in the perspective of 
  % a continuous rise of available data, not only regarding the human
  % sequencing. \\ 
\end{enumerate}
\vspace{-1.25mm}
%\dc{Mancano lib per pangenoma (anche lato benchmark)}
The theoretical research requires an experimental verification, for this purpose
some technologies that will be used during my Ph.D. must be mentioned. 
During both my bachelor's and master's degree, I mainly focused on the use of
the \textbf{C++ programming language}, thanks to the
availability of efficient libraries, such as the already 
cited \textit{SDSL}. However, in
addition to \textit{C++}, I had the opportunity to deepen 
\textbf{Python}, with libraries such as \textit{biopython}, and \textbf{Rust},
with recently developed libraries such as \textit{bio-rust} (even if still not
complete from an algorithmic point of view). Always from an experimental point
of view, there is currently a lack of libraries, in any programming language,
dedicated to the study of \textit{pangenome}, also with regard to benchmarking.
It is therefore my interest to contribute to the development of such open
source libraries. Another point of interest is the
analysis and the development of efficient algorithms based on parallel computing
(also on GPU), that are increasingly in use in both \textit{bioinformatics} and
\textit{computational systems biology}.  
\gdv{Secondo me bisogna estendere la parte di ricerca, mettendo qualcosa di più
  a lungo termine. Ad esempio pensare a tecniche di indicizzazione diverse, non
  necessariamente basate su BWT e bitvectors} 
\dc{Allo stato attuale abbiamo una pagina di goals. Non so se possa bastare}
To conclude this proposal, I also would point out the intention to remain in
contact with various researchers, with whom I have already collaborated during 
my master thesis, including Christina Boucher (University of Florida)
and Travis Gagie (Dalhousie University).

\printbibliography

\end{document}