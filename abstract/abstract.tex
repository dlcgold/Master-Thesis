\documentclass[a4paper,11pt, oneside,italian]{article}

\usepackage[utf8]{inputenc}
\usepackage{hyperref}
\usepackage{framed}
\usepackage{xcolor}
\usepackage{amsmath}
\usepackage{enumitem}
\usepackage{relsize}
\usepackage{microtype}
\usepackage{typearea}

\hypersetup{
  pdftitle = {Algoritmi per la trasformata di Burrows-Wheeler Posizionale con
    compressione run-length}, 
  pdfauthor = {Davide Cozzi},
  pdfsubject = {Riassunto della tesi},
  pdfpagemode = UseNone
}


\usepackage{fancyhdr}
% \pagestyle{fancy}
% \fancyhead[LE,RO]{\slshape \rightmark}
% \fancyhead[LO,RE]{\slshape \leftmark}
\fancyfoot[C]{\thepage}

\title{Algoritmi per la trasformata di Burrows-Wheeler
  Posizionale con compressione run-length} 
\author{Davide Cozzi\\\smaller matr.~829827}
\date{}
\makeatletter
\renewcommand{\paragraph}{%
  \@startsection{paragraph}{4}%
  {\z@}{0.75ex \@plus 1ex \@minus .2ex}{-1em}%
  {\normalfont\normalsize\bfseries}%
}
\makeatother

\begin{document}
\maketitle
\setlist{leftmargin = 2cm}
\noindent
%\subsection*{Introduzione}
Negli ultimi anni si è assistito ad un cambio di paradigma nel campo della
bioinformatica, ovvero il passaggio dallo studio della sequenza lineare di un
singolo genoma a quello di un insieme di genomi, provenienti da un gran numero
di individui, al fine di poter considerare anche le varianti
  geniche. Questo nuovo concetto è stato nominato per la prima volta da
Tettelin, nel 2005, con il termine di \textit{pangenoma}. Grazie ai risultati
ottenuti in pangenomica ci sono stati miglioramenti sia nel 
campo della biologia che in quello della medicina personalizzata, grazie al
fatto che, con il pangenoma, si migliora la precisione della rappresentazione di
multipli genomi e delle loro differenze. 

Il genoma umano di riferimento (GRCh38.p14), è composto da circa
3.1 miliardi di basi, con più di 88 milioni 
varianti tra i genomi sequenziati, secondo i risultati ottenuti nel 1000 Genome
Project. Considerando come la quantità dei dati di sequenziamento sia destinata
ad aumentare esponenzialmente nei prossimi anni, grazie al
miglioramento delle tecnologie di sequenziamento (Next Generation Sequencing e
Third-Generation Sequencing), risulta necessaria la costruzione di algoritmi e
strutture dati efficienti. 
In merito, uno degli approcci più usati per rappresentare il pangenoma è un
pannello di aplotipi, ovvero, computazionalmente, una matrice di $M$
righe, corrispondenti agli individui, e $N$ colonne, corrispondenti ai siti con
le varianti. Si specifica che, con il termine
aplotipo, si intende l'insieme di alleli, ovvero di varianti, che un organismo
eredita da ogni genitore.

In questo contesto trova spazio uno dei problemi fondamentali della
bioinformatica, ovvero quello del pattern matching. Inizialmente tale concetto
  era relativo allo studio di un piccolo pattern all'interno di un testo di
  grandi dimensioni, ovvero il genoma di riferimento. Ora, con l'introduzione 
del pangenoma, tale problema si è adattato alle nuove strutture
dati.

Lo scopo di questa tesi è ottimizzare il problema del pattern 
matching, inteso come ricerca di match massimali esatti (MEM) tra un aplotipo
esterno e un pannello di aplotipi, in una delle 
strutture dati più utilizzata: la \textit{trasformata di Burrows-Wheeler
  Posizionale (PBWT)}. Il progetto di tesi, svolto in collaborazione con il
prof. Gagie (Dalhousie University) e la prof.ssa Boucher 
(University of Florida), ha quindi permesso lo sviluppo di una variante
\textbf{run-length encoded} della \textbf{PBWT}, detta \textbf{RLPBWT}, che
permettesse di risolvere tale problema.
\subsection*{Stato dell'arte}
Si introducono ora i principali algoritmi e le principali strutture dati,
presenti allo stato dell'arte, posti a fondamento di questo progetto di tesi.

La prima struttura dati che bisogna citare è la  ben nota \textit{trasformata
  Burrows-Wheeler (BWT)}, una trasformata reversibile che ha permesso di
ottenere algoritmi efficienti per il pattern matching su un testo.
La BWT, inoltre, è fortemente legata al concetto di Suffix Array (SA), definito
come la lista lessicograficamente ordinata delle posizioni di partenza di 
tutti i suffissi di un testo, e a quello delle lunghezze del Longest Common
Prefix (LCP) tra ogni suffisso consecutivo indicizzato nel SA.

Durbin, nel 2014, propose una struttura dati ispirata alla BWT, detta
\textit{Trasformata di Burrows-Wheeler Posizionale (PBWT)}, la quale viene
costruita a partire da un pannello di aplotipi, rappresentato, riferendosi al
solo caso bi-allelico, tramite una matrice binaria. Il funzionamento di tale
struttura dati prevede la costruzione di due insiemi di array, tramite
l'\textit{ordinamento dei prefissi inverso} ad ogni colonna del pannello.
Tali insiemi sono detti \textit{insieme dei prefix array} e \textit{insieme dei
  divergence array}. Il primo, denotato $a$, contiene, per ogni colonna e ogni
posizione, l'indice dell'aplotipo nel pannello originale riordinato secondo
l'ordinamento inverso alla data colonna. Quindi, si ha, in ogni colonna, la
permutazione degli indici di riga secondo l'ordinamento alla colonna $k$-esima.
Il secondo insieme, invece, indica l'indice della colonna iniziale del suffisso
comune più lungo, che termina nella colonna nella colonna $k$, tra una riga e la
sua precedente secondo l'ordinamento inverso ottenuto per la $k$-esima colonna.
Il pannello ottenuto con le permutazioni dettate dal prefix array viene chiamato
matrice PBWT.

Cruciale è il fatto che tale struttura permetta di calcolare i MEM tra
un aplotipo esterno e il pannello in tempo $\mathcal{O}(MN)$, mentre 
una soluzione naive impiegherebbe un tempo proporzionale a
$\mathcal{O}(M^2N)$. Il tradeoff di tale algoritmo, conosciuto anche come
\textit{algoritmo 5 di Durbin}, è la richiesta in termini di spazio. L'autore
stesso infatti stima la richiesta di $13NM$ byte in memoria per poter eseguire
l'algoritmo e superare questo limite è l'obiettivo principale di questo progetto
di tesi. 

Tornando a parlare della BWT, una caratteristica di questa trasformata è la
tendenza a produrre una sequenza con caratteri uguali in posizioni
consecutive. Questo fenomeno è dovuto alle ripetizioni di determinate
sottostringhe nel testo, avendo che tali ripetizioni, per motivazioni
biologiche, sono frequenti in ambito genomico.
Al fine di sfruttare tali caratteri uguali consecutivi si 
è quindi ideata la \textit{trasformata Burrows-Wheeler Run-Length
  encoded (RLBWT)} dove una sequenza massimale di caratteri uguali, detta 
run, viene memorizzata in modo efficiente come coppia (carattere,
lunghezza della run). Ad esempio, la stringa \texttt{aaaaaa} sarebbe memorizzata
come \texttt{(a,6)}.
Alcuni paper recenti hanno proposto un nuovo tipo di
indicizzazione, il cosiddetto \textit{r-index}, per questa struttura
dati compressa. In tal modo, si ottiene un indice che non lineare sulla
lunghezza del testo ma sul numero di run della sua BWT. L'r-index include la
RLBWT e un suffix array sample, ovvero i valori del SA all’inizio e alla fine di
ogni run. 
Tali articoli hanno portato alla produzione di due tool, \textit{MONI} e 
\textit{PHONI}, alle cui tecniche è fortemente ispirata questa tesi. Queste due
soluzioni hanno entrambe l'obiettivo di 
calcolare i MEM tra un testo e un pattern, ovvero sottostringhe del pattern che
sono anche sottostringhe del testo, avendo che tale match non può essere esteso,
in entrambe le direzioni, senza introdurre un mismatch. 

Gli autori hanno rilevato come il calcolo dei MEM sia direttamente
correlato alla costruzione dell'array delle \textit{Matching Statistics (MS)}.
Tale array, lungo quanto il pattern e formato da coppie (posizione $pos$,
lunghezza $len$), annota, per ogni posizione del pattern, la lunghezza $len$ di
un match, anche non massimale, con una sottostringa nel testo avente
indice iniziale $pos$. Estraendo da tali match i MEM si possono poi ottenere
tutte altre occorrenze del medesimo MEM nel testo. 
Gli autori, per il calcolo delle MS, hanno
usato, in MONI, anche il concetto di \textit{threshold}, definito come
il minimo valore dell'array LCP tra due run consecutive dello stesso
carattere. Un ulteriore miglioramento si è registrato in \textit{PHONI} con
l'uso delle \textit{LCE query} per il calcolo dell'array MS. Infatti, tale
struttura, dati due indici del testo $i$ e $j$, restituisce il più lungo
prefisso comune tra il suffisso $i$-esimo e il suffisso $j$-esimo.

Si può quindi apprezzare il collegamento naturale che si ha tra la PBWT
e la BWT, avendo, ad esempio, che il prefix array della prima corrisponde
al suffix array della seconda mentre il divergence array è una
diversa rappresentazione dell'array LCP. Anche grazie a queste premesse, si è
basato il progetto di tesi sulla costruzione di una versione run-length encoded
della PBWT.

% Al fine di raggiungere tale obiettivo sono state usate altre nozioni. In primis,
% si sono sfruttate le cosiddette \textbf{strutture dati succinte}, ovvero
% strutture per le quali, assumendo che $\mathcal{X}$ sia il numero di bit
% ottimale per memorizzare dei dati, si richiede uno spazio in memoria pari a
% $\mathcal{X}+o(\mathcal{X})$. Nel dettaglio si sono usati i cosiddetti
% \textbf{bitvector sparsi}, strutture che richiedono in memoria $\approx
% m\left(2+\log\frac{n}{m}\right)$ bit, con $n$ lunghezza del bitvector e $m$
% numero di simboli $\sigma=1$ in esso. L'efficienza delle operazioni che si
% possono fare con tali strutture dati, incredibilmente efficienti dal punto di
% vista dello spazio occupato, sono uno dei punti cruciali del funzionamento della
% \textit{RLPBWT}.\\
% Infine, per memorizzare il pannello in modo compatto, si è usata una
% struttura dati, detta \textbf{Straight-Line Programs (\textit{SLP})}. Tale
% struttura è una \textbf{grammatica context-free}, che
% genera una e una sola parola, la quale permette sia di effettuare \textit{random
% access} al testo, non in tempo costante, che di calcolare le \textit{LCE
% query} in modo efficiente. 
\subsection*{Contributo}
L'idea con la quale si è sviluppata questa tesi vede, come punto di partenza,
alcuni primi risultati teorici, presenti in letteratura, che caratterizzavano
una variante run-length encoded della PBWT. Purtroppo, tale proposta presentava
alcune ridondanze nei dati memorizzati e non caratterizzava alcuna tecnica
algoritmica per il calcolo di MEM.

% Il primo approccio aveva come obiettivo l'ottenere un riadattamento ``diretto''
% dell'\textit{algoritmo 5 di Durbin}, pur tenendo in memoria informazioni legate
% principalmente alle run della \textit{matrice PBWT}. Per quanto si siano trovate
% soluzioni interessanti per gestire il \textit{mapping} tra una colonna e la sua
% successiva (ma anche tra una colonna e la sua precedente), per poter
% ``seguire'' una riga del pannello originale nella 
% \textit{matrice PBWT}, si dovute memorizzare anche quantità non relative alle
% run.  Infatti, è risultato necessario memorizzare l'intero
% \textit{divergence array} (in forma di \textit{LCP array} quindi
% memorizzando la lunghezza del prefisso comune in ordine inverso e non la colonna
% d'inizio dello stesso), al fine di poter computare i \textit{MEM} con
% un aplotipo esterno. Inoltre, l'assenza delle informazioni relative al
% \textit{prefix array} ha impedito di poter annotare quali righe del pannello
% presentassero un match massimale esatto, terminante in una certa colonna, con
% l'aplotipo query, avendo solo 
% informazioni relative alla cardinalità di tale sottoinsieme di righe. Le
% informazioni in memoria si è 
% stimato non fossero ottimali, non solo per il \textit{divergence array}, ma anche
% perché non veniva utilizzato alcun approccio tramite \textit{strutture dati
%   succinte}.
Per quanto l'intuizione iniziale fosse quella di adattare l'algoritmo 5 di
Durbin all'uso con una struttura run-length encoded, non si è trovato un modo
per memorizzare i dati del divergence array in modo proporzionale al numero di
run. Inoltre non si è trovata una soluzione efficiente per utilizzare i sample
del prefix array ad inizio e fine di ogni run.

Si è quindi deciso di cambiare approccio al problema, ispirandosi ai risultati
già ottenuti per la RLBWT. Tale approccio si basa anche sull'uso dei bitvector
sparsi, una struttura dati succinta che permette di memorizzare e interrogare
vettori binari efficientemente.
Quindi, le strutture necessarie all'indicizzazione delle run e al cosiddetto
mapping, ovvero il ``seguire'' una riga dalla sua posizione permutata in una
certa colonna alla posizione permutata in quella successiva nella matrice PBWT,
sono state implementate tramite bitvector sparsi, con un numero di simboli
$\sigma=1$ proporzionale al numero run.
Al fine di avvicinarsi alle idee proposte in \textit{MONI} e \textit{PHONI} è
quindi servito uno studio teorico preliminare per ridefinire: matching
statistics, MEM, threshold e LCE query. In particolare, grazie agli 
ultimi due concetti, si è potuto non memorizzare il divergence array.
Parlando di matching statistics, si ha che sono definite tramite un array, lungo
quanto il pattern e formato da coppie (riga $row$, lunghezza $len$), che, per
ogni colonna, tiene traccia del suffisso comune che non sia estendibile a
sinistra, terminate in quella colonna e lungo $len$, tra la riga $row$ del
pannello e l'aplotipo query. Una volta calcolato tale
array è possibile estrarre un MEM, terminate in colonna $k$ tra la query e la
riga $row$, ed estendere, tramite una struttura dati a supporto, tale risultato
a tutte le altre righe del pannello che presentano il medesimo MEM.
Il primo metodo di calcolo di tale array è stato basato sull'utilizzo delle
threshold. Ragionando sulla \textit{matrice PBWT}, una threshold è definita
come il minimo valore dell'\textit{array LCP} all'interno di una run
(comprendendo anche la testa, qualora esistente, della run successiva, essendo
il suo valore \textit{LCP} calcolato sfruttando anche la coda della run
corrente). L'insieme delle threshold per una certa colonna è stato memorizzato
come bitvector sparso. Inoltre, si è provveduto a tenere in memoria i
sample di prefix array a inizio e fine di ogni run.

Grazie all'uso delle threshold si è potuto sviluppare un algoritmo
efficiente, dal punto di vista della memoria richiesta, per il calcolo delle
matching statistics in due ``sweep'' sul pattern, calcolandone prima le
posizioni e poi, tramite random access al pannello, anche le
lunghezze. Tale soluzione richiedeva in memoria l'intero pannello, per il quale
si è scelto di usare uno Straight-Line Program (SLP), ovvero una grammatica
context-free che genera una e una sola parola, molto compatta in memoria, sulla
quale è possibile effettuare random access (in tempo non costante). Ad esempio,
un pannello $70000 \times 46538$, che normalmente richiede $\sim 3.3$GB in
memoria ($\sim 451$MB se compresso tramite GZIP), viene memorizzato, tramite
SLP, in $\sim 7$MB.

Riprendendo quanto fatto in \textit{PHONI} per
la RLBWT, si è, poi, deciso di risparmiare ulteriore spazio eliminando
l'uso delle \textit{threshold}. Per ottenere il calcolo dell'array della
matching statistics si sono quindi usate le LCE query, ridefinite in questo caso
come il suffisso comune più lungo tra due righe del pannello, eventualmente
fissando il termine a una colonna precisata. Il loro calcolo è permesso
in modo efficiente dall'uso dell'SLP. Calcolando la
lunghezza di tale suffisso comune, è possibile calcolare anche le
lunghezze delle matching statistics contemporaneamente al calcolo delle
righe della stessa. Quindi, si è ottenuto il calcolo completo di tale array in
un singolo ``sweep'' sul pattern, passando da un tempo proporzionale a
$\mathcal{O}(2N)$ a uno proporzionale a 
$\mathcal{O}(N)$. Si noti che il calcolo dei MEM viene fatto contemporaneamente
al calcolo delle lunghezze delle matching statistics quindi, con l'uso delle LCE
query, si è ridotto sia lo spazio necessario in memoria per la RLPBWT che il
tempo di calcolo dei MEM.

\subsection*{Risultati sperimentali}
L'obiettivo della fase sperimentale è stato quello di confrontarsi con
l'algoritmo 5 di Durbin, permettendo di verificare la
riduzione della memoria necessaria al calcolo di MEM. L'importanza di ottenere
dati quantitativi è anche dovuta al limite 
dato dalla non forte caratterizzazione asintotica, dal punto di vista della
complessità temporale, delle operazioni con le strutture dati succinte, fattore
che potrebbe comportare errori di 
valutazione. L'uso di tali strutture comporta, in ogni caso, un aumento atteso
dei tempi di calcolo.

Per questa sintesi, si è scelto di riportare alcuni risultati relativi ad un
pannello $70,000 \times 46,538$, che è stato interrogato con $30,000$ query, al
fine di ridurre la fluttuazione statistica dei risultati.
I risultati qui riportati, relativi alla PBWT, sono stati ottenuti usando
l'implementazione ufficiale della stessa. Per la RLPBWT, invece, si sono usate
le LCE query. 

In merito al tempo di calcolo si è ottenuto il risultato atteso avendo che con
la PBWT si ottiene il calcolo dei MEM in $\sim 411$s 
mentre con la RLPBWT in $\sim 1824$s, ovvero poco più del quadruplo del tempo.
Si è passato quindi ad analizzare i risultati in termini di memoria. Secondo le
stime di Durbin, gli array necessari al funzionamento dell'algoritmo 5
richiederebbero $\sim 13NM$ bytes, implicando, per il pannello in analisi, $\sim
40$GB di memoria. Anche sperimentalmente si sono registrati 
picchi di memoria prossimi ai $40$GB di memoria, confermando le stime. Invece,
per la RLPBWT il picco registrato è stato di appena $\sim 3$GB, richiedendo
circa il $93$\% di memoria in meno rispetto alla soluzione di Durbin.
\subsection*{Conclusioni}
In conclusione, si rileva come i risultati prefissati siano stati raggiunti,
producendo una struttura dati efficiente in memoria per la risoluzione del
calcolo dei \textit{MEM} tra un pannello di aplotipi e un aplotipo
esterno.  
Si segnala inoltre come siano possibili diversi sviluppi futuri, come
un'ulteriore ottimizzazione della struttura, sia in termini 
di gestione del mapping che, eventualmente, di gestione di più query
contemporaneamente. In merito a quest'ultimo si segnala, infatti, come Durbin
abbia proposto anche un algoritmo per il calcolo di MEM tra un pannello di
aplotipi e un pannello di query, fondendoli in un unico pannello su cui
costruire la PBWT e calcolare i match interni al pannello stesso. Tale
implementazione, nel setup di test 
descritto nella sezione precedente (che si noti comportare il perfetto input per
tale soluzione), risulta essere molto performante sia in
termini di tempo ($\sim 22$s) che di memoria utilizzata ($\sim
10,084$KB) durante l'esecuzione. 

Inoltre, sono possibili diverse generalizzazioni rispetto alle caratteristiche
del pannello. Per quanto i pannelli di aplotipi prodotti dal sequencing del
genoma umano raramente presentino siti multi-allelici si ha una 
presenza stimata, al momento, di circa il 2\% di siti tri-allelici, avendo che
tale percentuale risulti fortemente sottostimata. Una prima generalizzazione
quindi sarebbe quella di studiare pannelli multi-allelici, riformulando la
RLPBWT al fine di poter funzionare anche con pannelli costruiti su un
alfabeto arbitrario. Un'altra generalizzazione interessante riguarda
l'ammissione di 
dati mancanti all'interno del pannello stesso. La maggior parte delle soluzioni
attualmente sviluppate sono basate su una forte assunzione: non si hanno dati
mancanti. Una variante della RLPBWT che sia quindi in grado di
lavorare, eventualmente con algoritmi parametrici o algoritmi
  approssimati, 
su pannelli in cui sono presenti wildcard per rappresentare tali dati,
permetterebbe di fare studi più completi su dati reali, migliorandone gli usi,
ad esempio, nel campo della medicina personalizzata.\\ 
Si segnala, infine, come la RLPBWT sia potenzialmente in grado di
risolvere 
efficientemente anche altri task, come ad esempio la ricerca di k-MEM,
ovvero MEM con un aplotipo esterno che coinvolgano
esattamente $k$ righe nel pannello. \\
Le potenzialità di tale struttura sono quindi molteplici e, grazie al ridotto
consumo di memoria, si hanno le giuste premesse perché venga utilizzata per
gestire e interrogare grosse moli di dati reali, incrementando le capacità di
studio, previsione e inferenza che si possono avere grazie allo studio del
pangenoma.
\end{document}
% LocalWords:  pangenoma naive sottostringa
