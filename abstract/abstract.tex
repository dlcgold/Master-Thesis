\documentclass[a4paper,11pt, oneside,italian]{article}

\usepackage[utf8]{inputenc}
\usepackage{hyperref}
\usepackage{framed}
\usepackage{xcolor}
\usepackage{amsmath}
\usepackage{enumitem}
\usepackage{relsize}
\usepackage{microtype}
\usepackage{typearea}

\hypersetup{
  pdftitle = {Algoritmi per la trasformata di Burrows-Wheeler Posizionale con compressione run-length},
  pdfauthor = {Davide Cozzi},
  pdfsubject = {Riassunto della tesi},
  pdfpagemode = UseNone
}


\usepackage{fancyhdr}
% \pagestyle{fancy}
% \fancyhead[LE,RO]{\slshape \rightmark}
% \fancyhead[LO,RE]{\slshape \leftmark}
\fancyfoot[C]{\thepage}

\title{Algoritmi per la trasformata di Burrows-Wheeler
  Posizionale con compressione run-length} 
\author{Davide Cozzi\\\smaller matr.~829827}
\date{}
\makeatletter
\renewcommand{\paragraph}{%
  \@startsection{paragraph}{4}%
  {\z@}{0.75ex \@plus 1ex \@minus .2ex}{-1em}%
  {\normalfont\normalsize\bfseries}%
}
\makeatother

\begin{document}
\maketitle
\setlist{leftmargin = 2cm}
\noindent
\subsection*{Introduzione}
Negli ultimi anni si è assistito ad un cambio di paradigma nel campo della
\textit{bioinformatica}, ovvero il passaggio dallo studio della \textit{sequenza
  lineare di un singolo genoma} a quello di un insieme di
genomi, provenienti da un gran numero di diversi individui, al fine di poter
considerare anche le \textit{varianti geniche}. Questo
nuovo concetto è stato 
nominato per la prima 
volta da Tettelin, nel 2005, con il termine di \textbf{pangenoma}.\\
Grazie ai risultati ottenuti in \textit{pangenomica} ci sono stati
miglioramenti sia nel campo della \textit{biologia} che in quello
della \textit{medicina personalizzata}, soprattutto grazie ai
\textbf{genome-wide association studies (\textit{GWAS})}. Infatti, in tali
studi, si necessitano grandi collezioni di genomi al fine di poter
identificare associazioni tra \textit{varianti geniche} e la propensione
all'insorgenza di una
certa malattia o un suo particolare sintomo.\\
Il genoma umano,
prendendo come reference l'\textit{homo sapiens reference
genome GRCh38.p14}, è composto da circa
3.1 miliardi di basi, comportanti circa 59,000 geni. Inoltre, i risultati del
\textit{1000 Genome Project} hanno portato a identificare più di 88 milioni di
varianti tra i genomi sequenziati. Tra esse si hanno, ad esempio, 84.7 milioni
di \textbf{Single Nucleotide Polymorphisms (\textit{SNPs})}, ovvero differenze
di singole basi azotate, 3.6 milioni di piccole \textbf{inserzioni/delezioni
  (\textit{indel})} di basi e circa 60,000 \textbf{varianti strutturali},
ciascuna formata da più di 50 nucleotidi. Questi numeri fanno capire come siano
richiesti algoritmi e strutture 
dati efficienti in grado di gestire quest'incredibile mole di dati, considerando
che tale quantità è destinata ad aumentare nel prossimo futuro, grazie al
miglioramento delle tecnologie di sequenziamento (\textbf{Next Generation
  Sequencing (\textit{NGS})} e \textbf{Third-Generation Sequencing}).\\ 
In tale ottica, da un punto di vista computazionale, il
\textit{pangenoma} può essere rappresentato in molteplici modi, tramite
strutture dati che permettano di memorizzare tutte le varianti tra i genomi
dei diversi individui. Le due rappresentazioni principali sono:
il \textbf{grafo del pangenoma} e il \textbf{pannello di
  aplotipi}. In entrambi i casi sono necessarie metodologie efficienti sia per
la memorizzazione della struttura dati che per le interrogazioni alla stessa. In
questa tesi si è studiata la rappresentazione
tramite \textit{pannello di aplotipi}, dove con \textbf{aplotipo} si intende
l'insieme di alleli, ovvero di varianti, che un organismo eredita da ogni
genitore. L'informazione combinata di tutti gli aplotipi in un individuo è detta
invece \textbf{genotipo}.\\
In questo contesto trova spazio uno dei problemi fondamentali
della \textit{bioinformatica}, ovvero quello del \textbf{pattern
  matching}. Inizialmente tale concetto era relativo allo studio di un piccolo
pattern all'interno di un testo di grandi dimensioni. Ora, con l'introduzione
del \textit{pangenoma}, tale problema si è adattato alle nuove strutture
dati.\\
\textit{Lo scopo di questa tesi è ottimizzare il problema del
  pattern 
matching, tra un aplotipo esterno e un pannello di aplotipi, in una delle
strutture dati più utilizzat: la \textbf{trasformata di Burrows-Wheeler
  Posizionale (\textit{PBWT})}. Il 
progetto di tesi, svolto in collaborazione con il laboratorio BIAS (Università
degli Studi di Milano Bicocca), il professor
Gagie (Dalhousie University), la professoressa Boucher e il dottor Rossi
(University of Florida), ha quindi permesso lo sviluppo di una variante
\textbf{run-length encoded} della \textbf{PBWT} che permettesse di risolvere il
problema della ricerca di match massimali esatti (MEM) tra un pannello di
aplotipi e 
un aplotipo query.}
\subsection*{Preliminari}
Al fine di comprendere al meglio i metodi usati in questa tesi bisogna
introdurre alcuni concetti preliminari.\\
Il primo è quello della ben nota \textbf{trasformata Burrows-Wheeler
  (\textit{BWT})}, una trasformata reversibile che ha permesso di ottenere
algoritmi efficienti per il pattern matching. Questo è stato permesso grazie
all'indicizzazione tramite 
\textbf{FM-index}, un \textit{self-index} che permette di lavorare sulla
\textit{BWT} senza averla effettivamente in memoria (avendo in memoria solo
l'indice stesso). La \textit{BWT} è fortemente legata al concetto di
\textbf{Suffix Array (\textit{SA})}. Infatti, dato un testo $T$ lungo $n$ si ha
che $SA$ è la lista lessicograficamente ordinata delle posizioni di partenza di
tutti i suoi suffissi, avendo che $SA[i]=j$ se e solo se $T[j, |T|-1]$ è
l'$i$-esimo 
suffisso lessicograficamente minore di $T$, ovvero $T[SA[i],n-1] \prec 
T[SA[j],n-1]$. A questo punto si può dire che la \textit{BWT} del testo $T$ è
tale per cui $BWT[i]=T[SA[i]-1]$, se $SA[i]\neq 1$, o \$, altrimenti. Un altro
concetto spesso usato 
sono le lunghezze del cosiddetto \textbf{Longest Common Prefix (\textit{LCP})}
tra ogni suffisso consecutivo in $SA$.\\
Una caratteristica di questa trasformata è la tendenza a
produrre una sequenza con caratteri uguali in posizioni
consecutive. Questo fenomeno è dovuto alle ripetizioni di determinate
sottostringhe nel testo, avendo che tali ripetizioni, per motivazioni
biologiche, sono frequenti nell'ambito genomico. Al fine di sfruttare tali
caratteri uguali consecutivi si 
è quindi ideata la \textbf{trasformata Burrows-Wheeler Run-Length encoded
  (\textit{RLBWT})} dove una sequenza massimale di caratteri uguali, detta
\textit{run}, viene memorizzata in modo efficiente come coppia (carattere,
lunghezza della run). Ad esempio, la stringa \texttt{aaaaaa} sarebbe memorizzata
come \texttt{(a,6)}. Alcuni paper recenti hanno proposto un nuovo tipo di
indicizzazione, tramite il cosiddetto \textbf{r-index}, per questa struttura
dati compressa, al fine di ottenere un indice che non fosse lineare sulla
lunghezza del testo ma sul numero di run della sua \textit{BWT}.
Tale indice include la \textit{RLBWT} e un \textit{Suffix Array sample},
ovvero i valori del \textit{SA} all’inizio e alla fine di ogni run. Tali
articoli hanno portato alla produzione di due tool, \textit{MONI} e 
\textit{PHONI}, alle cui tecniche è fortemente ispirata questa tesi. Queste due
soluzioni hanno entrambe l'obiettivo di 
calcolare i \textbf{Maximal Exact Matches (\textit{MEM})} tra un testo $T$,
lungo $n$, e un pattern $P$, lungo $m$. Si ha infatti che una sottostringa del
pattern, di lunghezza $l$ e iniziante all'indice $i$, ovvero $P[i,i+l-1]$ è un
\textit{MEM} di $P$ in $T$ 
se e solo se $P[i,i+l-1]$ è anche una sottostringa di $T$ ed essa non può essere
estesa 
in nessuna direzione, ovvero né $P[i-1,i+l-1]$ né $P[i,i+l]$ sono
sottostringhe di $T$. L'algoritmo per il calcolo dei \textit{MEM} è direttamente
correlato alla costruzione dell'array delle \textbf{Matching Statistics
  (\textit{MS})}, 
ovvero un array lungo quanto il pattern, di coppie (posizione $pos$, lunghezza
$len$), tale 
che 
$T[MS[i].pos,MS[i].pos+MS[i].len-1]=P[i,i+MS[i].len-1]$ e che $P[i,i+MS[i].len]$ 
non occorre in $T$. Quindi si ha un match tra $P$ e $T$ lungo $MS[i].len$ a
partire da $MS[i].pos$ in $T$ e da $i$ in $P$ che non è ulteriormente
estendibile. Gli autori, per il calcolo delle \textit{Matching Statistics} hanno
usato, in \textit{MONI}, anche il concetto di \textit{threshold}, definito come
il minimo valore dell'\textit{array LCP} tra due run consecutive dello stesso
carattere. Un 
ulteriore miglioramento si ha avuto in \textit{PHONI} con l'uso delle
\textbf{LCE query} per il calcolo dell'array $MS$. Si ha che, dato un testo $T$,
tale che $|T|=n$, il risultato della \textbf{LCE query} tra 
due posizioni $i$ e $j$, tali che $0\leq i,j<n$, corrisponde al più lungo
prefisso comune tra le sottostringhe che hanno come indice di partenza $i$ e
$j$, avendo quindi il più lungo prefisso comune tra $T[i,n-1]$ e
$T[j,n-1]$.\\
I concetti appena espressi verranno riformulati in questa tesi, in quanto essa
tratta la costruzione di una versione \textbf{run-length encoded} della
\textbf{PBWT}, detta \textbf{RLPBWT}. Questa struttura, ideata da Durbin, nel
2014 viene costruita a partire da un 
pannello di aplotipi (nello specifico limitandosi al caso bi-allelico avendo che
il pannello è composto da simboli nell'alfabeto $\Sigma=\{0,1\}$) e 
assume in input un pannello $X$, composto da $M$ individui/righe e $N$
siti/colonne, e produce, tramite l'\textit{ordinamento dei prefissi inverso}, ad
ogni colonna $k$, due insiemi di array. Tali insiemi sono detti \textbf{insieme
  dei prefix array} e \textbf{insieme dei divergence array}. Il primo, denotato
$a$, contiene, per ogni colonna $k$ e ogni posizione $i$, l'indice dell'aplotipo
$m$ nel pannello originale riordinato secondo l'ordinamento inverso alla colonna
$k$-esima. Si ha quindi che $a_k[i]=m$ se e solo se $X_m$ 
(denotando la riga $m$-esima di $X$) è l'$i$-esimo aplotipo secondo
l'ordinamento inverso fatto in colonna $k$. Il secondo insieme, denotato con
$d$, indica l'indice della colonna iniziale del suffisso comune più lungo, che
termina nella colonna $k$, tra una riga e la sua precedente secondo
l'ordinamento inverso ottenuto per la $k$-esima colonna. Il pannello ottenuto
con le permutazioni dettate dal \textit{prefix array} viene chiamato
\textit{matrice PBWT}.\\
Si può quindi apprezzare il collegamento naturale che si ha tra la \textit{PBWT}
e la \textit{BWT}, avendo che il \textit{prefix array} della prima corrisponde
al \textit{suffix array} della seconda mentre il \textit{divergence array} è una
  diversa rappresentazione dell'\textit{array LCP}. \\ 
Cruciale è il fatto che tale struttura permetta di calcolare i \textit{MEM} tra
un aplotipo esterno e il pannello in tempo $\mathcal{O}(MN)$, mentre 
una 
soluzione naive impiegherebbe un tempo proporzionale a $\mathcal{O}(M^2N)$. Il
tradeoff di tale algoritmo, conosciuto anche come \textbf{algoritmo 5 di
  Durbin}, è la richiesta in termini di spazio. L'autore stesso infatti stima la
richiesta di $13NM$ byte in memoria per poter eseguire l'algoritmo e superare
questo limite è l'obiettivo principale di questo progetto di tesi.\\
Al fine di raggiungere tale obiettivo sono state usate altre nozioni. In primis,
si sono sfruttate le cosiddette \textbf{strutture dati succinte}, ovvero
strutture per le quali, assumendo che $\mathcal{X}$ sia il numero di bit
ottimale per memorizzare dei dati, si richiede uno spazio in memoria pari a
$\mathcal{X}+o(\mathcal{X})$. Nel dettaglio si sono usati i cosiddetti
\textbf{bitvector sparsi}, strutture che richiedono in memoria $\approx
m\left(2+\log\frac{n}{m}\right)$ bit, con $n$ lunghezza del bitvector e $m$
numero di simboli $\sigma=1$ in esso. L'efficienza delle operazioni che si
possono fare con tali strutture dati, incredibilmente efficienti dal punto di
vista dello spazio occupato, sono uno dei punti cruciali del funzionamento della
\textit{RLPBWT}.\\
Infine, per memorizzare il pannello in modo compatto, si è usata una
struttura dati, detta \textbf{Straight-Line Programs (\textit{SLP})}. Tale
struttura è una \textbf{grammatica context-free}, che
genera una e una sola parola, la quale permette sia di effettuare \textit{random
access} al testo, non in tempo costante, che di calcolare le \textit{LCE
query} in modo efficiente. 
\subsection*{Metodi}
Il processo per ottenere la \textit{RLPBWT} è stato incrementale,
iniziando con la creazione di una variante naive, basata sulle intuizioni avute
a fine 2021 da Gagie, il quale propose una prima variante della struttura senza
però specificare come effettuare query alla stessa. Tale proposta, inoltre,
presentava alcune ridondanze tra i dati memorizzati.\\
Il primo approccio aveva come obiettivo l'ottenere un riadattamento ``diretto''
dell'\textit{algoritmo 5 di Durbin}, pur tenendo in memoria informazioni legate
principalmente alle run della \textit{matrice PBWT}. Per quanto si siano trovate
soluzioni interessanti per gestire il \textit{mapping} tra una colonna e la sua
successiva (ma anche tra una colonna e la sua precedente), per poter
``seguire'' una riga del pannello originale nella 
\textit{matrice PBWT}, si dovute memorizzare anche quantità non relative alle
run.  Infatti, è risultato necessario memorizzare l'intero
\textit{divergence array} (in forma di \textit{LCP array} quindi
memorizzando la lunghezza del prefisso comune in ordine inverso e non la colonna
d'inizio dello stesso), al fine di poter computare i \textit{MEM} con
un aplotipo esterno. Inoltre, l'assenza delle informazioni relative al
\textit{prefix array} ha impedito di poter annotare quali righe del pannello
presentassero un match massimale esatto, terminante in una certa colonna, con
l'aplotipo query, avendo solo 
informazioni relative alla cardinalità di tale sottoinsieme di righe. Le
informazioni in memoria si è 
stimato non fossero ottimali, non solo per il \textit{divergence array}, ma anche
perché non veniva utilizzato alcun approccio tramite \textit{strutture dati
  succinte}. Si è quindi deciso di
cambiare approccio al problema, ispirandosi ai risultati già ottenuti per la
\textit{RLBWT}. In primis le strutture necessarie al mapping e
all'indicizzazione delle run sono state sostituite da \textit{bitvector
  sparsi}, creando una prima variante della \textit{RLPBWT} basata su
\textit{bitvector}. Ovviamente tale sostituzione non risolveva i limiti dati
dall'avere in memoria il \textit{divergence array} e l'algoritmo di ricerca dei
\textit{MEM} era ancora basato su quello di Durbin. \\
Al fine di avvicinarsi alle idee proposte in \textit{MONI} e \textit{PHONI} è
quindi servito uno studio teorico preliminare per ridefinire i concetti
di: \textit{matching statistics}, \textit{MEM} 
calcolabili da esse, \textit{threshold} e \textit{LCE query}. Da un punto
di vista formale, le \textit{matching statistics}
sono state definite, assumendo un pannello $X$, con $M$ aplotipi, $N$ siti e
riga arbitraria $i$ indicabile con la dicitura $x_i$, come un array, lungo $N$,
di 
coppie (riga $row$, lunghezza $len$) tale per cui
$x_{MS[i].row}[i-MS[i].len+1,i]=z[i-MS[i].len+1,i]$ mentre $z[i-MS[i].len,i]$
non è un suffisso terminante in colonna $i$ di un qualsiasi sottoinsieme di
righe di $X$. In pratica si ha un suffisso comune, lungo $MS[i].len$ e non
estendibile ulteriormente a sinistra, terminante in
colonna $i$, tra la query e la riga $MS[i].row$. Data questa definizione,
inoltre, si ha che $z[i-l+1:i]$ 
è un \textit{MEM} di lunghezza $l$ tra la query e la riga $MS[i].row$ del
pannello $X$ se e solo se  $MS[i].len=l\land(i=N-1\lor MS[i].len\geq
MS[i+1].len)$.\\ 
Al fine di poter calcolare tale array si è quindi pensato all'utilizzo del
concetto di \textit{threshold}, anch'esso riadattato alla \textit{matrice PBWT},
come minimo valore dell'\textit{array LCP} all'interno di una run (comprendendo
anche la testa, qualora esistente, della run successiva, essendo il suo valore
\textit{LCP} calcolato sfruttando anche la coda della run corrente). Anche per
tale informazione si è usato un \textit{bitvector sparso} (avendone, per ogni
colonna, uno lungo quanto una colonna della \textit{matrice PBWT} e con un
numero di $\sigma=1$ pari al numero delle run) e, analogamente a
quanto visto per l'\textit{r-index}, si è provveduto a 
tenere in memoria i \textit{sample di prefix array} a inizio e fine di ogni
run.\\ 
Grazie all'uso delle \textit{threshold} si è potuto sviluppare un algoritmo
efficiente dal punto di vista della memoria richiesta per il calcolo delle
\textit{matching statistics} in due ``sweep'' sul pattern, calcolandone prima le
posizioni e 
poi, tramite \textit{random access} al pannello, anche le
lunghezze. Tale soluzione richiedeva in memoria
l'intero pannello, in forma di vettori di
bitvector. Esplorando le tecniche presenti negli ultimi sviluppi avuti con la
\textit{RLBWT} si è giunti all'uso 
dell'\textit{SLP} e riprendendo quanto fatto in \textit{PHONI} per
la \textit{RLBWT}, si è anche deciso di risparmiare ulteriore spazio eliminando
l'uso delle \textit{threshold}. Per ottenere il calcolo dell'array della
\textit{matching 
  statistics} si sono quindi usare le \textit{LCE query}, ridefinite per il
caso posizionale. Dato un pannello $X$, $M\times N$, e due righe $x_i$ e $x_j$
tali che $0\leq i <m$ e $0\leq j <M$, con $i\neq j$, si definisce \textbf{LCE
  query} il suffisso comune più lungo tra le due righe, eventualmente fissando
il termine a una colonna precisata. Calcolando la
lunghezza di tale suffisso comune è possibile calcolare anche le
lunghezze delle \textit{matching statistics} contemporaneamente al calcolo delle
righe della stessa. Quindi, si è ottenuto il calcolo completo di tale array in
un singolo ``sweep'' sul pattern.\\
Al fine di risolvere il problema del calcolo dei \textit{MEM}, è
servito 
costruire una struttura dati a supporto che permettesse l'estensione, nel
pannello, a tutte le righe che presentassero il \textit{MEM} calcolato con le
\textit{matching statistics}. Infatti, con il calcolo di tale array, si aveva
nozione di una sola delle righe, $MS[i].row$, presentanti un match massimale
esatto, lungo 
$MS[i].len$, fino alla colonna $i$. Anche tale struttura, seguendo l'ormai
evidente correlazione tra \textit{BWT} e \textit{PBWT} (nonché delle rispettive
implementazioni \textit{run-length}), si è basata su concetti precedentemente
teorizzati, ovvero le \textbf{funzioni} $\boldsymbol\varphi$ e
$\boldsymbol\varphi^{\mathbf{-1}}$. Infatti, tali
funzioni, dato un certo indice di suffisso, ne restituivano i due valori
adiacenti nel \textit{SA}. In
modo analogo si è creata una struttura dati che, data una colonna $k$ e un
indice di riga, permettesse di ottenere i valori adiacenti all'indice di riga
dato nel \textit{prefix array} ottenuto in colonna $k$. Per le proprietà date
dalla costruzione della \textit{PBWT} tutte le righe che presentano un match
massimale esatto terminante in colonna $k$ hanno indici adiacenti nel
\textit{prefix array}. Quindi, ipotizzando un match massimale esatto terminante
in $i$, partendo da 
$MS[i].row$, si possono computare tutte le altre righe del pannello che
presentano il medesimo \textit{MEM}, terminante in una certa colonna, con
l'aplotipo query. Tale 
struttura è costituita 
da due insiemi (uno per $\varphi$ e uno per $\varphi^{-1}$), di cardinalità pari
  al numero di righe, di \textit{bitvector sparsi} con una quantità di simboli 
$\sigma=1$ pari alle volte che la riga è testa/coda di una run nella
\textit{matrice PBWT}. A questi insiemi
se ne aggiungono altri due, sempre di cardinalità
$M$, formati da vettori lunghi 
quanto il numero di volte che la rispettiva riga è testa/coda di una run.\\
Ricapitolando, per computare tutti i \textit{MEM}, si hanno in
memoria, per ogni colonna: un \textit{bitvector sparso} per identificare le run,
due \textit{bitvector sparsi} per permettere il mapping tramite il numero di
simboli $\sigma=0$ e $\sigma=1$ fino ad un certo 
indice (infatti il mapping tra due colonne adiacenti è basato su un algoritmo
che sfrutta il sorting stabile che si usa per creare la \textit{matrice PBWT}),
un booleano per capire se la prima run è composta da simboli $\sigma=0$ 
(avendo che poi le run hanno simboli alternati), il valore
complessivo di zeri nella colonna della \textit{matrice PBWT} e, eventualmente,
un \textit{bitvector sparso} per le \textit{threshold} (non in memoria se si
volessero usare le \textit{LCE query}). In aggiunta l'intera struttura ha in
memoria il pannello sotto forma di \textit{SLP} e la struttura atta a computare
le funzioni $\varphi$ e $\varphi^{-1}$ (nonché il solo \textit{prefix array}
$a_{N-1}$ per il calcolo delle due strutture dedicate).\\
\textit{Per concludere questa sezione si segnala che la
  variante 
della \textbf{RLPBWT} basata su \textbf{SLP} e \textbf{LCE query} è stata
considerata come il risultato finale ottenuto in questo progetto di tesi.}
\subsection*{Risultati}
L'obiettivo della tesi era quello di concentrarsi sui limiti
dell'\textit{algoritmo 5 di Durbin} dal punto di vista della memoria
richiesta. L'utilizzo di strutture dati succinte, la scelta di memorizzare il
pannello tramite \textit{SLP} e la necessità di ridurre al minimo le
informazioni in memoria hanno comportato inevitabilmente un aumento dei tempi di
calcolo. Tale aumento è davvero molto importante, passando
da $\sim 411$ s, ottenuti con l'implementazione originale di Durbin, a $\sim
1824$ s 
ottenuti con la \textit{RLPBWT} con \textit{SLP} e \textit{LCE query} prendendo
ad esempio 
un pannello $70000 \times 46538$ e volendo calcolare i \textit{MEM} con $30000$
query. \\
Analizzando però meglio i risultati relativi in termini di memoria richiesta si
sono ottenuti dati confortanti. In primis, risulta interessante approfondire la
memoria utilizzata dall'\textit{SLP} per il pannello, avendo che, per un
pannello 
$100000\times 358653$, che normalmente occupa $\sim 
35$\textit{gb} (nel formato \texttt{.macs} usato per l'input), si produce un
\textit{SLP} che pesa appena $\sim 15$\textit{mb}.\\
In secondo luogo si ricorda che, secondo le stime di Durbin, gli array necessari
al funzionamento dell'algoritmo 5 richiedono $\sim 13NM$ bytes e,
prendendo un pannello $70000 \times 46538$, sarebbero
necessari $\sim 40$\textit{gb} di memoria. Sperimentalmente si sono registrati
picchi di memoria prossimi a quel valore mentre, per la \textit{RLPBWT} con
\textit{SLP} e \textit{LCE query}, il picco è stato di appena $\sim
3$\textit{gb}. In pratica la soluzione \textit{run-length} ha richiesto il 7\%
della memoria della soluzione di Durbin.\\
Esplorando il codice di Durbin si è scoperta
l'esistenza di un algoritmo non approfondito nel paper. In questo caso il
calcolo dei \textit{MEM} con un pannello di query prevede la fusione di
quest'ultimo 
con il pannello principale e alla costruzione di un'unica \textit{matrice PBWT},
per poi procedere con il calcolo dei match massimali interni 
al pannello stesso. Tale soluzione può quindi giovare del fatto che non richieda
tutte le strutture necessarie al mapping tra le colonne e al sistema di
indicizzazione sul \textit{prefix array} atto a tenere traccia dei
match. Inoltre necessita solo delle informazioni (\textit{prefix array} e
\textit{divergence array}), calcolate dinamicamente, della colonna corrente.
Tale algoritmo è risultato essere superiore sia in termini di tempi
(confrontandosi coi risultati appena descritti si parla di $20$\textit{s} di
esecuzione) che di memoria (essendo stimata una memoria proporzionale a
$\mathcal{O}(N+M)$, confermata dal fatto che il picco di memoria in esecuzione
sia stato di appena $10,084$\textit{kb}). Del resto tale algoritmo produce i
risultati in ordine sparso e, constatando come in letteratura siano presenti
solo estensioni e varianti dell'\textit{algoritmo 5}, probabilmente impedisce un
facile riadattamento alla risoluzione di altre problematiche che non siano il
calcolo dei \textit{MEM} con una o più query.
\subsection*{Conclusioni}
In conclusione, si rileva come i risultati prefissati siano stati raggiunti,
producendo una struttura dati efficiente in memoria per la risoluzione del
calcolo dei \textit{MEM} tra un pannello di aplotipi e un aplotipo
esterno.  
Si segnala inoltre come siano possibili diversi sviluppi futuri, come
un'ulteriore ottimizzazione della struttura, sia in termini 
di gestione del mapping che, eventualmente, di gestione di più query
contemporaneamente. \\
Inoltre sono possibili diverse generalizzazioni rispetto alle caratteristiche
del pannello. Per quanto i pannelli di aplotipi prodotti dal sequencing del
genoma umano raramente presentino siti multi-allelici si ha una 
presenza stimata, al momento, di circa il 2\% di siti tri-allelici, avendo che
tale percentuale risulti fortemente sottostimata. Una prima generalizzazione
quindi sarebbe quella di studiare pannelli multi-allelici, riformulando la
\textit{RLPBWT} al fine di poter funzionare anche con pannelli costruiti su un
alfabeto arbitrario (dovendo rinunciare a diverse ottimizzazioni che permette il
caso binario). Un'altra generalizzazione interessante riguarda l'ammissione di
dati mancanti all'interno del pannello stesso. La maggior parte delle soluzioni
attualmente sviluppate sono basate su una forte assunzione: non si hanno dati
mancanti. Una variante della \textit{RLPBWT} che sia quindi in grado di
lavorare, eventualmente con \textit{algoritmi parametrici} o \textit{algoritmi
  approssimati}, 
su pannelli in cui sono presenti wildcard per rappresentare tali dati,
permetterebbe di fare studi più completi su dati reali, estendendone gli usi nei
campi della \textit{medicina personalizzata}, dei \textit{GWAS} etc$\ldots$.\\ 
Si segnala, infine, come la \textit{RLPBWT} sia potenzialmente in grado di
risolvere 
efficientemente anche altri task, come ad esempio la ricerca di \textbf{k-MEM},
ovvero \textit{MEM} con un aplotipo esterno che coinvolgano
esattamente $k$ righe nel pannello. \\
Le potenzialità di tale struttura sono quindi molteplici e, grazie al ridotto
consumo di memoria, si hanno le giuste premesse perché venga utilizzata per
gestire e interrogare grosse moli di dati reali, incrementando le capacità di
studio, previsione e inferenza che si possono avere grazie allo studio del
\textit{pangenoma}.
\end{document}
% LocalWords:  pangenoma naive
