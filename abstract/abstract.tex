\documentclass[a4paper,11pt, oneside]{article}
\usepackage[left=15mm, right=15mm, top=20mm, bottom=20mm]{geometry}
\usepackage[utf8]{inputenc}
\usepackage{amssymb}
\usepackage{amsthm}
\usepackage{graphics}
\usepackage{amsfonts}
\usepackage{listings}
\usepackage{amsmath}
\usepackage{amstext}
\usepackage{engrec}
\usepackage{rotating}
\usepackage{verbatim}
\usepackage[safe,extra]{tipa}
% \usepackage{showkeys}
\usepackage{multirow}
\usepackage{hyperref}
\usepackage{microtype}
\usepackage{fontspec}
\usepackage{enumerate}
\usepackage{physics}
\usepackage{braket}
\usepackage{mhchem}
\usepackage{marginnote}
\usepackage{pgfplots}
\usepackage{cancel}
\usepackage{polynom}
\usepackage{booktabs}
\usepackage{enumitem}
\usepackage{framed}
\usepackage{pdfpages}
\usepackage{pgfplots}
\usepackage{algorithm}
% \usepackage{algpseudocode}
\usepackage[cache=false]{minted}
\usepackage{mathtools}
\usepackage[noend]{algpseudocode}

\newcommand*{\bfrac}[2]{\genfrac{}{}{0pt}{}{#1}{#2}}

\usepackage{tikz}\usetikzlibrary{er}\tikzset{multi  attribute /.style={attribute
    ,double  distance =1.5pt}}\tikzset{derived  attribute /.style={attribute
    ,dashed}}\tikzset{total /.style={double  distance =1.5pt}}\tikzset{every
  entity /.style={draw=orange , fill=orange!20}}\tikzset{every  attribute
  /.style={draw=MediumPurple1, fill=MediumPurple1!20}}\tikzset{every
  relationship /.style={draw=Chartreuse2,
    fill=Chartreuse2!20}}\newcommand{\key}[1]{\underline{#1}}
\usetikzlibrary{arrows.meta}
\usetikzlibrary{decorations.markings}
\usetikzlibrary{arrows,shapes,backgrounds,petri}
\tikzset{
  place/.style={
    circle,
    thick,
    draw=black,
    minimum size=6mm,
  },
  transition/.style={
    rectangle,
    thick,
    fill=black,
    minimum width=8mm,
    inner ysep=2pt
  },
  transitionv/.style={
    rectangle,
    thick,
    fill=black,
    minimum height=8mm,
    inner xsep=2pt
  }
} 
\usetikzlibrary{automata,positioning,chains,fit,shapes}
\usetikzlibrary{circuits.logic.US}
\usetikzlibrary{positioning}
\usepackage{fancyhdr}
\pagestyle{fancy}
\fancyhead[LE,RO]{\slshape \rightmark}
\fancyhead[LO,RE]{\slshape \leftmark}
\fancyfoot[C]{\thepage}
\usepackage[usenames,dvipsnames]{pstricks}
\usepackage{epsfig}
\usepackage{pst-grad} % For gradients
\usepackage{pst-plot} % For axes
\usepackage[space]{grffile} % For spaces in paths
\usepackage{etoolbox} % For spaces in paths
\makeatletter % For spaces in paths
\patchcmd\Gread@eps{\@inputcheck#1 }{\@inputcheck"#1"\relax}{}{}
\makeatother
\usepackage{lipsum}
\DeclareSymbolFont{symbolsC}{U}{txsyc}{m}{n}
\DeclareMathSymbol{\strictif}{\mathrel}{symbolsC}{74}
\title{PhD Proposal}
\author{Davide Cozzi, 829827,
  \href{mailto:d.cozzi@campus.unimib.it}{d.cozzi@campus.unimib.it}} 
\date{}

\pgfplotsset{compat=1.13}
\begin{document}
\maketitle

\definecolor{shadecolor}{gray}{0.80}
\setlist{leftmargin = 2cm}
\newtheorem{teorema}{Teorema}
\newtheorem{definizione}{Definizione}
\newtheorem{esempio}{Esempio}
\newtheorem{corollario}{Corollario}
\newtheorem{lemma}{Lemma}
\newtheorem{osservazione}{Osservazione}
\newtheorem{nota}{Nota}
\newtheorem{esercizio}{Esercizio}
\algdef{SE}[DOWHILE]{Do}{doWhile}{\algorithmicdo}[1]{\algorithmicwhile\ #1}

\renewcommand{\sectionmark}[1]{\markright{\thesection.\ #1}}
\newcommand{\floor}[1]{\lfloor #1 \rfloor}
\newcommand{\MYhref}[3][blue]{\href{#2}{\color{#1}{#3}}}%
\section*{Introduction}
The problem of \textit{pattern matching} is one of the most studied topics in
the field of algorithmics and bioinformatics. For example the interest in such
problems is due to the need to align sequences or search for specific patterns
within the \textit{DNA}. \\
In this context, a large number of data structure algorithms have been
modeled. Among these, one of the most used is the \textbf{Burrows-Wheeler
  transform (\textit{BWT})} thanks to the studies of Ferragina and Manzini who
proposed its use together with the so-called FM-index \cite{fmindex}.\\
In recent years, in the field of bioinformatics, there is a change of
interest. While until a few years ago the research was focused on the study of
the \textit{genome}, we are now starting to deepen the topic of
\textit{pangenome}. Currently, the need to take into
account the high variability in population genomes as well as the specificity of
an individual genome in a personalized approach to medicine is rapidly pushing
the abandonment of the traditional paradigm of using a single reference genome
\cite{pangenome}. \\ 
In this context, various algorithms and various data structures have been
implemented in order to study the so called \textit{haplotypes} and
\textit{genotypes}. Briefly we could define \textit{haplotypes} as a combination
of allelic variants, inherited from a parent. Instead we can define the
\textit{genotype} as the complete set of genes contained in the DNA. From a
biological point of view it is indeed interesting to note that each individual
shares about 99\% of the genetic code, while the remaining 1\% differs mostly
with \textbf{Single Nucleotide Polymorphisms (\textit{SNPs})}, i.e. variations
of a single nucleotide in a precise \textit{locus} of the DNA\\ 
Thanks to the last developments in sequencing technologies, which have led both
to reduce the costs of single sequencing and to produce sequences of ever higher
quality in less and less time, the researchers were able to theorize the
\textit{pangenome graph}, replacing the old representation of the single genome
by a single sequence. \\
One of the most important data structure developed in order to handle the study
of haplotypes sequences is the \textbf{positional Burrows-Wheeler transform
  (\textit{PBWT})}, proposed by Durbin in 2014 \cite{pbwt_durbin}. With this
particular data structure it was possible both to study possible matches inside
a haplotype panel and to search for set maximal matches between an external
haplotype and a panel, as well as the set maximal matches inside the panel
itself. Furthermore, for example, variants have also been studied for the 
management of the multiallelic case \cite{mpbwt}. \\
In 2021 Rossi et al. proposed \textit{MONI} as a data structure to handle a
\textbf{run-length version of BWT (\textit{RLBWT})} with the ultimate intention
of indexing and using multiple genomes as a reference \cite{moni}. Together with
this data structure the authors proposed the concept of \textit{matching
  statistics} in order to efficiently compute the matches between a pattern and
a text. A recent improvement, regarding the \textit{RLBWT}, has been made
through the implementation of \textit{PHONI} \cite{phoni}, where where the
so-called \textit{longest-common-extension (LCE) queries} are used. \\
During the development of my master's thesis I worked in collaboration with the
authors of MONI in order to create a run length encoded variant of the PBWT. In
this work I used practically all concepts applied to the ``classic''
\textit{BWT} in \textit{MONI} and \textit{PHONI}. \\
During my PhD I will focus my studies on the development of new algorithms in
various topics such as variants of the \textit{PBWT} (and any related uses),
\textit{haplotyping/genotyping} or solutions concerning the study of
\textit{SNPs}.It is also my intention to deepen the more experimental themes
relating to pattern matching, in detail the new developments on BWT and new
indexing structures, as well as the theme of \textit{succinct data structures},
with particular attention to the use of \textit{bitvectors}. 
\section*{State of the art}
I will now present a brief overview of the main algorithms, data structures,
methods etc that will be the core of my studies during my PhD. 
\subsection*{BWT}
The \textbf{Burrows-Wheeler Transform (\textit{BWT})} \cite{bwt} was introduced
in 1994 in order to compress texts but it has had then wide use in
bioinformatics, above all thanks to the already cited \textit{FM-index}. Given a
text $T$, \$-terminated, such that $|T|=n$, we can define the $BWT_T$, denoting
with $SA_T$ the \textit{suffix array} of $T$, as:
$BWT_T[i] = T[SA_T[i]-1]$, if $SA_T[i]>0$, and $BWT_T[i] = \$ $ otherwise. Less
formally we can say that $BWT_T[i]$ is the character that precedes the $i$-th
suffix in the lexicographically order. It is important to note that this
transform is reversible so we can reconstruct the text $T$ from its transform
$BWT_T$ using the so-called \textbf{LF-mapping}. Given $BWT_T$ and an array,
called $F_T$, with all the characters of $T$ in the lexicographically order, we
can say that, thanks to the \textit{LF-mapping}, the $j$-th occurrence of a
certain character in $BWT_T$ corresponds to the $j$-th occurrence of the same
character in $F_T$, so we can reconstruct $T$ starting from its last character
\$. With the use of the \textit{LF-mapping} we can perform the
\textit{backward-search} in order to use the $BWT_T$ to look for a pattern $P$
within $T$. This can be done efficiently thank to the \textit{FM-index} which
consists of two functions. The first one is $C$ function, such that, given an
alphabet $\Sigma$ (that includes the ending character \$),
$C:\Sigma\to[1,n]$. This function, given a character $\sigma\in \Sigma$ returns
the number of occurrences of characters lexicographically than the one given as
argument in $T$. The second one is the $Occ$ function,
$Occ:\Sigma\times[1,n]\to[1,n]$, has as arguments a character $\sigma\in\Sigma$
and an index $i$ of $BWT_T$ and returns the count of occurrences of $\sigma$ in
$BWT_T[1,i]$ (\textbf{CONTROLLARE E RISCRIVERE MEGLIO}).\\
The use of \textit{BWT} has allowed the construction of efficient algorithms
both in the field of pattern matching and in that of sequence alignment. 
\subsection*{Bitvectors}
\textit{Bitvectors} are ones of the most important data structure when
mentioning succinct data structures. \\
A \textit{bitvector} is an array on $n$ bits which allows two particular
operations, called \textbf{rank} and \textbf{select}, in addition to the classic
operations possible on Boolean arrays, such as \textit{random access} in
\textit{constant time} (\textbf{DA VERIFICARE}). More in detail the rank
function allows you to calculate how many values of 1 are up to a certain
index. Instead the select function allows to obtain the index of any one present
in the bitvector. Formally, given a bitvector $B$, such that $|B|=n$, and given
an index $i$, such that $0\leq i<n$, we can define $rank_B(i)=\sum_{k=0}^{k<i}
B[k]$. Instead, about the select function, given an integer $i$, such that
$0<i\leq rank_B(n)$, where $n=|B|$,  we can define $select(i)=\min\{j \,| \,\,
rank_B(j+1)=i\}$.\\
From a purely theoretical point of view, with the additional cost of
$\mathcal{O}(n)$ bits in memory, these two operations can be supported with
constant time.  In more practical terms there are several implementations of the
same within \textbf{SDSL (\textit{Succint Data Structures Library})}
\cite{sdsl}, one of the most important C++ library used in bioinformatics. As
the implementation changes (for example \textit{plain bitvector, interleaved
  bitvector, sparse bivector} etc$\ldots$) the computational time of the two
operations varies (usually only one of the two is in constant time) and the
amount of additional bits needed.\\
An example of the use of bitvectors is to track the runs in the run-length
encoded implementations of \textit{BWT} and \textit{PBWT}, where we put one at
each head of run.

\subsection*{RLBWT}
Speaking of pangenome, linear indexing via FM-index is no longer the best
solution as it does not handle the large repetitions present in this new type of
sequences in the best possible way. In 2005 M\"{a}niken and Navarro defined
the \textbf{Run-Length Burrows–Wheeler Transform (\textit{RLBWT})}. Given a text
$T$, $RLBWT_T$ is a rappresentation of $BWT_T$ with a compact rappresentation of
consecutive equal characters, the so-called \textit{runs}. With this change of
perspective the algorithms have gone from being linear over the length of the
text, $n$, to being linear over the number of runs, $r$. \\
The new indexing method, which was then introduced by Gagie et
al. \cite{rindex}, was called \textbf{r-index} and corresponds to the
\textit{RLBWT} and a SA sampling at the begin and at the end of every run. The
algorithm for querying through the \textit{RLBWT} take advantage of other
methods, such as the use of \textbf{thresholds} (minimum
\textit{LCP} value between two consecutive runs of the same character) in
\textit{MONI}, and the use of \textbf{longest common extension (\textit{LCE})
  query} (to compute the right equal common extension between two position in
the text) cd\cite{lce} and \textbf{straight-line programs (\textit{SLP})}
\cite{slp} (for \textit{random access} and \textit{lce queries} over the
grammar-compressed text) in \textit{PHONI}. Both these solutions are used for
the computation of the \textbf{matching statistics (\textit{MS})}. Given a
pattern $P$ and a text $T$ the \textit{MS} of $P$ in respect to $P$ is an array
$M$ of pairs position/length, $|M|=|P|$, such that
$T[M[i].pos:M[i].pos+M[i].len-1=P[i:i+M[i].len]-1]$ and $P[i:i+M[i].len]$ does
not occur in $T$. Given \textit{MS} we can compute every \textbf{Maximal Exact
  Match (\textit{MEM})} of a pattern in a text. Given a text $T$ and a pattern
$P$ a substring of the pattern $P[i : i+l-1]$, of length $l$, is a \textit{MEM}
of $P$ in $T$ if $P[i:i+l-1]$ is a substring of $T$ but but neither
$P[i-1:i+l-1]$ nor $P[i:i+l]$ are. At the end, using a particular function
called $\varphi$ (and $\varphi^{-1}$), based on the use of the \textbf{inverse
  suffix array (\textit{ISA})} as well, it was possible to start from the
starting positon of a match extracted 
by \textit{MS} and find all the starting positions of all the copies of $P$ in
$T$. Infact, formally, given a starting position $p$, we have
$\varphi(p)=SA[ISA[p]-1]$, $NULL$ if $ISA[p]=0$, and
$\varphi^{-1}(p)=SA[ISA[p]+1]$, $NULL$ if $ISA[p]=|T|-1$, where $ISA[i]=j$ iff
$SA[j]=i$.\\   
Thanks to these and other methods it was possible to perform pattern matching
efficiently even on long sequences of nucleotides, such as those studied in
a pangenomic context. 
\subsection*{PBWT}
Based on the theories of BWT Durbin, in 2014, he devised the positional
\textbf{Burrows–Wheeler transform (\textit{PBWT}}) \cite{pbwt}, in order to
solve the problem of pattern matching on panels (matrices) of haplotypes,
denoted by $X$. In
detail we are talking about a panel with $M$ haplotypes and $N$ biallelic
sites. This data structure is based on a reversed-prefix ordering at each column
$k$ that produces two different multidimensional arrays. The first one is called
\textbf{prefix array}, denoted by $a$, and, for each column $k$, contains, for
each position $i$, the haplotype of index $m$ in the original panel. More
formally we can say that $a_k[i]=m$ iff $X_m$ is the $i$-th haplotype in the
reversed-prefix ordering at column $k$. Note that $x_m$ such that $a_k[i]=m$
could be denoted by $y_i^k$. The second bidimensional array is
called \textbf{divergence array}, denoted by $d$, and indicates the index of the
starting column of the longest common suffix, ending in column $k$, between a
row and its previous one, at reversed-prefix ordering at column $k$. More
formally we can define $d_k[i]=h$ iff $h$ is the smallest column index such that
$y_i^k[h,k)=y_{i-1}^k[h,k)$. (\textbf{RISCRIVERE MEGLIO})\\
Thanks to these two bidimensional arrays it is possible to compute all matches
within $X$ longer than a minimum length $L$, all set-maximal matches within $X$
in linear time, all set-maximal matches (which we could also call ``MEMs'') from
a new sequence $z$ to $X$ etc$\ldots$ 
\section*{Research goals}
For my master's thesis  I therefore tried to combine the ideas related to the
\textit{RLBWT} with those of the \textit{PBWT}, creating the \textbf{RLPBWT}. In
order to obtain this result I have to rethink the concept of \textit{Matching
  Statistics} for \textit{PBWT}, how to compute the \textit{SLP} for the panel,
how to use \textit{thresholds}/\textit{LCE queries}, how to obtain the same
behaviour of the $\varphi$ function etc$\ldots$\\
In detail, regarding \textit{MS}, instead of the \textit{pos} we track a
\textit{row} of the panel, regarding thresholds we use the index inside a run in
a column where we have the minimum value of divergence array and, regarding the
\textit{SLP}, we ``stretch'' the reverse panel (from the right to the left) in
in order to make \textit{LCE queries} possible. Instead, regarding the $\varphi$
function, I have implemented a new simple data structure to identify which row
is above and which row is below each row in the panel permutated via the
prefix array and stored in a run-length compressed way. (\textbf{RISCRIVERE
  MEGLIO ED EVENTUALMENTE APPROFONDIRE}).\\
Obviously there are some limitations, such as the study of biallelic panels
only and the lack of management of any missing data, which are very frequent in
the case of real data. First of all it will be interesting to implement a new
version of the \textit{RLPBWT} that can handle multiallelic data, adapting the
current use of bitvectors to manage the \textit{LF-mapping} also in that
case. On the other hand, it will be more complicated to manage the missing
data. This problem is known to be \textit{hard} so I should probably deal with
parametric algorithms or approximate algorithms, based on researches already
developed in \textit{BIAS}.
\bibliographystyle{unsrt}
\bibliography{abstract}
\end{document}