\documentclass[a4paper,11pt, oneside,italian]{article}

\usepackage[utf8]{inputenc}
\usepackage{hyperref}
\usepackage{framed}
\usepackage{xcolor}
\usepackage{amsmath}
\usepackage{enumitem}
\usepackage{relsize}
\usepackage{microtype}
\usepackage{typearea}
\newtheorem{teorema}{Teorema}
\newtheorem{definizione}{Definizione}
\newtheorem{esempio}{Esempio}
\newtheorem{corollario}{Corollario}
\newtheorem{lemma}{Lemma}
\newtheorem{osservazione}{Osservazione}
\newtheorem{nota}{Nota}
\newtheorem{esercizio}{Esercizio}
\hypersetup{
  pdftitle = {Algoritmi per la trasformata di Burrows--Wheeler posizionale con
    compressione run-length}, 
  pdfauthor = {Davide Cozzi},
  pdfsubject = {Discorso },
  pdfpagemode = UseNone
}


\usepackage{fancyhdr}
% \pagestyle{fancy}
% \fancyhead[LE,RO]{\slshape \rightmark}
% \fancyhead[LO,RE]{\slshape \leftmark}
\fancyfoot[C]{\thepage}

\title{Algoritmi per la trasformata di Burrows--Wheeler
  posizionale con compressione run-length} 
\author{Davide Cozzi\\\smaller matr.~829827}
\date{}
\makeatletter
\renewcommand{\paragraph}{%
  \@startsection{paragraph}{4}%
  {\z@}{0.75ex \@plus 1ex \@minus .2ex}{-1em}%
  {\normalfont\normalsize\bfseries}%
}
\makeatother

\def\SLP{\mbox{\rm {\sf SLP}}}
\def\rank{\mbox{\rm {\sf rank}}}
\def\lcs{\mbox{\rm {\sf lcs}}}
\def\lcp{\mbox{\rm {\sf lcp}}}
\def\lce{\mbox{\rm {\sf lce}}}
\def\LCE{\mbox{\rm {\sf LCE}}}
\def\ROT{\mbox{\rm {\sf ROT}}}
\def\select{\mbox{\rm {\sf select}}}
\def\col{\mbox{\rm {\sf col}}}
\def\NULL{\mbox{\rm {\sf null}}}
\def\len{\mbox{\rm {\sf len}}}
\def\pos{\mbox{\rm {\sf pos}}}
\def\row{\mbox{\rm {\sf row}}}
\def\LF{\mbox{\rm {\sf LF}}}
\def\FL{\mbox{\rm {\sf FL}}}
\def\W{\mbox{\rm {\sf w}}}
\def\A{\mbox{\rm {\sf A}}}
\def\A{\mbox{\rm {\sf A}}}
\def\SA{\mbox{\rm {\sf SA}}}
\def\LCP{\mbox{\rm {\sf LCP}}}
\def\ISA{\mbox{\rm {\sf ISA}}}
\def\IISA{\mbox{\rm {\sf (I)SA}}}

\def\PLCP{\mbox{\rm {\sf PLCP}}}
\def\PPLCP{\mbox{\rm {\sf (P)PLCP}}}
\def\RLCP{\mbox{\rm {\sf RLCP}}}
\def\RLBWT{\mbox{\rm {\sf RLBWT}}}
\def\MEM{\mbox{\rm {\sf MEM}}}
\def\KMEM{\mbox{\rm {\sf K-MEM}}}
\def\KSMEM{\mbox{\rm {\sf K-SMEM}}}
\def\MS{\mbox{\rm {\sf MS}}}
\def\LCP{\mbox{\rm {\sf LCP}}}
\def\NSV{\mbox{\rm {\sf NSV}}}
\def\PSV{\mbox{\rm {\sf PSV}}}
\def\RMQ{\mbox{\rm {\sf RMQ}}}
\def\BWT{\mbox{\rm {\sf BWT}}}
\def\BWM{\mbox{\rm {\sf BWM}}}
\def\ITR{\mbox{\rm {\sf index\_to\_run}}}
\def\GS{\mbox{\rm {\sf get\_symbol}}}
\def\PBWT{\mbox{\rm {\sf PBWT}}}
\def\MPBWT{\mbox{\rm {\sf mPBWT}}}
\def\MRLPBWT{\mbox{\rm {\sf mRLPBWT}}}
\def\DPBWT{\mbox{\rm {\sf dPBWT}}}
\def\RLPBWT{\mbox{\rm {\sf RLPBWT}}}
\def\SMEM{\mbox{\rm {\sf SMEM}}}
\def\M{\mbox{\rm {\sf M}}}
\def\C{\mbox{\rm {\sf C}}}
\def\Occ{\mbox{\rm {\sf Occ}}}
\def\L{\mbox{\rm {\sf L}}}
\def\F{\mbox{\rm {\sf F}}}
\def\DA{\mbox{\rm {\sf DA}}}
\def\DM{\mbox{\rm {\sf DM}}}
\def\PA{\mbox{\rm {\sf PA}}}
\def\LCE{\mbox{\rm {\sf LCE}}}
\def\UP{\mbox{\rm {\sf update}}}
\def\lceb{\mbox{\rm {\sf lce\_bounded}}}
\def\RM{\mbox{\rm {\sf reverse\_map}}}


\begin{document}
\maketitle
\setlist{leftmargin = 2cm}
\noindent
\subsection*{Slide 1}
Buongiorno, sono Davide Cozzi e oggi presento la mia tesi dal titolo:
``Algoritmi per la trasformata di Burrows--Wheeler posizionale con
compressione run-length''
\subsection*{Slide 2}
Negli ultimi anni si è assistito a un cambio di paradigma nel campo della
bioinformatica, ovvero il passaggio dallo studio della sequenza lineare di un
singolo genoma a quello di un insieme di genomi, provenienti da un gran numero
di individui, al fine di poter considerare anche le varianti geniche. Questo
nuovo concetto è stato introdotto da Tettelin, nel 2005, con il termine di
pangenoma. Grazie ai risultati ottenuti in pangenomica, ci sono stati
miglioramenti sia nel campo della biologia che in quello della medicina
personalizzata, grazie al fatto che, con il pangenoma, si migliora la precisione
della rappresentazione di multipli genomi e delle loro differenze. Il genoma
umano di riferimento (GRCh38.p14), è composto da circa 3.1 miliardi di basi, con
più di 88 milioni varianti tra i genomi sequenziati, secondo i risultati
ottenuti nel 1000 Genome Project. Considerando che, grazie al miglioramento
delle tecnologie di sequenziamento, la quantità dei dati di sequenziamento sia
destinata ad aumentare esponenzialmente nei prossimi anni, risulta necessaria la
costruzione di algoritmi e strutture dati efficienti per gestire una tale mole di
dati. A questo scopo, uno degli approcci più usati per rappresentare il
pangenoma è attraverso un pannello di aplotipi, ovvero, da un punto di vista
computazionale, una matrice di M righe, corrispondenti agli individui, e N
colonne, corrispondenti ai siti con le varianti. Si specifica che, con il
termine aplotipo, si intende l'insieme di alleli, ovvero di varianti che, a meno
di mutazioni, un organismo eredita da ogni genitore.
In questo contesto trova spazio uno dei problemi fondamentali della
bioinformatica, ovvero quello del pattern matching. Inizialmente tale problema
era relativo alla ricerca di una stringa (pattern) all'interno di un testo di
grandi dimensioni, cioè il genoma di riferimento. Ora, con l'introduzione del
pangenoma, il problema deve essere risolto sulle nuove strutture di
rappresentazione del pangenoma.
\subsection*{Slide 3}
Lo scopo di questa tesi è progettare strutture dati e algoritmi efficienti per
risolvere il problema del pattern matching, inteso come ricerca dei set-maximal
exact match (SMEM) tra un aplotipo esterno e un pannello di aplotipi, in una
delle strutture dati più utilizzata per la rappresentazione del pangenoma: la
trasformata di Burrows–Wheeler Posizionale (PBWT).\\
Questo progetto, svolto in collaborazione con il laboratorio BIAS e
con diversi ricercatori internazionali (University of Florida, Dalhousie
University e Tokyo Medical and Dental University), permetterà la gestione e lo
studio (ad esempio nei GWAS) dei
sempre più grandi dati provenienti dalle tecnologie di sequenziamento. Inoltre,
con tale progetto, si è confermata l'ovvia correlazione tra la BWT e la PBWT,
estendendo tale correlazione anche alle rispettive varianti run-length.
\subsection*{Slide 4}
In questa breve presentazione è impossibile entrare nei dettagli di tutti i
concetti teorici alla base di questo progetto. Tra di essi si hanno:
\begin{itemize}
  \item bitvector e bitvector sparsi, strutture succinte alla base del lavoro
  \item intvector compressi, strutture compresse che hanno permesso di lavorare
  con valori interi
  \item straight-line program ($\SLP$) e longest common extension ($\LCE$)
  query, una grammatica context-free compressa che permette random access e LCE
  query in tempo logaritmico
  \item trasformata di Burrows--Wheeler ($\BWT$), (inverse) suffix array
  ($\IISA$), (permuted) longest common prefix ($\PPLCP$), funzione $\varphi$,
  FM-index, 
  $\LF$-mapping, maximal exact matches ($\MEM$), e tutte le altre teorie allo
  stato dell'arte su questa trasformata
  \item trasformata di Burrows--Wheeler run-length encoded ($\RLBWT$),
  r-index, Toheold lemma, matching statistics ($\MS$) e tutti i più recenti
  studi sull'uso del 
  run-length encoded
  \item trasformata di Burrows--Wheeler posizionale ($\PBWT$) e set-maximal
  exact match ($\SMEM$), che invece per ovvie ragioni vedremo un po' più nel
  dettaglio
\end{itemize}
\subsection*{Slide 5}
In merito alla $\RLBWT$ bisogna citare due recenti lavori, che sfruttano tale
trasformata per calcolare le matching statistics e da qui calcolare $\MEM$.
Il primo è MONI (di Rossi et al.), che sfrutta il concetto di threshold (minimo
lcp in una run) e il random access al pannello per il calcolo delle matching
statistics. \\
Il secondo è PHONI (di Boucher, Rossi et al.), che sfrutta invece le $\LCE$
query per fare il calcolo in una singola passata sul pattern, ottimizzando ancor
di più la memoria necessaria.\\
Tali lavori sono da citare in quanto, in questo progetto, si sono create le
varianti ispirate ad entrambi i lavori per la $\PBWT$. A tal fine, come
vedremo, tutti i concetti teorici della $\RLBWT$ sono stati ripensati in ottica
posizionale.
\subsection*{Slide 6}
La Trasformata di Burrows–Wheeler Posizionale (PBWT), presentata da Durbin nel
2014, viene costruita a partire 
da un pannello di aplotipi, rappresentato, riferendosi al solo caso biallelico,
tramite una matrice binaria. La motivazione essenziale della PBWT è considerare
match, e quindi anche SMEM, dove anche le posizioni di inizio e fine sono
rispettate. Tale vincolo, da cui deriva il termine “posizionale”, non è
soddisfacibile dalla BWT ed è dovuto al fatto che ogni colonna (o indice della
query) rappresenta un preciso sito per una specifica variante genica.
Il funzionamento della PBWT prevede la costruzione di due insiemi di array,
tramite l’ordinamento dei prefissi inversi a ogni colonna del pannello, detti
insieme dei prefix array (che tiene traccia degli indici degli ordinamenti) e
insieme dei divergence array (che tiene traccia della colonna d'inizio del
prefisso inverso più lungo tra una riga e la precedente nel riordinamento ad una
certa colonna). Il pannello, permutato  
tramite l’insieme dei prefix array, è detto matrice PBWT. \\
Qui, ad esempio, vediamo la costruzione della trasformata alla colonna 6, basata
sul riordinamento fino alla quinta, e la conseguente produzione dei due array.
Si noti che il divergence può anche essere sostituito dal Reverse Longest Common
prefix, che memorizza la lunghezza del prefisso comune.
\subsection*{Slide 7}
La PBWT permette di calcolare gli SMEM, di cui un esempio è qui disponibile,
tra un aplotipo esterno e il pannello in tempo Avg. $\mathcal{O}(N + c)$ (dove c
è il numero complessivo di SMEM), mentre una soluzione semplice impiegherebbe
$\mathcal{O}(N^2M)$ tramite il famoso algoritmo 5 di Durbin che si basa sul
mantenere ed eventualmente estendere un intervallo sui prefix array che contiene
gli indici delle righe che hanno uno SMEM fino a quella colonna. Se l'intervallo
non è più estendibile si reporta lo SMEM e si sfrutta il divergence array per
computare il nuovo intervallo.
Il tradeoff di questo algoritmo è la richiesta in termini di spazio (13NM
bytes), dovuto ad ulteriori array necessari in memoria per il ``mapping'',
ovvero il forward step, tra una colonna e la successiva nella matrice
$\PBWT$. Superare questo limite è l’obiettivo principale di questo progetto di
tesi.
\subsection*{Slide 8}
Si ha qui una breve panoramica delle componenti atomiche che hanno permesso la
creazione delle varianti della RLPBWT:
\begin{itemize}
  \item componenti per il mapping tra una colonna e la successiva nella
  $\PBWT$, tramite bitvector sparsi (\texttt{MAP-BV}) o intvector compressi
  (\texttt{MAP-INT}). Tali componenti includono tutte le informazioni necessarie
  al mapping tra una colonna e la successiva, memorizzando l'indice di ogni run
  e gli 
  equivalenti dell'FM-index. Inoltre si memorizza un singolo bool per poter
  risalire la carattere di ogni singola run
  \item componenti per le threshold (\texttt{THR-BV}/\texttt{THR-INT}), dove si
  memorizza l'indice ogni threshold
  \item componente per i prefix array sample (\texttt{PERM}), ovvero i valori di
  prefix array ad inizio e fine di ogni run
  \item componenti per il random access, tramite pannello di bitvector
  (\texttt{RA-BV}) tramite $\SLP$ (\texttt{RA-SLP})
  \item componente per le $\LCE$ query con $\SLP$ (\texttt{LCE})
  \item componente per il calcolo delle funzioni $\varphi$ e $\varphi^{-1}$
  (\texttt{PHI}), che permettono data una colonna e un valore di prefix array,
  di sapere quale sia il valore precedente e quello successivo
  \item componente per il reverse longest common prefix (\texttt{RLCP}), che non
  scala sul numero di run
\end{itemize}
Il senso di queste multiple componenti si ritrova nel fatto che, parlando di
strutture dati succinte e compresse, è difficile stimare l'effettivo spazio
necessario basandosi solo sulle complessità asintotiche. Inoltre, dal punto di
vista temporale, si aggiunge il problema che gli algoritmi tratti dipendono
fortemente dalla caratteristica del dato. Al fine di esplorare a pieno le varie
soluzioni quindi, oltre ad aver studiato e implementato le varianti di MONI e
PHONI, si sono studiati gli usi di varie strutture dati sottostanti.
\subsection*{Slide 9}
Possiamo qui velocemente vedere alcune dei confronti tra le componenti con
multiple rappresentazioni in termini di complessità temporale. Avendo che
$M>>\rho$ si nota come gli intvector compressi si preannunciano più veloci.
Si nota inoltre come il random access (o il calcolo delle $\LCE$ query), per
quanto tale caso peggiore sia irrealistico nel nostro caso dovendosi in realtà
basare sulla stringa prodotta dall'SLP, sia nettamente meno performante con tale
grammatica compressa.
\subsection*{Slide 10}
Possiamo qui confrontare velocemente a sinistra le stime di memoria dell'uso di
un bitvector sparso e un intvector compressso, stime derivanti dai dati di SDSL
(la lib usata) e dal numero di run attese (proporzionale a quanto visto con i
dati del 1000 genome project). Si nota quindi che, per quanto all'inizio gli
intvector compressi richiedano meno memoria tale comportamento è destinato ad
invertirsi all'aumentare del numero di sample.\\
A destra invece possiamo notare il forte vantaggio dell'SLP, anticipando già i
risultati, rispetto ai pannelli del 1000 genome project.
\subsection*{Slide 11 e 12}
Si ha quindi lo schema che mostra le 8 strutture dati studiate. Di fatto le
soluzioni sono 3, che diventano 8 per il discorso di dualità visto
precedentemente.
Si hanno quindi:
\begin{enumerate}
  \item le strutture che vediamo in centro, basate sul rifacimento
  dell'algoritmo 5 di Durbin e sull'uso dell'RLCP. Non è in grado di sapere
  quali siano le righe che presentano un certo SMEM ma solo quante
  \item le soluzioni ispirate a MONI
  \item le soluzioni ispirate a PHONI
\end{enumerate}
Le ultime due soluzioni computano esattamente quali righe presentano uno SMEM.\\
È superfluo notare come le prime due soluzioni non siano effettivamente
utilizzabili ma sono state citate in quanto punto iniziale di questa tesi e dei
primi studi sull'uso del run-length.\\
\textbf{magari qui si può dire di più sulla pic}
\subsection*{Slide 13}
In questa slide possiamo visualizzare un semplice esempio di matching statistics
con la PBWT. Che sono così definite:
\begin{definizione}
  Dato un pannello $X$, di dimensioni $M\times N$, con $M$ individui e $N$ siti,
  e un aplotipo esterno/pattern $z$, tale che $|z|=N$, si definisce matching
  statistics di $z$ su $X$ un array $\MS$ di coppie $(\row,\len)$, di lunghezza
  $N$, tale che (avendo che $x_i$ indica l'$i$-esima riga del pannello $X$): 
  \begin{itemize}
    \item $x_{\MS[i].\row}[i-\MS[i].\len+1,i]=z[i-\MS[i].\len+1,i]$, ovvero si
    ha che 
    l'aplotipo query ha un match, terminante in colonna $i$, con la riga
    $\MS[i].\row$  
    \item $z[i-\MS[i].\len,i]$ non è un suffisso terminante in colonna $i$ di un
    qualsiasi sottoinsieme di righe di $X$. In altri termini, il match non deve
    essere ulteriormente estendibile a sinistra
  \end{itemize}
\end{definizione}
E ne segue il seguente lemma:
\begin{lemma}
  Dato un pannello $X$, di dimensioni $M\times N$, con $M$ individui e $N$
  siti, un aplotipo esterno/pattern $z$, tale che $|z|=N$, e il corrispondente
  array di matching statistics $\MS$ si ha che $z[i-l+1,i]$
  presenta uno $\SMEM$ di lunghezza $l$ in con la riga $\MS[i].\row$ del
  pannello $X$ sse: 
  \[\MS[i].\len=l\land(i=N-1\lor \MS[i].\len\geq \MS[i+1].\len)\]
\end{lemma}
Ad esempio, con $k=5$, abbiamo $\row=13$ e $\len=6$, infatti con la riga 6 si ha
un suffisso comune lungo 6, terminante in $k=5$, con la riga 13.
\subsection*{Slide 14}
Possiamo qui vedere, ad alto livello, come funzioni la struttura per il calcolo
delle funzioni $\varphi$ che permettono, dato un valore di prefix array e una
colonna, di computare il valore di prefix array precedente e quello
successivo. Il computo si basa sull'uso dei prefix array sample ed eventualmente
dell'ultimo prefix array considerando quando due linee consecutive si separano
durante le permutazioni. Quando si separano sono sicuramente una fine di run e
una testa di run e quindi si può sapere che fino a quella colonna sono
consecutive. Si tiene traccia quindi tramite bitvector di dove una riga sia
testa o coda di run e tramite intvector compresso dell'indice della riga sopra e
sotto. L'ultimo prefix array serve per quelle righe che non si ``spezzano'' mai.
In tal modo, come visibile nell'esempio per $k=0$ (dove si cerca chi sia sotto
$m$ e chi sopra $j$), si può computare con la
funzione rank a che colonna sia lo spazio e quindi accedere agli intvector per
computare l'indice di prefix array.\\
Quindi si circa in su e in giù a partire dallo SMEM di MS fino a che si ha il
medesimo SMEM (avendo che sono tutti consecutivi nel riordinamento in una certa
colonna)
\subsection*{Slide 15}
La sperimentazione, orchestrata tramite \texttt{snakemake}, è stata
effettuata su una macchina con processore 
Intel Xeon E5-2640 V4 ($2,40$GHz), $756$GB di RAM, $768$GB di swap e
sistema operativo Ubuntu 20.04.4 LTS.\\
Si sono confrontate l'implementazione in \Cplusplus $\,\,$della $\RLPBWT$ e
l'implementazione in C ufficiale della $\PBWT$.\\
Si segnala che la RLPBWT supporta lo studio multithread di stringhe ma è stato
usato un singolo thread per questi test a fini di avere risultati più
comparabili.\\
Vediamo qui le caratteristiche dei pannelli usati, relativi alla phase 3 del
1000 Genome Project. Si nota come il numero di run sia molto inferiore
all'altezza del pannello, fattore che conferma l'utilità del run-length
encoding.
\subsection*{Slide 16}

\end{document}


% LocalWords:  pangenoma naive sottostringa BWT sottostringhe Durbin prefix MEM
% LocalWords:  array matching threshold divergence PBWT SMEM query statistics
% LocalWords:  RLBWT aplotipi aplotipo Burrows Wheeler RLPBWT maximal exact LCP
% LocalWords:  LCE
