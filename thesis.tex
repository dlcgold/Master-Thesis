\documentclass[a4paper,12pt, oneside]{book}

% \usepackage{fullpage}
\usepackage[italian]{babel}
\usepackage[utf8]{inputenc}
\usepackage{amssymb}
\usepackage{amsthm}
\usepackage{graphics}
\usepackage{amsfonts}
\usepackage{listings}
\usepackage{amsmath}
\usepackage{amstext}
\usepackage{colortbl}
\usepackage{engrec}
\usepackage{rotating}
\usepackage{subcaption}
\usepackage{verbatim}
\usepackage[safe,extra]{tipa}
% \usepackage{showkeys}
\usepackage{multirow}
\usepackage{hyperref}
\usepackage{microtype}
% \usepackage{fontspec}
\usepackage{enumerate}
\usepackage{braket}
\usepackage{relsize}
\usepackage{marginnote}
\usepackage{pgfplots}
\usepackage{cancel}
\usepackage{polynom}
\usepackage{booktabs}
\usepackage{enumitem}
\usepackage{framed}
\usepackage{pdfpages}
\usepackage{pgfplots}
\usepackage{algorithm}
\usepackage[backend=biber, backref=true, sorting=none]{biblatex}
\usepackage{fvextra}
\usepackage{csquotes}
% \usepackage{natbib}
% \usepackage{algpseudocode}
% \usepackage[cache=false]{minted}
\usepackage{mathtools}
\usepackage[noend]{algpseudocode}
\usepackage{svg}
\usepackage{graphicx}
\usepackage{hyperref}
\usepackage{setspace}
\usepackage{geometry}
\usepackage{blindtext}
\usepackage{titleps}
\addbibresource{thesis.bib}
\usepackage{tikz}\usetikzlibrary{er}\tikzset{multi  attribute /.style={attribute
    ,double  distance =1.5pt}}\tikzset{derived  attribute /.style={attribute
    ,dashed}}\tikzset{total /.style={double  distance =1.5pt}}\tikzset{every
  entity /.style={draw=orange , fill=orange!20}}\tikzset{every  attribute
  /.style={draw=MediumPurple1, fill=MediumPurple1!20}}\tikzset{every
  relationship /.style={draw=Chartreuse2,
    fill=Chartreuse2!20}}\newcommand{\key}[1]{\underline{#1}}
\usetikzlibrary{arrows.meta}
\usetikzlibrary{decorations.markings}
\usetikzlibrary{arrows,shapes,backgrounds,petri} 
\usetikzlibrary{automata,positioning}
\usetikzlibrary{matrix}
% \renewcommand{\chaptermark}[1]{\markboth{#1}{}}
\usepackage{fancyhdr}
\pagestyle{fancy}

\fancyhead[LO,RE]{\slshape \leftmark}
% \fancyhead[CO,CE]{\slshape\rightmark}
\fancyhead[LE,RO]{\slshape\rightmark}
\fancyfoot[C]{\thepage}
% \fancyhf{}
% \fancyhead[LO,RE]{\slshape \leftmark}
% % \fancyhead[CO,CE]{\slshape\rightmark}
% \fancyhead[LE,RO]{\slshape\thepage}
% \renewcommand{\footrulewidth}{0pt}
% \fancyfoot[C]{\thepage}
% \title{Relazione}
% \fancypagestyle{plain}{% \fancyhf{} % clear all header and footer fields
% \fancyhead[RO,RE]{\thepage}%RO=right odd, RE=right even
% \renewcommand{\headrulewidth}{0pt}
% \renewcommand{\footrulewidth}{0.3pt}}


\pgfplotsset{compat=1.13}

\begin{document}

% \maketitle
\newgeometry{margin=1in} 
\begin{titlepage}
  

  \noindent
  \begin{minipage}[t]{0.19\textwidth}
    \vspace{-4mm}{\includegraphics[scale=1.15]{img/logo_unimib.pdf}}
  \end{minipage}
  \begin{minipage}[t]{0.81\textwidth}
    {
      \setstretch{1.42}
      {\textsc{Università degli Studi di Milano - Bicocca}} \\
      \textbf{Scuola di Scienze} \\
      \textbf{Dipartimento di Informatica, Sistemistica e Comunicazione} \\
      \textbf{Corso di Laurea Magistrale in Informatica} \\
      \par
    }
  \end{minipage}
  
  \vspace{40mm}
  
  \begin{center}
    {\LARGE{
        \setstretch{1.2}
        \textbf{Algoritmi per la trasformata di}}}
    \vspace{1mm}
    {\LARGE{
        \setstretch{1.2}
        \textbf{Burrows-Wheeler Posizionale con}}}
    \vspace{1mm}
    {\LARGE{
        \setstretch{1.2}
        \textbf{compressione run-length, RLPBWT}}}
    
  \end{center}
  
  \vspace{48mm}

  \noindent
  {\large \textbf{Relatore:} \textit{Prof.ssa Raffaella Rizzi}} \\

  \noindent
  {\large \textbf{Correlatore:} \textit{}}
  
  \vspace{15mm}

  \begin{flushright}
    \textbf{\large Tesi di Laurea Magistrale di:} \\
    \large{\textit{Davide Cozzi}}\\
    \large{\textit{Matricola 829827}}
  \end{flushright}
  
  \vspace{40mm}
  \begin{center}
    {\large{\bf Anno Accademico 2021-2022}}
  \end{center}

  \restoregeometry
  
\end{titlepage}
\restoregeometry
\definecolor{shadecolor}{gray}{0.80}
\setlist{leftmargin = 2cm}
\newtheorem{teorema}{Teorema}
\newtheorem{definizione}{Definizione}
\newtheorem{esempio}{Esempio}
\newtheorem{corollario}{Corollario}
\newtheorem{lemma}{Lemma}
\newtheorem{osservazione}{Osservazione}
\newtheorem{nota}{Nota}
\newtheorem{esercizio}{Esercizio}

\algdef{SE}[DOWHILE]{Do}{doWhile}{\algorithmicdo}[1]{\algorithmicwhile\ #1}

\renewcommand{\chaptermark}[1]{%
  \markboth{\chaptername
    \ \thechapter.\ #1}{}}
\renewcommand{\sectionmark}[1]{\markright{\thesection.\ #1}}
\newcommand{\floor}[1]{\lfloor #1 \rfloor}

\newcommand{\MYhref}[3][blue]{\href{#2}{\color{#1}{#3}}}%
\newcommand{\hiddenchapter}[1]{
  \stepcounter{chapter*}
  \chapter*{\arabic{chapter}\hspace{1em}{#1}}
}
% \pagenumbering{roman}

\newpage
\thispagestyle{plain}
\begin{flushleft}
  \huge{\textbf{Abstract}}
\end{flushleft}
\vspace{10mm}
\textit{}
{
  \pagestyle{plain}
  \tableofcontents
  \cleardoublepage
}
\chapter{Introduzione}
\section{Motivazioni Biologiche}
\section{Bitvector sparsi}
\section{Straight-Line Program}
\subsection{Random access}
\subsection{Longest Common Extension}
\section{Trasformata di Burrows-Wheeler}
\subsection{Trasformata di Burrows-Wheeler run-length}
\subsection{Matching Statistics}
\subsection{R-index}
\subsection{MONI}
\subsection{PHONI}
\section{Trasformata di Burrows-Wheeler posizionale}
\subsection{Implementazione originale}
\subsubsection{Gli algoritmi di Durbin}
\subsubsection{Limiti spaziali}
\subsection{Varianti della PBWT}
\subsubsection{PBWT multi-allelica}
\subsubsection{PBWT con struttura LEAP}
\subsubsection{PBWT dinamica}
\subsubsection{PBWT bidirezionale}
\subsubsection{Recenti sviluppi}
\chapter{Metodo}
\section{Introduzione agli strumenti usati}
\subsection{SDSL}
\subsection{BigRepair}
\subsection{ShapedSlp}
\subsubsection{Ricostruzione del panel}
\section{Introduzione alle varianti della RLPBWT}
\subsection{Perché un'implementazione run-length}
\section{Mapping nella RLPBWT}
\section{RLPBWT naive}
\subsection{Algoritmo per match massimali}
\section{RLPBWT con bitvectors}
\subsection{Algoritmo per match massimali}
\section{RLPBWT con pannello}
\subsection{Algoritmo con matching statistics}
\section{RLPBWT con SLP}
\subsection{Algoritmo con matching statistics}
\section{Funzione Phi}
\subsection{Costruzione della struttura di supporto}
\subsection{Estensione dei match}
\chapter{Risultati}
\section{Ambiente di benchmark}
\subsection{Descrizione input}
\section{Analisi temporale}
\section{Analisi spaziale}
\chapter{Conclusioni}
\section{Sviluppi futuri}
\subsection{K-mems}
\subsection{RLPBWT multi-allelica}
\subsection{RLPBWT con dati mancanti}
%% BIBLIOGRAFIA
\addcontentsline{toc}{chapter}{Bibliografia e sitografia}
\printbibliography[title={Bibliografia e sitografia}]
\appendix
\chapter{Pseudocodici}
\chapter{Tabelle}
\begin{comment}
  \newpage
  \thispagestyle{empty}
  \begin{flushleft}
    \textit{}.%\\
    \[\sim\cdot\sim\]
    % \vspace{6mm}
    \textit{}.%\\ 
    \[\sim\cdot\sim\]
    % \vspace{6mm}
    \textit{}.\\
    \vspace{2mm}
    \textit{}.%\\
    
    \[\sim\cdot\sim\]
    % \vspace{6mm}
    \textit{}.\\
    \vspace{2mm}
    \textit{}.\\
    \vspace{2mm}
    \textit{}.
    \[\sim\cdot\sim\]
    % \vspace{6mm}
    \textit{}.
  \end{flushleft}
\end{comment}
\end{document}

% LocalWords:  divergence prefix