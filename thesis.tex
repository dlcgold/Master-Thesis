%\documentclass[a4paper,12pt, oneside]{book}
\documentclass[a4paper,12pt, oneside, draft]{book}
% \usepackage{fullpage}
\usepackage[italian]{babel}
\usepackage[utf8]{inputenc}
\usepackage{amssymb}
\usepackage{amsthm}
\usepackage{graphics}
\usepackage{amsfonts}
\usepackage{listings}
\usepackage{amsmath}
\usepackage{amstext}
\usepackage{colortbl}
\usepackage{engrec}
\usepackage{appendix}
\usepackage{multicol}
\usepackage{rotating}
\usepackage{subcaption}
\usepackage{verbatim}
\usepackage{stackengine}
\usepackage[safe,extra]{tipa}
% \usepackage{showkeys}
\usepackage{multirow}
\usepackage{hyperref}
\usepackage{microtype}
% \usepackage{fontspec}
\usepackage{enumerate}
\usepackage{braket}
\usepackage{relsize}
\usepackage{marginnote}
\usepackage{pgfplots}
\usepackage{cancel}
\usepackage{polynom}
\usepackage{booktabs}
\usepackage{enumitem}
\usepackage{framed}
\usepackage{pdfpages}
\usepackage{pgfplots}
\usepackage[chapter]{algorithm}
%\usepackage[ruled,vlined,linesnumbered]{algorithm2e}
%\usepackage[ruled,vlined]{algorithm2e}
\makeatletter
\renewcommand{\ALG@name}{Algoritmo}
\renewcommand{\listalgorithmname}{Lista degli algoritmi}
\makeatother
%\usepackage[backend=biber, backref=true, sorting=none]{biblatex}
\usepackage{fvextra}
\usepackage{csquotes}
% \usepackage{natbib}
% \usepackage{algpseudocode}
\usepackage[cache=false]{minted}
\usepackage{mathtools}
\usepackage[noend]{algpseudocode}
\usepackage{svg}
\usepackage{graphicx}
\usepackage{hyperref}
\usepackage{setspace}
\usepackage{geometry}
\usepackage{blindtext}
\usepackage{titleps}
% \makeatletter
% \long\def\@makecaption#1#2{%
%   \vskip\abovecaptionskip
%     \bfseries #1: #2\par
%   \vskip\belowcaptionskip}%
% \makeatother
% \lstset{ % General setup for the package
%     language=Perl,
%     basicstyle=\small\sffamily,
%     frame=lines,
%     tabsize=4,
%     columns=fixed,
%     showstringspaces=false,
%     showtabs=false,
%     keepspaces,
%     commentstyle=\color{red},
%     keywordstyle=\color{blue}
% }
\usepackage{tikz}\usetikzlibrary{er}\tikzset{multi  attribute /.style={attribute
    ,double  distance =1.5pt}}\tikzset{derived  attribute /.style={attribute
    ,dashed}}\tikzset{total /.style={double  distance =1.5pt}}\tikzset{every
  entity /.style={draw=orange , fill=orange!20}}\tikzset{every  attribute
  /.style={draw=MediumPurple1, fill=MediumPurple1!20}}\tikzset{every
  relationship /.style={draw=Chartreuse2,
    fill=Chartreuse2!20}}\newcommand{\key}[1]{\underline{#1}}
\usetikzlibrary{arrows.meta}
\usetikzlibrary{decorations.markings}
\usetikzlibrary{arrows,shapes,backgrounds,petri} 
\usetikzlibrary{automata,positioning}
\usetikzlibrary{matrix}

\usepackage[textsize=scriptsize, textwidth = 25mm]{todonotes}
\newcommand{\dc}[1]{\todo[backgroundcolor=yellow]{\textbf{DC} #1}}
\newcommand{\gdv}[1]{\todo[backgroundcolor=blue]{\textbf{GDV} #1}}
\newcommand{\pb}[1]{\todo[backgroundcolor=red]{\textbf{PB} #1}}
\newcommand{\yp}[1]{\todo[backgroundcolor=green]{\textbf{YP} #1}}
\newcommand{\rr}[1]{\todo[backgroundcolor=pink]{\textbf{RR} #1}}

\def\SLP{\mbox{\rm {\sf SLP}}}
\def\rank{\mbox{\rm {\sf rank}}}
\def\lcs{\mbox{\rm {\sf lcs}}}
\def\lcp{\mbox{\rm {\sf lcp}}}
\def\lce{\mbox{\rm {\sf lce}}}
\def\LCE{\mbox{\rm {\sf LCE}}}
\def\select{\mbox{\rm {\sf select}}}
\def\col{\mbox{\rm {\sf col}}}
\def\NULL{\mbox{\rm {\sf null}}}
\def\len{\mbox{\rm {\sf len}}}
\def\pos{\mbox{\rm {\sf pos}}}
\def\row{\mbox{\rm {\sf row}}}
\def\LF{\mbox{\rm {\sf LF}}}
\def\FL{\mbox{\rm {\sf FL}}}
\def\A{\mbox{\rm {\sf A}}}
\def\SA{\mbox{\rm {\sf SA}}}
\def\LCP{\mbox{\rm {\sf LCP}}}
\def\ISA{\mbox{\rm {\sf ISA}}}
\def\PLCP{\mbox{\rm {\sf PLCP}}}
\def\RLBWT{\mbox{\rm {\sf RLBWT}}}
\def\MEM{\mbox{\rm {\sf MEM}}}
\def\LCP{\mbox{\rm {\sf LCP}}}
\def\NSV{\mbox{\rm {\sf NSV}}}
\def\PSV{\mbox{\rm {\sf PSV}}}
\def\RMQ{\mbox{\rm {\sf RMQ}}}
\def\BWT{\mbox{\rm {\sf BWT}}}
\def\PBWT{\mbox{\rm {\sf PBWT}}}
\def\RLPBWT{\mbox{\rm {\sf RLPBWT}}}
\def\SMEM{\mbox{\rm {\sf SMEM}}}
\def\M{\mbox{\rm {\sf M}}}
\def\L{\mbox{\rm {\sf L}}}
\def\F{\mbox{\rm {\sf F}}}
\def\DA{\mbox{\rm {\sf DA}}}
\def\DM{\mbox{\rm {\sf DM}}}
\def\PA{\mbox{\rm {\sf PA}}}
\def\LCE{\mbox{\rm {\sf LCE}}}
\newcommand{\Oh}{\mathcal{O}}

% \renewcommand{\chaptermark}[1]{\markboth{#1}{}}
\usepackage{fancyhdr}
\pagestyle{fancy}

\fancyhead[LO,RE]{\slshape \leftmark}
% \fancyhead[CO,CE]{\slshape\rightmark}
\fancyhead[LE,RO]{\slshape\rightmark}
\fancyfoot[C]{\thepage}
% \fancyhf{}
% \fancyhead[LO,RE]{\slshape \leftmark}
% % \fancyhead[CO,CE]{\slshape\rightmark}
% \fancyhead[LE,RO]{\slshape\thepage}
% \renewcommand{\footrulewidth}{0pt}
% \fancyfoot[C]{\thepage}
% \title{Relazione}
% \fancypagestyle{plain}{% \fancyhf{} % clear all header and footer fields
% \fancyhead[RO,RE]{\thepage}%RO=right odd, RE=right even
% \renewcommand{\headrulewidth}{0pt}
% \renewcommand{\footrulewidth}{0.3pt}}

% \AtBeginDocument{%
%   \renewcommand{\thelisting}{\Alph{chapter}.\arabic{listing}}
%   % \renewcommand\thetable{\Alph{section}.\arabic{table}}
%   % \renewcommand\thefigure{\Alph{section}.\arabic{figure}}
% }
% %Automatically change the driver counter for reset:
% \makeatletter
% \g@addto@macro\appendix{%
%   \counterwithin*{listing}{section}%
% }
% \makeatother
% c C plus plus
\def\Cplusplus{C\raisebox{0.5ex}{\tiny\textbf{++}}}
\definecolor{nordred}{RGB}{191, 97, 106}
\definecolor{nordgreen}{RGB}{135, 157, 116}

\pgfplotsset{compat=1.13}
\setlist[itemize]{leftmargin=.5in}
\setlist[enumerate]{leftmargin=.5in}
\begin{document}

% \maketitle
\newgeometry{margin=1in} 
\begin{titlepage}
  

  \noindent
  \begin{minipage}[t]{0.19\textwidth}
    \vspace{-4mm}{\includegraphics[scale=1.15]{img/logo_unimib.pdf}}
  \end{minipage}
  \begin{minipage}[t]{0.81\textwidth}
    {
      \setstretch{1.42}
      {\textsc{Università degli Studi di Milano - Bicocca}} \\
      \textbf{Scuola di Scienze} \\
      \textbf{Dipartimento di Informatica, Sistemistica e Comunicazione} \\
      \textbf{Corso di Laurea Magistrale in Informatica} \\
      \par
    }
  \end{minipage}
  
  \vspace{40mm}
  
  \begin{center}
    {\LARGE{
        \setstretch{1.2}
        \textbf{Algoritmi per la trasformata di}}}
    \vspace{1mm}
    {\LARGE{
        \setstretch{1.2}
        \textbf{Burrows--Wheeler posizionale con}}}
    \vspace{1mm}
    {\LARGE{
        \setstretch{1.2}
        \textbf{compressione run-length}}}
    
  \end{center}
  
  \vspace{48mm}

  \noindent
  {\large \textbf{Relatore:} \textit{Prof.~Raffaella Rizzi}} \\

  \noindent
  {\large \textbf{Correlatore:} \textit{Dr.~Yuri Pirola}}
  
  \vspace{15mm}

  \begin{flushright}
    \textbf{\large Tesi di Laurea Magistrale di:} \\
    \large{\textit{Davide Cozzi}}\\
    \large{\textit{Matricola 829827}}
  \end{flushright}
  
  \vspace{40mm}
  \begin{center}
    {\large{\bf Anno Accademico 2021-2022}}
  \end{center}

  \restoregeometry
  
\end{titlepage}
\restoregeometry
\definecolor{shadecolor}{gray}{0.80}
\setlist{leftmargin = 2cm}
\newtheorem{teorema}{Teorema}
\newtheorem{definizione}{Definizione}
\newtheorem{esempio}{Esempio}
\newtheorem{corollario}{Corollario}
\newtheorem{lemma}{Lemma}
\newtheorem{osservazione}{Osservazione}
\newtheorem{nota}{Nota}
\newtheorem{esercizio}{Esercizio}

\algdef{SE}[DOWHILE]{Do}{doWhile}{\algorithmicdo}[1]{\algorithmicwhile\ #1}
\renewcommand{\bibname}{Bibliografia}
\renewcommand{\chaptermark}[1]{%
  \markboth{\chaptername
    \ \thechapter.\ #1}{}}
\renewcommand{\sectionmark}[1]{\markright{\thesection.\ #1}}
\newcommand{\floor}[1]{\lfloor #1 \rfloor}

\newcommand{\MYhref}[3][blue]{\href{#2}{\color{#1}{#3}}}%
\newcommand{\hiddenchapter}[1]{
  \stepcounter{chapter*}
  \chapter*{\arabic{chapter}\hspace{1em}{#1}}
}
\newcommand\xrowht[2][0]{\addstackgap[.5\dimexpr#2\relax]{\vphantom{#1}}}
% \pagenumbering{roman}
\begin{titlepage}
  \begin{flushright}
    \textit{E pensare che\\
      mi iscrissi ad informatica\\
      per fare il sistemista!}
  \end{flushright}
\end{titlepage}
\newpage
\newpage
% \thispagestyle{plain}
% \begin{flushleft}
%   \huge{\textbf{Abstract}}
% \end{flushleft}
% \vspace{10mm}
% Negli ultimi anni, a partire dall'articolo di Durbin del 2014, la
\textbf{Trasformata di Burrows-Wheeler Posizionale (\textit{PBWT})} è stata al
centro delle ricerche riguardanti 
il disegno di algoritmi efficienti per il pattern matching su grandi collezioni
di aplotipi. Come indicato da Durbin stesso, una \textbf{rappresentazione
  run-length encoded della PBWT} risulta essere molto efficiente dal punto di
vista della memorizzazione della stessa.\\
In questa tesi, svolta in collaborazione con il
laboratorio di ricerca \textbf{BIAS} del \textbf{Dipartimento di Informatica
  Sistemistica e Comunicazione \textit{(DISCo})}, con professori e ricercatori
dell'\textbf{University of Florida (\textit{UFL})} e della \textbf{DalHousie
  University}, si è quindi implementata una variante della \textbf{RLPBWT}, 
ispirata ai risultati già ottenuti con la \textbf{variante run-length encoded
  della BWT} tradizionale, che permettesse di risolvere il problema del matching
tra un aplotipo esterno e un pannello di aplotipi.\\
A tal fine si sono selezionate le
informazioni minimali da memorizzare per ogni run, utilizzando strutture dati
succinte (come gli sparse bit-vectors) al fine di ottimizzare la complessità
spaziale della struttura dati, e costruendo un efficiente algoritmo per
effettuare query alla struttura stessa. ​ 
{
  \pagestyle{plain}
  \tableofcontents
  \cleardoublepage
}
\chapter{Introduzione}
Negli ultimi anni si è assistito ad un cambio di paradigma nel campo della
bioinformatica, ovvero il passaggio dallo studio della sequenza lineare di un
singolo genoma a quello di un insieme di genomi, provenienti da un gran numero
di individui, al fine di poter considerare anche le varianti
geniche. Questo nuovo concetto è stato nominato per la prima volta, nel 2005,
da Tettelin \cite{tettelin} con il termine di \textit{pangenoma}. Grazie ai
risultati ottenuti in pangenomica, ci sono stati miglioramenti sia nel 
campo della biologia che in quello della medicina personalizzata, grazie al
fatto che, con il pangenoma, si migliora la precisione della rappresentazione di
multipli genomi e delle loro differenze. \\
Il genoma umano di riferimento (GRCh38.p14), è composto da circa
3.1 miliardi di basi, con più di 88 milioni 
varianti tra i genomi sequenziati, secondo i risultati ottenuti nel 1000 Genome
Project \cite{tutorial}. Considerando come la quantità dei dati di
sequenziamento sia destinata 
ad aumentare esponenzialmente nei prossimi anni, grazie al
miglioramento delle tecnologie di sequenziamento (Next Generation Sequencing e
Third-Generation Sequencing), risulta necessaria la costruzione di algoritmi e
strutture dati efficienti per gestire una tale informazione. 
In merito, uno degli approcci più usati per rappresentare il pangenoma è un
pannello di aplotipi \cite{pancon}, ovvero, computazionalmente, una matrice di
$M$ righe, corrispondenti agli individui, e $N$ colonne, corrispondenti ai siti
con le varianti. Si specifica che, con il termine
aplotipo, si intende l'insieme di alleli, ovvero di varianti, che, a meno di
mutazioni, un organismo eredita da ogni genitore.\\
In questo contesto trova spazio uno dei problemi fondamentali della
bioinformatica, ovvero quello del pattern matching. Inizialmente tale concetto
era relativo allo studio di un piccolo pattern all'interno di un testo di
grandi dimensioni, ovvero il genoma di riferimento. Ora, con l'introduzione 
del pangenoma, tale problema si è adattato alle nuove strutture
dati.\\
Lo scopo di questa tesi è ottimizzare il problema del pattern 
matching, inteso come ricerca dei \textbf{set-maximal exact match}
  ($\SMEM$) tra un aplotipo 
esterno e un pannello di aplotipi, in una delle 
strutture dati più utilizzata: la \textbf{trasformata di Burrows--Wheeler
  Posizionale} ($\PBWT$) \cite{pbwt}. Il progetto di tesi, svolto in
collaborazione con il 
prof. Gagie (Dalhousie University) e la prof.ssa Boucher 
(University of Florida), tra gli autori dei principali risultati ottenuti per la
\textbf{trasformata di Burrows--Wheeler run-length encoded} ($\RLBWT$)
\cite{rlbwt} 
\cite{gagie2020} \cite{moni} \cite{phoni}, ha quindi permesso lo sviluppo di
diverse strutture dati composte per la variante 
\textbf{run-length encoded} della \textbf{PBWT} ($\RLPBWT$),
efficienti dal punto di vista delle memoria utilizzata.\\
Parte delle strutture dati e degli algoritmi presentati in questa tesi sono
inseriti in un articolo, dal titolo \textit{Compressed data structures for
  population-scale positional Burrows--Wheeler transforms} \cite{rlpbwt}. Al
momento della 
stesura di questa tesi in fase di review per il giornale \textit{Briefings in
Bioinformatics} (\textit{Oxford Academic Press}).
\subsection*{Struttura della tesi}
Nel Capitolo \ref{prechap} si introdurranno i concetti di base, di ambito
computazionale e bioinformatico, necessari a
comprendere questa tesi. Nel Capitolo \ref{metchap} verranno discussi i
contributi di questa tesi, descrivendo le soluzioni algoritmiche e le
metodologie utilizzate per raggiungere gli obiettivi prefissati. Nel dettaglio
verranno presentate varie strutture dati che saranno le componenti necessarie
alla produzione delle strutture dati per la $\RLPBWT$ e al calcolo degli
$\SMEM$. Nel Capitolo 
\ref{reschap} si discuteranno i risultati ottenuti durante la
sperimentazione sui dati reali della \textit{phase 3} del \textbf{1000 Genome
  Project}, progetto, che ha avuto inizio nel 2008, il quale ha visto lo sforzo
della comunità scientifica internazionale per catalogazione delle variazioni
geniche umane. Infine, nel Capitolo \ref{conchap}, si trarranno le conclusioni 
di questo progetto di tesi discutendone, infine, i prospetti futuri.
\dc{L'intera introduzione va estesa}
\chapter{Preliminari}
\label{prechap}
In questo capitolo verranno specificati tutti i concetti fondamentali, allo
stato dell'arte, atti a comprendere i metodi usati in questa tesi.
Si introdurranno i concetti di:
\begin{itemize}
  \item bitvector
  \item straight-line program e longest common extension query
  \item suffix array e longest common prefix
  \item trasformata di Burrows--Wheeler e la sua variante run-length encoded
  \item trasformata di Burrows--Wheeler posizionale
\end{itemize}
L'unione di tutte queste strutture e di queste tecniche ha permesso la creazione
della $\RLPBWT$.\\
\noindent
\textit{A livello di notazione, si specifica inoltre che, con la notazione
  $T[i,j]$ si intende la sottostringa del testo/sequenza/riga/colonna $T$,
  iniziante all'indice $i$ e terminante all'indice $j$ incluso. Qualora si
  avesse $j>i$ si identifica la sottostringa nulla $\varepsilon$.}

% sezione introduzione biologia
%\input{include/bio}

% sezione bitvector
\section{Bit vector}
\textbf{TUTTE LE TABELLE VANNO VERIFICATE!!!}\\
Nell'ambito delle \textit{strutture dati succinte}, una delle strutture dati
principali, ormai sviluppatasi in molteplici varianti, è quella denominata
\textbf{bit vector}.
\begin{definizione}
  Si definisce un \textbf{bit vector} $B$ come un array di lunghezza $n$,
  popolato da elementi binari. Formalmente si ha quindi:
  \[B[i]=\{0,1\},\,\,\forall i \mbox{ t.c. } 0\leq i < n\]
  In alternativa si potrebbe avere come formalismo:
  \[B[i]=\{\bot,\top\},\,\,\forall i \mbox{ t.c. } 0\leq i < n\]
\end{definizione}
Nel corso degli ultimi anni si sono sviluppate diverse varianti dei \textit{bit
  vector}, finalizzate ad offrire diversi costi di complessità spaziale e
diversi tempi computazionali per le principali funzioni offerte.\\
Il primo vantaggio di questa struttura dati, nelle varianti che si andranno poi
a nominare, è quella di garantire \textit{random access} in tempo costante pur
sfruttando varie tecniche per la memorizzazione efficiente della stessa in
memoria. Lo spazio necessario per l'implementazione, presente in
\textit{SDSL} \cite{sdsl}, delle principali varianti è visualizzabile in tabella
\ref{tab:bvspace}. Il secondo vantaggio consiste nel fatto che i \textit{bit
  vector} permettono l'implementazione efficiente di due funzioni:
\begin{enumerate}
  \item la \textbf{funzione rank}
  \item la \textbf{funzione select}
\end{enumerate}
Tali funzioni, al costo di $\mathcal{O}(n)$ bit aggiuntivi, possono essere
supportate in tempo costante. Questo è però un discorso prettamente teorico,
infatti si vedrà come, nelle implementazioni in \textit{SDSL}, le complessità
temporali delle due funzioni non siano mai entrambe costanti.
\subsection{Funzione rank}
La prima funzione che si approfondisce è la \textbf{funzione rank}. Tale
funzione permette di calcolare il \textit{rango} di un dato elemento del
bit vector $B$, $|B|=n$. In altri termini, data una certa posizione $i$ del
\textit{bit vector}, la funzione restituisce il numero di 1 presenti fino a
quella data posizione, escluda. Più formalmente si ha:
\[rank_B(i)=\sum_{k=0}^{k<i}B[k],\,\,\forall i \mbox{ t.c. } 0\leq i < n\]
Come detto, da un punto di vista teorico, al costo di $\mathcal{O}(n)$ bit
aggiuntivi in memoria tale funzione sarebbe supportata in tempo
$\mathcal{O}(1)$. Questo però non risulta vero nelle principali
implementazioni. La complessità temporale varia infatti a seconda
dell'implementazione, anche in conseguenza al fatto che si ha una quantità
diversa di bit aggiuntivi salvati in memoria.
La tabella con le complessità temporali stimate della \textit{funzione rank},
per le varianti di \textit{bit vector} implementate in \textit{SDSL}, è
visualizzabile in tabella \ref{tab:rank}.

\subsection{Funzione select}
La seconda funzione fondamentale è la \textbf{funzione select}. Tale funzione
permette, dato un valore intero $i$, di calcolare l'indice dell'$i$-esimo valore
pari a 1 nel \textit{bit vector} $B$, tale che $|B|=n$. Più formalmente si ha
che:
\[select_B(i)=\min\{j < n\,|\,\,rank_B(j+1)=1\},\,\,\forall i \mbox{ t.c. }
  0<i\leq rank_B(n)\]
Anche in questo caso vale lo stesso discorso fatto per la \textit{funzione rank}
in merito alla complessità temporale e ai bit aggiuntivi. La tabella con le
complessità temporali stimate della \textit{funzione select}, 
per le varianti di \textit{bit vector} implementate in \textit{SDSL}, è
visualizzabile in tabella \ref{tab:select}.\\
Si può quindi vedere un semplice esempio esplicativo.
\begin{esempio}
  Ipotizziamo di avere il seguente bit vector $B$, di lunghezza $n=14$:
  \begin{center}
    \begin{tikzpicture} [nodes in empty cells,
      nodes={minimum width=0.6cm, minimum height=0.6cm},
      row sep=-\pgflinewidth, column sep=-\pgflinewidth]
      border/.style={draw}
      
      \matrix(vector)[matrix of nodes,
      row 1/.style={nodes={draw=none, minimum width=0.3cm}},
      nodes={draw}]
      {
        \tiny{0} & \tiny{1} & \tiny{2} & \tiny{3} & \tiny{4} & \tiny{5}&\tiny{6}
        & \tiny{7} & \tiny{8} & \tiny{9} & \tiny{10} & \tiny{11} & \tiny{12} &
        \tiny{13}\\  
        $\mathbf{1}$ & $0$ & $0$ & $\mathbf{1}$ & $0$ & $\mathbf{1}$ & $0$ &
        $\mathbf{1}$ & $0$ & $\mathbf{1}$ & $0$ & $0$ & $\mathbf{1}$ & $0$\\ 
      };
    \end{tikzpicture}
  \end{center}
  Si ha che, per esempio:
  \[\mathtt{rank}(6)=3\]
  \[\mathtt{select}(5) =9\]
\end{esempio}
Da un punto di vista pratico si vedrà nel corso di questa tesi come l'uso di
tali strutture, nel dettaglio l'uso dei \textit{bit vector plain} e dei
\textit{bit vector sparsi}, sia fondamentale sia nella costruzione delle
\textit{strutture run-length encoded} che nelle interrogazioni alle stesse.

% sezione slp
\section{Straight-line program}
\label{slpsec}
In ambito bioinformatico, una delle principali problematiche è la
gestione di testi molto estesi. Si pensi, ad esempio, al caso umano, dove il
primo cromosoma, il più lungo, conta circa $247.249.719$
\textit{pb} (paia di basi), nonostante l'uomo
non sia l'essere vivente con il genoma più esteso. Fatta questa breve
premessa, è facile comprendere l'importanza degli algoritmi e delle strutture
dati per la compressione di testi.\\
Per questa tesi si è pensato all'uso dei cosiddetti \textbf{straight-line
  program} ($\SLP$). In termini 
generici, un $\SLP$ è una \textit{grammatica context-free} che 
genera una e una sola parola \cite{slpsurvey}. Si parla, quindi, di
\textit{grammar-based compression}.
\begin{definizione}
  Sia dato un alfabeto finito $\Sigma$ di simboli terminali. Sia data una
  stringa $s=a_1,a_2,\ldots, a_n\in\Sigma^{*}$, lunga $n$ e costruita
  sull'alfabeto $\Sigma$, avendo $a_i\in\Sigma$, $\forall\, i \mbox{
    t.c. }1\leq i\leq n$. Si denota con $alph(s)=\{a_1,a_2,\ldots
  a_n\}$ l'insieme dei simboli della stringa $s$.\\
  Si definisce $\SLP$, costruito sull'alfabeto $\Sigma$, una grammatica
  context-free $\mathcal{A}$ tale che: 
  \begin{equation}
    \label{eq:slpdef}
    \mathcal{A}=\left(\mathcal{V}, \Sigma, \mathcal{S}, \mathcal{P}\right)
  \end{equation}
  Dove:
  \begin{itemize}
    \item $\mathcal{V}$ è l'insieme dei simboli non terminali
    \item $\Sigma$ è l'insieme dei simboli terminali
    \item $\mathcal{S}\in \mathcal{V}$ è il simbolo iniziale non terminale
    \item $\mathcal{P}$ è l'insieme delle produzioni, avendo che:
    \begin{equation}
      \label{eq:slpprod}
      \mathcal{P}\subseteq \mathcal{V}\times\left(\mathcal{V}\cup
        \Sigma\right)^{*}
    \end{equation}
  \end{itemize}
  Tale grammatica, per essere un $\SLP$, deve soddisfare due proprietà:
  \begin{enumerate}
    \item si ha una e una sola produzione $(A,\alpha)\in \mathcal{P}$,
    $\forall\, A\in \mathcal{V}$ e con $\alpha\in
    \left(\mathcal{V}\cup\Sigma\right)^{*}$ (si 
    noti che la produzione $(A,\alpha)$ può anche essere indicata con
    $A\to\alpha$) 
    \item la relazione $\{(A,B)\,|\,\,(A,\alpha)\in\mathcal{P},\,\,B\in
    alph(\alpha)\}$ è aciclica
    %\dc{verificare questo secondo punto}
  \end{enumerate}
  Si ha che la grandezza dell'$\,\SLP$ è calcolabile come:
  \begin{equation}
    \label{eq:slplen}
    |\mathcal{A}| = \sum_{(A,\alpha)\in\mathcal{P}}|\alpha|
  \end{equation}
  Il linguaggio $\mathcal{A}$ generato da un $\SLP$ consiste in una singola
  parola, denotata da $eval(\mathcal{A})$. 
\end{definizione}
A partire dall'$\,\SLP$ $\mathcal{A}$ si genera un \textit{albero
  di derivazione}, che, nel dettaglio, è un albero radicato e ordinato
dove la radice è etichettata con $\mathcal{S}$, ogni nodo
interno con un simbolo di $\mathcal{V}\cup\Sigma$ e ogni foglia
con un simbolo di $\Sigma$.
\begin{esempio}
  \label{ese:slpgagie}
  Si prenda, ad esempio \cite{slpgagie}, la seguente stringa:
  \[s=\mbox{GATTAGATACAT}\,\$\mbox{GATTACATAGAT}\]
  Si potrebbe produrre il seguente $\SLP$:
  \begin{multicols}{3}
    \begin{itemize}
      \item $\mbox{S}\to \mbox{ZWAY}\,\$\mbox{ZYAW}$
      \item $\mbox{Z}\to \mbox{WX}$
      \item $\mbox{Y}\to \mbox{CV}$
      \item $\mbox{X}\to \mbox{TA}$
      \item $\mbox{W}\to \mbox{GV}$
      \item $\mbox{V}\to \mbox{AT}$
    \end{itemize}
  \end{multicols}
  Al quale corrisponde il seguente albero di derivazione:
  \begin{figure}[H]
    \centering
    \includegraphics[width=\textwidth]{img/slpgagie.pdf}
  \end{figure}
  Si noti che il simbolo
  iniziale non terminante, ovvero la radice, è indicato con un cerchio giallo, i
  simboli non terminanti, ovvero i nodi interni, sono indicati dai cerchi blu
  mentre i simboli terminanti, ovvero le foglie, sono indicati dai quadrati
  verdi.
\end{esempio}
Nel 2020, Gagie et al. \cite{slpgagie} 
proposero un articolo, a cui si rimanda per approfondimenti, in merito a
miglioramenti prestazionali per il random access all'$\,\SLP$,
anche tramite l'uso dei bitvector sparsi.\\
Si stima che il tempo necessario al random access su un testo $T$, compresso
tramite $\SLP$ e lungo $n$, sia: 
\begin{equation}
  \label{eq:slptime}
  \mathcal{O}\left(\log n\right)
\end{equation}
L'uso degli \textit{SLP} è stato cruciale, come si vedrà più
avanti in questa tesi, per la costruzione della variante run-length encoded sia
della $\BWT$ che della $\PBWT$.
\subsection{Longest common extension}
Oltre a permettere il \textit{random access} al testo compresso, 
l'uso degli $\SLP$ permette di effettuare 
un'altra operazione in modo efficiente, ovvero il calcolo delle \textbf{longest
  common extension} ($\LCE$) \textbf{query}.
\begin{definizione}
  Dato un testo $T$, tale che $|T|=n$, il risultato della $\LCE$ query tra
  due posizioni $i$ e $j$, tali che $0\leq i,j<n$, corrisponde al più lungo
  prefisso comune tra le sottostringhe che hanno come indice di partenza $i$ e
  $j$, ovvero il più lungo prefisso comune tra $T[i,n-1]$ e $T[j,n-1]$.
\end{definizione}
Sfruttando l'$\,\SLP$ del testo $T$ è quindi possibile effettuare due
random access agli indici $i$ e $j$ del testo compresso, per poi ``risalire''
l'albero di derivazione al fine di computare il prefisso comune tra le due
sottostringhe.\\ 
Avendo l'$\,\SLP$ di un testo $T$ lungo $n$, si stima che il calcolo di una
$\LCE$ query di lunghezza $l$ sia
effettuabile in tempo: 
\begin{equation}
  \label{eq:lcetime}
  \mathcal{O}\left(\log n+l\right)\approx\mathcal{O}\left(\log n\right)
\end{equation}
Si noti che, normalmente, la lunghezza dell'$\,\LCE$ è trascurabile rispetto
alla lunghezza 
del testo.\\ 
In questa tesi, i due concetti di $\SLP$ ed $\LCE$ query verranno generalizzati
all'uso su matrici, permettendo una rappresentazione compatta in 
memoria per un pannello di aplotipi, garantendo random access.
\subsubsection{Librerie}
Da un punto di vista implementativo, l'oggetto contenente  l'$\,\SLP$ del
pannello viene costruito e interrogato mediante l'uso della libreria
\textbf{ShapedSlp}\footnote{\url{ttps://github.com/itomomoti/ShapedSlp}},
implementazione dei risultati teorici ottenuti da 
Gagie et al. \cite{slpgagie}. Inoltre, tale libreria basa il suo funzionamento
sull'uso di un'altra libreria, detta
\textbf{BigRePair}\footnote{\url{https://gitlab.com/manzai/bigrepair}}, che 
implementa ulteriori risultati teorici di Gagie et al. \cite{rpair} in merito
alla compressione, via uso di grammatiche, di file con frequenti ripetizioni
(come, nel nostro caso, i pannelli binari di aplotipi).\\
In termini di pipeline, si procede:
\begin{enumerate}
  \item generando la grammatica tramite BigRePair, che accetta
  come file di input un file \texttt{txt} ``raw'' oppure file in formato
  standard nel campo della bioinformatica, come i \texttt{FASTA}
  \item generando l'$\,\SLP$ tramite ShapedSlp specificatamente a
  partire dai risultati di BigRePair (si segnala che la libreria
  accetta anche grammatiche prodotte da altri software).
\end{enumerate}

% sezione suffix array
\section{Suffix Array}
Nel 1976, Manber e Myers \cite{sa} proposero una struttura dati per la
memorizzazione di 
stringhe e la loro interrogazione, efficiente sia in termini di uso della
memoria che di complessità temporale. Tale struttura venne denotata
\textbf{Suffix Array (\textit{SA})}.
\begin{definizione}
  Dato un testo $T$, \$-terminato (assumendo che il simbolo \$ sia sempre il
  simbolo lessicograficamente minore nell'alfabeto di studio), tale che $|T|=n$,
  si definisce \textbf{suffix 
    array} di $T$, denotato con $SA_T$, un di interi array lungo $n$, tale che
  $SA_T[i]=j$ sse il suffisso di ordine $j$, ovvero $T[j,n-1]$, è
  l'$i$-esimo suffisso nell’ordinamento lessicografico dei suffissi di $T$. Ne
  segue che, presi $i,i'\in \mathbb{N}$
  tali che $0\leq i < i' < n$ allora vale che, indicando con $\prec$
  l'ordinamento lessicografico:
  \begin{equation}
    \label{eq:saord}
    T[SA_T[i],n-1] \prec T[SA_T[i'],n-1]
  \end{equation}
  Il \textbf{suffix array} è quindi una permutazione dei numeri interi in
  $\{0,n-1\}$. 
\end{definizione}
\begin{esempio}
  Si prenda la stringa:
  \[s=\mbox{mississippi\$},\,\,|s|=12\]
  Si producono quindi i seguenti suffissi e il loro riordinamento:
  \begin{table}[H]
    \footnotesize
    \centering
    \begin{tabular}{c|l}
      \textbf{Indice del suffisso} & \textbf{Suffisso}\\
      \hline
      0 & mississippi\$\\
      1 & ississippi\$\\
      2 & ssissippi\$\\
      3 & sissippi\$\\
      4 & issippi\$\\
      5 & ssippi\$\\
      6 & sippi\$\\
      7 & ippi\$\\
      8 & ppi\$\\
      9 & pi\$\\
      10 & i\$\\
      11 & \$\\
    \end{tabular}
    \quad $\implies$\quad
    \begin{tabular}{c|l} 
      \textbf{Indice del suffisso} & \textbf{Suffisso}\\ 
      \hline
      11 & \$\\
      10 & i\$\\
      7 & ippi\$\\
      4 & issippi\$\\
      1 & ississippi\$\\
      0 & mississippi\$\\
      9 & pi\$\\
      8 & ppi\$\\
      6 & sippi\$\\
      3 & sissippi\$\\
      5 & ssippi\$\\
      2 & ssissippi\$\\
    \end{tabular}
  \end{table}
  Ottenendo quindi che:
  \[SA_T=[11,10,7,4,1,0,9,8,6,3,5,2]\]
\end{esempio}
\subsection{Longest common prefix}
L'uso del \textit{suffix array} è spesso accompagnato da un'altra struttura
dati, detta \textbf{Longest Common Prefix}.
\begin{definizione}
  Si definisce il \textbf{Longest Common Prefix (\emph{LCP})} di un testo $T$
  lungo $n$,
  denotato con $LCP_T$, come un array lungo $n+1$, contenente la
  lunghezza del prefisso comune tra ogni coppia di suffissi consecutivi
  nell'ordinamento lessicografico dei suffissi, ovvero l'ordinamento specificato
  da $SA_T$. Più formalmente
  $LCP_T$ è un array tale che, avendo $0\leq i\leq n$ e indicando con $lcp(x,y)$
  il più lungo prefisso comune tra le stringhe $x$ e $y$:
  \begin{equation}
    \label{eq:lcpdef}
    LCP_T[i]=
    \begin{cases}
      -1&\mbox{ se } i=0 \lor i=n\\
      \left|lcp(T[SA_T[i-1], n],T[SA_T [i], n])\right|&\mbox{ altrimenti}
    \end{cases}
  \end{equation}
\end{definizione}
\begin{esempio}
  Riprendendo l'esempio precedente si avrebbe quindi:
  \begin{table}[H]
    \centering
    \footnotesize
    \begin{tabular}{c|c|c|l} 
      \textbf{Indice} & $\mathbf{SA_T}$ & $\mathbf{LCP_T}$ & \textbf{Suffisso}\\ 
      \hline
      0 & 11 & -1 & \$\\
      1 & 10 & 0 & i\$\\
      2 & 7 & 1 & \underline{i}ppi\$\\
      3 & 4 & 1 & \underline{i}ssippi\$\\
      4 & 1 & 4 & \underline{issi}ssippi\$\\
      5 & 0 & 0 & mississippi\$\\
      6 & 9 & 0 & pi\$\\
      7 & 8 & 1 & \underline{p}pi\$\\
      8 & 6 & 0 & sippi\$\\
      9 & 3 & 2 & \underline{si}ssippi\$\\
      10 & 5 & 1 & \underline{s}sippi\$\\
      11 & 2 & 3 & \underline{ssi}ssippi\$\\
      12 & - & -1 & -
    \end{tabular}
  \end{table}
\end{esempio}
Senza entrare in ulteriori dettagli relativi all'algoritmo di pattern matching
tramite \textit{SA} e \textit{LCP}, in quanto non centrali per il resto della
trattazione, risulta comunque interessante riportare le complessità
temporali. Si ha quindi che, per l'algoritmo di query su \textit{SA} senza l'uso
dell'\textit{LCP}, si ha, per un testo lungo $n$ e un pattern lungo $m$:
\begin{equation}
  \label{eq:satime}
  \mathcal{O}(m\log n)
\end{equation}
Con l'uso dell'\textit{LCP} questo si riduce a:
\begin{equation}
  \label{eq:salcptime}
  \mathcal{O}(m+\log n)
\end{equation}
Per ulteriori approfondimenti in merito agli algoritmi di pattern matching
basati su \textit{suffix array} e ai relativi \textit{acceleratori}, si rimanda
al testo di Gusfield \cite{gusfield1997}.
\subsection{SA inverso}
Ai fini di poter comprendere future definizioni si presenta anche la
permutazione inversa dei valori del \textit{suffix array}, detta
\textbf{Inverse Suffix Array (\textit{ISA})}. Grazie a tale permutazione
inversa, dato un indice di suffisso, è possibile sapere in quale posizione si
trovi tale suffisso nel \textit{suffix array}.  
\begin{definizione}
  Dato il \textbf{suffix array} $SA_T$, costruito su un testo $T$ di lunghezza
  $n$, si definisce l'\textbf{inverse suffix array}, denotato con $ISA_T$, come:
  \begin{equation}
    \label{eq:isadef}
    ISA_T[i]=j\iff SA_T[j]=i,\,\,\forall\, i\in\{0,n-1\}
  \end{equation}
\end{definizione}

\begin{esempio}
  Riprendendo l'esempio precedente si avrebbe quindi:
  \begin{table}[H]
    \centering
    \footnotesize
    \begin{tabular}{c|c|c|l} 
      \textbf{Indice} & $\mathbf{SA_T}$ & $\mathbf{ISA_T}$ & \textbf{Suffisso}\\ 
      \hline
      0 & 11 & 5 & \$\\
      1 & 10 & 4 & i\$\\
      2 & 7 & 11 & ippi\$\\
      3 & 4 & 9 & issippi\$\\
      4 & 1 & 3 & ississippi\$\\
      5 & 0 & 10 & mississippi\$\\
      6 & 9 & 8 & pi\$\\
      7 & 8 & 2 & ppi\$\\
      8 & 6 & 7 & sippi\$\\
      9 & 3 & 6 & sissippi\$\\
      10 & 5 & 1 & ssippi\$\\
      11 & 2 & 0 & ssissippi\$\\
    \end{tabular}
  \end{table}
\end{esempio}
\subsection{LCP permutato}
Un'altra permutazione che bisogna introdurre è il \textbf{permuted
  longest-common-prefix array (\textit{PLCP})} \cite{plcp}.
Tale permutazione
permette una rappresentazione succinta in memoria dell'\textit{LCP}
\cite{plcp2},
permettendo di ottenere gli stessi risultati di quest'ultimo. Un'altro vantaggio
è che la sua ricostruzione richiede un minor costo computazionale.
\dc{L'intera sottosezione potrebbe essere quasi totalmente rimossa ma almeno al
  definizione serve per il calcolo di tutte le occorrenze di un MEM, come in
  PHONI} 
\begin{definizione}
  Si definisce il \textbf{permuted longest-common-prefix array}, denotato con
  $PLCP_T$, costruito a partire da un testo $T$ di lunghezza $n$, come un
  array tale per cui \cite{phoni}:
  \begin{equation}
    \label{eq:plcpdef1}
    PLCP_T[p]=
    \begin{cases}
      -1&\mbox{se }ISA_T[p]=0\\
      LCP_T[ISA_T[p]]&\mbox{altrimenti}
    \end{cases},\,\,\forall\, p\in\{0,n-1\}
  \end{equation}
  Quindi, i valori sono in ordine di posizione, ovvero l'ordine originale dato
  dagli indici dei suffissi, e non
  lessicografico. In altri termini, si ha una permutazione dei valori di $LCP_T$
  tale per cui \cite{plcp}:
  \begin{equation}
    \label{eq:plcpdef2}
    PLCP_T[SA_T[p]] = LCP_T[p],\,\,\,\forall\, p\in\{1,n-1\}
  \end{equation}
\end{definizione}
\begin{esempio}
  Riprendendo l'esempio precedente si avrebbe quindi:
  \begin{table}[H]
    \centering
    \footnotesize
    \begin{tabular}{c|c|c|c|c|l} 
      \textbf{Indice} & $\mathbf{SA_T}$ & $\mathbf{ISA_T}$ & $\mathbf{LCP_T}$
      & $\mathbf{PLCP_T}$ & \textbf{Suffisso}\\  
      \hline
      0 & 11 & 5 & -1 & 0 & \$\\
      1 & 10 & 4 & 0 & 4 & i\$\\
      2 & 7 & 11 & 1 & 3 & \underline{i}ppi\$\\
      3 & 4 & 9 & 1 & 2 & \underline{i}ssippi\$\\
      4 & 1 & 3 & 4 & 1 & \underline{issi}ssippi\$\\
      5 & 0 & 10 & 0 & 1 & mississippi\$\\
      6 & 9 & 8 & 0 & 0 & pi\$\\
      7 & 8 & 2 & 1 & 1 & \underline{p}pi\$\\
      8 & 6 & 7 & 0 & 1 & sippi\$\\
      9 & 3 & 6 & 2 & 0 & \underline{si}ssippi\$\\
      10 & 5 & 1 & 1 & 0 & \underline{s}sippi\$\\
      11 & 2 & 0 & 3 & -1 & \underline{ssi}ssippi\$\\
      12 & - & - & -1 & - & - 
    \end{tabular}
  \end{table}
\end{esempio}
Ciò che permette una rappresentazione compatta del \textit{PLCP} è descritto nel
seguente lemma \cite{plcp3}.
\begin{lemma}
  Dato un testo $T$, tale che $|T|=n$, si ha che:
  \begin{equation}
    \label{eq:plcpdef3}
    PCLP_T[i]\geq PLCP_T[i-1]-1,\,\,\forall\, i\in\{1,n-1\}
  \end{equation}
\end{lemma}
Grazie a tale lemma si può memorizzare l'\textit{PLCP sparso}.
\begin{definizione}
  Dato un intero
  $q$, per il quale calcolo (basato sul lemma precedente) si rimanda al paper di
  Kasai \cite{plcp3}, si definisce \textbf{array PCLP sparso}, lungo
  $\left\lfloor\frac{n}{q}\right\rfloor$ e denotato $PLCP_q$, l'array che
  memorizza ogni $q$-esimo valore del \textit{PLCP}, avendo che:
  \begin{equation}
    \label{eq:plcpdef4}
    PLCP_q[i]=PLCP_T[iq]
  \end{equation}
\end{definizione}
\dc{Capire se serve variante sparsa e se serve esempio}
\subsection{Funzione phi}
L'ultimo concetto che si introduce sono le \textbf{funzioni}
$\boldsymbol\varphi$ e $\mathbf{\boldsymbol\varphi^{-1}}$, usate per poter
identificare i valori precedenti e successivi di 
un dato valore in $SA_T$. Essere sono utili al fine di poter sia ricostruire
efficientemente il \textit{PLCP} di un testo $T$ (per i dettagli si rimanda
all'articolo di K\"{a}rkk\"{a}inen \cite{plcp}) che di permettere, come si vedrà
più  avanti nella sezione \ref{secbwt}, il riconoscimento di tutte le occorrenze
di un \textbf{match massimale esatto (\textit{MEM})} in $T$ \cite{phoni}.
\begin{definizione}
  Dato un testo $T$ di lunghezza $n$ si definiscono le funzioni, che di fatto
  sono permutazioni dei valori di $SA_T$, $\boldsymbol\varphi$ e
  $\mathbf{\boldsymbol\varphi^{-1}}$ come \cite{phoni}: 
  \begin{equation}
    \label{eq:phidef1}
    \varphi(p)=
    \begin{cases}
      null&\mbox{se } ISA_T[p]=0\\
      SA_T[ISA_T[p]-1]&\mbox{altrimenti}
    \end{cases},\,\,\forall\, p\in\{0,n-1\}
  \end{equation}
  \begin{equation}
    \label{eq:phiinvdef1}
    \varphi(p)^{-1}=
    \begin{cases}
      null&\mbox{se } ISA_T[p]=n-1\\
      SA_T[ISA_T[p]+1]&\mbox{altrimenti}
    \end{cases},\,\,\forall\, p\in\{0,n-1\}
  \end{equation}
  Si noti che si ha il valore $null$ quando, rispettivamente, si studia il
  primo e l'ultimo valore del \textit{suffix array} in quanto non hanno, sempre
  rispettivamente, l'antecedente e il successore.   Infatti, semplicemente, tali
  funzioni restituiscono i due valori, se 
  esistenti, di $SA_T$ adiacenti ad un valore del suffix array dato.\\
  Analogamente, sempre coi medesimi vincoli, possono essere definite come
  \cite{plcp}: 
  \begin{equation}
    \label{eq:phidef2}
    \varphi[SA[p]]=SA[p-1]
  \end{equation}
  \begin{equation}
    \label{eq:phiinvdef2}
    \varphi^{-1}[SA[p]]=SA[p+1]
  \end{equation}

\end{definizione}
\begin{esempio}
   Riprendendo l'esempio precedente si avrebbe quindi:
   % \begin{table}[H]
   %   \centering
   %   \footnotesize
   %   \begin{tabular}{c|c|c|c|c|c|c|l} 
   %     \textbf{Indice} & $\mathbf{SA_T}$ & $\mathbf{ISA_T}$ & $\mathbf{LCP_T}$
   %     & $\mathbf{PLCP_T}$ & $\mathbf{\boldsymbol\varphi}$
   %     & $\mathbf{\boldsymbol\varphi^{-1}}$ & \textbf{Suffisso}\\  
   %     \hline
   %     0 & 11 & 5 & -1 & 0 & 1 & 9 & \$\\
   %     1 & 10 & 4 & 0 & 4 & 4 & 0 & i\$\\
   %     2 & 7 & 11 & 1 & 3 & 5 & $null$ & ippi\$\\
   %     3 & 4 & 9 & 1 & 2 & 6 & 5 & issippi\$\\
   %     4 & 1 & 3 & 4 & 1 & 7 & 1 & ississippi\$\\
   %     5 & 0 & 10 & 0 & 1 & 3 & 2 & mississippi\$\\
   %     6 & 9 & 8 & 0 & 0 & 8 & 3 & pi\$\\
   %     7 & 8 & 2 & 1 & 1 & 10 & 4 & ppi\$\\
   %     8 & 6 & 7 & 0 & 1 & 9 & 6 & sippi\$\\
   %     9 & 3 & 6 & 2 & 0 & 0 & 8 & sissippi\$\\
   %     10 & 5 & 1 & 1 & 0 & 11 & 7 & ssippi\$\\
   %     11 & 2 & 0 & 3 & 0/-1 & $null$ & 10 & ssissippi\$\\
   %   \end{tabular}
   % \end{table}
   \begin{table}[H]
     \centering
     \footnotesize
     \begin{tabular}{c|c|c|c|c|l} 
       \textbf{Indice} & $\mathbf{SA_T}$ & $\mathbf{ISA_T}$
       & $\mathbf{\boldsymbol\varphi}$
       & $\mathbf{\boldsymbol\varphi^{-1}}$ & \textbf{Suffisso}\\  
       \hline
       0 & 11 & 5 & 1 & 9 & \$\\
       1 & 10 & 4 & 4 & 0 & i\$\\
       2 & 7 & 11 & 5 & $null$ & ippi\$\\
       3 & 4 & 9 & 6 & 5 & issippi\$\\
       4 & 1 & 3 & 7 & 1 & ississippi\$\\
       5 & 0 & 10 & 3 & 2 & mississippi\$\\
       6 & 9 & 8 & 8 & 3 & pi\$\\
       7 & 8 & 2 & 10 & 4 & ppi\$\\
       8 & 6 & 7 & 9 & 6 & sippi\$\\
       9 & 3 & 6 & 0 & 8 & sissippi\$\\
       10 & 5 & 1 & 11 & 7 & ssippi\$\\
       11 & 2 & 0 & $null$ & 10 & ssissippi\$\\
     \end{tabular}
   \end{table}
  Infatti, ad esempio, il valore $9$ in $SA_T$ è preceduto dal valore
  $\varphi(9)=0$ ed è seguito dal valore $\varphi^{-1}(9)=8$.
\end{esempio}


% sezione BWT
\section{Trasformata di Burrows-Wheeler}
Introdotta nel 1994 da Burrows e Wheeler con lo scopo di comprimere testi, la
\textbf{Burrows-Wheeler Transform} \cite{bwt} è divenuta ormai uno standard nel
campo dell'\textit{algoritmica su stringhe} e della \textit{bioinformatica},
grazie ai suoi molteplici vantaggi sia dal punto di vista della complessità
temporale che da quello della complessità spaziale.\\
Nel dettaglio la \textit{BWT} è una \textit{trasformata reversibile} che
permette una \textit{compressione lossless}, quindi senza perdita
d'informazione. Tale trasformazione vien costruita a partire dal riordinamento
dei caratteri del testo in input, fattore che ha 
portato all'evidenza per cui caratteri uguali tendono ad essere posti
consecutivamente all'interno della stringa prodotta dalla trasformata.
\begin{definizione}
  Dato un testo $T$ \$-terminato, tale che $|T|=n$, si definisce la
  \textbf{Burrows-Wheeler Transform (\textit{BWT})} di $T$, denotata con
  $BWT_T$, come un array di caratteri lungo $n$ dove l'elemento $i$-esimo è il
  carattere che precede l'$i$-esimo suffisso $T$ nel riordinamento
  lessicografico. Più formalmente si ha che, con $0\leq i<n$:
  \[BWT_T[i]=
    \begin{cases}
      T[SA_T[i]-1]&\mbox{ se } SA_T[i]\neq 1\\
      \$&\mbox{ altrimenti}
    \end{cases}
  \]
\end{definizione}
In termini più pratici, la \textit{BWT} di un testo è calcolabile riordinando
lessicograficamente tutte le possibili \textbf{rotazioni} del testo $T$.
\begin{definizione}
  Si definisce \textbf{rotazione $\mathbf{i}$-esima}, denotata con $rot_T(i)$ di
  un testo $T$, tale che $|T|=n$, come la stringa ottenuta dalla concatenazione
  del suffisso $i$-esimo con la restante porzione del testo. Più formalmente si
  ha che, avendo $0\leq i<n$:
  \[rot_T(i)=T[i:n-1]\cdot T[0:i-1]\]
\end{definizione}
Data questa definizione quindi la \textit{BWT} del testo $T$ risulta essere
l'ultima colonna della matrice che si ottiene riordinando tutte le
\textit{rotazioni} di $T$, che altro non sono che i suffissi già riordinati per
il calcolo del \textit{SA} a cui viene concatenata la parte restante del
testo.\\
Un altro array spesso utilizzato insieme alla \textit{BWT} è il cosiddetto
\textbf{array $\mathbf{F}$}, lungo $|T|$, che altro non è che l'array formato
dalla prima colonna della matrice delle rotazioni. In termini ancora più
semplicistici l'array $F$ è banalmente l'array formato dal riordinamento
lessicografico dei caratteri del testo $T$.\\
Per chiarezza si vede un esempio. 
\begin{esempio}
   Si prenda la stringa:
  \[s=\mbox{mississippi\$},\,\,|s|=12\]
  Si produce la seguente matrice delle rotazioni riordinate:
  \begin{table}[H]
    \centering
    \footnotesize
    \begin{tabular}{c|c|c|c|c} 
      \textbf{Indice} & $\mathbf{SA_T}$ & $\mathbf{F_T}$ & \textbf{Rotazione}
      & $\mathbf{BWT_T}$\\ 
      \hline
      0 & 11 & \$ & \$mississippi & i\\
      1 & 10 & i & i\$mississipp & p\\
      2 & 7 & i & ippi\$mississ & s\\
      3 & 4 & i & issippi\$miss & s\\
      4 & 1 & i & ississippi\$m & m\\
      5 & 0 & m & mississippi\$ & \$\\
      6 & 9 & p & pi\$mississip & p\\
      7 & 8 & p & ppi\$mississi & i\\
      8 & 6 & s & sippi\$missis & s\\
      9 & 3 & s & sissippi\$mis & s\\
      10 & 5 & s & ssippi\$missi & i\\
      11 & 2 & s & ssissippi\$mi & i\\
    \end{tabular}
  \end{table}
  Avendo quindi:
  \[F_T=\mbox{\$iiiimppssss}\mbox{ e }BWT_T=\mbox{ipssm\$pissii}\]
\end{esempio}
L'importanza di questa trasformata è dovuta soprattutto al fatto che sia
\textit{reversibile}, implicando quindi che a partire da $BWT_T$ è possibile
ricostruire $T$. Questo è possibile grazie ad una proprietà intrinseca della
trasformata che viene riassunta nel cosiddetto \textbf{LF-mapping}.
\begin{definizione}
  Dato un testo $T$, tale che $|T|=n$, data la sua $BWT_T$ e il suo array $F_T$
  si definisce \textbf{LF-mapping} come la proprietà per la quale l'$i$-esima
  occorrenza di un carattere $\sigma$ in $BWT_T$ corrisponde all'$i$-esima
  occorrenza dello stesso carattere in $F_T$.
\end{definizione}
Grazie a questa definizione è possibile partire dall'ultimo carattere del testo,
\$, e ricostruire l'intero testo a ritroso. Si vede quindi un breve esempio.
\begin{esempio}
  Si riprende l'esempio precedente, avendo:
  \[BWT_T=\mbox{ipssm\$pissii}\mbox{ e }F_T=\mbox{\$iiiimppssss}\]
  % ricostruisci testo
  
\end{esempio}

\subsection{Trasformata di Burrows-Wheeler run-length}
\subsection{Matching Statistics}
\subsection{R-index}
\subsection{PHONI}

% sezione PBWT
\section{Trasformata di Burrows-Wheeler posizionale}
\label{secpbwt}
Presentata nel 2014 da Richard Durbin \cite{pbwt}, la \textbf{Positional
  Burrows-Wheeler Transform (\textit{PBWT})}, traducibile con
\textit{trasformata di Burrows-Wheeler posizionale}, è una struttura efficiente
per la memorizzazione e l'interrogazione di \textbf{pannelli di aplotipi}.\\
La costruzione di tali pannelli avviene tramite il riconoscimento delle
variazioni di un singolo nucleotide tra le sequenze genomiche di diversi
individui, ovvero dei cosiddetti \textbf{Single-Nucleotide Polymorphism
  (\textit{SNP})}. Ogni variazione, identificata per un certo nucleotide in una
posizione specifica, viene 
detto \textbf{allele}. La combinazione di tutte le \textit{varianti alleliche},
ereditate, a meno di mutazioni, da ogni genitore, forma l'\textbf{aplotipo} di
un certo individuo. Come visibile in figura \ref{fig:haplo} \cite{haplo}, la 
costruzione parte dai
vari sequenziamenti (nell'immagine relativi a diversi cromosomi ma il
procedimento è uguale partendo da diversi individui) da cui si identificano le
varianti alleliche. Da queste ultime si costruiscono gli aplotipi da cui si
estraggono i cosiddetti \textit{tag SNPs}, ovvero le possibili alternative per
una certa variante allelica. Questi ultimi, tendenzialmente rappresentati per
l'uomo da due caratteri vista la sua natura \textit{diploide}, formano,
l'alfabeto del pannello. 
L’informazione combinata di tutti gli aplotipi in un individuo è detta,
invece, \textbf{genotipo}.
\begin{figure}
  \centering
  \includegraphics[scale = 0.3]{img/haplo.jpg}
  \caption{Schema di ottenimento del pannello di aplotipi.}
  \label{fig:haplo}
\end{figure}
Formalmente si considera un pannello $X$ di $M$ aplotipi $x_i$, con $i=0,\ldots,
M-1$, su $N$ siti, indicizzati tramite $k=0,\ldots, N-1$, tale per cui tutti i
siti sono considerati biallelici. 
Da un punto di vista computazionale, quest'ultima
assunzione comporta che il pannello $X$ è costruito sull'alfabeto ordinato
$\Sigma =\{0,1\}$, con $0\prec 1$. Si ha la sostituzione dei \textit{tag SNPs},
per un certo sito, con tale alfabeto. Ne segue che:
\begin{equation}
  \label{eq:pbwtdip}
  x_i[k]=\{0,1\}
\end{equation}
Prima di proseguire con la trattazione è bene fornire la descrizione di alcuni
formalismi utilizzati:
\begin{itemize}
  \item si denota, per una qualsiasi riga $x_i$, con $x_i[k_1,k_2)$ la
  \textbf{sottostringa} di $x_i$ che inizia alla colonna $k_1$ e termina alla
  colonna $k_2-1$
  \item date due righe $x_i$ e $x_j$, si definisce un \textbf{match} tra le due
  righe, iniziante
  alla colonna $k_1$ e terminante alla colonna $k_2-1$, sse:
  \begin{equation}
    \label{eq:pbwtmatch}
    x_i[k_1,k_2)=x_j[k_1,k_2)
  \end{equation}
  \item un match tra due righe $x_i$ e $x_j$, come definito al punto precedente,
  è definito \textbf{localmente massimale} sse non si ha alcuna estensione a
  destra o sinistra che comporti un ulteriore match, avendo quindi che:
  \begin{equation}
    \label{eq:pbwtmem}
    (k_1=0\lor x_i[k_1-1]\neq x_j[k_1-1])\land (k_2=N\lor x_i[k_2]\neq x_j[k_2])
  \end{equation}
  \item comparando una sequenza $z$ ad un pannello di aplotipi $X$ si definisce
  che $s$ ha un \textbf{set-maximal exact match (SMEM)} con $x_i$, che inizia
  alla colonna $k_1$ e termina alla colonna $k_2-1$, sse tale match è
  \textit{localmente massimale} e non si ha alcun altro match di $z$ con un
  altro $x_j$ che include ed estende l'intervallo $[k_1,k_2)$. Si ha che $z$
  può avere uno \textit{SMEM}, tra $k_1$ e $k_2-1$, con più di un aplotipo del
  pannello 
\end{itemize}
Si noti che il match tra le due sequenze nella \textit{PBWT} è tale sse iniziano
entrambi nella stessa colonna e terminano nella stessa colonna. Questo vincolo,
da cui deriva il termine ``posizionale'' e che, di fatto, impedisce l'uso degli
algoritmi tradizionali visti con la \textit{BWT}, è dato dal fatto che una
colonna rappresenta un preciso sito di una variante genica. \\
La costruzione di questa struttura dati si basa, ad ogni colonna $k$, sul
riordinamento lessicografico delle sequenze di aplotipi basato sull'ordinamento
inverso dei prefissi terminanti in colonna $k-1$. I valori presenti in colonna
$k$, dopo il riordinamento, altro non sono che i valori che andranno a popolare
la cosiddetta \textbf{matrice PBWT}, che rappresenta la vera e propria
trasformata. Si noti che avere le sequenze 
ordinate in base ai prefissi invertiti alla $k$-esima colonna permette di
identificare i match con maggior facilità in quanto, ad ogni colonna, aplotipi
con suffisso comune (o prefisso comune in ordine inverso) saranno in posizioni
consecutive all'interno della trasformata.\\
La computazione di tutti i riordinamenti non presenta difficoltà dal punto di
vista computazionale in quanto, conoscendo l'ordinamento in colonna $k$, si può
derivare facilmente l'ordinamento in colonna $k+1$, studiando solo i valori
riordinati alla colonna precedente.
\begin{definizione}
  Dato un aplotipo $i$, appartenente al pannello $X$, e un indice di colonna
  $k$, si definisce il \textbf{prefix array} $a_k$ come una permutazione degli
  indici $0,\ldots, M-1$ tale che $a_k[i]=j$ sse $x_j$ è l'$i$-esimo aplotipo di
  $X$ nell'ordinamento inverso dei prefissi ottenuto alla colonna $k$. Quindi
  $a_k[i]=m$, con $m<M$, altro non è che l'indice della sequenza $x_m$ del
  pannello $X$ da cui deriva il prefisso $i$-esimo nell'ordine inverso
  in colonna $k$.
\end{definizione}
Data questa definizione ne segue che la \textit{matrice PBWT} si ottiene
direttamente andando a vedere, per ogni colonna, gli indici del \textit{prefix
  array} e prendendo i valori del pannello $X$ secondo l'ordine espresso da
esso.\\ 
Per comodità di rappresentazione definiamo formalmente i valori della
\textit{matrice PBWT} con il seguente formalismo:
\begin{equation}
  \label{eq:pbwty}
  y_i^k[j]=x_{a_k[i]}[j]
\end{equation}
avendo quindi che $y_i^k$ denota la sequenza $i$-esima secondo l'ordinamento
ottenuto per la colonna $k$. Si può quindi accedere al valore $j$-esimo, ovvero
il valore in colonna $j$, di tale
sequenza. Possiamo quindi meglio spiegare perché risulti 
semplice computare i vari \textit{prefix array}. Infatti, si ha
che l'ordinamento degli elementi per $a_{k+1}$ si ottiene a partire
dall'ordinamento in $a_k$. Si considerano, infatti, i valori $y_i^k[k]$ e la
precedenza del valore 0 sul valore 1 per riordinare in modo stabile tali
valori.\\ 
\dc{Un po' confusionario}
Come anticipato, prefissi simili saranno consecutivi nei riordinamenti fino alla
colonna $k$-esima. Quindi, risulta utile tenere traccia della posizione iniziale
dei match tra prefissi vicini. 
\begin{definizione}
  Si definisce \textbf{divergence array} l'array $d_k$ tale che $d_k[i]$ è
  l'indice colonna iniziale del match massimale a sinistra terminante in $k$ tra
  l'$i$-esimo aplotipo e il suo precedente nell'ordinamento ottenuto alla
  colonna $k$-esima. Formalmente, dato $i>0$, si definisce 
  $d_k[i]$ come il più piccolo $j$ tale che:
  \begin{equation}
    \label{eq:pbwtdiv}
    y_i^k[j,k)=y_{i-1}^k[j,k)
  \end{equation}
  Ne segue che
  \begin{equation}
    \label{eq:pbwtdiv2}
    y_i^k[k-1]\neq y_{i-1}^k[k-1] \implies d_k[i]=k
  \end{equation}
  Per definizione, non avendo una riga precedente con cui effettuare il
  confronto: 
  \begin{equation}
    \label{eq:pbwtdiv3}
   d_k[0]=k
  \end{equation}
\end{definizione}
Si può quindi dimostrare che l'inizio di qualsiasi match massimale, terminante in
colonna $k$, tra qualsiasi $y_i^k$ e $y_j^k$, con $i<j$, è calcolabile facilmente
avendo che è dato da:
\begin{equation}
  \label{eq:pbwtint}
  \max_{i<m\leq j}d_k[m]
\end{equation}
Si noti che al posto del \textbf{divergence array} si può usare anche una
variante del \textbf{Longest Common Prefix (\textit{LCP}) array}.
\begin{definizione}
  Si definisce \textbf{Longest Common Prefix (\textit{LCP}) array} l'array $l_k$
  che, anziché 
  memorizzare l'indice d'inizio del match massimale a sinistra da due aplotipi
  consecutivi nell'ordinamento ottenuto alla colonna $k$-esima, tiene traccia
  della lunghezza di tale match. Formalmente si ha, quindi, che:
  \begin{equation}
    \label{eq:pbwtlcp}
    l_k[i]=k-d_k[i]
  \end{equation}
\end{definizione}
Fatte queste premesse possiamo quindi fornire una definizione formale di
\textbf{PBWT}.
\begin{definizione}
  Dato un pannello di $M$ aplotipi con $N$
  siti $X=\{x_1,x_2,\ldots,x_M\}$, si definisce la \textbf{PBWT} di $X$ come una
  collezione di $N+1$ coppie di array $(a_k,d_k)$, con $0\leq k\leq N$, dove
  ogni $a_k$ è detto \textbf{prefix array} e ogni $d_k$ è detto
  \textbf{divergence array}.  
\end{definizione}
L'algoritmo per la costruzione di $a_{k+1}$ e $d_{k+1}$ a partire da $a_k$ e
$d_k$ è disponibile all'algoritmo \ref{algo:durbin1}. Si può quindi notare come
il costo della costruzione dei due insiemi di array sia:
\begin{equation}
  \label{eq:pbwtadtime}
  \mathcal{O}(NM)
\end{equation}
\dc{Serve altro? Serve spiegare i dettagli dell'algoritmo?}
\begin{algorithm}
  \small
  \begin{algorithmic}[1]
    \Function{BuildPrefixAndDivergenceArrays}{$k$, $M$, $a_k$, $d_k$}
    \State $u\gets 0$, $v\gets 0$
    \State $p\gets k+1$, $q\gets k+1$
    \State $a\gets []$, $b\gets []$, $d\gets []$, $e\gets []$
    \For {\textit{every} $i\in[0,M-1]$}
    \If {$d_k[i]>p$}
    \State $p\gets d_k[i]$
    \EndIf
    \If {$d_k[i]>q$}
    \State $q\gets d_k[i]$
    \EndIf
    \If {$y_i^k[k]=0$}
    \State $a[u]\gets a_k[i]$, $d[u]\gets p$
    \State $u\gets u+1$, $p\gets 0$
    \Else
    \State $b[v]\gets a_k[i]$, $e[v]\gets q$
    \State $v\gets v+1$, $q\gets 0$
    \EndIf
    \EndFor
    \State $a_{k+1}\gets concatenate(a,b)$
    \State $d_{k+1}\gets concatenate(d,e)$ 
    \EndFunction
  \end{algorithmic}
  \caption{Algoritmo di Durbin per la costruzione di $a_{k+1}$ e $d_{k+1}$ a
  partire da $a_{k}$ e $d_{k}$.}
  \label{algo:durbin1}
\end{algorithm}
Ai fini della trattazione dell'algoritmo di match con un'aplotipo esterno
ricordiamo un'ulteriore definizione.
\begin{definizione}
  Si definisce $\alpha_k$ come l'\textbf{inverso della permutazione data dal
    prefix array} $a_k$, avendo che:
  \[\alpha_k[i]=j \iff a_k[j]=i\]
\end{definizione}
Grazie a queste prime definizioni è possibile notare alcune prime forti
correlazioni, fattore chiave nello sviluppo di questa tesi, che sussistono tra
\textit{BWT} e \textit{PBWT} (e le rispettive varianti \textit{run-length
  encoded}. Nella seguente tabella si ricordano queste prime correlazioni:
\begin{table}[H]
  \centering
  \begin{tabular}{c|c}
    \textbf{BWT} & \textbf{PBWT}\\
    \hline
    $SA_T$ & $a_k$\\
    $ISA_T$ & $\alpha_k$\\
    $LCP_T$ & $d_k$ o $l_k$\\            
  \end{tabular}
\end{table}

\begin{esempio}
  \label{es:pbwt1}
  Si assuma il seguente pannello $X$ e di voler calcolare $y^6$:
  \begin{table}[H]
    \centering
    \scriptsize
    \begin{tabular}{c|ccccccccccccccc}
      X & 00 & 01 & 02 & 03 & 04 & 05 & 06 & 07 & 08 & 09 & 10 & 11 & 12 & 13
      & 14 \\
      \hline
      00 & 1 & 0 & 0 & 1 & 0 & 0 & 0 & 0 & 0 & 0 & 0 & 1 & 1 & 0 & 1 \\
      01 & 1 & 0 & 0 & 1 & 1 & 0 & 0 & 1 & 0 & 0 & 0 & 0 & 0 & 1 & 1 \\
      02 & 1 & 0 & 0 & 1 & 1 & 0 & 0 & 1 & 0 & 0 & 0 & 1 & 0 & 0 & 1 \\
      03 & 1 & 0 & 0 & 1 & 1 & 0 & 0 & 1 & 0 & 0 & 0 & 1 & 0 & 0 & 1 \\
      04 & 0 & 1 & 0 & 1 & 0 & 1 & 0 & 0 & 0 & 0 & 0 & 1 & 0 & 0 & 1 \\
      05 & 0 & 1 & 0 & 1 & 0 & 1 & 0 & 0 & 0 & 0 & 0 & 1 & 0 & 0 & 1 \\
      06 & 0 & 1 & 0 & 1 & 0 & 1 & 0 & 0 & 0 & 0 & 0 & 1 & 0 & 0 & 1 \\
      07 & 0 & 1 & 0 & 1 & 0 & 1 & 0 & 0 & 0 & 0 & 0 & 0 & 1 & 0 & 1 \\
      08 & 0 & 1 & 0 & 0 & 1 & 0 & 0 & 0 & 0 & 1 & 1 & 1 & 0 & 0 & 1 \\
      09 & 0 & 1 & 0 & 1 & 0 & 0 & 0 & 0 & 1 & 0 & 0 & 0 & 0 & 1 & 1 \\
      10 & 0 & 1 & 0 & 1 & 0 & 0 & 0 & 0 & 1 & 0 & 0 & 0 & 0 & 1 & 1 \\
      11 & 0 & 1 & 0 & 0 & 1 & 0 & 0 & 0 & 0 & 0 & 1 & 1 & 0 & 0 & 0 \\
      12 & 0 & 1 & 0 & 0 & 1 & 0 & 0 & 0 & 1 & 0 & 1 & 1 & 0 & 0 & 1 \\
      13 & 0 & 1 & 0 & 0 & 1 & 0 & 0 & 0 & 1 & 0 & 1 & 1 & 0 & 0 & 1 \\
      14 & 0 & 1 & 0 & 0 & 0 & 0 & 0 & 0 & 1 & 0 & 0 & 0 & 1 & 0 & 1 \\
      15 & 0 & 1 & 0 & 0 & 0 & 0 & 0 & 0 & 1 & 0 & 0 & 0 & 1 & 0 & 1 \\
      16 & 0 & 1 & 0 & 1 & 0 & 0 & 0 & 0 & 0 & 0 & 0 & 1 & 1 & 0 & 1 \\
      17 & 1 & 1 & 0 & 0 & 0 & 1 & 0 & 0 & 0 & 0 & 0 & 1 & 1 & 0 & 1 \\
      18 & 0 & 1 & 1 & 0 & 1 & 0 & 0 & 0 & 0 & 0 & 0 & 1 & 0 & 0 & 1 \\
      19 & 0 & 1 & 1 & 0 & 1 & 0 & 1 & 0 & 0 & 0 & 0 & 0 & 1 & 0 & 1 
    \end{tabular}
  \end{table}
  Si inizia riordinando il pannello con l'ordine inverso alla
  quinta colonna, avendo che $y^6$ altro non è che la sesta colonna del pannello
  così riordinato. Ne segue che $a_6$ è la colonna degli indici, che è stata
  ottenuta con la permutazione data dall'ordinamento, e $d_6$ la
  colonna iniziale in cui terminano i match tra righe consecutive nel
  rioridnamento le sottolineature (evidenziati nell'immagine seguente dalle
  sottolineature):   
  \begin{figure}[H]
    \centering
    \includegraphics[scale = 0.315]{img/matrix1.pdf}
  \end{figure}
  \noindent
  Avendo, nel complesso:
  \[a_6=[14,15,0,9,10,16,8,11,12,13,18,19,1,2,3,17,4,5,6,7]\]
  \[\alpha_6=[2,12,13,14,16,17,18,19,6,3,4,7,8,9,0,1,5,15,10,11]\]
  \[d_6=[6,0,4,2,0,0,5,0,0,0,3,0,4,0,0,6,4,0,0,0]\]
  \[l_6=[0,6,2,4,6,6,1,6,6,6,3,6,2,6,6,0,2,6,6,6]\]
  Con il calcolo di tutti gli $a_k$ si otterrebbe la seguente \textbf{matrice
    PBWT}: 
  \begin{table}[H]
  \centering
  \scriptsize
  \begin{tabular}{c|ccccccccccccccc}
    X & 00 & 01 & 02 & 03 & 04 & 05 & 06 & 07 & 08 & 09 & 10 & 11 & 12 & 13
    & 14 \\
    \hline
    00 & 1 & 1 & 0 & 1 & 1 & 0 & 0 & 0 & 1 & 0 & 0 & 1 & 1 & 1 & 1 \\
    01 & 1 & 1 & 0 & 1 & 1 & 0 & 0 & 0 & 1 & 0 & 0 & 1 & 1 & 1 & 1 \\
    02 & 1 & 1 & 0 & 1 & 1 & 1 & 0 & 0 & 0 & 1 & 1 & 1 & 0 & 1 & 1 \\
    03 & 1 & 1 & 0 & 1 & 1 & 0 & 0 & 0 & 1 & 0 & 0 & 1 & 1 & 0 & 1 \\
    04 & 0 & 1 & 0 & 1 & 0 & 1 & 0 & 0 & 1 & 0 & 0 & 1 & 1 & 0 & 1 \\
    05 & 0 & 1 & 0 & 1 & 0 & 1 & 0 & 0 & 0 & 0 & 0 & 1 & 0 & 0 & 1 \\
    06 & 0 & 1 & 0 & 1 & 0 & 1 & 0 & 0 & 0 & 0 & 0 & 1 & 0 & 0 & 0 \\
    07 & 0 & 1 & 0 & 1 & 1 & 1 & 0 & 0 & 0 & 0 & 0 & 0 & 1 & 0 & 1 \\
    08 & 0 & 1 & 0 & 0 & 1 & 0 & 0 & 0 & 1 & 0 & 0 & 0 & 1 & 0 & 1 \\
    09 & 0 & 1 & 0 & 1 & 0 & 0 & 0 & 0 & 1 & 0 & 0 & 0 & 0 & 0 & 1 \\
    10 & 0 & 1 & 0 & 1 & 1 & 0 & 0 & 0 & 0 & 0 & 0 & 1 & 1 & 0 & 1 \\
    11 & 0 & 1 & 0 & 0 & 1 & 0 & 1 & 1 & 0 & 0 & 0 & 1 & 0 & 0 & 1 \\
    12 & 0 & 1 & 0 & 0 & 1 & 0 & 0 & 1 & 0 & 0 & 0 & 0 & 0 & 0 & 1 \\
    13 & 0 & 1 & 0 & 0 & 0 & 0 & 0 & 1 & 0 & 0 & 0 & 0 & 0 & 0 & 1 \\
    14 & 0 & 1 & 0 & 0 & 0 & 0 & 0 & 0 & 0 & 0 & 0 & 0 & 0 & 0 & 1 \\
    15 & 0 & 0 & 0 & 0 & 0 & 0 & 0 & 0 & 0 & 0 & 0 & 0 & 0 & 0 & 1 \\
    16 & 0 & 0 & 0 & 1 & 0 & 0 & 0 & 0 & 0 & 0 & 0 & 1 & 0 & 0 & 1 \\
    17 & 1 & 0 & 1 & 0 & 0 & 0 & 0 & 0 & 0 & 0 & 1 & 1 & 0 & 0 & 1 \\
    18 & 0 & 0 & 1 & 0 & 0 & 0 & 0 & 0 & 0 & 0 & 1 & 1 & 0 & 0 & 1 \\ 
    19 & 0 & 1 & 0 & 0 & 0 & 0 & 0 & 0 & 0 & 0 & 1 & 1 & 0 & 0 & 1
  \end{tabular}
\end{table}
\end{esempio}
\subsection{Match massimali con aplotipo esterno}
Durbin, nel suo articolo, propone diversi algoritmi per l'uso effettivo della
sua trasformata. Ad esempio, viene proposto un algoritmo per il calcolo
di match interni ad $X$ più lunghi di una lunghezza minima $L$ e uno per la
ricerca di tutti i \textit{set-maximal match} interni ad $X$ in tempo lineare.\\
Di interesse per questa tesi è però il cosiddetto \textbf{algoritmo 5}. Tale
algoritmo 
si propone di trovare tutti i \textit{set-maximal match} tra il panello $X$ e un
aplotipo esterno $z$, assumendo $|z|=N$, sempre avendo che una colonna $k$ del
pannello e una posizione $k$ della query rappresentano il medesimo sito.\\ 
L'idea dietro l'algoritmo è quella di usare tre indici: $e_k$, $f_k$ e
$g_k$. Nel dettaglio $e_k$ tiene traccia dell'inizio del più lungo match,
terminante in colonna $k$, tra $z$ e un qualche $y_i^k$. L'intervallo
$[f_k,g_k)\subseteq[0,\ldots,M)$ invece identifica il sotto-intervallo di
$a_k$ contenente gli indici degli aplotipi appartenetenti a tale match. Si noti
come si riprenda quindi l'idea, vista con la \textbf{backward search} per la
\textit{BWT}, di studiare un intervallo $[f_k,g_k)$ su $SA_T$ per identificare i
match tra un pattern e un testo. 
\begin{definizione}
  Dato un pannello $X$, con $M$ aplotipi/righe e $N$ siti/colonne, e un aplotipo
  query $z$, tale che $|z|=N$, si definisce un \textbf{Set-Maximal Exact Match
    (\textit{SMEM})}, iniziante in colonna $e_k$ e terminante il colonna
  $k$, tra 
  la query $z$ e le righe del pannello indicizzate dai valori compresi
  nell'intervallo $[f_k,g_k)$ in $a_k$ sse:
  \begin{equation}
    \label{eq:pbwtsmem}
    z[e_k,k)=y_i^k[e_k,k)\land z[e_k-1]\neq y_i^k[e_k-1], \forall\, i\mbox{
      t.c. }f_k\leq i < g_k
  \end{equation}
  Si noti che $g_k=M$ sse $y_{M-1}^k$ appartiene alle righe per le quali si ha
  tale \textit{SMEM}.
\end{definizione}
Bisogna quindi come aggiornare $e_k$, $f_k$ e $g_k$ passando dalla
colonna $k$ alla colonna $k+1$. Si procede esattamente come visto per la
\textbf{backward search}, selezionando, per calcolare  $[f_{k+1},g_{k+1})$, il
sottointervallo di $[f_k,g_k)$ in cui si hanno aplotipi che possono essere estesi
a destra con il simbolo $z[k+1]$. L'idea è quella per cui, avendo
$f_{k+1}<g_{k+1}$ allora sicuramente ho ancora delle righe che presentano un
match che parte da $e_{k+1}=e_{k}$ e termina in $k$ che può essere esteso in
$k+1$. In caso contrario, avendo $f_{k+1}=g_{k+1}$, non si hanno match
estendibili e quindi si può concludere che quelli terminanti in colonna $k$
erano match massimali. In questo secondo caso bisogna poi aggiornare $e_{k+1}$,
ottenedo i relativi $f_{k+1}$ e $g_{k+1}$, al fine di trovare la nuova colonna
da cui parte lo \textit{SMEM} successivo e le righe del pannello per le quali si
ha tale \textit{SMEM}. \\
Bisogna, quindi, capire come funzioni la
variante del \textbf{backward-step} visto per la \textit{BWT}. Tale mapping,
guidato dal carattere corrente dell'aplotipo query, permette di ottenere
$f_{k+1}$ e $g_{k+1}$ a partire da $f_k$ e $g_k$ .\\ 
Per effettuare il mapping abbiamo bisogno di tre componenti, che,
intuitivamente, svolgono la medesima funzione di $C$ e $Occ$ per la
\textit{BWT}: 
\begin{enumerate}
  \item l'array $c$ tale per cui $c[k]=j$ sse la colonna $k$ contiene $j$
  occorrenze di 0
  \item l'array $u_k$ tale per cui, alla colonna $k$-esima, $u_k[i]=j$ sse $j$ è
  il numero di occorrenze di 0 prima dell'indice $i$ nella colonna $k$
  \item l'array $v_k$ tale per cui, alla colonna $k$-esima, $v_k[i]=j$ sse $j$ è
  il numero di occorrenze di 1 prima dell'indice $i$ nella colonna $k$ 
\end{enumerate}
Tali valori possono essere computati e memorizzati in fase di costruzione della
\textbf{PBWT}, come visibile direttamente nell'algoritmo \ref{algo:durbin1} per
quanto riguarda $u$ e $v$. Per quanto riguarda $c$
si ha che potrebbe essere banalmente calcolato anch'esso in fase di costruzione
della \textbf{PBWT}, tenendo ogni volta traccia del numero di 0 incontrati
nella colonna $k$-esima.\\
Sfruttando i valori di questi 3 array possiamo quindi effettuare lo step/mapping
alla colonna successiva,
definito, per comodità, da una funzione:
\begin{equation}
  \label{eq:pbwtw1}
  w_k:\{0,\ldots,M\}\times\Sigma\to \{0,\ldots,M\}
\end{equation}
tale per cui:
\begin{equation}
  \label{eq:pbwtw2}
  w_k(i,\sigma)=
  \begin{cases}
    u_k[i]&\mbox{ se }\sigma=0\\
    v_k[i]+c[k]&\mbox{ se }\sigma=1
  \end{cases}
\end{equation}
Tale funzione è rappresentabile in pseudocodice come nell'algoritmo
\ref{algo:pbwtlf}. \\
Risulta interessante notare, come confermato anche dall'algoritmo di costruzione
stesso, che si ha:
\begin{equation}
  \label{eq:pbwt3}
  a_{k+1}\left[w_k\left(i,y_i^k[k]\right)\right]=a_k[i]
\end{equation}
Avendo che tale mapping permette di ``seguire'' una determinata riga
all'interno delle varie permutazioni dettate dai vari $a_k$.
\begin{algorithm}
  \small
  \begin{algorithmic}[1]
    \Function{$w$}{$k,i,s, c_k, u_k, v_k$}
    \If{$s = 0$}
    \State \textbf{return} $u_k[i]$
    \Else
    \State \textbf{return} $c_k+v_k[i]$
    \EndIf
    \EndFunction
  \end{algorithmic}
  \caption{Algoritmo per il mapping nella PBWT.}
  \label{algo:pbwtlf}
\end{algorithm}
\begin{esempio}
  Vediamo un piccolo esempio chiarificatore, riprendendo l'esempio
  \ref{es:pbwt1}, ricordando che:
  \[a_5=[14,15,17,0,4,5,6,7,9,10,16,8,11,12,13,18,19,1,2,3]\]
  \[\alpha_5=[3,17,18,19,4,5,6,7,11,8,9,12,13,14,0,1,10,2,15,16]\]
  \[a_6=[14,15,0,9,10,16,8,11,12,13,18,19,1,2,3,17,4,5,6,7]\]
  \[\alpha_6=[2,12,13,14,16,17,18,19,6,3,4,7,8,9,0,1,5,15,10,11]\]
  Si ha, ad esempio, con $k=5$ e $i=2$, che:
  \[a_{6}\left[w_5\left(2,y_2^5[5]\right)\right]=a_5[2]\]
  Avendo:
  \[w_5\left(2,y_2^5[5]\right)=w_5\left(2,1\right)=v_5[2]+c[5]=0+15=15\]
  Si ha che:
  \[a_{6}[15]=17=a_5[2]\]
  Dimostrando come con tale funzione si possa, di fatto, ``seguire'' la riga 17,
  capendo da quale posizione arrivi della colonna permutata precedente.
\end{esempio}
\noindent
Pensando alla permutazione inversa del \textbf{prefix array}, si
ottiene un risultato interessante, permettendo di ottenere risultati per la
colonna $k+1$ a partire dalla colonna $k$-esima:
\begin{equation}
  \label{eq:pbwtw4}
  \alpha_{k+1}[i]=w_k(\alpha_k[i],x_i[k])
\end{equation}
\begin{esempio}
  Si riprendono i dati dell'esempio precedente e si vuole calcolare, sempre con
  $k=5$ e $i=2$:
  \[\alpha_{6}[2]=w_5(\alpha_5[2],x_2[5])=w_5(18,0)=13\]
  Come volevasi dimostrare.
\end{esempio}
\dc{Capire se dire altro su $w()$}
L'ultima equazione ci suggerisce quindi che la funzione $w$
consente il corretto aggiornamento di $f_k$ e $g_k$.
Definendo, infatti:
\begin{equation}
  \label{eq:pbwtq5}
  f_{k+1}=w_k(f_k,z[k])
\end{equation}
si ha che $f_{k+1}$ sarà l'indice, in $a_{k+1}$, della prima sequenza $y_j^k$,
con $j\geq f_k$, per la quale $y_j^k[k]=z[k]$. Analogamente, pensando alla prima
sequenza per cui si ha un mismatch dopo l'aggiornamento dell'intervallo, si
calcola: 
\begin{equation}
  \label{eq:pbwtq6}
  g_{k+1}=w_k(g_k,z[k])
\end{equation}
Si hanno quindi, dopo il calcolo dei potenziali $f_{k+1}$ e $g_{k+1}$ due
possibili casi: 
\begin{enumerate}
  \item si ha che $f_{k+1}<g_{k+1}$. In questo caso si hanno ancora match che
  partono da $e_k$ e terminano in $k$ che si estendono anche in $k+1$. In altri
  termini si 
  ha che un sottointervallo non nullo di $[f_k, g_k)$ è relativo a righe che
  presentano $z[k+1]$ come simbolo in colonna $k+1$. In tal caso, si
  prosegue con l'iterazione, avendo $e_{k+1}=e_k$
  \item si ha che $f_{k+1}=g_{k+1}$. In questo caso non si hanno match che
  partono da $e_k$ e terminano in $k$ che sono anche estendibili in
  $k+1$. Bisogna quindi 
  annotare i match terminanti in $k-1$, nell'intervallo $[f_k,g_k)$ su $a_k$,
  e poi ricalcolare i nuovi $e_{k+1}$, $f_{k+1}$ e $g_{k+1}$. Il punto
  fondamentale per poter calcolare i nuovi indici è 
  che, virtualmente, l'aplotipo $z$ si trova, in colonna $k$, o subito prima o
  subito dopo il blocco di aplotipi indicizzati da $[f_k,g_k)$ su $a_k$, secondo
  l'ordinamento dato dalla medesima colonna. Di conseguenza si può inferire che,
  essendo $z$
  nell'ordinamento in $k$ o subito prima di $f_{k}$ o subito dopo $g_k$ ed
  avendo $f_{k+1}=g_{k+1}$: 
  \begin{equation}
    \label{eq:pbwtsmem1}
    y_{f_{k+1}-1}^{k+1}\prec z\prec y_{f_{k+1}}^{k+1}
  \end{equation}
  Ne segue direttamente che:
  \begin{equation}
    \label{eq:pbwtsmem2}
    e_{k+1}\leq d_{k+1}[f_{k+1}]
  \end{equation}
  Avendo che il nuovo indice di partenza del match sarà almeno nella colonna
  indicata da $d_{k+1}[f_{k+1}]$, essendo esso calcolato tra $
  y_{f_{k+1}-1}^{k+1}$ e $ y_{f_{k+1}}^{k+1}$, tra le quali sequenze è
  virtualmente compresa la query $z$. \\
  Si considera quindi, come punto di partenza:
  \begin{equation}
    \label{eq:pbwtsmem3}
    e_{k+1}=d_{k+1}[f_{k+1}]-1
  \end{equation}
  Studiando, di conseguenza, $z[e_{k+1}]$, si hanno due casi possibili, dati dal
  fatto che, per la nozione di \textit{divergence array} e di ordinamento dei
  prefissi inversi con $0\prec 1$:
  \begin{equation}
    \label{eq:pbwtsmem4}
    y_{f_{k+1}-1}^{k+1}[e_{k+1}]=0\neq y_{f_{k+1}}^{k+1}[e_{k+1}]=1
  \end{equation}
  Si ha quindi che:
  \begin{enumerate}
    \item se tale valore è 0 allora $z$ ha un match migliore
    con $y_{f_{k+1}-1}^{k+1}$ rispetto che con $y_{f_{k+1}}^{k+1}$. Si aggiorna
    quindi $e_{k+1}$, decrementandolo, fino a che si ha match tra $z[e_{k+1}-1]$
    e $y_{f_{k+1}-1}^{k+1}[e_{k+1}-1]$. Infine si decrementa $f_{k+1}$ fino a
    che $d_{k+1}[f_{k+1}]\leq e_{k+1}$, trovando quelle righe per il quale il
    \textit{divergence array} non supera il valore di $e_{k+1}$. Si ottengono in
    tal modo le sequenze, nel riordinamento in $k+1$, che hanno un match da
    $e_{k+1}$ a $k+1$. Invece $g_{k+1}$ resta fisso, avendo che
    $y_{g_{k+1}}^{k+1}$ presenta un
    mismatch in colonna $k+1$
    \item se tale valore è 1 allora, per l'ordinamento, $z$ ha un match migliore
    con $y_{f_{k+1}}^{k+1}$ rispetto che con $y_{f_{k+1}-1}^{k+1}$. Si aggiorna
    quindi $e_{k+1}$, decrementandolo, fino a che si ha match tra $z[e_{k+1}-1]$
    e $y_{f_{k+1}-1}^{k+1}[e_{k+1}-1]$. Infine si incrementa $g_{k+1}$ fino a
    che $d_{k+1}[g_{k+1}]\leq e_{k+1}$, per lo stesso ragionamento del caso
    precedente. Si noti che si permette di ottenere $g_{k+1}=M$ avendo che tale
    valore risulta escluso in $[f_{k+1},g_{k+1})$. In tal modo si segnala che la
    riga indicizzata con $a_{k+1}[M-1]$, in colonna $k+1$, presenta un
    match. Invece $f_{k+1}$ resta fisso, avendo che $y_{f_{k+1}}^{k+1}$ 
    presenta un mismatch in colonna $k+1$ 
  \end{enumerate}
\end{enumerate}
In termini di inizializzazione, per permettere il funzionamento dell'algoritmo,
si hanno: 
\[f_0=g_0=e_0=0\]
Quindi il primo step sarà già un caso in cui $f_k=g_k$ qualora $x_0[0]=y_0^0\neq
z[0]$. 
\begin{esempio}
  \label{es:algo5}
  Mostrare un esempio completo di esecuzione richiederebbe troppo spazio quindi
  ci si limita a mostrare cosa succede nel caso in cui, ad un certo punto
  dell'esecuzione si hanno $f_{k+1}=g_{k+1}$.\\
  Si assuma il pannello e la \textit{matrice PBWT} visti all'esempio
  \ref{es:pbwt1} con una query $z$. Nel complesso si identificano i
  seguenti match:
  \begin{figure}[H]
    \centering
    \includegraphics[scale = 0.365]{img/pbwtmatch.pdf}
  \end{figure}
  Si assuma di essere in colonna $k=6$, avendo, dopo i calcoli fatti in colonna
  $k=5$: 
  \begin{itemize}
    \item $f_6=6$
    \item $g_6=10$
    \item $e_6=0$
  \end{itemize}
  Avendo quindi che, a partire dalla colonne $0$ fino alla colonna $6-1=5$, si
  hanno le righe nel range $[6,10)$ di $a_{6}$ che matchano con $z[0,5]$. Tali
  righe sono, nel dettaglio, quelle indicizzate $\{8, 11, 12, 13\}$.\\
  Bisogna quindi aggiornare $f_7$ e $g_7$. Si assuma che $z[6]=1$ e che:
  \[y^6=00000000000100000000,\,\,\,c[6]=19\]
  Si calcolano quindi:
  \[f_7=w_6(6,1)=v_6[6]+c[6]=0+19=19\]
  \[g_7=w_6(10,1)=v_6[10]+c[6]=0+19=19\]
  Avendo quindi $f_7=g_7$ si procede, in primis, annotando i match terminanti in
  $k=5$.\\
  Seguendo l'algoritmo si ha quindi un primo aggiornamento di $e_{k+1}$, che
  viene inizializzato a, avendo in memoria $d_7$ con random access in tempo
  costante: 
  \[e_7=d_7[19]-1=7-1=6\]
  Questo viene fatto in quanto, come detto, l'aplotipo $z$ si trova o subito
  prima del blocco di aplotipi $[f_k,g_k)$.\\
  Essendo inoltre $z[e_7]=z[6]=1$ si procede aggiornando $g_7$ e tenendo fermo
  $f_7$, avendo $z[6]=1$. Si procede quindi inizializzando il
  nuovo $g_7$:
  \[g_7=f_7+1=20\]
  ricordando che $g_k$ può ``superare'' le dimensioni del pannello essendo
  escluso in $[f_k,g_k)$.\\
  A questo punto si segue la linea specificata da $f_7$ in $a_7$ a ritroso,
  partendo da $e_7-1$, fino a che si hanno match con $z$, aggiornando così il
  valore di $e_7$.\\
  In questo caso non si hanno altre operazioni, in quanto $g_7=M$ ma, qualora
  non lo fosse stato, si sarebbe incrementato $g_7$ fino a che il corrispondente
  $d_7[g_7]$ sarebbe stato minore o uguale di $e_7$, identificando tutte le
  nuove righe che hanno un match da con $z[e_7, 6]$.
  \dc{Forse sono necessarie immagini?}
\end{esempio}
L'\textit{algoritmo 5}, consultabile all'algoritmo \ref{algo:dur5} secondo i
calcoli di Durbin, ha complessità:
\begin{equation}
  \label{eq:pbwtsmem5}
  \mathcal{O}(N+c)
\end{equation}
tale risultato è stimato tale in quanto si ritiene che il
numero di accessi ai loop interni sia limitato dalla costante rappresentante il
numero di match, $c$. Nonostante ciò tale complessità temporale è ancora in
corso di studio in quanto si hanno in letteratura evidenze della sua non
correttezza. Un esempio è il paper di Naseri \cite{dpbwt}, dove si afferma che
l'intuizione per cui tale costante $c$ limiti superiormente gli accessi ai loop
innestati sia falsa. Si noti che nell'articolo non viene però precisata una
nuova misura per la complessità dell'algoritmo ma solo che la stima di Durbin è
empiricamente accettabile come \textit{caso medio}:
\begin{equation}
  \label{eq:pbwtsmem6}
  Avg.\,\,\,\mathcal{O}(N+c)
\end{equation}
In ogni caso, una soluzione na\"{i}ve, impiegherebbe tempo:
\begin{equation}
  \label{eq:pbwtsmem7}
  \mathcal{O}(N^2M)
\end{equation}
Si comprende, quindi, come tale algoritmo e tale struttura siano stati
rivoluzionari per lo studio di pannelli di aplotipi.
\begin{algorithm}
  \begin{algorithmic}[1]   
    \Function{Find\_Set\_Maximal\_Matches\_From\_Z}{$z$}
    \For {$k\gets 0$ \textbf{to} $N$}
    \State $e,f,g\gets \mbox{\textit{Update\_Z\_Matches}}(k, z, e, f, g)$
    \EndFor
    \EndFunction
    \State
    \Function{Update\_Z\_Matches}{$k, z, e, f, g$}
    \State $f'\gets w(k, f, z[k])$
    \State $g'\gets w(k, g, z[k])$
    \If{$f'<g'$}
    \Comment\textit{{se $k$ è $N-1$ match da $e_k$ a $N-1$}}
    \State $e'\gets e_k$
    \Else
    \Comment{\textit{match da $e_k$ a $k$}}
    \State $e'\gets d_{k+1}[f']-1$
    \If{$z[e']=0$ \textbf{and} $f'>0$}
    \State $f'\gets g'-1$
    \State \textbf{while} $z[e'-1]=y_{f'}^{k+1}[e'-1]$ \textbf{do} $e'\gets
    e'-1$
    \State \textbf{while} $d_{k+1}[f']\leq e'$ \textbf{do} $f'\gets f'-1$
    \Else
    \State $g'\gets f'+1$
    \State \textbf{while} $z[e'-1]=y_{f'}^{k+1}[e'-1]$ \textbf{do}  $e'\gets
    e'-1$ 
    \State \textbf{while} $g'<M$ \textbf{and} $d_{k+1}[g']\leq e'$ \textbf{do}
    $g'\gets g'+1$ 
    \EndIf
    \EndIf
    \State \textbf{return} $e',f',g'$
    \EndFunction
  \end{algorithmic}
  \caption{Algoritmo 5 di Durbin.}
  \label{algo:dur5}
\end{algorithm}
\subsubsection{Limiti spaziali}
Bisogna affrontare la tematica della complessità in spazio di tale
algoritmo. Si ipotizzi di non ricalcolare, colonna per colonna, comportando
un'incremento dal punto di vista temporale, tutti gli array 
necessari a costruire la \textit{PBWT} e a permettere di computare la funzione
$w(i,\sigma)$.\\
Ricapitolando, per poter eseguire l'algoritmo 5, si necessita di avere in
memoria, con \textit{random access} in tempo costante:
\begin{itemize}
  \item il \textbf{pannello} $X$, di dimensione $NM$
  \item l'insieme dei \textbf{prefix array} $a$, di dimensione $NM$
  \item l'insieme dei \textbf{divergence array} $d$, di dimensione $NM$
  \item i \textbf{vettori} $u_k$ e $v_k$, per ogni colonna $k$, complessivamente
  di dimensione $2NM$ 
  \item il \textbf{vettore} $c$, di dimensione $N$
\end{itemize}
Possiamo quindi dire che si ha una complessità in memoria pari a:
\begin{equation}
  \label{eq:pbwtsize}
  \mathcal{O}(NM)
\end{equation}
Nel dettaglio, Durbin stesso ha proposto una stima quantitativa di tale memoria
richiesta,
ovvero\footnote{\scriptsize{\url{https://github.com/richarddurbin/pbwt/blob/0de8d02df1b77146ded81e9e196991fdab520767/pbwtMatch.c\#L252}}}:
\begin{equation}
  \label{eq:pbwtsize2}
  13NM\mbox{ bytes}
\end{equation}
Per poter capire meglio la problematica conseguente a tali richieste di spazio,
prendiamo, ad esempio, un pannello di 
medie dimensioni, con $N= 6.196.151$ e $M=4.908$. Ne segue che, secondo la stima
di 
Durbin, si necessitano 368GB di memoria. Inoltre, una stima
sperimentale di tale richiesta di memoria può essere confermata con l'esecuzione
dell'implementazione ufficiale della
\textit{PBWT}\footnote{\url{https://github.com/richarddurbin/pbwt}}. Infatti,
monitorando  
con \texttt{time} il picco di memoria durante l'esecuzione, si ha che esso
corrisponde a 369GB, comprensivi anche di tutto ciò che è ``a
contorno'' all'algoritmo stesso. I dati quindi sembrano confermare le stime di
Durbin, confermando l'alto uso di memoria richiesto dall'\textit{algoritmo
  5}. Questa è 
stata la motivazione principale per cui si è sviluppata, in questa tesi
magistrale, una versione \textbf{run-length encoded} della struttura dati che
permettesse di effettuare query con un aplotipo esterno.
\subsection{Varianti della PBWT}
Negli anni immediatamente successivi all'articolo di Durbin, una miriade di
articoli e ricerche sono state svolte per migliorare la \textit{PBWT}, crearne
varianti o utilizzarla per portare a compimento vari studi. Non essendo tali
lavori direttamente correlati a questa tesi non 
verranno approfonditi ma, soprattutto nell'ottica dei prospetti futuri, è bene
citarne i principali.
\subsubsection{PBWT multiallelica}
La prima variante che si introduce è la \textbf{PBWT multiallelica
  (\textit{mPBWT})}, proposta da Naseri et al. nel 2019 \cite{mpbwt}. Questo
lavoro estende la \textit{PBWT} di Durbin generalizzandola ad un alfabeto
arbitrario. \\
Dal punto di vista delle motivazioni biologiche, questa soluzione risulta
fondamentale, oltre che per lo studio di specie multialleliche (soprattutto nel
mondo vegetale) in quanto gli studi riportano come, nell'uomo, la presenza di
siti triallelici sia sotto stimata. \\
Da un punto di vista prettamente algoritmico si sono quindi estesi i concetti di
$c$, $u_k$ e $v_k$ visti nella \textit{PBWT} per ottenere un vero e proprio
\textit{FM-index} in grado di lavorare su alfabeto arbitrario $\Sigma$, con
conseguente forte aumento dello spazio richiesto in memoria. Da un punto di
vista della complessità temporale, invece, si ha che le le stime asintotiche
degli 
algoritmi devono tenere conto anche della grandezza dell'alfabeto stesso, avendo
però che, essendo esso tendenzialmente di dimensioni ridotte, questo fatto non
comporti, in media, particolari problematiche dal punto di vista dei tempi di
calcolo. Le complessità temporali della \textit{mPBWT} infatti sono incrementate
di un fattore $t$, con $t=\left|\Sigma\right|$, e se tale valore è assunto
costante ad inizio computazione, avendo che difficilmente si ha $t>>2$, la
complessità temporale non subisce variazioni considerevoli.
\subsubsection{PBWT con struttura LEAP}
Sempre nel 2019, Naseri et al.\cite{leap} proposero anche una variante della
\textit{PBWT} 
che permettesse il calcolo non solo dei match massimali, come per l'algoritmo 5
di Durbin, ma anche qualsiasi match di lunghezza maggiore uguale ad una
lunghezza arbitraria $L$. Tale algoritmo fu nominato
\textbf{PBWT-query}. Inoltre, nello 
stesso articolo, proposero 
una struttura dati aggiuntiva, detta \textbf{LEAP (\textit{Linked
    Equal/Alternating Position})}, che, al costo della 
memorizzazione di otto array aggiuntivi che permettessero di effettuare dei
salti nella \textit{matrice PBWT} (salvando gli indici del precedente/prossimo
valore nella colonna uguale/diverso) e di memorizzare gli indici dei valori
nel \textit{divergence array} relativi a tali indici, ottimizzava i tempi
dell'algoritmo per la \textit{PBWT-query} ottenendo l'algoritmo detto
\textbf{(\textit{L-PBWT-query})}.
\dc{Serve dire di più sulla struttura LEAP?}
Da un punto di vista computazionale si noti che la complessità dell'algoritmo
per la \textit{\textit{PBWT query}}, con match di lunghezza minima $L$ è:
\begin{equation}
  \label{eq:leap1}
  \mathcal{O}(N+c(R-L+1))
\end{equation}
Avendo:
\begin{itemize}
  \item $R$ lunghezza media dei match
  \item $c$ numero totale dei match
\end{itemize}
In merito ai tempi dell'algoritmo \textit{L-PBWT-query} si ha invece che è, al
costo di $8NM$ interi aggiuntivi in memoria, con $N$ e $M$ dimensioni del
pannello: 
\begin{equation}
  \label{eq:leap2}
  \mathcal{O}(N+c)
\end{equation}
\subsubsection{PBWT dinamica}
Sanaullah et al., nel 2021, proposero la \textbf{Dynamic PBWT (\textit{d-PBWT})}
\cite{dpbwt} col fine di superare le limitazioni imposte dalle strutture
statiche usate nella \textit{PBWT} di Durbin. Si è quindi pensato di sostituire
l'uso degli array, statici, con l'uso di \textit{linked list}, dinamiche.\\
Grazie alle \textit{linked list}, si è reso possibile l'aggiornamento
efficiente della \textit{matrice PBWT} all'aggiunta di un nuovo aplotipo nel
pannello o alla rimozione di uno già presente nel pannello.\\
Da un punto di vista computazionale, è interessante notare come le
implementazioni degli algoritmi di Durbin presentino la medesima complessità
asintotica. Infatti, ad esempio, la creazione della \textit{d-PBWT} richiede
tempo:
\begin{equation}
  \label{eq:dpbwt}
  \mathcal{O}(NM)
\end{equation}
Invece, l'aggiunta e la rimozione di un aplotipo sono entrambe in tempo:
\begin{equation}
  \label{eq:dpbwt1}
  Avg.\,\,\,\mathcal{O}(N)
\end{equation}
\subsubsection{PBWT con wildcard}
La tematica dei dati mancanti è una tematica aperta in
\textit{bioinformatica}. I sequenziatori infatti presentano un range d'errore
dal 1\% al 15\%, si ha a volte un basso \textit{coverage} (ovvero il numero di
read che contengono la base sequenziata per un certo locus del genoma) e la fase
di assemblaggio del genoma può comportare errori. Questo, in fase di produzione
dei pannelli, implica che, in determinati casi, non si sappia quale sia l'allele
corretto per un individuo riferendosi ad un sito. \\
Nel 2020, Williams e Mumey \cite{williams} proposero quindi l'uso della
\textbf{PBWT con 
  wildcard} al fine di disegnare un algoritmo in grado di trovare i match
interni ad un pannello biallelico con dati mancanti, rappresentati come
\textit{wildcard} mediante il simbolo ``*'' (avendo quindi
$\Sigma=\{0,1,*\}$).\\ 
In termini computazionali gli autori sono riusciti a formulare un algoritmo in
grado tutti i match interni (ovvero i \textit{blocchi}) massimali al pannello in
tempo, con $T$ numero di blocchi: 
\begin{equation}
  \label{eq:pbwtwild}
  \mathcal{O}(NMT)
\end{equation}
\dc{Serve dire altro?}
\subsubsection{IMPUTE5}
Per citare un uso della \textit{PBWT} si può introdurre il concetto di
\textbf{genotype 
  imputation}, ovvero il processo con il quale si predicono genotipi non ancora
osservati in un campione di individui, usando un pannello di aplotipi. Questo
tipo di studio si basa sui dati prodotti dai \textbf{GWAS (\textit{Genome-wide
    association studies})}, studi il cui scopo è quello di di esaminare multipli
genomi alla ricerca di associazioni tra varianti genetiche e malattie (o
outcome specifici delle stesse), identificando varianti genomiche che sono
statisticamente associate al rischio per una malattia.\\ 
A tal fine, nel 2020, Rubinacci et al. proposero \textbf{IMPUTE5}
\cite{impute5}, un metodo basato sulla \textit{PBWT} per la \textit{genotype
  imputation}, in grado di studiare pannelli di grandi dimensioni.
\dc{Aggiungere qualcosa?}
\subsection{Una prima proposta run-length encoded}
\label{subsectravis}
A fine 2021, Gagie et al. \cite{tricks} hanno iniziato a teorizzare una variante
\textbf{run-length encoded} della \textbf{PBWT}, basandosi sui risultati
già ottenuti sulla \textit{BWT} classica con la \textit{RLBWT}.\\
Pensando alla costruzione della \textit{PBWT}, con $M$ individui e $N$ siti, si
ha che ogni colonna della 
\textit{matrice PBWT} è ottenuta tramite la permutazione data dal \textit{prefix
  array}. Denotiamo tale permutazione, alla colonna $k$, con $\pi_k$, $\forall\,
1\leq k<N$. 
Ipotizziamo ora di voler studiare la riga $i$-esima del pannello originale. Si
ha che, al variare della colonna $k$ sulla \textit{matrice PBWT}, la posizione
della riga $i$ è ricostruibile applicando le varie permutazioni
% (che, nei
% termini già presentati nella sezione, sarebbe uguale ad effettuare il
% \textit{mapping} dalla colonna $K$ alla colonna $k+1$)
:
\begin{equation}
  \label{eq:pbwttrick}
  i, \pi_1(i), \pi_2(\pi_1(i)),\ldots,
  \pi_{N-1}(\cdots(\pi_2(\pi_1(i)))\cdots)
\end{equation}
Il punto fondamentale si ritrova nel fatto che l'autore asserisce:
\begin{center}
  \textit{Notice $\pi_k$ can be stored in space proportional to the number of
    runs in the $k$th column of the PBWT$\ldots$} 
\end{center}
Nell'articolo si propone quindi una struttura dati formata da $N$ ``tabelle''
dove, la $j$-esima riga della $k$ tabella contiene: 
\begin{itemize}
  \item l'indice $p$ di inizio della $j$-esima run nella colonna $k$ della
  \textit{matrice PBWT}
  \item il valore $\pi_k(p)$, avendo che:
  \begin{equation}
    \label{eq:pbwttrick1}
    \pi_k(p)=
    \begin{cases}
      p-v_k[p]&\mbox{if } y_p^k[k]=0\\
      c[k]+v_k[p]-1&\mbox{if } y_p^k[k]=1\\
    \end{cases}
  \end{equation}
  \item l'indice della run contenente il simbolo $pi_k(p)$ nella colonna $k+1$
  della \textit{matrice PBWT}
  \item un booleano per capire se la prima run è composta da simboli $\sigma=0$
  o $\sigma =1$
\end{itemize}
Il paper presenta anche il metodo per l'estrazione della $i$-esima riga:
\begin{enumerate}
  \item si cerca della prima ``tabella'' la riga relativa alla run, con indice
  di testa $p$, contenente
  l'indice $i$. Si noti che la prima ``tabella'', relativa alla colonna $k=0$
  non presenta permutazioni e quindi 
  l'indice $i$ del pannello è anche l'indice $i$ della \textit{matrice PBWT}
  \item si calcola la permutazione per l'indice $i$ (alla prima operazione
  si avrà $k=1$):
  \begin{equation}
    \label{eq:pbwttrick2}
    \pi_k(i)=\pi_k(p)+i-p
  \end{equation}
  \item si cerca poi la riga relativa alla run contenente il simbolo $\pi_k(p)$
  nella ``tabella'' successiva e si scansionano le righe di tale tabella a
  partire da quella appena identificata fino a trovare la run che contiene
  $\pi_k(i)$ (alla prima
  operazione si avrà $k=1$). Infine, si estrae il simbolo relativo a tale run 
  \item si ripete la procedura dal punto 2) per ogni colonna $k$
\end{enumerate}
Inoltre,
vengono proposti ulteriori ottimizzazioni, basate sul metodo detto
\textbf{fractional cascading}. Con tale rappresentazione, si riesce a ridurre il
numero di run che devono essere scansionate nel passaggio 3. Questo risultato
ha un tradeoff in termini di spazio. Infatti, aumentando il numero totale di
righe in tutte le tabelle di un fattore al più $\left(1+\frac{1}{d}\right)$, è
possibile garantire che si avranno al più $d$ iterazioni, in ogni tabella, per
ottenere l'estrazione del simbolo desiderato.\\
\textit{Si segnala che, nel paper, non vengono specificati metodi per
effettuare query a 
tale struttura dati, indicando solo che dovrebbe essere possibile interrogare
tale struttura a tabelle}.\\
Per ulteriori dettagli si rimanda al paper di riferimento \cite{tricks}.
\begin{esempio}
   Supponendo di voler ricostruire la riga $i=9$, si assuma la \textit{matrice
     PBWT}, avendo che in rosso sono segnati i simboli appartenenti alla riga 9: 
  \begin{table}[H]
    \centering
    \footnotesize
    \begin{tabular}{c|ccccccccccccccc}
      X & 01 & 02 & 03 & 04 & 05 & 06 & 07 & 08 & 09 & 10 & 11 & 12 \\
      \hline
      00 & 1 & 1 & 0 & 0 & 0 & 1 & 0 & 0 & 1 & 1 & 1 & 1 \\
      01 & 1 & 1 & 0 & 0 & 0 & 1 & 0 & 0 & 1 & 1 & {\color{nordred}\textbf{1}}
                                                               & 1 \\
      02 & 1 & 1 & 1 & 0 & 0 & 0 & 1 & 1 & 1 & 0 & 1 & 1 \\
      03 & 1 & 1 & 0 & {\color{nordred}\textbf{0}} & {\color{nordred}\textbf{0}}
                                 & {\color{nordred}\textbf{1}} & 0 & 0 & 1 & 1
                                                          & 0 & 1 \\
      04 & 1 & 0 & 1 & 0 & 0 & 1 & 0 & 0 & 1 & 1 & 0 & 1 \\
      05 & 1 & 0 & 1 & 0 & 0 & 0 & 0 & 0 & 1 & {\color{nordred}\textbf{0}} & 0
                                                               & 1 \\
      06 & 1 & 0 & 1 & 0 & 0 & 0 & 0 & 0 & 1 & 0 & 0 & 0 \\
      07 & 1 & 1 & 1 & 0 & 0 & 0 & 0 & 0 & 0 & 1 & 0 & 1 \\
      08 & 0 & 1 & {\color{nordred}\textbf{0}} & 0 & 0 & 1 & 0 & 0 & 0 & 1 & 0
                                                               & 1 \\
      09 & {\color{nordred}\textbf{1}} & 0 & 0 & 0 & 0 & 1 & 0 & 0 & 0 & 0 & 0
                                                               & 1 \\
      10 & 1 & 1 & 0 & 0 & 0 & 0 & 0 & 0 & 1 & 1 & 0 & 1 \\
      11 & 0 & 1 & 0 & 1 & 1 & 0 & 0 & 0 & 1 & 0 & 0 & 1 \\
      12 & 0 & 1 & 0 & 0 & 1 & 0 & 0 & 0 & 0 & 0 & 0 & 1 \\
      13 & 0 & 0 & 0 & 0 & 1 & 0 & 0 & 0 & 0 & 0 & 0 & 1 \\
      14 & 0 & 0 & 0 & 0 & 0 & 0 & 0 & 0 & {\color{nordred}\textbf{0}} & 0 & 0
                                                               & 1 \\
      15 & 0 & 0 & 0 & 0 & 0 & 0 & 0 & {\color{nordred}\textbf{0}} & 0 & 0 & 0
                                                               & 1 \\
      16 & 1 & 0 & 0 & 0 & 0 & 0 & {\color{nordred}\textbf{0}} & 0 & 1 & 0 & 0
                                                               & 1 \\
      17 & 0 & {\color{nordred}\textbf{0}} & 0 & 0 & 0 & 0 & 0 & 1 & 1 & 0 & 0
                                                               & 1 \\
      18 & 0 & 0 & 0 & 0 & 0 & 0 & 0 & 1 & 1 & 0 & 0
                                         & {\color{nordred}\textbf{1}} \\ 
      19 & 0 & 0 & 0 & 0 & 0 & 0 & 0 & 1 & 1 & 0 & 0 & 1 \\
    \end{tabular}
  \end{table}
  Si costruiscono quindi le seguenti tabelle \cite{tricks}:
   \begin{figure}[H]
    \centering
    \includegraphics[scale=0.3]{img/trick.jpg}
  \end{figure}
  Nelle tabelle, in rosso, si hanno le varie $\pi_k(i)$ calcolate nel processo,
  ottenute, 
  se necessario, iterando a partire dalle $\pi_k(p)$, segnalate in azzurro. \\
  Si hanno infatti i seguenti calcoli, ovvero i vari $\pi_j(i)=\pi_j(p)+i-p$ 
  relativi alle permutazioni in colonna $k$, per l'estrazione della riga $9$:
  \begin{multicols}{2}
    \begin{itemize}
      \item $\pi_1(9)=17+9-9=17$
      \item $\pi_2(17)=4+17-13=8$
      \item $\pi_3(8)=4+8-8=3$
      \item $\pi_4(3)=0+3-0=3$
      \item $\pi_5(3)=0+3-0=3$
      \item $\pi_6(3)=16+3-3=16$
      \item $\pi_7(16)=2+16-3=15$
      \item $\pi_8(15)=2+15-3=14$
      \item $\pi_9(14)=3+14-12=5$
      \item $\pi_{10}(5)=1+5-5=1$
      \item $\pi_{11}(1)=17+1-0=18$
      % \item $\pi_{12}(3)=16+3-3=16$        
    \end{itemize}
  \end{multicols}
  Sfruttando quindi il valore booleano (non rappresentato nelle tabelle ma
  esistente) che ci dice con che simbolo inizia una colonna e sapendo che,
  essendo un pannello binario si alternano le run con simboli $\sigma=0$ e
  $\sigma=1$, si può ricostruire la riga 9 del pannello originale:
  \[x_9=100001000011\]
\end{esempio}

% LocalWords:  pseudocodice aplotipi mismatch



\chapter{Metodo}
\label{metchap}
In questo capitolo verranno illustrate le metodologie usate in questa tesi,
trattando, sia dal punto di vista teorico che sperimentale, tutte le soluzioni
che hanno portato alla costruzione di diverse varianti della \textbf{RLPBWT}.\\
Nel dettaglio, si approfondiranno tutte le varianti della \textbf{RLPBWT}
ottenute durante lo studio, evidenziandone pro e contro.

% sezione varianti RLPBWT
%\section{Motivazioni}
Prima di proseguire con la spiegazione dettagliata delle varianti della
\textbf{RLPBWT} è bene dare 
una prima motivazione al perché si sia ritenuto utile sviluppare una variante
\textbf{run-length encoded} della \textbf{PBWT}.\\
Citando direttamente il paper di Durbin del 2014 \cite{pbwt}, in cui si
introduce la struttura:
\begin{center}
  \textit{Furthermore we can also expect the y arrays to be strongly run-length
    compressible. This is because population genetic structure means that there
    is local correlation in values due to linkage disequilibrium, which means
    that haplotypes with similar prefixes in the sort order will tend to have
    the same allele values at the next position, giving rise to long runs of
    identical values in the y array. So the PBWT can easily be stored in smaller
    space than the original data.} 
\end{center}
Dove, con la dicitura \textit{y arrays}, si indicano le colonne già permutate
della \textbf{matrice PBWT}.\\
Quindi il risultato atteso è quello per cui aplotipi simili, che ad ogni step
saranno consecutivi nel riordinamento, è molto probabile presentino lo stesso
allele nella colonna di cui si sta in quel momento calcolando la
permutazione. Ne segue che, all'interno della \textbf{matrice PBWT}, è molto
probabile che si abbiano, consecutivamente, lunghe run di $0$ e di $1$.\\
Si noti quindi che si ottiene quindi il medesimo risultato atteso che si ha con
la \textbf{BWT}, avendo che caratteri uguali è probabile che vengano posti in
modo consecutivo all'interno della \textit{BWT} stessa. Si hanno quindi le
stesse premesse che hanno portato alla \textbf{RLBWT}, considerando inoltre che,
come in quel caso, non si tratta solo di memorizzare la struttura con
compressione run-length ma di lavorare direttamente con la struttura dati
compressa, risolvendo il problema del pattern matching senza decomprimere la
struttura dati.
\section{Introduzione alle varianti della RLPBWT}
Lo sviluppo di questo progetto di tesi è stato tale per cui si sono sviluppate
varie implementazioni della \textbf{RLPBWT}. Tali varianti non sono da
intendersi ugualmente valide ma corrispondono al percorso evolutivo che c'è
stato nell'ultimo anno di studio e ricerca in merito. Riassumendo il tutto si
vedranno:
\begin{itemize}
  \item una prima implementazione \textit{naive}, detta appunto \textbf{RLPBWT
    naive}, che corrisponde al primo tentativo di studio. Questa soluzione non
  permette di sapere quali righe del pannello stanno matchando ma solo quali
  \item si è quindi iniziato ad introdurre l'uso dei \textit{bitvectors}, con la
  \textbf{RLPBWT con bitvectors}, il cui funzionamento è pressoché analogo alla
  versione \textit{naive} al più dell'uso di tali strutture succinte per il
  funzionamento del mapping. Questa soluzione non
  permette di sapere quali righe del pannello stanno matchando ma solo quali
  \item il primo sostanziale ``cambio di paradigma'', si ha avuto con la
  \textbf{RLPBWT con pannello denso}, variante in cui, oltre all'uso dei
  \textit{bitvectors} si è proceduto al calcolo dei match tramite
  \textit{matching statistics} e \textit{LCE query}. Questa soluzione permette
  di sapere l'indice di una sola riga per la quale si sta avendo un match con il
  pattern 
  \item migliorando la soluzione precedente con l'uso dell'\textit{SLP} per la
  memorizzazione del pannello si è ottenuta la \textbf{RLPBWT con SLP}. Questa
  soluzione permette di sapere l'indice di una sola riga per la quale si sta
  avendo un match con il pattern 
  \item con l'implementazione della \textbf{funzione} $\mathbf{\varphi}$ per la
  \textbf{RLPBWT} si è permesso di estendere i risultati delle ultime due
  varianti in modo da ottenere tutti i match con tutti gli indici delle righe
  per cui si hanno tali match con il pattern
\end{itemize}
Si può quindi iniziare ad apprezzare il percorso evolutivo e incrementale
vissuto con questo progetto.

\subsection{Componente per il mapping}
La prima componente che si descrive è quella relativa
al mapping tra una colonna e la sua successiva. In altri termini, bisogna
studiare come effettuare il forward-step, che viene qui nominato
\textit{mapping}, nella $\RLPBWT$. Bisogna memorizzare, 
per 
ogni colonna $k$ e in modo proporzionale al numero di run della stessa, le
informazioni atte a ottenere i medesimi risultati ottenibili con la funzione
$\W(i,\sigma)$, secondo la notazione di Durbin \cite{pbwt}. 
\subsubsection{Mapping con intvector compressi}
La prima variante che si descrive è quella denominata \texttt{MAP-INT}.\\
L'ispirazione iniziale per tale componente è stata data dall'articolo di Gagie
et al \cite{tricks}, nonostante si abbiano, di fatto, diverse modifiche
strutturali. Riprendendo quanto descritto al termine della sezione
\ref{secpbwt}, si è deciso di
memorizzare gli indici delle teste di run. Ovviamente questa informazione non è
sufficiente per poter sapere 
se una run sia composta da simboli $\sigma=0$ o simboli
$\sigma=1$. Fortunatamente, essendo lo studio limitato, come per la
$\PBWT$ di Durbin, a pannelli costruiti su alfabeto binario $\Sigma=\{0,1\}$, si
è 
potuto sfruttare il fatto che le run si alternano tra un carattere e
l'altro. Basta quindi tenere in memoria un valore booleano nominato
$start_k$, che permetta di 
capire se, in colonna $k$, la prima run sia una run di simboli
$\sigma=0$. Infatti, le run di 
indice pari presentano lo stesso simbolo della prima run e quindi, dato un
qualsiasi indice di run, è possibile sapere quale sia il simbolo corrispondente
a tale run. L'implementazione di questo concetto banale si ritrova nella
funzione $\GS$, la quale è
visualizzabile all'algoritmo \ref{algo:extrchar} e richiede tempo costante.
\begin{algorithm}
  \footnotesize
  \begin{algorithmic}[1]
    \Function{get\_symbol}{$k, \,\,r$}
    \Comment $k$ indice di colonna, $r$ indice di run
    \If{$start_k$}
    \State \textbf{if} $r\bmod 2 = 0$ \textbf{then} \textbf{return} $0$
    \textbf{else} \textbf{return} $1$
    \Else
    \State \textbf{if} $r\bmod 2 = 0$ \textbf{then} \textbf{return} $1$
    \textbf{else} \textbf{return} $0$
    \EndIf
    \EndFunction
  \end{algorithmic}
  \caption{Algoritmo per estrazione simbolo da una run in una colonna.}
  \label{algo:extrchar}
\end{algorithm}

Si memorizzano gli indici delle teste di run in un array $p_k$, di lunghezza
pari al numero di run in colonna $k$. Si memorizza un indice $i$ sse:
\begin{equation}
  \label{eq:naive1}
  y_{i-1}^k[k]\neq y_i^k[k]
\end{equation}
Il passaggio successivo è stato quello di capire se le informazioni necessarie
al mapping fossero tutte necessarie, ovvero se, data la colonna $k$
nella matrice $\PBWT$, fossero necessari interamente $c[k]$, $u_k$ e $v_k$. In
merito 
al valore $c[k]$, esso è calcolabile, ipotizzando di avere solo $p_k$, in
tempo $\mathcal{O}(r)$, dove $r$ è 
il numero di run della colonna $k$-esima. Però, si è deciso che si potesse
calcolarlo 
in fase di costruzione delle $\RLPBWT$ e memorizzarlo esattamente come per
la $\PBWT$. In merito, invece, ai vettori $u_k$ e $v_k$ si è cercato un modo
per ottenerne una rappresentazione che implicasse avere un solo valore per ogni
run della colonna. In altri termini, si è cercato di capire se fosse possibile
tenere in memoria $r$ valori che permettessero di effettuare comunque il
mapping, a partire da un indice arbitrario $i\in\{0,\ldots,M-1\}$. L'alternanza
data dal caso binario ha permesso di trovare una 
semplice soluzione, avendo che i valori di $u_k$ e $v_k$ crescono infatti in
modo 
alternato. Infatti, a seconda del simbolo $\sigma$ rappresentato in una data
run, si ha che solo i valori dell'array relativo a tale simbolo, nel range di
indici di quella run, verranno incrementati, ad ogni passo, di un'unità. Facendo
un semplice esempio, se siamo in una run di 0 e iteriamo 
virtualmente all'interno di tale run, solo i valori di $u_k$, in quel
range di indici, cresceranno di volta in volta di uno mentre per $v_k$, nello
stesso range, si avrà sempre lo stesso valore.
\begin{esempio}
  Si vede un esempio per chiarire meglio quanto espresso in merito a $u_k$ e
  $v_k$.\\
  Sia data la seguente colonna:
  \[y^5=00101111000000000000\]
  Si hanno, oltre a $c[5]=15$:
  \begin{table}[H]
    \footnotesize
    \centering
    \begin{tabular}{c||cc|c|c|cccc|cccccccccccc}
      & 0 & 1 & 2 & 3 & 4 & 5 & 6 & 7 & 8 & 9 & 10 & 11 & 12 & 13 & 14 & 15 & 16
      & 17 & 18 & 19\\
      \hline
      \hline
      $y^5$ & 0 & 0 & 1 & 0 & 1 & 1 & 1 & 1 & 0 & 0 & 0 & 0 & 0 & 0 & 0 & 0 & 0
      & 0 & 0 & 0\\
      \hline
      \hline
      $u_5$ & 0 & 1 & 2 & 2 & 3 & 3 & 3 & 3 & 3 & 4 & 5 & 6 & 7 & 8 & 9 & 10
      & 11 & 12 & 13 & 14\\
      \hline
      $v_5$ & 0 & 0 & 0 & 1 & 1 & 2 & 3 & 4 & 5 & 5 & 5 & 5 & 5 & 5 & 5 & 5 & 5
      & 5 & 5 & 5
    \end{tabular}
  \end{table}
  Dove si nota l'alternanza di crescita dei valori sopra descritta.
\end{esempio}
Grazie a questo comportamento è possibile memorizzare, per ogni indice di
testa di run $i$, tale che $i\neq 0$, solo il valore di $u_k[i]$ o $v_k[i]$,
rispettivamente se sia una run su simboli $\sigma=1$ o $\sigma=0$.
% Infatti, se
% si analizza una run di zeri si avrà che solo i valori di $v_k$, nel 
% range della run, verranno incrementati ad ogni step.
Per $i=0$, banalmente, si ha $u_k[i]=v_k[i]=0$.\\
Memorizzando i valori di $u_k$ e $v_k$ in un array $uv_k$, tale che $|uv_k|=r$,
con $r$ numero di run, e dato $i\in\{0,\ldots, r-1\}$, a seconda che la colonna
presenti o meno la prima run  
con simboli $\sigma=0$, si possono estrarre, in tempo costante, i valori di
$u_k$ e $v_k$ per una data testa di run. Nel dettaglio, dato $i\in{0,\ldots,
  r-1}$:
\begin{itemize}
  \item se $i=0$ si ha che $u_k[p[i]]=v_k[p[i]]=uv_k[0]=0$
  \item se $i\mod 2 =0$ si hanno due casi:
  \begin{itemize}
    \item la prima run è di simboli $\sigma=0$ e quindi si ottiene
    $u_k[p[i]]=uv_k[i-1]$ e $v_k[p[i]]=uv_k[i]$
    \item la prima run è di simboli $\sigma=1$ e quindi si ottiene
    $u_k[p[i]]=uv_k[i]$ e $v_k[p[i]]=uv_k[i-1]$
  \end{itemize}
  \item se $i\mod 2 \neq 0$ si hanno due casi, che sono l'inverso della
  situazione descritta precedentemente:
  \begin{itemize}
    \item la prima run è di simboli $\sigma=0$ e quindi si ottiene
    $u_k[p[i]]=uv_k[i]$ e $v_k[p[i]]=uv_k[i-1]$
    \item la prima run è di simboli $\sigma=1$ e quindi si ottiene
    $u_k[p[i]]=uv_k[i-1]$ e $v_k[p[i]]=uv_k[i]$   
  \end{itemize}
\end{itemize}
Tale operazione è eseguibile in tempo costante lo pseudocodice relativo a
quanto appena detto è consultabile all'algoritmo \ref{algo:uvnaive}.
\begin{algorithm}
  \small
  \begin{algorithmic}[1]
    \Function{uvtrick}{$k,\,\, i$}
    \Comment $k$ indice di colonna, $i$ indice di run
    \If{$i = 0$}
    \State \textbf{return} $(0,\,\,0)$
    \ElsIf{$i\mod 2=0$}
    \State $u\gets uv_k[i-1],\,\,v\gets uv_k[i]$
    \State \textbf{if} $start_k$ \textbf{then} \textbf{return} $(u,\,\,v)$
    \textbf{else} 
    \textbf{return} $(v,\,\,u)$
    % \If{$start_k$}
    % \State \textbf{return} $(u,\,\,v)$
    % \Else
    % \State \textbf{return} $(v,\,\,u)$
    % \EndIf
    \Else
    \State $u\gets uv_k[i],\,\,v\gets uv_k[i-1]$
    \State \textbf{if} $start_k$ \textbf{then} \textbf{return} $(u,\,\,v)$
    \textbf{else} 
    \textbf{return} $(v,\,\,u)$
    % \If{$start_k$}
    % \State \textbf{return} $(u,\,\,v)$
    % \Else
    % \State \textbf{return} $(v,\,\,u)$
    % \EndIf
    \EndIf
    \EndFunction
  \end{algorithmic}
  \caption{Algoritmo per uvtrick con \texttt{MAP-INT}.}
  \label{algo:uvnaive}
\end{algorithm}

Da un punto di vista implementativo, gli array di interi $p_k$ e $uv_k$ vengono
memorizzati in vettori di interi bit-compressed (intvector compressi),
disponibili bella libreria 
SDSL. Dato un array $v$, 
tale che $x$ è il massimo valore, si memorizzano le componenti di $v$ in un
vettore di interi con componenti memorizzate in $b$ bit, avendo:
\begin{equation}
  \label{eq:bc1}
  b=\lceil\log(x-1)\rceil+1
\end{equation}
Ricapitolando, per la componente \texttt{MAP-INT}, si hanno in memoria, per
ogni colonna $k$:
\begin{itemize}
  \item $start_k$, ovvero il booleano atto a capire il simbolo della prima
  run
  \item $p_k$, ovvero gli indici delle teste di run. 
  \item $uv_k$, ovvero i valori compatti di $u_k$ e $v_k$ per le teste di run
  \item $c[k]$, ovvero il numero totale di simboli $\sigma=0$ nella colonna
  $k$ della matrice $\PBWT$
\end{itemize}
Facendo una prima stima della memoria occupata da questa componente si ha che,
avendo $\rho$ numero medio di run per colonna e considerando che un elemento di
un intvector 
compresso necessita $\frac{\lceil\log(M-1)\rceil+1}{8}$ byte, essa è:
\begin{equation}
  \label{eq:mapintmem}
  \approx N\left(\rho\frac{\lceil\log(M-1)\rceil+1}{4}+5\right)\mbox{ byte}
\end{equation}
\dc{Ricontrollare conto}
\begin{algorithm}
  \small
  \begin{algorithmic}[1]
    \Function{build\_map\_int}{$col,\,\, pref$}
    \Comment $pref=a_k$
    \State $c\gets 0,\,\,u\gets 0,\,\,v\gets 0,\,\,u'\gets 0,\,\, v'\gets
    0,\,\,run\gets 0$
    \State $start \gets \top,\,\,beg_{run}\gets \top,\,\,push_{zero}\gets
    \bot,\,\,push_{one}\gets \bot$
    \State $p\gets [],\,\,uv\gets []$
    \For {\textit{every} $k\in\left[0,\,\, M\right)$}
    \State \textbf{if} $k=0\land col[pref[k]]=1$ \textbf{then} $start \gets \bot$ 
    % \If{$k=0\land col[pref[k]]=1$}
    % \State $start \gets \bot$
    % \EndIf
    \State \textbf{if} $col[pref[k]]=0$ \textbf{then} $c\gets c+1$
    % \If{$col[pref[k]]=0$}
    % \State $c\gets c+1$
    % \EndIf
    \EndFor
    \State \textbf{if} $start$ \textbf{then} $push_{one}\gets \top$
    \textbf{else} $push_{zero}\gets \top$
    % \If{$start$}
    % \State $push_{one}\gets \top$
    % \Else
    % \State $push_{zero}\gets \top$
    % \EndIf
    \For{\textit{every} $k\in[0,M)$}
    \If{$beg_{run}$}
    \State $u\gets u',\,\,v\gets v'$
    \State $beg_{run}\gets \bot$
    \EndIf
    \State \textbf{if} $col[pref[k]]=1$ \textbf{then}  $v'\gets v'+1$
    \textbf{else} $u'\gets u'+1$
    % \If{$col[pref[k]]=1$}
    % \State $v'\gets v'+1$
    % \Else
    % \State $u'\gets u'+1$
    % \EndIf
    \If{$k=0\lor col[pref[k]]\neq col[pref[k-1]]$}
    \State $run\gets k$
    \EndIf
    \If{$k=M-1\lor col[pref[k]]\neq col[pref[k+1]]$}
    \If {$push_{one}$}
    \State $push(p, run),\,\,push(uv, v)$
    \State $swap(push_{one}, push_{zero})$
    \Else
    \State $push(p, run),\,\,push(uv, u)$
    \State $swap(push_{one}, push_{zero})$
    \EndIf
    \State $beg_{run}\gets \top$
    \EndIf
    \EndFor
    \State \textbf{return} $(start,\,\, c,\,\, p,\,\, uv)$
    \EndFunction
  \end{algorithmic}
  \caption{\footnotesize{Algoritmo per la costruzione della componente
  \texttt{MAP-INT} per la colonna $k$.}}
  \label{algo:buildnaive}
\end{algorithm}
\begin{esempio}
  Sia data la seguente colonna:
  \[y^5=00101111000000000000\]
  Per la componente \texttt{MAP-INT} della colonna 5, si hanno in memoria:
  \[p_5=[0,2,3,4,8]\]
  \[uv_5=[0,2,1,3,5]\]
  \[c[5]=15\]
\end{esempio}
La costruzione della componente \texttt{MAP-INT} per una certa colonna,
analizzabile nell'algoritmo 
\ref{algo:buildnaive}, ha costo temporale $\mathcal{O}(M)$, avendo che la
costruzione 
avviene scorrendo la colonna $k$ permutata dal 
prefix array $a_k$.\\
Per
l'implementazione dei vari algoritmi si ha necessità di usare anche indici
con valori $i\in\{0, M-1\}$ e non solo $i\in\{0,r-1\}$. Una delle operazioni
fondamentali è quindi 
quella, dato un indice $i\in\{0,\ldots,M-1\}$, di computare a quale run esso
appartenga, in una certa colonna $k$. Tale operazione può essere svolta usando
una semplice variante della ricerca binaria che, anziché
ritornare l'indice di un elemento qualora esista nell'array, restituisce
l'ultimo indice iniziale del sottointevallo usato dalla ricerca binaria,
calcolato prima dell'interruzione dell'esecuzione dell'algoritmo. Pur essendo,
di fatto, una funzione $\rank$ 
si è deciso, per comodità di lettura, di chiamare tale funzione
$\ITR$, la quale, come visualizzabile 
all'algoritmo \ref{algo:itr}, ha complessità in tempo:
\begin{equation}
  \label{eq:itrcomp}
  \mathcal{O}(\log (r))
\end{equation}
\begin{algorithm}
  \footnotesize
  \begin{algorithmic}[1]
    \Function{index\_to\_run}{$k,\,\, i$}
    \Comment $k$ indice di colonna, $i$ indice di riga della colonna $k$
    \State \textbf{if} $i\geq p_k[|p_k|-1]$ \textbf{then} \textbf{return}
    $|p_k|-1$ 
    % \If{$i\geq p_k[|p_k|-1]$}
    % \State \textbf{return} $|p_k|-1$
    % \EndIf
    \State $b\gets 0,\,\, e \gets |p_k|$
    \State $run\gets \frac{e-b}{2}$
    \While {$run\neq e \land p_k[run] \neq i$ }
    \If {$i< p_k[run]$ }
    \State $e\gets run$
    \Else
    \If {$run+1=e \lor  p_k[run+1] > i$}
    \State \textbf{break}
    \EndIf
    \State $b\gets run+1$
    \EndIf
    \State $run\gets b+\frac{e-b}{2}$
    \EndWhile
    \State \textbf{return} $run$
    \EndFunction
  \end{algorithmic}
  \caption{Algoritmo per convertire un indice di colonna in indice di run, con
    \texttt{MAP-INT}.} 
  \label{algo:itr}
\end{algorithm}

Ipotizzando di avere
un indice $i\in\{0,\ldots,M-1\}$ è possibile risalire ai valori
$u_k[i]$ e $v_k[i]$, sfruttando l'offset dell'indice rispetto alla
testa della run a cui appartiene. Banalmente, ipotizzando di essere in una run
di simboli $\sigma$ con testa di run all'indice $p$, si avranno, avendo
ottenuto $u_k[p]$ e $v_k[p]$ da $uv_k[p]$:
\begin{equation}
  \small
  \label{eq:naive2}
  \begin{cases}
    v_k[i]=v_k[p]\\
    u_k[i]=u_k[p]+(i-p)
  \end{cases},\mbox{sse } y^k_p[k]=0
  \quad
  \begin{cases}
    u_k[i]=u_k[p]\\
    v_k[i]=v_k[p]+(i-p)
  \end{cases},\mbox{sse } y^k_p[k]=1
\end{equation}
Tenendo eventualmente conto dell'offset $off$, qualora si abbia un simbolo
$\sigma$ uguale a quello della run in analisi, è quindi possibile
riadattare l'algoritmo per il 
mapping visto per la $\PBWT$ di Durbin dalla colonna $k$ alla colonna
$k+1$. Tale soluzione è riportata all'algoritmo \ref{algo:lfoff}. 
Ricordando che si può risalire ai valori $u[p]$ e $v[p]$ in tempo costante,
anche il mapping da una colonna alla successiva avviene in tempo costante.
\begin{algorithm}
  \begin{algorithmic}[1]
    \Function{w}{$k,\,\, i, \,\,\sigma,\,\,o$}
    \Comment $k$ indice di colonna, $i$ indice di riga, $\sigma$ simbolo
    \State $run\gets \ITR(k,i)$
    \If{$\sigma=0\land \GS(start_k, run)=1$}
    \State $off\gets 0$
    \ElsIf {$\sigma=1\land \GS(start_k, run)=0$} 
    \State $off\gets 0$
    \Else
    \State $off\gets i-p_k[run]$
    \EndIf
    \State $(u,v)\gets uvtrick(k,\,\,i)$
    \If{$p_k[i]+off=M$}
    \State \textbf{if} $\GS(start_k, i)=0$ \textbf{then} $v\gets v-1$
    \textbf{else} $u\gets u-1$
    % \If{$get\_symbol(start_k, i)=0$}
    % \State $v\gets v-1$
    % \Else
    % \State $u\gets u-1$
    % \EndIf
    \EndIf
    \State \textbf{if} $\sigma = 0$ \textbf{then} \textbf{return} $u+off$
    \textbf{else} \textbf{return} $c[k]+v+off$
    % \If{$\sigma = 0$}
    % \State \textbf{return} $u+off$
    % \Else
    % \State \textbf{return} $c[k]+v+off$
    % \EndIf
    \EndFunction
  \end{algorithmic}
  \caption{Algoritmo per il mapping con \texttt{MAP-INT}.}
  \label{algo:lfoff}
\end{algorithm}
\subsubsection{Mapping con bitvector}
La seconda variante della componente di mapping sfrutta, al posto degli
intvector compressi, i bitvector sparsi, da cui la nomenclatura
\texttt{MAP-BV}.\\
L'idea è quindi quella di sostituire, data una colonna $k$, quanto necessario a
rappresentare le run e quanto
necessario a permettere il mapping (ovvero i vettori $p_k$ e $uv_k$ della
\texttt{MAP-INT}).\\ 
In primis, per poter localizzare le run nella $k$-esima colonna, si è scelto di
usare un bitvector sparso, che denominiamo per praticità $h_k$, tale
che $|h_k|=M$. Formalmente si ha che:
\begin{equation}
  \label{eq:bv1}
  h_k[i]=
  \begin{cases}
    1&\mbox{se } y^k_{i}[k]\neq y^k_{i+1}[k]\lor i=M-1\\
    0&\mbox{altrimenti}
  \end{cases},\forall i\in \{0,\ldots,M-1\}
\end{equation}
Informalmente, quindi, si ha che si ha $h_k[j]=1$ sse $j$ è l'indice di fine di
una run.\\
La ratio dietro l'uso di questa struttura dati succinta è quella per cui ci si
aspettano poche run all'interno di una colonna della 
matrice $\PBWT$, per quanto già discusso nella sezione
\ref{secpbwt}. Avendo poche run, ci si aspetta do avere anche ``pochi'' 1
all'interno di 
$h_k$, di conseguenza si è optato per usare i bitvector sparsi per la
memorizzazione in memoria di ogni $h_k$. Si ricorda che, secondo quanto
riportato per la libreria \textit{SDSL} \cite{sdsl}, tale variante richiede in
memoria, indicando con $r$ il numero di run:
\begin{equation}
  \label{eq:bv2}
  \approx r\left(2+\log\frac{M}{r}\right)\mbox{ bit}
\end{equation}
Pensando ad una correlazione tra \texttt{MAP-INT} e \texttt{MAP-BV}, si ha che
$\rank_{h_k}$ fa le veci della funzione $\ITR$ mentre
$\select_{h_k}$ equivale ad accedere ai valori di $p_k$.\\
Più elaborata è la rappresentazione dei vettori $u_k$ e $v_k$. In questo caso si
è deciso, a differenza della rappresentazione unica vista con la
\texttt{MAP-INT}, di optare per due bitvector sparsi. In particolare,
per il vettore $u_k$, tale che $|u_k|=c[k]$, si ha che, $\forall
i\in\{0,\ldots,|u_k|-1\}$: 
\begin{equation}
  \label{eq:bv3}
  u_k[i]=
  \begin{cases}
    1&\mbox{se }i \mbox{ è il numero di simboli che contiene la
    }\rank_{u_k}(i)\mbox{-esima run di 0}\\
    0&\mbox{altrimenti}
  \end{cases}
\end{equation}
Analogamente si definisce $v_k$, avendo $|v_k|=M-c[k]$ e $\forall
i\in\{0,\ldots,|v_k|-1\}$, come: 
\begin{equation}
  \label{eq:bv4}
  v_k[i]=
  \begin{cases}
    1&\mbox{se }i \mbox{ è il numero di simboli che contiene la
    }\rank_{v_k}(i)\mbox{-esima run di 1}\\
    0&\mbox{altrimenti}
  \end{cases}
\end{equation}
Si noti che:
\begin{equation}
  \label{eq:bv5}
  \rank_{h_k}(|h_k|-1)+1=(\rank_{u_k}(|u_k|-1)+1)+(\rank_{v_k}(|v_k|-1)+1)
\end{equation}
Ovvero il numero di 1 presenti in $h_k$ è pari alla somma di quelli presenti in
$u_k$ e $v_k$. Si noti che i vari $+1$ sono
dovuti al fatto che la funzione 
$\rank(i)$ esclude dal computo la posizione $i$ stessa e tutti e tre i
bitvector, 
per costruzione, presentano $\sigma=1$ in ultima posizione. Ne segue che, anche
per questi ultimi due bitvector, la scelta di 
usare bitvector sparsi per la loro memorizzazione sia giustificata,
empiricamente, dalla poca quantità attesa di simboli $\sigma=1$.
\begin{esempio}
  \label{es:bv1}
  Sia data la seguente colonna:
  \[y^5=00101111000000000000\]
  Si ha quindi che:
  \[h_5=01110001000000000001\]
  Avendo appunto un numero di run pari a:
  \[\rank_{h_5}(|h_5|-1)+1=4+1=5\]
  In merito alle run composte da simboli $\sigma=0$ si ha che:
  \[u_5=011000000000001\]
  Avendo infatti che si segnalano:
  \begin{itemize}
    \item la prima run composta da due simboli $\sigma=0$
    \item la seconda run composta da un solo simbolo $\sigma=0$
    \item la terza run composta da dodici simboli $\sigma=0$
  \end{itemize}
  Parlando invece di $v_5$ si ha:
  \[v_5=10001\]
  Avendo che:
  \begin{itemize}
    \item la prima run è composta da un solo simbolo $\sigma=1$
    \item la seconda run è composta da quattro $\sigma=1$
  \end{itemize}
  Si conferma, inoltre, quanto detto nell'equazione \ref{eq:bv5}, avendo:
  \[\rank_{h_5}(|h_5|-1)+1=5 =
    (\rank_{u_5}(13)+1)+(\rank_{v_5}(4)+1)=(2+1)+(1+1)=5\] 
\end{esempio}
Lo pseudocodice relativo alla costruzione della componente \texttt{MAP-BV} per
la colonna $k$-esima è disponile all'algoritmo \ref{algo:cosbv}. Anche in questo
caso la costruzione avviene scorrendo la colonna $k$ permutata dal
prefix array $a_k$. \\ 
Assumendo che la complessità in tempo delle costruzioni delle
strutture a supporto per le funzioni $\rank$ e $\select$ dei tre
bitvector sparsi sia limitata superiormente dalla loro lunghezza massima, ovvero
$M$, si ha che la costruzione della componente \texttt{MAP-BV}, per una singola
colonna, avviene in tempo:
\begin{equation}
  \label{eq:bvcos}
  \mathcal{O}(M)
\end{equation}
\begin{algorithm}
  \small
  \begin{algorithmic}[1]
    \Function{build\_map\_bv}{$col,\,\, pref$}
    \Comment $pref=a_k$
    \State $c\gets 0,\,\,u\gets 0,\,\,v\gets 0,\,\,u'\gets 0,\,\, v'\gets
    0,\,\,curr_{lcs}\gets 0$
    \State $start \gets \top,\,\,beg_{run}\gets \top,\,\,push_{zero}\gets
    \bot,\,\,push_{one}\gets \bot$
    \For {\textit{every} $k\in\left[0,\,\, M\right)$}
    \State \textbf{if} $k=0\land col[pref[k]]=1$ \textbf{then} $start \gets
    \bot$  
    % \If{$k=0\land col[pref[k]]=1$}
    % \State $start \gets \bot$
    % \EndIf
    \State \textbf{if} $col[pref[k]]=0$ \textbf{then}  $c\gets c+1$
    % \If{$col[pref[k]]=0$}
    % \State $c\gets c+1$
    % \EndIf
    \EndFor
    \State $runs\gets[0..0]$
    \Comment bitvector sparso per le run, di lunghezza $M+1$
    \State $zeros\gets[0..0]$
    \Comment bitvector sparso per $u_k$, di lunghezza $c[k]$
    \State $ones\gets[0..0]$
    \Comment bitvector sparso per $v_k$, di lunghezza $M-c$
    \State \textbf{if} $start$ \textbf{then} $push_{one}\gets \top$
    \textbf{else} $push_{zero}\gets \top$ 
    % \If{$start$}
    % \State $push_{one}\gets \top$
    % \Else
    % \State $push_{zero}\gets \top$
    % \EndIf
    \For {\textit{every} $k\in\left[0,\,\, M\right)$}
    \If{$beg_{run}$}
    \State $u\gets u',\,\,v\gets v',\,\,beg_{run}\gets \bot$
    \EndIf
    \State \textbf{if} $col[pref[k]]=1$ \textbf{then} $v'\gets v'+1$
    \textbf{else} e $u'\gets u'+1$   
    % \If{$col[pref[k]]=1$}
    % \State $v'\gets v'+1$
    % \Else
    % \State $u'\gets u'+1$
    % \EndIf
    \If{$k=M-1\lor col[pref[k]]\neq col[pref[k+1]]$}
    \State $runs[k]\gets 1$
    \If{$push_{one}$}
    \State \textbf{if} $v\neq 0$ \textbf{then} $ones[k-1]=1$
    % \If{$v\neq 0$}
    % \State $ones[k-1]=1$
    % \EndIf
    \State $swap(push_{zero},\,\,push_{one})$
    \Else
    \State \textbf{if} $u\neq 0$ \textbf{then} $zeros[k-1]=1$
    % \If{$u\neq 0$}
    % \State $zeros[k-1]=1$
    % \EndIf
    \State $swap(push_{zero},\,\,push_{one})$
    \EndIf
    \State $beg_{run}\gets \top$
    \EndIf
    \EndFor
    \State \textbf{if} $|zeros|\neq 0$ \textbf{then} $zeros[|zeros|-1]\gets 1$
    \State \textbf{if} $|ones|\neq 0$ \textbf{then} $ones[|ones|-1]\gets 1$
    
    % \If{$|zeros|\neq 0$}
    % \State $zeros[|zeros|-1]\gets 1$
    % \EndIf
    % \If{$|ones|\neq 0$}
    % \State $ones[|ones|-1]\gets 1$
    % \EndIf
    \State \textit{costruzione delle strutture per $\rank$/$\select$ dei tre
    bitvector} 
    \State \textbf{return}
    $(start,\,\,c,\,\,runs,\,\,zeros,\,\,ones)$  
    \EndFunction
  \end{algorithmic}
  \caption{\footnotesize{Algoritmo per la costruzione della componente
  \texttt{MAP-BV} per la colonna $k$.}}
  \label{algo:cosbv}
\end{algorithm}
% Si analizza ora il calcolo della funzione $w_(i,\sigma)$.\\
% Una formulazione della stessa, usando i bitvector sparsi, sarebbe:
% \begin{equation}
%   \label{eq:wtravis}
%   w_k(i,\sigma)=y^k_{\select_{b_k}(\rank_b^k(i-1)+1)}[k]
%   =y^k_{\select_{b_k}(\rank_b^k(i))}[k]
% \end{equation}
% Avendo:
% \begin{equation}
%   \label{eq:wtravis2}
%   \rank_{b_k}=
%   \begin{cases}
%     \rank_{u_k}&\mbox{se }y_i^k[k]=\sigma=1\\
%     \rank_{v_k}&\mbox{se }y_i^k[k]=\sigma=0\\
%   \end{cases}\quad
%   \select_{b_k}=
%   \begin{cases}
%     \select_{u_k}&\mbox{se }y_i^k[k]=\sigma=1\\
%     \select_{v_k}&\mbox{se }y_i^k[k]=\sigma=0\\
%   \end{cases}
% \end{equation}
\dc{Commentato nel \TeX $\,$ c'è la formula di Travis, in caso da adattare sui
  significati qui di u e v}
Bisogna, in primis, spiegare come, dato un indice di aplotipo
$i\in\{0,\ldots,M-1\}$ e una 
colonna $k$, estrarre $u'_k[i]$ e $v'_k[i]$, ovvero come se si stesse usando la
$\PBWT$, a partire dagli attuali $u_k[i]$ e $v_k[i]$ (i due bitvector
sparsi). A differenza della componente \texttt{MAP-INT}, risulta difficile
formulare un'unica formula per il calcolo di tali valori. Si è quindi deciso di
offrire una spiegazione operazionale.\\
In primis, 
se $i=0$, si ha che $u'_k[0]=v'_k[0]=0$. In caso contrario bisogna 
calcolare la run 
in cui si trova l'indice $i$. Questo si ottiene direttamente sfruttando $h_k$:
\begin{equation}
  \label{eq:bv7}
  run = \rank_{h_k}(i)
\end{equation}
Una volta calcolato l'indice di run si hanno tre possibilità:
\begin{enumerate}
  \item si ha che $run=0$ e una run di simboli $\sigma=b$, con $b\in\{0,1\}$. 
  In tal caso, semplicemente:
  \begin{equation}
    \label{eq:bv8}
    (u,v)=
    \begin{cases}
      (i,0)&\mbox{se } b=0\\
      (0,i)&\mbox{altrimenti}
    \end{cases}
  \end{equation}
  \item si ha che $run=1$ e una run di simboli $\sigma=b$, con $b\in\{0,1\}$. In
  tal caso, bisogna individuare l'indice di inizio della seconda
  run, sfruttando $h_k$:
  \begin{equation}
    \label{eq:bv9}
    beg = \select_{h_k}(1)+1
  \end{equation}
  A questo punto si ha il numero di simboli della prima run, e,
  calcolando la distanza tra l'indice di riga e quello di inizio della prima
  run, si ottiene che:
  \begin{equation}
    \label{eq:bv10}
    (u,v)=
    \begin{cases}
      (beg,i-beg)&\mbox{se } b=0\\
      (i-beg,beg)&\mbox{altrimenti}
    \end{cases}
  \end{equation}
  \item si ha che $run=j$, con $j\in\{2,r-1\}$. Anche in questo caso si procede
  calcolando l'indice di inizio della run:
  \begin{equation}
    \label{eq:bv11}
    beg = \select_{h_k}(run)+1
  \end{equation}
  e l'offset rispetto all'indice $i$ dato da:
  \begin{equation}
    \label{eq:bv12}
    offset = i-beg
  \end{equation}
  Poi, sfruttando la solita dicotomia fornita dal caso binario in studio, si
  hanno due casi: 
  \begin{enumerate}
    \item si è in una run di indice pari e si sfruttano poi $u_k$ e $v_k$ per
    sapere l'indice della precedente run con 
    simboli $\sigma=0$:
    \begin{equation}
      \label{eq:bv13}
      pre_u=\select_{u_k}\left(\left\lfloor\frac{run}{2}\right\rfloor\right)+1
    \end{equation}
    Analogamente si calcola l'indice della run con simboli simboli $\sigma=1$:
    \begin{equation}
      \label{eq:bv14}
      pre_v=\select_{v_k}\left(\left\lfloor\frac{run}{2}\right\rfloor\right)+1
    \end{equation}
    Si noti che si usa $\frac{run}{2}$ in quanto, essendo in una run di indice
    pari si hanno precedentemente lo stesso numero di run per $\sigma=0$ e per
    $\sigma=1$ e quindi si considera lo stesso numero di run nei due
    bitvector sparsi $u_k$ e $v_k$.\\
    A questo punto, sempre per il ragionamento per cui solo uno tra $u$ e $v$
    non è costante all'interno di una run si ha che o $pre_u$ o $pre_v$ è
    costante mentre l'altro valore deve essere calcolato considerando l'offset:
    \begin{equation}
      \label{eq:bv15}
      (u,v)=
      \begin{cases}
        (pre_u+offset,pre_v)&\mbox{se } b=0\\
        (pre_u,pre_v+offset)&\mbox{altrimenti}
      \end{cases}
    \end{equation}
    \item ci si trova in una run di indice dispari, quindi non si hanno
    precedentemente lo stesso numero di run per i due simboli. Bisogna quindi
    calcolare quante siano tali run. Se la prima run è di zeri si ha che:
    \begin{equation}
      \label{eq:bv16}
      run_u=\select_{u_k}\left(\left\lfloor\frac{run}{2}\right\rfloor\right)+1
    \end{equation}
    \begin{equation}
      \label{eq:bv17}
      run_v=\select_{v_k}\left(\left\lfloor\frac{run}{2}\right\rfloor\right)
    \end{equation}
    Mentre se la prima run non è di zeri si devono invertire i due valori. Si
    sa 
    quindi quali run considerare sui due bitvector sparsi $u_k$ e
    $v_k$.\\ 
    Posso quindi procedere come nel caso precedente, avendo:
    \begin{equation}
      \label{eq:bv18}
      pre_u=\select_{u_k}(run_u)+1
    \end{equation}
    \begin{equation}
      \label{eq:bv19}
      pre_v=\select_{v_k}(run_v)+1
    \end{equation}
    Infine, si calcola:
    \begin{equation}
      \label{eq:bv20}
      (u,v)=
      \begin{cases}
        (pre_u,pre_v+offset)&\mbox{se } b=0\\
        (pre_u+offset,pre_v)&\mbox{altrimenti}
      \end{cases}
    \end{equation}
  \end{enumerate}
\end{enumerate}
\dc{Sistemare}
\begin{esempio}
  Si prendano i dati e i risultati ottenuti all'esempio \ref{es:bv1}. Si
  vogliono calcolare $u[i]$ e $v[i]$ per $i=6$.\\
  In primis si hanno:
  \[run=\rank_{h_5}(6)=3\]
  \[beg = \select_{h_5}(3)+1=3+1=4\]
  \[offset = i-beg=6-4=2\]
  Quindi ci si trova nel terzo caso e, nel dettaglio, avendo una run di
  indice dispari. Si calcolano quindi:
  \[run_u=\select_{u_5}\left(\left\lfloor\frac{3}{2}\right\rfloor\right)+1=
    \select_{u_5}(1)+1 =1+1=2\] 
  \[run_v=\select_{v_5}\left(\left\lfloor\frac{3}{2}\right\rfloor\right)=
    \select_{v_5}(1)=0\] 
  Tali valori non andranno invertiti avendo $start^5=\top$.\\
  Si calcolano quindi:
  \[pre_u=\select_{u_5}(2)+1=2+1=3\]
  \[pre_v=\select_{v_5}(0)+1=0+1=1\]
  Avendo infatti, in totale, tre simboli $\sigma=0$ e un simbolo $\sigma=1$
  prima dell'indice 6.\\ 
  Concludendo, avendo $start^5=\top$:
  \[(u,v)=(pre_u, pre_v + offset)=(3,1+2)=(3,3)\]
\end{esempio}
Lo pseudocodice per il calcolo di $u_k[i]$ e $v_k[i]$ è disponibile
all'algoritmo \ref{algo:uvbv}. In merito alla 
complessità in tempo, si ha che essa è
limitata superiormente dal costo della funzione $\rank$ su
bitvector sparsi, essendo la funzione $\select$ disponibile in
tempo costante. Ne segue che, avendo $r$ run nella colonna $k$, si ha un tempo
proporzionale a:
\begin{equation}
  \label{eq:bvuvtime}
  \mathcal{O}\left(\log\frac{M}{r}\right)
\end{equation}
Non dovendo considerare esplicitamente l'offset, come nel caso della
\texttt{MAP-INT}, 
il mapping dalla colonna $k$ alla 
colonna $k+1$ viene fatto come nel caso della
$\PBWT$, come visualizzabile all'algoritmo \ref{algo:lfr}, che presenta
quindi la medesima complessità del calcolo di $u[i]$ e $v[i]$, ovvero quello
visto all'equazione \ref{eq:bvuvtime}.
\begin{algorithm}
  %\small
  \begin{algorithmic}[1]
    \Function{uvtrick}{$k,\,\, i$}
    \Comment $k$ indice di colonna, $i$ indice di riga
    \State \textbf{if} $i=0$ \textbf{then}  \textbf{return} $(0,\,\,0)$
    % \If{$i = 0$}
    % \State \textbf{return} $(0,\,\,0)$
    % \EndIf
    \State $run \gets rank_h^{k}(i)$
    \If{$run=0$}
    \State \textbf{if} $start_k$ \textbf{then} \textbf{return} $(i,\,\, 0)$
    \textbf{else} \textbf{return} $(0, \,\,i)$
    % \If{$start_k$}
    % \State \textbf{return} $(i,\,\, 0)$
    % \Else
    % \State \textbf{return} $(0, \,\,i)$
    % \EndIf
    \ElsIf{$run=1$}
    \If{$start_k$}
    \State \textbf{return} $(select_h^{k}(run)+1,\,\, i-(select_h^{k}(run)+1))$
    \Else
    \State \textbf{return} $(i-(select_h^{k}(run)+1),\,\, select_h^{k}(run)+1)$
    \EndIf
    \Else
    \If{$run\bmod 2 = 0$}
    \State $pre_u\gets
    select_u^{k}\left(\left\lfloor\frac{run}{2}\right\rfloor\right)+1$ 
    \State $pre_v\gets
    select_v^{k}\left(\left\lfloor\frac{run}{2}\right\rfloor\right)+1$ 
    \State $offset \gets i -(select_h^{k}(run)+1)$
    \If{$start_k$}
    \State \textbf{return} $(pre_u+offset,\,\, pre_v)$
    \Else
    \State \textbf{return} $(pre_u, \,\,pre_v+offset)$
    \EndIf
    \Else
    \State $run_u\gets \left(\left\lfloor\frac{run}{2}\right\rfloor\right)+1$
    \State $run_v\gets \left\lfloor\frac{run}{2}\right\rfloor$
    \State \textbf{if} $\neg start_k$ \textbf{then} $swap(run_u, run_v)$
    % \If{$\neg start_k$}
    % \State $swap(run_u, run_v)$
    % \EndIf
    \State $pre_u\gets select_u^{k}(run_u)+1$
    \State $pre_v\gets select_v^{k}(run_v)+1$
    \State $offset \gets i -(select_h^{k}(run)+1)$
    \If{$start_k$}
    \State \textbf{return} $(pre_u, \,\,pre_v+offset)$
    \Else
    \State \textbf{return} $(pre_u+offset, \,\,pre_v)$
    \EndIf
    \EndIf
    \EndIf
    \EndFunction
  \end{algorithmic}
  \caption{Algoritmo per uvtrick con \texttt{MAP-BV}.}
  \label{algo:uvbv}
\end{algorithm}
\begin{algorithm}
  \begin{algorithmic}[1]
    \Function{w}{$k,\,\, i, \,\,\sigma$}
    \Comment $k$ indice di colonna, $i$ indice di riga, $\sigma$ simbolo
    \State $c\gets c[k]$
    \State $(u, v) \gets uvtrick(k,\,\,i)$
    \State \textbf{if} $\sigma = 0$ \textbf{then} \textbf{return} $u$
    \textbf{else}  \textbf{return} $c+v$
    % \If{$\sigma = 0$}
    % \State \textbf{return} $u$
    % \Else
    % \State \textbf{return} $c+v$
    % \EndIf
    \EndFunction
  \end{algorithmic}
  \caption{Algoritmo per il mapping con \texttt{MAP-BV}.}
  \label{algo:lfr}
\end{algorithm}
Facendo una prima stima della memoria occupata da questa componente si ha che,
avendo $\rho$ numero medio di run per colonna, essa è:
\begin{equation}
  \label{eq:mapintmem}
  \approx N\left(\frac{\rho\left(2+\log\frac{M}{\rho}\right)}{4}+21\right)\mbox{ byte} 
\end{equation}
\dc{Ricontrollare conto}
È possibile fare ora un piccolo confronto, ad alto livello, tra la memoria
richiesta da un bitvector sparso e un intvector compresso.\\
Si noti che confrontare analiticamente le stime in memoria delle due componenti
possibili per il mapping, non è banale, valutando l'equazione
(relativo ad un solo bitvector sparso, comprendente le strutture per
le funzioni $\rank$ e $\select$, e ad un solo intvector
compresso): 
\begin{equation}
  \label{eq:mapbvintmem}
  \rho\left(2+\log\frac{M}{\rho}\right)+128=\rho\lceil\log(M-1)\rceil+1
\end{equation}
\dc{Ricontrollare conto}
Solo valutazioni sperimentali hanno mostrato come, con valori di $M$ e $\rho$
relativi ai pannelli di aplotipi studiati, l'uso degli intvector compressi sia
più vantaggioso in termini di memoria richiesta. Una stima derivata da tali
conti è visibile in figura \ref{fig:bvvsint}, dove si evince che, qualora la
media di run resti proporzionale a quanto visto con 4.908 sample
(ovvero una media di 12 run come verrà
analizzato nel Capitolo \ref{reschap}), l'uso dei
bitvector sparso diventerebbe favorevole solo con pannelli con più di circa
17.530 sample. 
\begin{figure}
  \centering
  \includegraphics[scale = 0.7]{img/bv_vs_iv.png}
  \caption{Confronto tra la stima di memoria in bit necessaria ai bitvector
    sparsi e agli 
    intvector compressi, al variare dell'altezza del pannello. Con la linea
    verde si segnala il numero di sample dopo il quale si avrebbe un vantaggio
    in memoria nell'usare i bitvector sparsi.}
  \label{fig:bvvsint}
\end{figure}
\dc{Serve altro?}
\section{Componente per le threshold}
Come discusso per \textit{MONI}, l'uso delle \textbf{threshold} è parte
fondamentale di uno dei due modi per ottenere le \textit{matching
  statistics}. \\
\begin{definizione}
  Data la colonna $k$-esima della \textbf{matrice PBWT}, $y^k$, memorizzata
  tramite compressione \textbf{run-length} e data la run $j$-esima, indicizzata
  da $i$ a $i'$, si definisce \textbf{threshold} come l'indice del minimo valore
  \textit{LCP}, che ricordiamo essere calcolato sull'ordinamento inverso,
  compreso negli indici della run, compreso l'eventuale 
  $LCP_k[i'+1]$, qualora $i'\neq M-1$. Si noti che quest'ultimo valore, se
  esistente, deve essere considerato in quanto per il suo calcolo, come
  specificato nei preliminari alla sezione \ref{secpbwt}, si prende in
  considerazione $y^k_{i'}$ e $y^k_{i'+1}$.
\end{definizione}
% Con tale informazione, unita ai \textit{prefix array sample}, si può quindi
% ottenere un comportamento analogo a quanto si ottiene con l'\textbf{R-index} per
% la \textit{RLBWT}.\\
% Sia infatti data $t$ la posizione della \textit{threshold} nella run corrente,
% in colonna $k$, e
% si supponga che tale run, con testa all'indice $h$, non sia associata al simbolo
% desiderato, ovvero $z[k]$. Si supponga che, con il mapping, si sia arrivati
% all'indice $i$ della colonna $k$. Si supponga inoltre che la run successiva
% abbia testa in indice $e$. Si hanno due casi possibili, denotando con
% $LCS(x,y)$ il \textit{longest common suffix} tra le stringhe $X$ e $Y$ e con
% $a_k$ il \textit{prefix array} in colonna $K$:
% \begin{enumerate}
%   \item $i<t$ allora, per definizione di \textit{threshold}:
%   \[LCS(z[0,k], x_{a_{k}[h-1]}[0,k])\geq LCS(z[0,k], x_{a_{k}[e]}[0,k])\]
%   Quindi si ha che $MS[k].row=a_{k}[h-1]$ e il mapping potrà proseguire
%   dall'indice $h-1$
%   \item  $i\geq t$ allora, per definizione di \textit{threshold}:
%   \[LCS(z[0,k], x_{a_{k}[h-1]}[0,k])\leq LCS(z[0,k], x_{a_{k}[e]}[0,k])\]
%   Quindi si ha che $MS[k].row=a_{k}[e]$ e il mapping potrà proseguire
%   dall'indice $e$
% \end{enumerate}
% Qualora una colonna presenti solo simboli $\sigma\neq z[k]$, per convenzione, si
% imposta che $MS[k].row = M$ e si ricomincia, in colonna $k+1$, dall'ultima
% posizione, indicizzata nel pannello originale dal valore finale del
% \textit{prefix array sample} dell'ultima run.\\
Da un punto di vista implementativo, come anticipato, si hanno due soluzioni,
una basata su \textit{intvector} e una basata su \textit{bitvector sparsi}. In
entrambi i casi il calcolo si può effettuare in parallelo a quello di
\texttt{MAP-INT} e \texttt{MAP-BV}.
\subsection{Threshold con intvector}
In questo caso la memorizzazione delle threshold avviene in modo molto semplice,
usando un \textit{vettore di interi bit-compressed}. Data una colonna $k$ della
\textit{PBWT matrix}, con $r$ numero di run, si calcola $t_k$ tale che
$t_k[i]=j$ sse $j$ è l'indice della threshold dell'$i$-esima run.\\
Lo pseudocodice per la costruzione della componente
\texttt{THR-INT} della colonna $k$ è consultabile all'algoritmo
\ref{algo:thrint} 
e, dovendo scorrere la colonna permutata dal prefix array $a_k$ e dovendo
accedere ai valori di $l_k$, tale operazione ha complessità in tempo
proporzionale a:
\begin{equation}
  \label{eq:thrint}
  \mathcal{O}(M)
\end{equation}
Si noti che, qualora minimo LCP si trovi nella testa della run successiva (da
considerare in quanto calcolato anche grazie all'ultimo elemento della run
corrente), si può tranquillamente memorizzare l'indice della testa della run
successiva come \textit{threshold}.
\begin{algorithm}
  \begin{algorithmic}
    \Function{Build\_thr\_int}{$col,\,\,pref,\,\,div$}
    \Comment $pref=a_k,\,\,div=l_k$
    \State $curr_{lcs}\gets 0,\,\,tmp_{thr}\gets 0$
    \State $t\gets[]$
    \For {\textit{every} $k\in\left[0,\,\, M\right)$}
    \If{$k=0\lor col[pref[k]]\neq col[pref[k-1]]$}
    \State $curr_{lcs}\gets div[k],\,\,tmp_{thr}\gets k$
    \EndIf
    \If{$div[k]<curr_{lcs}$}
    \State $curr_{lcs}\gets div[k],\,\,tmp_{thr}\gets k$
    \EndIf
    \If{$k=M-1\lor col[pref[k]]\neq col[pref[k+1]]$}
    \If{$k\neq M-1\land div[k+1]<div[tmp_{thr}]$}
    \State $push(t, k+1)$
    \Else 
    \State $push(t, tmp_{thr})$
    \EndIf
    \EndIf
    \EndFor
    \State \textbf{return} $t$  
    \EndFunction
  \end{algorithmic}
  \caption{Algoritmo per la costruzione della componente \texttt{THR-INT}.}
  \label{algo:thrint}
\end{algorithm}
\subsection{Threshold con bitvector}
In questo caso le posizioni delle \textit{threshold} vengono
memorizzate tramite un \textit{bitvector sparso} per ogni colonna $k$, denotato
$thr_k$, avendo che $thr_k[i]=1$ sse $i$ è l'indice di una \textit{threshold}.
Qualora il minimo \textit{LCP} si ritrovi nell'indice della testa della run
successiva, la posizione della threshold verrà comunque memorizzata all'indice
della coda della run corrente. Purtroppo questa è una situazione di ambiguità,
avendo che, seguendo la definizione sopra, avendo la \textit{threshold} a fine
run, bisognerebbe scegliere la testa della run successiva, qualora l'indice $i$
si trovi esattamente a fine run. Invece, qualora la
\textit{threshold} sia a fine run a causa del fatto che il minimo \textit{LCP}
si trovi nella testa della run successiva, bisogna scegliere la coda della run
precedente. L'unico modo per disambiguare è quindi effettuare \textit{random
  access} al pannello per vedere quale sia la
soluzione migliore, ovvero quale tra la coda della run precedente e la testa
della run successiva siano relative alla riga del pannello originale con un
suffisso comune alla query più lungo.\\
Purtroppo non è possibile salvare la threshold direttamente nella testa della
run successiva in quanto questa potrebbe essere anche la posizione della
threshold della run successiva e avere due threshold sovrapposte impedirebbe di
capire a quale run appartiene una certa threshold, tramite la funzione
\textit{rank}. \\
Tale bitvector deve essere quindi aggiunto alle informazioni memorizzate per
ogni singola colonna. Lo pseudocodice per la costruzione della componente
\texttt{THR-BV} della colonna $k$ è consultabile all'algoritmo \ref{algo:thrbv}
e, dovendo scorrere la colonna permutata dal prefix array $a_k$ e dovendo
accedere ai valori di $l_k$, tale operazione ha complessità in tempo
proporzionale a:
\begin{equation}
  \label{eq:thrbv}
  \mathcal{O}(M)
\end{equation}
\begin{algorithm}
  \begin{algorithmic}
    \Function{Build\_thr\_bv}{$col,\,\,pref,\,\,div$}
    \Comment $pref=a_k,\,\,div=l_k$
    \State $curr_{lcs}\gets 0,\,\,tmp_{thr}\gets 0$
    \State $thrs\gets[0..0]$
    \Comment bitvector sparso di lunghezza $M$
    \For {\textit{every} $k\in\left[0,\,\, M\right)$}
    \If{$k=0\lor col[pref[k]]\neq col[pref[k-1]]$}
    \State $curr_{lcs}\gets div[k],\,\,tmp_{thr}\gets k$
    \EndIf
    \If{$div[k]<curr_{lcs}$}
    \State $curr_{lcs}\gets div[k],\,\,tmp_{thr}\gets k$
    \EndIf
    \If{$k=M-1\lor col[pref[k]]\neq col[pref[k+1]]$}
    \If{$k\neq M-1\land div[k+1]<div[tmp_{thr}]$}
    \State $thrs[k]\gets 1$
    \Else
    \State $thrs[tmp_{thr}]\gets 1$
    \EndIf
    \EndIf
    \EndFor
    \State \textit{costruzione delle strutture rank/select per thr}
    \State \textbf{return} $thr$  
    \EndFunction
  \end{algorithmic}
  \caption{Algoritmo per la costruzione della componente \texttt{THR-BV}.}
  \label{algo:thrbv}
\end{algorithm}
\subsection{Componente per i prefix array sample}
Come introdotto parlando delle matching statistics, qualora si abbia un
cambio di riga da memorizzare, si seleziona sempre quella relativa alla coda
della run precedente o quella relativa alla testa della run successiva. Risulta
quindi necessario, in colonna $k$, memorizzare i valori di $a_k$ all'inizio e
alla fine di ogni run. Anche in questo caso si sono scelti gli intvector
compressi. Tali valori sono un sample dei valori che permettono 
le permutazioni che costruiscono la matrice $\PBWT$ e, quindi, tale
componente prende il nome di \texttt{PERM}.\\
All'algoritmo \ref{algo:buildperm} è possibile analizzare lo pseudocodice del
metodo usato per calcolare tale componente per la colonna
$k$-esima. L'algoritmo, dovendo iterare l'intera colonna della \textit{matrice
  PBWT} ha costo, in tempo:
\begin{equation}
  \label{eq:timeperm}
  \mathcal{O}(M)
\end{equation}
La costruzione può essere fatta in contemporanea a quelle delle componenti già
descritte, ovvero: 
\texttt{MAP-INT}/\texttt{MAP-BV} e \texttt{THR-INT}/\texttt{THR-BV}.\\
\noindent
In termini di memoria necessaria per questa componente, si hanno le medesime
considerazioni fatte nel caso della componente \texttt{MAP-INT}, per l'uso degli
intvector compressi.
\begin{algorithm}
  \small
  \begin{algorithmic}[1]
    \Function{build\_perm}{$col,\,\, pref$}
    \Comment $pref = a_k$
    \State $tmp_{beg}\gets 0,\,\,beg_{run}\gets \top$
    \State $samples_{beg} \gets []$
    \Comment vettore per i prefix array sample ad inizio di ogni run
    \State $samples_{end}\gets []$
    \Comment vettore per i prefix array sample a fine di ogni run
    \For {\textit{every} $k\in\left[0,\,\, height\right)$}
    \If{$beg_{run}$}
    \State $tmp_{beg}\gets pref[k]$
    \State $beg_{run}\gets \bot$
    \EndIf
    \If{$k=height-1\lor col[pref[k]]\neq col[pref[k+1]]$}  
    \State $push(samples_{beg}, tmp_{beg})$
    \State $push(samples_{end}, pref[k])$
    \State $beg_{run}\gets \top$
    \EndIf
    \EndFor
    \State \textbf{return} $(samples_{beg},\,\, samples_{end})$  
    \EndFunction
  \end{algorithmic}
  \caption{{\footnotesize{Algoritmo per la costruzione della componente
  \texttt{PERM} per la colonna $k$.}}}
  \label{algo:buildperm}
\end{algorithm}
\section{Componenti per il random access e le LCE query}
Le ultime componenti da descrivere sono quelle atte a garantire il
\textit{random access} 
al testo e, nel caso dell'uso degli \textit{SLP}, permettere il computo delle
\textit{LCE query}.\\
Parlando di strutture per il random access una differenza sostanziale tra
l'uso di un \textit{vettore di bitvector}, \texttt{RA-BV}, e quello
dell'\textit{SLP}, \texttt{RA-SLP}, è data dai tempi di accesso ai singoli
elementi. Infatti, parlando di \texttt{RA-BV}, si ha accesso in tempo costante
ad un qualsiasi elemento del pannello mentre, nel caso di \texttt{RA-SLP},
denotando con $s$ la lunghezza della parola generata dall'\textit{SLP}, si ha
che l'accesso ad ogni elemento è in tempo:
\begin{equation}
  \label{eq:timera}
  \mathcal{O}(\log s)
\end{equation}
La seconda differenza, già ampiamente introdotta e di fatto scontata, è data
dalla dimensione delle due strutture dati, avendo che \texttt{RA-BV} memorizza
$\sim NM$ bit, dove il $\sim$ è dato dai costi in memoria aggiuntivi dati
dall'avere un vettore che memorizza i bitvector. Parlando invece di
\texttt{RA-SLP} non si può avere una stima teorica dello spazio necessario ma,
come si vedrà nel capitolo \ref{reschap}, i risultati quantitativi daranno prova
della capacità di compressione degli \textit{SLP}.\\
Parlando della componente \texttt{LCE} bisogna solo descrivere il modo con cui
si ottiene la singola stringa che verrà compressa tramite \textit{SLP}.
In primis, le
librerie per la costruzione di tale struttura assumono un input
``monodimensionale'', ovvero una singola sequenze lineare. Inoltre, anche per
permettere la costruzione efficiente della \textit{PBWT}, e conseguentemente
della \textit{RLPBWT}, il pannello in input risulta essere trasposto, avendo che
le righe nel file in input rappresentano i siti e non gli individui. Bisogna
quindi in primis trasporre tale pannello. Per procedere ulteriormente bisogna
però ricordare che sull'\textit{SLP} si avrà 
necessità di effettuare \textit{LCE query} che però, si anticipa, nel nostro
pannello, devono essere fatte tra due righe da destra a sinistra (a differenza
di quanto visto nel caso standard dove si confrontavano prefissi comuni). Per
rendere possibile questa operazione quindi il pannello deve essere sia 
salvato come un'unica riga, per ottenerne l'\textit{SLP}, che ``da destra a
sinistra'', per permettere le \textit{LCE query}. Si procede quindi concatenando
ogni riga, selezionandole consecutivamente e leggendone i singoli elementi da
destra a sinistra.
\begin{esempio}
  Si vede quindi un breve esempio.\\
  Si assuma di avere il seguente pannello nel file in input.
  \[
    X=\left[
      \begin{matrix}
        0 & 0 & 1 & 0 & 0\\
        1 & 1 & 1 & 0 & 1\\
        0 & 1 & 1 & 1 & 1\\
        0 & 0 & 0 & 1 & 0
      \end{matrix}
    \right]
  \]
  Dove però come detto le righe sono i siti e le colonne i sample. Per ottenere
  l'\textit{SLP} biosgna quindi, in primis, trasporre la matrice:
  \[
    X^T=\left[
      \begin{matrix}
        0 & 1 & 0 & 0\\
        0 & 1 & 1 & 0\\
        1 & 1 & 1 & 0\\
        0 & 0 & 1 & 1\\
        0 & 1 & 1 & 0
      \end{matrix}
    \right]
  \]
  A questo punto bisogna considerare l'ordine in cui si vorranno effettuare le
  \textit{LCE query}.
  Ad esempio, prendendo la seconda e la terza riga, facendo partire il confronto
  dall'ultima colonna, avremmo una \textit{LCE
    query} lunga 3, terminante nella prima colonna esclusa:
  \begin{figure}[H]
    \centering
    \includegraphics[scale = 0.38]{img/slppanel.pdf}
  \end{figure}
  Si procede quindi salvano la sequenza lineare relativa al pannelli come
  descritto sopra,
  ottenendo, con colorate gli stessi risultati della query fatta
  sopra:
  \[0010\,\,{\color{nordgreen}011}{\color{nordred}0}\,\,
    {\color{nordgreen}011}{\color{nordred}1} \,\,1100\,\,0110\]
  \textit{Si noti che qui si sono segnalate le varie righe con uno spazio ma
    solo per praticità ``visiva''.}
\end{esempio}
In termini di complessità, si ricorda che, come per il \textit{random access},
per il calcolo delle \textit{LCE query} con \textit{SLP} si ha un tempo
proporzionale a: 
\begin{equation}
  \label{eq:timelce}
  \mathcal{O}(\log s)
\end{equation}
\subsection{Componente per la struttura Phi}
\label{secphi}
L'ottenimento dell'array matching statistics permette di sapere solo
l'indice di una della righe del pannello per le quali si ha uno $\SMEM$ con
l'aplotipo query. Analogamente a quanto discusso in PHONI \cite{phoni},
anche per la $\RLPBWT$ si è pensato a due funzioni, $\varphi$ e
$\varphi^{-1}$, per il riconoscimento di tutte le
righe del pannello per le quali si ha il medesimo $\SMEM$. La componente che
permette il calcolo 
di tali funzioni è la componente denotata \texttt{PHI} e si dovrà considerare
l'assenza in memoria dei valori degli array $\RLCP$.\\
L'intuizione alla base del ragionamento è molto semplice. Nell'ordinamento alla
colonna $k$-esima, dato da $a_k$, tutte le righe, per le quali si ha un certo
$\SMEM$, sono poste consecutivamente, a causa dell'ordinamento
lessicografico inverso.
\begin{definizione}
  Dati un pannello $X$, di dimensioni $N\times M$, e una colonna $k$, avendo
  prefix array $a_k$ e permutazione inversa del prefix array $\alpha_k$, si
  definiscono formalmente: 
  \[\varphi_k(p)=
    \begin{cases}
      \NULL&\mbox{se }\alpha_k[p]=0\\
      a_k[\alpha_k[p]-1]&\mbox{altrimenti}
    \end{cases},\forall p\in\{0,M-1\}
  \]
  \[\varphi^{-1}_k(p)=
    \begin{cases}
      \NULL&\mbox{se }\alpha_k[p]=M-1\\
      a_k[\alpha_k[p]+1]&\mbox{altrimenti}
    \end{cases},\forall p\in\{0,M-1\}
  \]
  In altri termini, si ha che:
  \[\varphi_k(a_k[j])=
    \begin{cases}
     \NULL&\mbox{se }j=0\\
      a_k[j-1]&\mbox{altrimenti}
    \end{cases},\forall j\in\{0,M-1\}
  \]
  \[\varphi^{-1}_k(a_k[j])=
    \begin{cases}
      \NULL&\mbox{se }j=M-1\\
      a_k[j+1]&\mbox{altrimenti}
    \end{cases},\forall j\in\{0,M-1\}
  \]
  Quindi, dato un elemento di $a_k$, le due funzioni restituiscono il valore
  antecedente ad esso, se esistente, e il valore successivo ad esso, se
  esistente, nel prefix array. In caso di inesistenza di tale valore,
  rispettivamente ad inizio
  e fine del prefix array, si
  restituisce $\NULL$.
\end{definizione}
\dc{VERIFICARE DEFINIZIONE IN QUANTO ``NUOVA''}
\begin{esempio}
  Si ipotizzi di avere, come per l'esempio \ref{es:pbwt1}:
  \[a_6=[14,15,0,9,10,16,8,11,12,13,18,19,1,2,3,17,4,5,6,7]\]
  \[\alpha_6=[2,12,13,14,16,17,18,19,6,3,4,7,8,9,0,1,5,15,10,11]\]
  Si fissa quindi $p=3$ e si ottengono:
  \[\varphi_6(3)=a_6[\alpha_6[3]-1]=a_6[14-1]=a_6[13]=2\]
  \[\varphi^{-1}_6(3)=a_6[\alpha_6[3]+1]=a_6[14+1]=a_6[15]=17\]
\end{esempio}
Avendo, quindi, $\MS[i].\row=p$ e $\MS[i].\len=l$ basta iterare le righe a
partire da 
$p$ in $a_i$, righe che denotiamo con l'indice $q$, fino a che si ha 
$\LCE_k(x_p, x_q)\geq l$. Ovviamente bisogna iterare in entrambe le
direzioni. Tutte le righe $x_q$ che soddisfano tale condizione presentano uno
$\SMEM$ di lunghezza $l$ con 
l'aplotipo query. L'algoritmo \ref{algo:phiext} rappresenta quanto
appena descritto, avendo che la funzione $\lceb$ limita il calcolo
della 
$\LCE$ alla lunghezza desiderata $l$, escludendo computazioni inutili oltre tale
lunghezza. Tale funzione è riproducile, per mezzo di iterazioni, anche sulla
componente \texttt{RA-BV}. La complessità temporale di questo algoritmo
varia a seconda della componente per il random access (e della conseguente
presenza della 
componente \texttt{LCE}). Inoltre, è difficile poter dare una stima asintotica
in quanto varia sul numero di righe $\nu$ che presentano un certo
$\SMEM$. Quindi, si ha, con la 
componente \texttt{RA-BV}, un tempo proporzionale a:
\begin{equation}
  \label{eq:phiaccbv}
  \mathcal{O}(\nu N)
\end{equation}
Mentre con l'uso della componente \texttt{LCE}, avendo il pannello in memoria
sotto forma di $\SLP$, si ha complessità in tempo:
\begin{equation}
  \label{eq:phiaccbv2}
  \mathcal{O}(\nu\log (NM))
\end{equation}
Entrambe le stime assumono, solo per il momento, che sia possibile calcolare il
valore delle funzioni $\varphi$ e $\varphi^{-1}$ in tempo $\mathcal{O}(1)$,
avendo in 
memoria l'insieme dei prefix array (e l'insieme delle permutazioni inverse dei
prefix array) con random access in tempo costante.\\
\dc{Stime molto per eccesso, con RA-BV si scala su $l$ di fatto}
\begin{algorithm}
  \small
  \begin{algorithmic}[1]
    \Function{extend\_matches}{$k, row, len$}
    \State $haplos\gets []$
    \State $check_{down}\gets \top,\,\,check_{up}\gets \top$
    \While {$check_{down}$}
    \State $down_{row}\gets \varphi^{-1}(row, k)$
    \If{$\lceb(k, row, down_{row}, len)$}
    \State $push(haplos, down_{row})$
    \State $row \gets down_{row}$
    \Else
    \State $check_{down}\gets \bot$
    \EndIf
    \EndWhile
    \While {$up_{down}$}
    \State $up_{row}\gets \varphi(row, k)$
    \If{$\lceb(k, row, up_{row}, len)$}
    \State $push(haplos, up_{row})$
    \State $row \gets up_{row}$
    \Else
    \State $check_{up}\gets \bot$
    \EndIf
    \EndWhile
    \State \textbf{return} $haplos$
    \EndFunction
  \end{algorithmic}
  \caption{\footnotesize{Algoritmo per il calcolo di ogni $\SMEM$ in colonna $k$
  tramite la 
  componente \texttt{PHI}.}}
  \label{algo:phiext}
\end{algorithm}
\noindent
Si è presentata la definizione formale delle due funzioni ma, con la $\RLPBWT$,
si hanno in memoria solo i prefix array sample  e nessuna informazione in merito
alla permutazione inversa del 
prefix array. Non si ha in memoria nemmeno il $\RLCP$, che, in via teorica, come
per la $\BWT$, potrebbe rendere efficiente il computo delle funzioni $\varphi$
e $\varphi^{-1}$. Quindi, si è pensato ad una struttura dati, basata
anch'essa su bitvector sparsi e intvector compressi, che permettesse il calcolo
delle due funzioni senza mantenere informazioni complete in memoria. 
\subsubsection{Costruzione della struttura di supporto}
L'idea, per la costruzione della struttura a supporto delle
funzioni $\varphi$ e $\varphi^{-1}$, si
basa sul fatto che, data una colonna $k$ e dati due valori consecutivi $p$ e $q$
in $a_k$ (avendo $a_k[i]=p$ e $a_k[i+1]=q$), essi rimarranno consecutivi anche
in $a_{k+o}$ (prefix array dell'arbitraria colonna $k+o$), fino a che
che $x_{p}[k+o]\neq x_{q}[k+o]$, ovvero fino a che, in colonna $k+o$, tali righe
corrisponderanno a due simboli diversi, consecutivi nella matrice
$\PBWT$. Cruciale è che, in quella colonna, 
$p$ sarà memorizzato come prefix array sample della fine della run $r$
mentre $q$ come prefix array sample dell'inizio della run $r+1$. Grazie
a questa informazione, si può costruire una struttura che, data una colonna
arbitraria e un arbitrario valore di prefix array, permetta di
computare $\varphi$ e $\varphi^{-1}$.\\
Tale struttura dati è composta da:
\begin{itemize}
  \item un vettore di bitvector sparsi per $\varphi$, che denotiamo con
  $\varPhi$, tale che $\varPhi[i][j]=1$ sse la riga $i$ indicizza una testa di
  run alla colonna $j$, nella matrice $\PBWT$. Si ha quindi che $\varPhi$
  ha dimensione $M\times N$
  \item un vettore di bitvector sparsi per $\varphi^{-1}$, che
  denotiamo con $\varPhi^{-1}$, tale che $\varPhi[i][j]=1$ sse la riga $i$
  indicizza una coda di run alla colonna $j$, nella matrice $\PBWT$. Si ha
  quindi che $\varPhi^{-1}$ ha dimensione $M\times N$
  \item un vettore di intvector compressi, denotato $\varPhi_{supp}$, a supporto
  del vettore $\varPhi$, che memorizza, per ogni simbolo $\sigma=1$ di tale
  vettore, il 
  prefix array sample della coda della run precedente o l'altezza
  del pannello, $M$, qualora non si abbia alcuna run precedente
  \item un vettore di intvector compressi, denotato $\varPhi^{-1}_{supp}$,
  a supporto del vettore $\varPhi^{-1}$, che memorizza, per ogni simbolo
  $\sigma=1$
  di tale vettore,
  il prefix array sample della testa della run successiva o l'altezza
  del pannello, $M$, qualora non si abbia alcuna run successiva
\end{itemize}
Si ha che la lunghezza della riga $i$-esima di $\varPhi_{supp}$ è
uguale al numero di simboli $\sigma=1$ presenti nella riga $i$-esima di
$\varPhi$. Analogamente 
si ha per $\varPhi^{-1}_{supp}$. In entrambi i casi, inoltre, si hanno $M$
righe. Queste osservazioni si ripercuotono sul costo in memoria della componente
\texttt{PHI}, avendo che può essere stimata coi costi in memoria dei bitvector
sparsi e degli intvector compressi, come fatto per le precedenti componenti. Si
noti, però, che in questo caso i bitvector sparsi sono ulteriormente
ottimizzati,
avendo un rapporto davvero basso di simboli $\sigma=1$ sul totale di
simboli $N$ (e non $M$
come negli altri casi, segnalando che, in pannelli reali, $N>>M$).\\ 
Al fine della costruzione, bisogna sfruttare $a_{N-1}$ per poter
identificare quelle coppie di valori consecutivi non presenti nei vari
prefix array sample, in modo che sia possibile effettuare le query per
qualsiasi valore di prefix array in input.\\
L'algoritmo \ref{algo:phicos} riporta la costruzione della struttura,
iterando prima i vari prefix array sample e completando i
risultati con $a_{N-1}$. Tale algoritmo ha complessità in tempo, nel caso
peggiore, pari a:
\begin{equation}
  \label{eq:phicos}
  \mathcal{O}(NM)
\end{equation}
Tale caso peggiore si ha qualora ogni colonna della matrice $\PBWT$ abbia un
numero di 
run pari all'altezza stessa della colonna (un caso irrealistico). Indicando con
$\rho$ il numero medio 
di run per colonna, si ha che la complessità nel caso medio è:
\begin{equation}
  \label{eq:phicos2}
  \varTheta(N\rho)
\end{equation}
\dc{CAPIRE SE COMMENTARE ULTERIORMENTE LA COSTRUZIONE}
\begin{algorithm}
  \footnotesize
  \begin{algorithmic}[1]
    \Function{Build\_phi}{$cols, panel, prefix$}
    \Comment  $prefix=a_{N-1}$ 
    \State $\varPhi\gets [[0..0]..[0..0]],\,\,\varPhi^{-1}\gets
    [[0..0]..[0..0]]$ 
    \Comment vettori di bitvector sparsi per $\varphi$ e $\varphi^{-1}$
    \State $\varPhi_{supp} = [],\,\,\varPhi_{supp}^{-1} = []$
    \Comment vettori di intvector compressi di supporto per $\varphi$ e
    $\varphi^{-1}$  
    \For {\textit{every} $k\in [0,|cols|)$}
    \Comment costruzione da prefix array sample
    \For {\textit{every} $i\in [0,|samples_{beg}|)$}
    \State $\varPhi[sample_{beg}^{k}[i]][k]\gets 1$
    \If{$i=0$}
    \State $push(\varPhi_{supp}[sample_{beg}^{k}[i]], panel_{height})$
    \Else
    \State $push(\varPhi_{supp}[sample_{beg}^{k}[i]],sample_{end}^{k}[i-1])$
    \EndIf

    \State $\varPhi^{-1}[sample_{end}^{k}[i]][k]\gets 1$
    \If{$i=|sample_{beg}^k|-1$}
    \State $push(\varPhi_{supp}^{-1}[sample_{end}^{k}[i]], panel_{height})$
    \Else
    \State $push(\varPhi_{supp}^{-1}[sample_{end}^{k}[i]],sample_{beg}^{k}[i+1])$
    \EndIf
    \EndFor
    \EndFor
    \For {\textit{every} $k\in [0,|prefix|)$}
    \Comment costruzione da ultimo prefix array
    \If{$\varPhi[k][|\varPhi[k]|-1] = 0$}
    \State $\varPhi[k][|\varPhi[k]|-1]\gets 1$
    \If{$k=0$}
    \State $push(\varPhi_{supp}[prefix[k]], panel_{height})$
    \Else
    \State $push(\varPhi_{supp}[prefix[k]] ,prefix^k[i-1])$
    \EndIf
    \EndIf
    \If{$\varPhi^{-1}[k][|\varPhi[k]|-1] = 0$}
    \State $\varPhi^{-1}[k][|\varPhi[k]|-1]\gets 1$
    \If{$k=|prefix|-1$}
    \State $push(\varPhi^{-1}_{supp}[prefix[k]], panel_{height})$
    \Else
    \State $push(\varPhi^{-1}_{supp}[prefix[k]],prefix^k[i+1])$
    \EndIf
    \EndIf
    \EndFor
    \State \textit{costruzione della struttura $\rank$ per ogni bitvector
    sparso} 
    $\varPhi$ e $\varPhi^{-1}$
    \EndFunction
  \end{algorithmic}
  \caption{Algoritmo per la costruzione della componente \texttt{PHI}.}
  \label{algo:phicos}
\end{algorithm}
Dal punto di vista delle query, data una colonna $k$ e un valore di
prefix array $p$, per la funzione $\varphi$ si effettua  $\rank^\varphi(k)$
sulla riga $p$ di $\varPhi$, avendo che:
\[\varphi_k(p)=
  \begin{cases}
    \NULL&\mbox{se }\varPhi_{supp}^p[\rank^\varphi_p(k)]=M\\
    \varPhi_{supp}^p[\rank^\varphi_p(k)]&\mbox{altrimenti }
  \end{cases}
\]
Analogamente, per la funzione $\varphi^{-1}$ si effettua la
$rank^{\varphi^{-1}}(k)$ 
sulla riga $p$ di $\varPhi^{-1}$, avendo che:
\[\varphi_k^{-1}(p)=
  \begin{cases}
    \NULL&\mbox{se }\varPhi^{-1\,\,p}_{supp}[\rank^{\varphi^{-1}}_p(k)]=M\\
    \varPhi^{-1\,\,p}_{supp}[\rank^{\varphi^{-1}}_p(k)]&\mbox{altrimenti }
  \end{cases}
\]
In termini di complessità, si ha che tale calcolo è limitato dalla complessità
della funzione $\rank$, avendo che il numero $m$ di simboli $\sigma=1$ in ogni
bitvector, lungo $N$, equivale al numero di volte in cui la corrispondente riga
è testa/coda di una run: 
\begin{equation}
  \label{eq:queryphi}
  \mathcal{O}\left(\log\frac{N}{m}\right)
\end{equation}
\dc{Serve altro?}
\begin{esempio}
  Si supponga di avere la seguente situazione nella matrice $\PBWT$:
  \begin{figure}[H]
    \centering
    \includegraphics[scale = 0.8]{img/phi.pdf}   
  \end{figure}
  \noindent
  Dove, a parità di colore, si ha lo stesso simbolo tra due indici
  consecutivi. \\
  In colonna $k$, che per praticità assumiamo essere $k=0$, si vorrebbe avere
  informazione in merito a $\varphi_k(j)$ e $\varphi^{-1}_k(m)$. \\
  Si nota che, per definizione della struttura dati, si ha (limitandoci alle
  colonne della figura):
  \[\varPhi_j=[0,0,0,1,0, \ldots]\]
  \[\varPhi^{-1}_m=[0,0,0,1,0,\ldots]\]
  In quanto, in entrambi i casi, rispettivamente per la riga $j$ e per la riga
  $m$, 
  in colonna $k+3$, si ha che $j$ è il valore del prefix array di una testa di
  run 
  mentre $m$ di una coda di run. In colonna $k+3$ si conoscono anche,
  rispettivamente, $i$, valore del prefix array della coda della run precedente
  a quella di $j$, e $n$, valore del prefix array della testa della run
  successiva quella di $m$. Si ottengono quindi:
  \[\varPhi_{supp}=[i,\ldots]\]
  \[\varPhi^{-1}_{supp}=[n,\ldots]\]
  Si vogliono quindi calcolare $\varphi_0(j)$ e  $\varphi^{-1}_0(m)$. Si ha:
  \[\varPhi_{supp}^j[\rank^\varphi_j(0)]=\varPhi_{supp}^j[0]=i\]
  \[\varPhi^{-1\,\,m}_{supp}[\rank^{\varphi^{-1}}_m(0)]=\varPhi^{-1\,\,m}_{supp}[0]=n\]
  Si noti che uguali risultati si avrebbero per $k+1$, $k+2$ e $k+3$.
\end{esempio}
\dc{SISTEMARE UN PO' TUTTO}
\section{Calcolo degli SMEM con LCP}
Questa prima soluzione per il calcolo degli SMEM con un aplotipo esterno è
quella che può essere effettuata tramite le strutture:
\begin{itemize}
  \item \texttt{MAP-INT + LCP}
  \item \texttt{MAP-BV + LCP}
\end{itemize}
I due algoritmi riprendono esattamente quanto discusso nell'\textit{algoritmo 5
  di Durbin}. Tali algoritmi, di 
fatto, non sfruttano l'uso delle matching statistics e sono limitati dal non
poter calcolare quali righe presentano un solo SMEM, calcolando solo quante
siano. Il secondo limite è dato dal fatto che necessitano di avere interamente
in memoria l'\textit{LCP} array. Questo comporta avere in memoria una
struttura non run-length encoded occupante $4NM$ bytes.\\
Il metodo procede, quindi, con l'aggiornamento dei tre indici $e_k$, $f_k$ e
$g_k$, avendo che gli ultimi due possono assumere qualsiasi valore in
$\{0,\ldots, M\}$,
come con la \textit{PBWT} classica. Avendo memorizzato solo informazioni
relative alle \textit{run} bisogna quindi, ogni volta, ricondurre l'indice alla
run corretta.
% Si ricordano quindi i tempi di tale operazione, avendo $r$ numero
% di run per la colonna $k$:
% \begin{itemize}
%   \item con la \texttt{MAP-INT + LCP} si ha tempo proporzionale a:
%   \begin{equation}
%     \label{eq:itrcomp}
%     \mathcal{O}(\log (r))
%   \end{equation}
%   \item con la \texttt{MAP-BV + LCP} si ha tempo proporzionale a:
%   \begin{equation}
%     \label{eq:itrbvcomp}
%     \mathcal{O}\left(\log\frac{M}{r}\right)
%   \end{equation}
% \end{itemize}
Inoltre Durbin sfruttava il \textit{random access} al pannello, avendo in
memoria sia il pannello che il \textit{prefix array}, al fine di aggiornare il
valore di $e_k$. In entrambe le struttura dati, però,
non si ha in memoria né il \textit{prefix array} né il pannello ma solo solo la
rappresentazione compatta della \textit{matrice PBWT}. Si è quindi dovuto
pensare ad un metodo che ricomponga data una riga $x_j$ del pannello $X$ a
partire da un elemento, indicizzato con $a_{k+1}[i]=j$, con $0\leq i<M$, alla
colonna $k+1$, della 
\textit{matrice PBWT}, muovendosi da destra a sinistra e seguendo in modo
inverso la permutazione che produce il \textit{prefix array}. In altri termini,
tale metodo permette un \textit{mapping inverso} che segua una riga del
pannello originale nella \textit{matrice PBWT}.\\ 
Per ottenere l'indice alla colonna $k$-esima da cui ``proviene'' la riga $j$,
indicizzata all'indice $i$ in
colonna $k+1$, si inizia analizzando il valore $c[k]$. Infatti, se $i<c[k]$,
allora sicuramente, in colonna $k$, è un indice corrispondente a $\sigma=0$
quello dal quale proviene, ricordando come la costruzione della colonna $k+1$
nella \textit{matrice PBWT} si abbia grazie ad ordinamento stabile. Si sfruttano
così o l'array $p_k$ o le funzioni 
$rank_{h_k}$ e $select_{h_k}$ per risalire all'indice in colonna $k$, calcolando
prima l'indice di run e l'eventuale offset, per il quale il mapping porta
all'indice $i'$ in colonna $k+1$, seguendo ``virtualmente'' la riga $x_j$ del
pannello originale. 
Per quanto riguarda la \texttt{MAP-INT + LCP} si ha lo pseudocodice per il
mapping inverso consultabile all'algoritmo \ref{algo:lfrev} mentre per quanto
riguarda la \texttt{MAP-BV + LCP} si ha l'algoritmo
\ref{algo:lfrevbv}. Parlando in termini di complessità in tempo si ha che, nel
caso della componente \texttt{MAP-INT}, si ha, con $r$ numero di run alla
colonna $k$, un caso peggiore proporzionale a:
\begin{equation}
  \label{eq:revint}
  \mathcal{O}(r)
\end{equation}
Nel caso, invece, in cui si ha la componente \texttt{MAP-BV}, si ha:
\begin{equation}
  \label{eq:revbv}
  \mathcal{O}\left(\log\frac{M}{r}\right)
\end{equation}
\dc{Approfondire?}
\begin{algorithm}
  \begin{algorithmic}[1]
    \Function{reverse\_map}{$k, \,\,i$}
    \Comment $k$ indice di colonna, $i$ indice di riga
    \If{$k=0$}
    \Comment by design
    \State \textbf{return} $0$
    \EndIf
    \State $k\gets k-1$
    \State $c\gets rlpbwt[k].c$
    \State $u\gets 0$, $v\gets 0$, $offset\gets 0$, $run \gets 0$,
    $found\gets \bot$
    \If {i < c}
    \State $u\gets i$
    \State $prev_0\gets 0$, $next_0\gets 0$
    \For {\textit{every} $j\in [0,|p_k|)$}
    \State $(prev_0,\_) \gets uvtrick(k,j)$
    \State $(next_0,\_) \gets uvtrick(k,j+1)$
    \If{$prev_0\leq u<next_0$}
    \State $run\gets j$,$found\gets \top$
    \State \textbf{break}
    \EndIf
    \EndFor
    \If{$\neg found$}
    \State $run \gets |p_k|-1$
    \EndIf
    \State $(curr_u,\_)\gets uvtrick(k, run)$, $offset\gets u-curr_u$
    \State \textbf{return} $p_k[run]+offset$
    \Else

    \State $v\gets i-c$
    \State $prev_1\gets 0$, $next_1\gets 0$
    \For {\textit{every} $j\in [0,|p_k|)$}
    \State $(\_,prev_1) \gets uvtrick(k,j)$
    \State $(\_,next_1) \gets uvtrick(k,j+1)$
    \If{$prev_1\leq v<next_1$}
    \State $run\gets j$,$found\gets \top$
    \State \textbf{break}
    \EndIf
    \EndFor
    \If{$\neg found$}
    \State $run \gets |p_k|-1$
    \EndIf
    \State $(curr_v,curr_u)\gets uvtrick(k, run)$, $offset\gets v-curr_v$
    \State \textbf{return} $p_k[run]+offset$
    \EndIf
    \EndFunction
  \end{algorithmic}
  \caption{Algoritmo per il mapping inverso con la \texttt{MAP-INT + LCP}.}
  \label{algo:lfrev}
\end{algorithm}

\begin{algorithm}
  \begin{algorithmic}[1]
    \Function{reverse\_map}{$k, \,\,i$}
    \Comment $k$ indice di colonna, $i$ indice di riga
    \If{$k=0$}
    \Comment by design
    \State \textbf{return} $0$
    \EndIf
    \State $k\gets k-1$
    \State $c\gets rlpbwt[k].c$
    \If{$i<c$}
    \If{$start_k$}
    \State $run\gets rank_u^{k}(i)\cdot 2$
    \Else
    \State $run\gets rank_u^{k}(i)\cdot 2+1$
    \EndIf
    \State $i_{run}\gets 0$
    \If{$run\neq 0$}
    \State $i_{run}\gets select_h^{k}(run)+1$
    \EndIf
    \State $(prev_0,\,\,\_)\gets uvtrick(k,\,\,i_{run})$
    \State \textbf{return} $i_{run}+(i-prev_0)$
    \Else
    \If{$start_k$}
    \State $run\gets rank_v^{k}(i)\cdot 2+1$
    \Else
    \State $run\gets rank_v^{k}(i)\cdot 2$
    \EndIf
    \State $i_{run}\gets 0$
    \If{$run\neq 0$}
    \State $i_{run}\gets select_h^{k}(run)+1$
    \EndIf
    \State $(\_,\,\,prev_1)\gets uvtrick(k,\,\,i_{run})$
    \State \textbf{return} $i_{run}+(i-(c+prev_1))$
    \EndIf
    \EndFunction
  \end{algorithmic}
  \caption{Algoritmo per il mapping inverso con la \texttt{MAP-BV + LCP}.}
  \label{algo:lfrevbv}
\end{algorithm}
Si procede quindi riadattando l'algoritmo di Durbin all'uso delle \textit{run},
ottenendo, ad ogni step, i medesimi valori per $e_k$, $f_k$ e $g_k$. Le uniche
differenze sono:
\begin{itemize}
  \item il calcolo del mapping necessità dell'estrazione dei valori $u$ e $v$,
  tenendo conto esplicito degli offset nel caso della \texttt{MAP-INT + LCP}
  \item non si ha \textit{random access} al pannello quindi bisogna procedere
  ogni volta con il'inverso del mapping e il calcolo del simbolo a partire
  dall'indice della run
  \item non si ha il \textit{prefix array} in memoria quindi non è possibile
  sapere quali siano le righe che stanno matchando fino alla colonna $k$ ma solo
  quante, sapendo che sono $g_k-f_k$
\end{itemize}
Anche in questo caso lo pseudocodice è consultabile all'algoritmo
\ref{algo:matchlcp}. Calcolare la complessità di tale algoritmo non è semplice,
come già visto nel caso dell'algoritmo 5 di Durbin. In modo analogo si può
comunque intuire come i vari cicli interni siano limitati superiormente dalla
larghezza del pannello e dai tempi di mapping. Questo si può stimare in quanto
le occorrenze dei cicli interni sono proporzionali al numero di SMEM e al numero
di ``step'' all'indietro necessari a ri-computare il nuovo intervallo, numero di
step che scala sul numero di caratteri in overlap tra due SMEM
consecutivi. Fatta questa premessa si può stimare che il calcolo degli SMEM con
la struttura \texttt{MAP-INT + LCP} è proporzionale, con $\rho$ numero medio di
run per una colonna, a:
\begin{equation}
  \label{eq:lcpmatchint}
  \mathcal{O}(N\log \rho)
\end{equation}
Nel caso, invece, della struttura \texttt{MAP-BV + LCP} si ha:
\begin{equation}
  \label{eq:lcpmatchbv}
  \mathcal{O}\left(N\log\frac{M}{\rho}\right)
\end{equation}
\dc{APPROFONDIRE SPIEGAZIONE ALGORITMI}

\begin{algorithm}
  \footnotesize
  \begin{algorithmic}[1]
    \Function{external\_matches}{$z$}
    \Comment si assume $|z|=N$
    \State $f\gets 0,\,\,f_{run}\gets 0,\,\,f'\gets 0$
    \State $g\gets 0,\,\,g_{run}\gets 0,\,\,g'\gets 0$
    \State $e\gets 0,\,\,nh\gets 0$
    \For {\textit{every} $k\in\left[0,\,\, |z|\right)$}
    \State $f_{run}\gets index\_to\_run(f,k)$ \textbf{oppure} $f_{run}\gets
    rank_h^k(f)$ 
    \State $g_{run}\gets index\_to\_run(g,k)$ \textbf{oppure} $f_{run}\gets
    rank_h^k(g)$ 
    \State $f'\gets w(k,\,\, f,\,\, z[k]),\,\,g'\gets w(k,\,\,
    g,\,\, z[k]),\,\,nh\gets g-f$
    \If{$f'<g'$}
    \State $f\gets f',\,\,g\gets g'$
    \Else
    \If{$k\neq 0$}
    %\State \textbf{report} \textit{matches in} $[e,\,\, k-1]$ \textit{with} $l$
    \State \textit{memorizzazione degli SMEM tra le colonne} $[e,\,\, k-1]$
    \textit{con} $nh$ aplotipi
    \EndIf
    \State \textbf{if} $f'=|l_{k+1}|$ \textbf{then} $e\gets k+1$ \textbf{else}
    $e\gets k-l_{k+1}[f']$ 
    % \If {$f'=|l_{k+1}|$}
    % \State $e\gets k+1$
    % \Else
    % \State  $e\gets k-l_{k+1}[f']$
    % \EndIf
    \If{$(z[e]=0\land f'>0)\lor f'=M$}
    \State $f'\gets g'-1$
    \If{$e\geq 1$}
    \State $f_{rev}\gets f',\,\,k'\gets k+1$
    \While { $k'\neq e-1$}
    \State  $f_{rev}\gets reverse\_map(k',\,\,f_{rev}),\,\,k'\gets k'-1$
    \EndWhile
    \State $run\gets index\_to\_run(f_{rev},k')$ \textbf{oppure} $run\gets
    rank_h^{k'}(f_{rev})$
    \State $symb\gets get\_symbol(start_{k'}, run)$ 
    \While {$e>0\land z[e-1]=symb$}
    \State $e\gets e-1,\,\,f_{rev}\gets reverse\_map(e, \,\,f_{rev})$
    \State $run\gets index\_to\_run(f_{rev}, e-1)$ \textbf{oppure} $run\gets
    rank_h^{e-1}(f_{rev})$
    \State $symb\gets get\_symbol(start_{e-1}, run)$ 
    \EndWhile
    \EndIf
    \State \textbf{while} $f'>0\land (k+1)-l_{k+1}[f]\leq e$ \textbf{do}
    $f'\gets f'-1$ 
    \State $f\gets f',\,\,g\gets g'$
    \Else
    \State $g'\gets f'-1$
    \If{$e\geq 1$}
    \State $f_{rev}\gets f',\,\,k'\gets k+1$
    \While {$k'\neq e-1$}
    \State  $f_{rev}\gets reverse\_map(k',\,\,f_{rev}),\,\,k'\gets k'-1$
    \EndWhile
    \State $run\gets index\_to\_run(f_{rev},k')$ \textbf{oppure} $run\gets
    rank_h^{k'}(f_{rev})$
    \State $symb\gets
    get\_symbol(start_{k'}, 
    run)$ 
    \While {$e>0\land z[e-1]=symb$}
    \State $e\gets e-1,\,\,f_{rev}\gets reverse\_map(e, \,\,f_{rev})$
    \State $run\gets index\_to\_run(f_{rev},e-1)$ \textbf{oppure} $run\gets
    rank_h^{e-1}(f_{rev})$
    \State $symb\gets get\_symbol(start_{e-1}, run)$ 
    \EndWhile
    \EndIf
    \State \textbf{while} $e<M\land (k+1)-l_{k+1}[g']\leq e$ \textbf{do}
    $g'\gets g'+1$  
    \State $f\gets f',\,\,g\gets g'$
    \EndIf
    \EndIf
    \EndFor
    \If{$f<g$}
    \State $nh\gets g-f$
    \State \textit{memorizzazione degli SMEM tra le colonne} $[e,\,\, |z|-1]$
    \textit{con} $nh$ aplotipi
    \EndIf
    \EndFunction
  \end{algorithmic}
  \caption{\footnotesize{Calcolo degli SMEM con aplotipo esterno per
  \texttt{MAP-INT/BV + LCP}, con eventuali usi diversificati per
  \texttt{MAP-INT} e \texttt{MAP-BV} segnalati con ``oppure''.}} 
  \label{algo:matchlcp}
\end{algorithm}


% \begin{algorithm}
%   \footnotesize
%   \begin{algorithmic}[1]
%     \Function{external\_matches}{$z$}
%     \Comment $|z|=N$
%     \State $f\gets 0,\,\,f_{run}\gets 0,\,\,f'\gets 0$
%     \State $g\gets 0,\,\,g_{run}\gets 0,\,\,g'\gets 0$
%     \State $e\gets 0,\,\,nh\gets 0$
%     \For {\textit{every} $k\in\left[0,\,\, |z|\right)$}
%     \State $f_{run}\gets rank_h^k(f),\,\,g_{run}\gets rank_h^k(g)$
%     \State $f'\gets w(k,\,\, f,\,\, z[k]),\,\,g'\gets w(k,\,\, g,\,\, z[k])$
%     \State $nh\gets g-f$
%     \If{$f'<g'$}
%     \State $f\gets f',\,\,g\gets g'$
%     \Else
%     \If{$k\neq 0$}
%     \State \textit{memorizzazione degli SMEM tra le colonne} $[e,\,\, k-1]$
%     \textit{con} $nh$ aplotipi
%     \EndIf
%     \If{$f'=|l_{k+1}|$}
%     \State $e\gets k+1$
%     \Else
%     \State $e\gets k-l_{k+1}[f']$
%     \EndIf
    
%     \If{$(z[e]=0\land f'>0)\lor f'=M$}
%     \State $f'\gets g'-1$
%     \If{$e\geq 1$}
%     \State $f_{rev}\gets f',\,\,k'\gets k+1$
%     \While {$k'\neq e-1$}
%     \State $f_{rev}\gets reverse\_map(k', \,\,f_{rev}),\,\,k'\gets k'-1$
%     \EndWhile
%     \State $run\gets rank_h^{k'}(f_{rev}),\,\,symb\gets get\_symbol(start_{k'},
%     run)$ 
%     \While {$e>0\land z[e-1]=symb$}
%     \State $f_{rev}\gets reverse\_map(e, \,\,f_{rev})$
%     \State $run\gets rank_h^{e-1}(f_{rev})$
%     \State $symb\gets get\_symbol(start_{e-1}, run)$
%     \EndWhile
%     \EndIf
%     \State \textbf{while} $f'>0\land (k+1)-l_{k+1}[f]\leq e$ \textbf{do}
%     $e\gets e-1$ 
%     \State $f\gets f',\,\,g\gets g'$
%     \Else
%     \State $g'\gets f'-1$
%     \If{$e\geq 1$}
%     \State $f_{rev}\gets f',\,\,k'\gets k+1$
%     \While {$k'\neq e-1$}
%     \State $f_{rev}\gets reverse\_map(k', \,\,f_{rev}),\,\,k'\gets k'-1$
%     \EndWhile
%     \State $run\gets rank_h^{k'}(f_{rev}),\,\,symb\gets get\_symbol(start_{k'},
%     run)$ 
%     \While {$e>0\land z[e-1]=symb$}
%     \State $f_{rev}\gets reverse\_map(e, \,\,f_{rev})$
%     \State $run\gets rank_h^{e-1}(f_{rev})$
%     \State $symb\gets get\_symbol(start_{e-1}, run)$
%     \EndWhile
%     \EndIf
%     \State \textbf{while} $e<M\land (k+1)-l_{k+1}[e]\leq e$ \textbf{do}
%     $e\gets e+1$  
%     \State $f\gets f',\,\,g\gets g'$
%     \EndIf
%     \EndIf
%     \EndFor
%     \If{$f<g$}
%     \State $nh\gets g-f$
%     \State \textit{memorizzazione degli SMEM tra le colonne} $[e,\,\, |z|-1]$
%     \textit{con} $nh$ aplotipi  
%     \EndIf
%     \EndFunction
%   \end{algorithmic}
%   \caption{\footnotesize{Calcolo degli SMEM con aplotipo esterno per
%   \texttt{MAP-BV + LCP}.}} 
%   \label{algo:matchpanelbv}
% \end{algorithm}
\section{Calcolo degli SMEM con matching statistics}
L'obbiettivo di questa tesi era quello di applicare i metodi e gli algoritmi già
studiati per la \textit{BWT}, riferendosi al calcolo dei \textit{MEM} a partire
dall'array delle \textit{matching statistics}, alla \textit{PBWT}.\\
Nelle sei strutture dati dedicate al calcolo degli SMEM tramite \textit{matching
  statistics} si riconoscono le due modalità già descritte con \textit{MONI} e
\textit{PHONI}:
\begin{enumerate}
  \item calcolare l'array MS in due passaggi sfruttando il \textit{random
    access} al pannello per calcolare i vari $MS[i].len$
  \item calcolare l'array MS in un passaggio sfruttando le \textit{LCE query}
  sia per scegliere i vari $MS[i].row$ che per calcolare, in contemporanea, i
  vari $MS[i].len$ 
\end{enumerate}
\subsection{Calcolo dell'array MS con threshold}
Questa prima soluzione, necessitando sia della componente
\texttt{THR-INT}/\texttt{THR-BV} che della componente
\texttt{RA-BV}/\texttt{RA-SLP}, è relativa alle seguenti strutture dati:
\begin{itemize}
  \item \texttt{MAP-INT + THR-INT + RA-BV + PERM + PHI}
  \item \texttt{MAP-INT + THR-INT + RA-SLP + PERM + PHI}
  \item \texttt{MAP-BV + THR-BV + RA-BV + PERM + PHI}
  \item \texttt{MAP-BV + THR-BV + RA-SLP + PERM + PHI}
\end{itemize}
Tra le quali le uniche differenze si riscontrano nei tempi d'esecuzione e nella
memoria richiesta.\\
% Si noti che unendo la componente \texttt{PERM} alla componente
% \texttt{THR-INT}/\texttt{THR-BV} si ottiene una variante dell'\textbf{R-index}
% visto per la \textit{RLBWT}.\\
Vediamo il funzionamento dell'algoritmo.
Sia data $t$ la posizione della \textit{threshold} nella run corrente,
in colonna $k$, e
si supponga che tale run, con testa all'indice $h$, non sia associata al simbolo
desiderato, ovvero $z[k]$. Si supponga che, con il mapping, si sia arrivati
all'indice $i$ della colonna $k$. Si supponga inoltre che la run successiva
abbia testa in indice $e$. Si hanno due casi possibili, denotando con
$LCS(x,y)$ il \textit{longest common suffix} tra le stringhe $X$ e $Y$ e con
$a_k$ il \textit{prefix array} in colonna $k$:
\begin{enumerate}
  \item $i<t$ allora, per definizione di \textit{threshold}:
  \[LCS(z[0,k], x_{a_{k}[h-1]}[0,k])\geq LCS(z[0,k], x_{a_{k}[e]}[0,k])\]
  Quindi si ha che $MS[k].row=a_{k}[h-1]$ e il mapping potrà proseguire
  dall'indice $h-1$
  \item  $i\geq t$ allora, per definizione di \textit{threshold}:
  \[LCS(z[0,k], x_{a_{k}[s-1]}[0,k])\leq LCS(z[0,k], x_{a_{k}[e]}[0,k])\]
  Quindi si ha che $MS[k].row=a_{k}[e]$ e il mapping potrà proseguire
  dall'indice $e$
\end{enumerate}
Qualora una colonna presenti solo simboli $\sigma\neq z[k]$, per convenzione, si
imposta che $MS[k].row = M$ e si ricomincia, in colonna $k+1$, dall'ultima
posizione, indicizzata nel pannello originale dal valore finale del
\textit{prefix array sample} dell'ultima run.
\begin{esempio}
  \label{es:thr}
  Si vede quindi un esempio di funzionamento delle threshold.\\
  Si prenda pannello visto all'esempio \ref{es:pbwt1} e si effettui la
  permutazione secondo $a_2$:
  \begin{table}[H]
    \centering
    \footnotesize
    \begin{tabular}{c|cc|c|cccccccccccc}
      X & 00 & 01 & 02 & 03 & 04 & 05 & 06 & 07 & 08 & 09 & 10 & 11 & 12 & 13
      & 14 \\
      \hline
      00 & 1 & 0 & 0 & 1 & 0 & 0 & 0 & 0 & 0 & 0 & 0 & 1 & 1 & 0 & 1 \\
      01 & 1 & 0 & 0 & 1 & 1 & 0 & 0 & 1 & 0 & 0 & 0 & 0 & 0 & 1 & 1 \\
      02 & 1 & 0 & 0 & 1 & 1 & 0 & 0 & 1 & 0 & 0 & 0 & 1 & 0 & 0 & 1 \\
      03 & 1 & 0 & 0 & 1 & 1 & 0 & 0 & 1 & 0 & 0 & 0 & 1 & 0 & 0 & 1 \\
      04 & 0 & 1 & 0 & 1 & 0 & 1 & 0 & 0 & 0 & 0 & 0 & 1 & 0 & 0 & 1 \\
      05 & 0 & 1 & 0 & 1 & 0 & 1 & 0 & 0 & 0 & 0 & 0 & 1 & 0 & 0 & 1 \\
      06 & 0 & 1 & 0 & 1 & 0 & 1 & 0 & 0 & 0 & 0 & 0 & 1 & 0 & 0 & 1 \\
      07 & 0 & 1 & 0 & 1 & 0 & 1 & 0 & 0 & 0 & 0 & 0 & 0 & 1 & 0 & 1 \\
      08 & 0 & 1 & 0 & 0 & 1 & 0 & 0 & 0 & 0 & 1 & 1 & 1 & 0 & 0 & 1 \\
      09 & 0 & 1 & 0 & 1 & 0 & 0 & 0 & 0 & 1 & 0 & 0 & 0 & 0 & 1 & 1 \\
      10 & 0 & 1 & 0 & 1 & 0 & 0 & 0 & 0 & 1 & 0 & 0 & 0 & 0 & 1 & 1 \\
      11 & 0 & 1 & 0 & 0 & 1 & 0 & 0 & 0 & 0 & 0 & 1 & 1 & 0 & 0 & 0 \\
      12 & 0 & 1 & 0 & 0 & 1 & 0 & 0 & 0 & 1 & 0 & 1 & 1 & 0 & 0 & 1 \\
      13 & 0 & 1 & 0 & 0 & 1 & 0 & 0 & 0 & 1 & 0 & 1 & 1 & 0 & 0 & 1 \\
      14 & 0 & 1 & 0 & 0 & 0 & 0 & 0 & 0 & 1 & 0 & 0 & 0 & 1 & 0 & 1 \\
      15 & 0 & 1 & 0 & 0 & 0 & 0 & 0 & 0 & 1 & 0 & 0 & 0 & 1 & 0 & 1 \\
      16 & 0 & 1 & 0 & 1 & 0 & 0 & 0 & 0 & 0 & 0 & 0 & 1 & 1 & 0 & 1 \\
      17 & 0 & 1 & 1 & 0 & 1 & 0 & 0 & 0 & 0 & 0 & 0 & 1 & 0 & 0 & 1 \\
      18 & 0 & 1 & 1 & 0 & 1 & 0 & 1 & 0 & 0 & 0 & 0 & 0 & 1 & 0 & 1 \\
      19 & 1 & 1 & 0 & 0 & 0 & 1 & 0 & 0 & 0 & 0 & 0 & 1 & 1 & 0 & 1 \\
    \end{tabular}
  \end{table}
  Si prenda la seconda run, di simboli $\sigma=1$, indicizzata tra 17 e 18. \\
  Si supponga che, tramite il mapping, si sia arrivati alla riga 17 ma che si
  abbia $z[2]=0$. la scelta è quindi tra la coda della run precedente, avendo
  che $a_2[16]=16$ o la testa della run successiva, avendo che $a_2[19]=17$. Si
  può notare come il minimo \textit{LCP} si trovi, per la 
  run, all'indice 18 (a causa del fatto che il minimo \textit{LCP} è all'indice
  19, quello della testa della run successiva). Si può quindi proseguire o con
  la riga. Questo significa che il suffisso comune più lungo con la query si ha
  con la riga 16 del pannello, per definizione di threshold, avendo che questa
  sarà memorizzata nell'array $MS$:
  \[MS[2].row=16\]
  Successivamente, tramite \textit{random access} al testo, confrontando la riga
  $x_{16}$ e la query $z$, fino alla colonna $k=2$, si potrà calcolare che
  $MS[2].len=3$. 
\end{esempio}
\textbf{SISTEMARE ESEMPIO}\\
Una volta computato tutti i valori $MS[i].row$ per calcolare i vari $MS[i].len$
si scorre da sinistra a destra calcolando la lunghezza dello SMEM facendo random
access al pannello e confrontando la query $z$ con la riga $MS[i].row$. Si
assuma infatti di aver calcolato $MS[i-1].len$ e di voler calcolare $MS[i].len$.
Si hanno tre casi possibili:
\begin{enumerate}
  \item $MS[i].row=M$ e in tal caso, avendo segnalata l'inesistenza di alcuno
  SMEM, si ha che $MS[i].len=0$
  \item $MS[i].row=MS[i-1].row$, avendo $i\neq 0$ e $MS[i-1].len\neq 0$, allora
  si sta seguendo la stessa riga che si seguiva in colonna $i-1$ e quindi,
  banalmente, $MS[i].len=MS[i-1].len+1$
  \item in qualsiasi altro caso bisogna confrontare, a partire dalla colonna
  $i$, la query 
  $z$ con la riga $MS[i].row$ del pannello da destra a sinistra, fino a che non
  si trova un mismatch, calcolando la lunghezza $l$ del suffisso comune tra esse
  e memorizzando tale valore, tramite $MS[i].len=l$
\end{enumerate}
In fase di costruzione delle lunghezze è possibile anche riportare gli
\textit{SMEM}, terminanti in colonna $i$, qualora:
\begin{itemize}
  \item $MS[i].len\geq MS[i+1].len \land MS[i].len\neq 0$
  \item si è arrivati a fine array, avendo $i=N-1\land MS[i].len\neq 0$
\end{itemize}
Come si può verificare nell'esempio \ref{es:ms}.\\
L'algoritmo per il match tramite \textit{threshold} è visualizzabile
all'algoritmo \ref{algo:matchthr}. Anche in questo caso la stima delle
complessità non è di facile ottenimento. Dividendo nelle varie parti l'algoritmo
si ha che:
\begin{itemize}
  \item il calcolo dei valori $row$ dell'array $MS$ varia a seconda dell'uso
  della componente \texttt{MAP-INT} o \texttt{MAP-BV}. Il costo della funzione
  \textit{down}, variabile a seconda della componente \texttt{RA} e
  dell'eventuale componente \texttt{LCE}, risulta trascurabile vista la bassa
  frequenza d'uso, in termini probabilistici. Si ha quindi che, con $\rho$
  numero medio di run per colonna, usando \texttt{MAP-INT} si ha tempo
  proporzionale a:
  \begin{equation}
    \label{eq:msthr1int}
    \mathcal{O}(N\log\rho)
  \end{equation}
  Mentre con \texttt{MAP-BV} è proporzionale a:
  \begin{equation}
    \label{eq:msthr1bv}
    \mathcal{O}\left(N\log\frac{M}{\rho}\right)
  \end{equation}
  
  \item il calcolo dei valori $len$ dell'array $MS$ è il più complesso da
  stimare in termini di complessità asintotica. Questa difficoltà è dovuta dal
  fatto che gli accessi al pannello vengono fatti solo quando $MS.row[i]\neq
  MS.row[i-1]$. Per semplicità denotiamo con $\gamma$ il numero di accessi al
  pannello e, nel caso della componente \texttt{RA-BV}, si ha che il calcolo
  complessivo delle lunghezza è proporzionale a:
  \begin{equation}
    \label{eq:msthr2bv}
    \mathcal{O}(N\gamma)
  \end{equation}
  Mentre con \texttt{RA-SLP} è proporzionale, con $s$ lunghezza della stringa
  unica prodotta dall'\textit{SLP}, a:
  \begin{equation}
    \label{eq:msthr2slp}
    \mathcal{O}\left(N\gamma\log s\right)
  \end{equation}
  \item si ha infine il calcolo effettivo degli \textit{SMEM} e di tutte le
  righe del pannello per le quali si hanno tali match. Denotando con $\delta$ il
  numero complessivo di accessi alla componente \texttt{RA-BV} o di usi della
  componente \texttt{LCE}, qualora disponibile, e con $\mu$ il numero di
  \textit{SMEM}, si ha che, nel primo caso, la complessità di tale operazione è
  proporzionale a:
  \begin{equation}
    \label{eq:msthr3bv}
    \mathcal{O}(\mu\delta)
  \end{equation}
  mentre nel caso di uso dell'\textit{SLP} e della componente \texttt{LCE},
  avendo sempre $s$ lunghezza della stringa unica prodotta dall'\textit{SLP}:
  \begin{equation}
    \label{eq:msthr3slp}
    \mathcal{O}(\mu\delta\log s)
  \end{equation}
\end{itemize}
Facendo una stima complessiva si può ipotizzare come la struttura \texttt{MAP-BV
+ THR-BV + RA-SLP + PERM + PHI}, a causa della maggior lentezza in fase di
mapping e di accesso al pannello per il calcolo delle lunghezze, sia quella con
prestazioni peggiori mentre, per il ragionamento inverso, la struttura
\texttt{MAP-INT + THR-INT + RA-BV + PERM + PHI} sia quella con le migliori
performance dal punto di vista del tempo macchina.\\
In termini di memoria, invece,  la struttura
\texttt{MAP-INT + THR-INT + RA-SLP + PERM + PHI} risulta essere la più
vantaggiosa mentre la struttura
\texttt{MAP-BV + THR-BV + RA-BV + PERM + PHI} la peggiore, per le stime viste
nelle sezioni precedenti dedicate alle singole componenti. 
\begin{algorithm}
  \scriptsize
  \begin{algorithmic}[1]
    \Function{external\_matches}{$z$}
    \State $ms_{row}\gets [0..0],\,\,ms_{len}\gets [0..0]$
    \Comment vettore $MS$ di lunghezza $|z|$
    \State $curr_{row}\gets
    rlpbwt[0].samples_{end}[|rlpbwt[0].samples_{end}|-1]$
    \State $curr_{index}\gets curr_{row}$
    \State  $curr_{run}\gets index\_to\_run(curr_{index},0)$ \textbf{oppure}
    $curr_{run}\gets rank_h^0(curr_{index})$    
    \State $symb\gets get\_symbol(start_0, curr_{run})$
    \Comment \textbf{Costruzione righe dell'array $MS$}
    \For {\textit{every} $k\in[0, |z|)$}

    \If{$z[i]=symb$}
    \State $ms_{row}[k]\gets curr_{row}$
    \State \hspace{-1.1mm}\textbf{if} $k\neq |z|-1$ \textbf{then}
    $(curr_{index},\,\,curr_{run},\,\,symb)\gets UPDATE(k, curr_{index},z)$ 
    \Else
    \State  $curr_{thr}\gets t_k[curr_{run}]$ \textbf{oppure}
    $curr_{thr}\gets rank_t^k(curr_{index})$ 
    \State $force_{down} \gets \top$\textit{ sse l'indice è sovrapposto ad una
    threshold non in coda di run}
     \State $force_{down} \gets \top$\textit{ sse l'indice è sovrapposto ad una
    threshold in coda di run e $DOWN(\ldots)=\top$}
    \If{$|samples_{beg}^k|=1$}
    \State $ms_{row}[k]\gets M$
    \If{$k\neq |z|-1$}
    \State $curr_{row}\gets
    rlpbwt[k+1].samples_{end}[|rlpbwt[k+1].samples_{end}|-1]$
    \State $curr_{index}\gets M-1$
    \State $curr_{run}\gets index\_to\_run(curr_{index},k+1)$ \textbf{oppure}
    $curr_{run}\gets rank_h^{k+1}(curr_{index})$
    \State $symb\gets get\_symbol(start_{k+1}, curr_{run})$
    \EndIf
    \ElsIf{$(curr_{run}\neq 0 \land curr_{run}=curr_{thr}\land \neg down)\lor
    curr_{run}=|samples_{beg}^k|-1$} 
    \State $curr_{index}\gets p_k[curr_{run}]-1$ \textbf{oppure}
    $curr_{index}\gets select_h^{k}(curr_{run})$
    \State $curr_{row}\gets samples_{end}^k[curr_{run}-1]$
    \State $ms_{row}[k]\gets curr_{row}$
    \State \textbf{if} $k\neq |z|-1$ \textbf{then}
    $(curr_{index},\,\,curr_{run},\,\,symb)\gets UPDATE(k, curr_{index},z)$ 
    % \If{$k\neq |z|-1$}
    % \State $(curr_{index},\,\,curr_{run},\,\,symb)\gets UPDATE(k, curr_{index},
    % z)$ 
    % \EndIf
    \Else
    \State $curr_{index}\gets  p_k[curr_{run}+1]$ \textbf{oppure}
    $curr_{index}\gets select_h^{k}(curr_{run}+1)+1$
    \State $curr_{row}\gets samples_{beg}^k[curr_{run}+1]$
    \State $ms_{row}[k]\gets curr_{row}$
    \State \textbf{if} $k\neq |z|-1$ \textbf{then} $(curr_{index},\,\,curr_{run},
    \,\,symb)\gets UPDATE(k, curr_{index}, z)$ 
    % \If{$k\neq |z|-1$}
    %  \State $(curr_{index},\,\,curr_{run},\,\,symb)\gets UPDATE(k, curr_{index},
    % z)$ 
    % \EndIf
    \EndIf
    \EndIf
    \EndFor
    \For {\textit{every} $k\in[0,|z|)$}
    \Comment \textbf{Costruzione lunghezze dell'array $MS$}
    \If{$ms_{row}[k] = M$}
    \State $ms_{len}[k]\gets 0$
    \ElsIf{$k\neq 0\land ms_{row}[i]=ms_{row}[i-1]\land
    ms_{len}[i-1]\neq 0$}
    \State $ms_{len}[i]\gets ms_{len}[i-1]+1$
    \Else
    \Comment $ra$ effettua il random access con la componente \texttt{RA-BV} o
    \texttt{RA-SLP} 
    \State $tmp_{index}\gets i,\,\,tmp_{len}\gets 0$
    \While {$tmp_{index}\geq 0 \land z[tmp_{index}]=ra(ms_{row}[k],
    tmp_{index})$}
    \State $tmp_{index}\gets tmp_{index}-1,\,\,tmp_{len}\gets tmp_{len}+1$
    \EndWhile
    \State $ms_{len}[k]\gets tmp_{len}$
    \EndIf
    \EndFor
    \For {\textit{every} $k\in[0,|z|)$}
    \Comment \textbf{Calcolo dei match da $MS$}
    \If{$(ms_{len}[k]>1 \land ms_{len}[k]\geq ms_{len}[k+1])\lor(k = |z|-1 \land
    ms_{len}[k]\neq 0$}
    \State \textit{report dello SMEM terminante in colonna $k$}
    \State \textit{SMEM di lunghezza $ms_{len}[k]$ con la riga $ms_{row}[k]$ e
    quelle estese da essa tramite} \texttt{PHI}
    \EndIf
    \EndFor
    \EndFunction

    
    \Function {down}{$pos, prev, next$}
    \State \textit{si usano le LCE queries o il random access per calcolare il
    suffisso comune più lungo tra quelli delle righe}
    \State \textit{$pos$/$prev$ e
    $pos$/$next$ fino alla colonna precedente a quella corrente} 
    \State \textit{se il secondo è maggiore o uguale al primo ritorna $\top$,
    altrimenti $\bot$} 
    \EndFunction
  \end{algorithmic}
  \caption{\footnotesize{Calcolo degli SMEM con aplotipo esterno con componenti
  \texttt{MAP-INT/BV},
  \texttt{THR-INT/BV} (i cui usi diversificati di entrambe le componenti sono
  segnalati con ``oppure''), \texttt{RA-BV/SLP}, \texttt{PERM} e \texttt{PHI}.}}    
  \label{algo:matchthr}
\end{algorithm}
\begin{algorithm}
  \footnotesize
  \begin{algorithmic}[1]
    \Function{update}{$k, curr_{index}, z$}
    \State $curr_{index}\gets lf(k, curr_{index}, z[k])$
    \State $curr_{run}\gets index\_to\_run(curr_{index},k+1)$ \textbf{oppure}
    $curr_{run}\gets rank_h^{k+1}(curr_{index})$
    \State $symb\gets get\_symbol(start_{k+1}, curr_{run})$
    \State \textbf{return} $(curr_{index},\,\,curr_{run},\,\,symb)$
    \EndFunction
  \end{algorithmic}
  \caption{Algoritmo per l'update con componenti \texttt{MAP-INT} e
  \texttt{MAP-BV}.} 
  \label{algo:updatems}
\end{algorithm}

\dc{MANCANO COMPLESSITÀ}
\subsection{Calcolo dell'array MS con LCE query}
Come anticipato, grazie all'uso delle \textit{LCE query} è possibile calcolare
l'array 
delle \textit{matching statistics} in un solo scorrimento da sinistra a
destra. Infatti è possibile usare tali query per calcolare non solo quale nuova
sequenza scegliere in caso di mismatch con l'aplotipo query in colonna $i$, come
si faceva con l'uso delle \textit{threshold}, ma anche di computare la lunghezza
del suffisso comune tra essa e l'aplotipo query, calcolando nello stesso momento
sia $MS[i].row$ che $MS[i].len$.\\
Tale soluzione è quindi relativa alle seguenti strutture dati:
\begin{itemize}
  \item \texttt{MAP-INT + LCE + PERM + PHI}
  \item \texttt{MAP-BV + LCE + PERM + PHI}
\end{itemize}
Con la notazione:
\begin{equation}
  \label{eq:lcecol}
  LCE(k, x, y)
\end{equation}
Si indica il calcolo della \textit{LCE query} tra le righe di indice $x$ e
indice $y$ terminante in colonna $k-1$ (quindi escludendo la colonna
$k$-esima).\\ 
Si illustra ora come computare l'array delle \textit{matching statistics}. Anche
in questo caso, per convenzione, si inizia la computazione dell'ultima 
riga della prima colonna.
Si assuma di avere calcolato l'array $MS$ di una query $z$ rispetto al pannello
$X$. le cui righe si identificano tramite $x_i, \forall i\in\{0,M\}$, fino alla
colonna $k-1$. Sia $i$ 
l'indice di riga sulla \textit{matrice PBWT} al quale si è arrivati mediante il
mapping, avendo che tale riga è quella che ha il più lungo suffisso comune con
$z[1,k-1]$. Si assuma che l'indice $i$ appartenga alla run $r$, di simboli
$\sigma$, testa di indice $h$ e coda di indice $e-1$. Si hanno diversi casi:
\begin{enumerate}
  \item $z[k]=\sigma$, quindi la riga $i$ può essere usata per estendere il
  match, avendo che $MS[k].row=MS[k-1].row$ e $MS[k].len=MS[k-1].len+1$, e per
  proseguire col mapping in colonna $k+1$
  \item $z[k]\neq\sigma$ e si ha una sola run in colonna $k$, avendo quindi che
  non si possono avere match. Per convenzione, si
  imposta che $MS[k].row = M$ e $MS[k].len=0$. Infine si ricomincia, in colonna
  $k+1$, dall'ultima posizione, indicizzata nel pannello originale dal valore
  finale del \textit{prefix array sample} dell'ultima run
  \item $z[k]\neq\sigma$ ma si hanno anche altre run, dovendo quindi scegliere
  la nuova riga da seguire. Si ha che il più lungo suffisso di $z[1,k]$ che è
  anche suffisso di $x_1[1,k],\ldots, x_m[1,k]$ è uno tra:
  \begin{itemize}
    \item $x_{a_k[h-1]}$, se $h\neq 0$, ovvero la riga del pannello
    corrispondente alla fine della run precedente a $r$ nella \textit{matrice
      PBWT}, se esistente
    \item $x_{a_k[e+1]}$, se $e\neq M-1$, ovvero la riga del pannello
    corrispondente all'inizio della run successiva a $r$ nella \textit{matrice
      PBWT}, se esistente
  \end{itemize}
  Avendo quindi i \textit{prefix array sample} che ci dicono a quale riga nel
  pannello corrispondano tali valori e conoscendo $MS[k-1].row$ è possibile
  calcolare $LCE(k,MS[k-1].row, a_k[h-1])$ e $LCE(k,MS[k-1].row, a_k[e+1])$. A
  questo punto si sceglie il suffisso comune più lungo tra le due, ovvero il
  maggiore tra i valori ritornati dalla \textit{LCE query} e si sceglie la riga
  corrispondente per proseguire. Si ha quindi o $MS[k].row=a_k[h-1]$ o
  $MS[k].row=a_k[e+1]$. In merito alla lunghezza, assumendo che la lunghezza
  maggiore delle due \textit{LCE query} sia $l$, si ha che:
  \[MS[k].len=\min(MS[k-1].len, l)+1\]
  In quanto la LCE query potrebbe restituire un valore più lungo dell'effettivo
  match con al query $z$ quindi si sceglie il minimo tra le due lunghezze,
  ottenendo l'effettiva lunghezza del suffisso comune tra $z$ e la nuova riga
  scelta fino a $k-1$, e lo si 
  incrementa di uno, contando il match ottenuto in colonna $k$
\end{enumerate}
\dc{Sistemare esempio.}
\begin{esempio}
  Riprendiamo l'esempio \ref{es:thr}, visto per il calcolo
  tramite \textit{threshold}. \\
  Senza usare le \textit{threshold}, nella medesima situazione si dovrebbe
  calcolare, avendo che $MS[1].row=19$ e $MS[1].len =2$:
  \[LCE(2, x_{19}, x_{16}) = \mbox{"01"} \implies|LCE(2, x_{19}, x_{16})|=2\]
  \[LCE(2, x_{19}, x_{17}) = \mbox{"1"} \implies|LCE(2, x_{19}, x_{17})|=1\]
  Come verificabile dal pannello presente all'esempio \ref{es:pbwt1}.\\
  Si ha quindi che $MS[2].row=16$. Inoltre, sempre per quanto detto sopra:
  \[MS[2].len=\min(MS[1].len, 2)+1=2+1=3\]
\end{esempio}
Con questa soluzione, il cui pseudocodice è consultabile all'algoritmo
\ref{algo:matchlce}, quindi: 
\begin{itemize}
  \item non si necessita di tenere in memoria le informazioni per le
  \textit{threshold}
  \item si permette il calcolo dell'array $MS$ in una singola scansione del
  pattern 
  \item non si necessita di memorizzare l'intero array $MS$ ma solamente quattro
  variabili relative alla coppia
  $(row,len)$ corrente e quella precedente 
\end{itemize}
Dal punto di vista della complessità temporale, per il calcolo dell'array $MS$,
si hanno solo due casistiche 
possibili, al variare della componente di mapping. 
Nel caso della componente \texttt{MAP-INT}, avendo $s$ lunghezza della singola
parola prodotta dall'\textit{SLP} e $\rho$ numero medio di run per colonna, si
ha un tempo proporzionale, dovendo iterare la query, fare il mapping e usare la
componente \texttt{LCE}, a: 
\begin{equation}
  \label{eq:mslce1}
  \mathcal{O}(N(\log \rho+\log s))
\end{equation}
Mentre nel caso dell'uso della componente \texttt{MAP-BV} si ha tempo
proporzionale a:
\begin{equation}
  \label{eq:mslce2}
  \mathcal{O}\left(N\left(\log \frac{M}{\rho}+\log s\right)\right)
\end{equation}
Infine, per il calcolo di tutte le righe del pannello per cui si ha uno
\textit{SMEM} si può fare riferimento all'equazione \ref{eq:msthr3slp}, avendo
la medesima situazione.\\
Si deduce quindi come la struttura \texttt{MAP-INT + LCE + PERM + PHI} sia, a
livello di tempo macchina, la soluzione più vantaggiosa usando la componente
\texttt{LCE}. Tale soluzione risulta, sempre nel contesto delle strutture basate
sulla componente \texttt{LCE}, essere anche la soluzione più vantaggiosa in
termini di memoria.\\
Si vedrà, sperimentalmente, nel capitolo \ref{reschap}, il
confronto con le altre strutture dati.
\begin{algorithm}
  \scriptsize
  \begin{algorithmic}[1]
    \Function{matches\_ms\_lce}{$z$}
    \State $ms_{row}\gets [0..0],\,\,ms_{len}\gets [0..0]$
    \Comment array $MS$ di lunghezza $|z|$
    \State $curr_{row}\gets
    rlpbwt[0].samples_{end}[|rlpbwt[0].samples_{end}|-1],\,\,curr_{index}\gets
    curr_{row}$ 
    \State $curr_{run}\gets index\_to\_run(curr_{index},0)$ \textbf{oppure}
    $curr_{run}\gets rank_h^0(curr_{index})$  
    \State $symb\gets get\_symbol(start_0, curr_{run})$
    \Comment \textbf{Costruzione dell'array $MS$}
    \For {\textit{every} $k\in[0, |z|)$}
    \If{$z[i]=symb$}
    \State $ms_{row}[k]\gets curr_{row}$
    \State \textbf{if} $k=0$ \textbf{then} $ms_{len}[k] \gets 1$ \textbf{else}
    $ms_{len}[k] \gets ms_{len}[k-1]+1$
    \State \hspace{-1.1mm}\textbf{if} $k\neq |z|-1$ \textbf{then}
    $(curr_{index},\,\,curr_{run},\,\,symb)\gets UPDATE(k, curr_{index},z)$ 
    % \If{$k\neq |z|-1$}
    % \State $(curr_{index},\,\,curr_{run},\,\,symb)\gets UPDATE(k, curr_{index},
    % z)$ 
    % \EndIf
    \Else
    \If{$|samples_{beg}^k|=1$}
    \State $ms_{row}[k]\gets M$
    \State $ms_{len}[k]\gets 0$
    \If{$k\neq |z|-1$}
    \State $curr_{row}\gets
    rlpbwt[k+1].samples_{end}[|rlpbwt[k+1].samples_{end}|-1]$
    \State $curr_{index}\gets M-1$
    \State $curr_{run}\gets index\_to\_run(curr_{index},k+1)$ \textbf{oppure}
    $curr_{run}\gets rank_h^{k+1}(curr_{index})$
    \State $symb\gets get\_symbol(start_{k+1}, curr_{run})$
    \EndIf
    \Else
    \If{$curr_{run}=|samples_{beg}^k|-1$}
    \State $curr_{index}\gets p_k[curr_{run}-1]$ \textbf{oppure}
    $curr_{index}\gets select_h^k(curr_{run})$
    \State $prev_{row}\gets
    samples_{end}^k[curr_{run}-1]$ 
    \State $lce\gets LCE(k, curr_{row}, prev_{row})$
    \State $ms_{row}[k]\gets prev_{row},\,\,curr_{row}\gets prev_{row}$
    \State \textbf{if} $k=0$ \textbf{then} $ms_{len}[k] \gets 1$ \textbf{else}
    $ms_{len}[k] \gets min(ms_{len}[k-1], |lce|)+1$ 
    % \If{$k=0$}
    % \State $ms_{len}[k] \gets 1$
    % \Else
    % \State $ms_{len}[k] \gets min(ms_{len}[k-1], |lce|)+1$
    % \EndIf
    \State \textbf{if} $k\neq |z|-1$ \textbf{then}
    $(curr_{index},\,\,curr_{run},\,\,symb)\gets UPDATE(k, curr_{index},z)$  
    % \If{$k\neq |z|-1$}
    % \State $(curr_{index},\,\,curr_{run},\,\,symb)\gets UPDATE(k, curr_{index},
    % z)$  
    % \EndIf    
    \ElsIf{$curr_{run}=0$}
    \State $curr_{index}\gets p_k[curr_{run}+1]$ \textbf{oppure}
    $curr_{index}\gets select_h^k(curr_{run}+1)+1$
    \State $next_{row}\gets samples_{beg}^k[curr_{run}+1]$ 
    \State $lce\gets LCE(k, curr_{row}, next_{row})$
    \State $ms_{row}[k]\gets next_{row},\,\,curr_{row}\gets next_{row}$
    \State \textbf{if} $k=0$ \textbf{then} $ms_{len}[k] \gets 1$ \textbf{else}
    $ms_{len}[k] \gets min(ms_{len}[k-1], |lce|)+1$
    % \If{$k=0$}
    % \State $ms_{len}[k] \gets 1$
    % \Else
    % \State $ms_{len}[k] \gets min(ms_{len}[k-1], |lce|)+1$
    % \EndIf
    \State \textbf{if} $k\neq |z|-1$ \textbf{then} $(curr_{index},\,\,
    curr_{run},\,\,symb)\gets UPDATE(k, curr_{index},z)$  
    % \If{$k\neq |z|-1$}
    % \State $(curr_{index},\,\,curr_{run},\,\,symb)\gets UPDATE(k, curr_{index},
    % z)$ 
    % \EndIf
    \Else
    \State $prev_{row}\gets samples_{end}^k[curr_{run}-1],\,\,next_{row}\gets
    samples_{beg}^k[curr_{run}+1]$ 
    \State $lce\gets \max (|LCE(k, curr_{row}, prev_{row})|, |LCE(k,
    curr_{row}, next_{row})|)$
    \State $curr_{row}\gets lce_{row}$
    \Comment $lce_{row}$ segnala l'indice della riga con \textit{LCE query} più
    lunga 
    \State $ms_{row}[k]\gets curr_{row}$
    \State \textbf{if} $k=0$ \textbf{then} $ms_{len}[k] \gets 1$ \textbf{else}
    $ms_{len}[k] \gets min(ms_{len}[k-1], |lce|)+1$
    % \If{$k=0$}
    % \State $ms_{len}[k] \gets 1$
    % \Else
    % \State $ms_{len}[k] \gets min(ms_{len}[k-1], |lce|)+1$
    % \EndIf
    \State \textbf{if} $k\neq |z|-1$ \textbf{then}
    $(curr_{index},\,\,curr_{run},\,\,symb)\gets UPDATE(k, curr_{index},z)$ 
    % \If{$k\neq |z|-1$}
    % \State $(curr_{index},\,\,curr_{run},\,\,symb)\gets UPDATE(k,curr_{index},
    % z)$ 
    % \EndIf
    \EndIf
    \EndIf
    \EndIf
    \EndFor
    
    \For {\textit{every} $k\in[0,|z|)$}
    \Comment \textbf{Calcolo dei match da $MS$}
    \If{$(ms_{len}[k]>1 \land ms_{len}[k]\geq ms_{len}[k+1])\lor(k = |z|-1 \land
    ms_{len}[k]\neq 0$}
    \State \textit{report degli SMEM di lunghezza $ms_{len}[k]$, terminanti in
    colonna $k$}
    \State \textit{con la riga $ms_{row}[k]$ e quelle estese da essa tramite
    la componente \texttt{PHI}} 
    \EndIf
    \EndFor
    \EndFunction
    
  \end{algorithmic}
  \caption{\footnotesize{Calcolo degli SMEM con aplotipo esterno con componenti
  \texttt{MAP-INT/BV} (i cui usi diversificati sono segnalati con ``oppure''),
  \texttt{LCE}, \texttt{PERM} e \texttt{PHI}.}}
  \label{algo:matchlce}
\end{algorithm}
\dc{Sistemare pseudocodice per non avere salvato intero $MS$}
\dc{Manca strima complessità}

% LocalWords:  pseudocodice


% sezione RLPBWT naive
% \section{RLPBWT naive}
Un primo approccio alla \textbf{compressione run-length} è stato quello di
semplicemente ``adattare'' quanto presentato da Durbin. Soprattutto a causa di
questo fattore tale approccio è stato nominato \textbf{RLPBWT naive}.\\
L'idea è stata quella di capire quali informazioni fossero necessarie al fine di
poter calcolare i match. Si è quindi partiti studiando quanto memorizzato da
Durbin stesso, pensando ad eventuali alternative.\\
Il dato fondamentale che la \textit{PBWT} tiene in memoria è \textit{il pannello
  $X$, con random access}. Ovviamente memorizzare l'intero pannello non era
possibile. D'altro canto l'idea dietro la 
\textbf{RLPBWT} è quella di memorizzare con \textit{compressione run-length}
la \textit{matrice PBWT}. La soluzione iniziale è stata quindi quella di
memorizzare gli indici delle \textit{teste di run}, ovvero gli indici iniziali
di ogni run. Ovviamente questa informazione non è sufficiente per poter sapere
se una run sia composta da simboli $\sigma=0$ o simboli
$\sigma=1$. Fortunatamente, essendo lo studio limitato, come per la
\textit{PBWT}, a pannelli costruiti su alfabeto binario, $\Sigma=\{0,1\}$, si è
potuto sfruttare il fatto che le run si alternano tra un carattere e
l'altro. Basta quindi tenere in memoria anche un valore booleano che permetta di
capire se la prima run sia una run di simboli $\sigma=0$. Infatti le run di
indice pari presentano lo stesso simbolo della prima run e quindi, dato un
qualsiasi indice di run, è possibile sapere quale sia il simbolo di tale run.\\
Il passaggio successivo è stato quello di capire se le informazioni necessarie
al mapping fossero tutte necessarie. In altri termini se, data la colonna $k$
nella \textit{matrice PBWT}, fossero necessari $c[k]$, $u_k$ e $v_k$. In merito
al valore $c[k]$, per quanto calcolabile in tempo $\mathcal{O}(r)$, dove $r$ è
il numero di run della colonna $k$-esima, si è deciso che si potesse calcolarlo
in fase di costruzione delle \textit{RLPBWT} e memorizzarlo esattamente come per
la \textit{PBWT}. In merito invece ai vettori $u_k$ e $v_k$ si è cercato un modo
per ottenerne una rappresentazione che implicasse avere un solo valore per ogni
run della colonna. In altri termini si è cercato di capire se fosse possibile
tenere in memoria $r$ valori che permettessero di effettuare comunque il
mapping, a partire da un indice arbitrario $i\in\{0,\ldots,N-1\}$. Anche in
questo caso l'alternanza data dal caso binario ha permesso di trovare una
semplice soluzione. I valori di $u_k$ e $v_k$ crescono infatti in modo
alternato. A seconda del simbolo $\sigma$ rappresentato in una data run infatti
si avrà che solo i valori dell'array relativo a tale simbolo, nel range di
indici di quella run, verranno incrementati ad ogni passo di una unità. Per fare
un semplice esempio, se siamo in una run di 0 e iteriamo virtualmente
all'interno di tale run, solo i valori di $u_k$, in quel
range di indici, cresceranno di volta in volta di uno mentre per $v_k$, nello
stesso range, si avrà sempre lo stesso valore.
\begin{esempio}
  Si vede un esempio per chiarire meglio quanto espresso in merito a $u_k$ e
  $v_k$.\\
  Sia data la seguente colonna:
  \[y^5=00101111000000000000\]
  Si hanno, oltre a $c[5]=15$:
  \begin{table}[H]
    \centering
    \begin{tabular}{c||cc|c|c|cccc|cccccccccccc}
      & 0 & 1 & 2 & 3 & 4 & 5 & 6 & 7 & 8 & 9 & 10 & 11 & 12 & 13 & 14 & 15 & 16
      & 17 & 18 & 19\\
      \hline
      \hline
      $y^5$ & 0 & 0 & 1 & 0 & 1 & 1 & 1 & 1 & 0 & 0 & 0 & 0 & 0 & 0 & 0 & 0 & 0
      & 0 & 0 & 0\\
      \hline
      \hline
      $u_5$ & 0 & 1 & 2 & 2 & 3 & 3 & 3 & 3 & 3 & 4 & 5 & 6 & 7 & 8 & 9 & 10
      & 11 & 12 & 13 & 14\\
      \hline
      $v_5$ & 0 & 0 & 0 & 1 & 1 & 2 & 3 & 4 & 5 & 5 & 5 & 5 & 5 & 5 & 5 & 5 & 5
      & 5 & 5 & 5
    \end{tabular}
  \end{table}
\end{esempio}
Grazie a questa alternanza è quindi possibile memorizzare, per ogni indice di
testa di run $p$, tale che $p\neq 0$, solo il valore di $u_k[p]$ o $v_k[p]$,
rispettivamente se sia una run su simboli $\sigma=1$ o $\sigma=0$ (visto che,
esempio, se si analizza una run di zeri si avrà che solo i valori di $v_k$, nel
range della run, verranno incrementati ad ogni step). Per $p=0$ banalmente si ha
che $u_k[0]=v_k[0]=0$.\\


% % sezione RLPBWT bitvectors
% \section{RLPBWT con bitvectors}
Al fine di compiere un ulteriore passo verso la formulazione di una struttura
dati efficiente dal punto di vista dello spazio in memoria per la
\textbf{RLPBWT}, si è provveduto a modificare la versione \textit{naive} al fine
di introdurre l'uso dei \textbf{bitvectors}. Questo è stato fatto al fine di
ottenere una rappresentazione in memoria della stessa che fosse ancora più
efficiente. Come si vedrà questa versione intermedia non comporterà un
miglioramento effettivo del consumo di memoria ma permetterà di avere la base su
cui costruire le versioni successive.\\
L'idea è quindi quella di sostituire, data una colonna $k$, quanto necessario a
rappresentare le run (ovvero il vettore $p_k$ della variante naive) e quanto
necessario a permettere il mapping (ovvero il vettore $uv_k$).\\
In primis, per poter localizzare le run nella $k$-esima colonna, si è scelto di
usare un \textit{bitvector}, che denominiamo per praticità $h_k$, tale che
$|h_k|=N$. Formalmente si ha che:
\[h_k[i]=
  \begin{cases}
    1&\mbox{se } y^k[i]\neq y^k[i+1]\lor i==N-1\\
    0&\mbox{altrimenti}
  \end{cases},\forall i\in \{0,\ldots,N-1\}
\]
Informalmente, quindi, si ha che si ha 1 in $h_k$ in tutti gli indici
corrispondenti alla fine di una run.\\
Empiricamente ci si aspettano ``poche'' run all'interno di una colonna della
\textit{matrice PBWT}, per quanto già discusso nella sezione
\ref{secpbwt}. Avendo poche run ci si aspetta anche ``pochi'' 1 all'interno di
$h_k$, di conseguenza si è optato per usare gli \textbf{sparse bitvector} per la
memorizzazione in memoria di ogni $h_k$, ricordando che, secondo quanto
riportato per la libreria \textit{SDSL} \cite{sdsl}, tale variante richiede in
memoria, indicando con $R$ il numero di run:
\[\approx R\left(2+\log\frac{|h_k|}{R}\right)\mbox{ bit}\]
\textbf{VERIFICARE CHE SIANO BITS}\\
Più elaborata è la rappresentazione dei vettori $u_k$ e $v_k$. In questo caso si
è deciso, a differenza della rappresentazione unica vista nella \textit{RLPBWT
  naive}, di optare per due \textit{sparse bitvector}. In particolare, per il
vettore $u_k$, tale che $|u_k|=c[k]$, si ha che:
\[u_k[i]=
  \begin{cases}
    1&\mbox{se }i \mbox{ è il numero di simboli che contiene la
    }rank_{u_k}(i)\mbox{-esima run di 0}\\
    0&\mbox{altrimenti}
  \end{cases},
\]
\vspace{-5mm}
\[\forall i\in\{0,\ldots,|u_k|-1\}\]
Analogamente si definisce $v_k$, tale che $|v_k|=N-c[k]$ come:
\[v_k[i]=
  \begin{cases}
    1&\mbox{se }i \mbox{ è il numero di simboli che contiene la
    }rank_{v_k}(i)\mbox{-esima run di 1}\\
    0&\mbox{altrimenti}
  \end{cases},
\]
\vspace{-5mm}
\[\forall i\in\{0,\ldots,|v_k|-1\}\]
Si noti che:
\[rank_{h_k}(|h_k-1|)+1=([rank_{u_k}(|u_k-1|)+1)+[rank_{v_k}(|v_k-1|)+1\]
Ovvero il numero di 1 presenti in $h_k$ è pari alla somma di quelli presenti in
$u_k$ e $v_k$. Ne segue che, anche per questi ultimi due vettori, la scelta di
usare \textit{sparse bitvector} per la loro memorizzazione sia giustificata
dalla poca quantità, empiricamente, di simboli $\sigma=1$.\\
Si vede un esempio chiarificatore.
\begin{esempio}
  Sia data la seguente colonna:
  \[y^5=00101111000000000000\]
  Si ha quindi che:
  \[h_5=01110001000000000001\]
  Avendo appunto un numero di run pari a:
  \[rank_{h_5}(|h_5|)+1=4+1=5\]
  In merito alle run composte da simboli $\sigma=0$ si ha che:
  \[u_5=011000000000001\]
  Avendo infatti che si segnalano:
  \begin{itemize}
    \item la prima run composta da due simboli $\sigma=0$
    \item la seconda run composta da un solo simbolo $\sigma=0$
    \item la terza run composta da dodici simboli $\sigma=0$
  \end{itemize}
  Parlando invece di $v_5$ si ha:
  \[v_5=10001\]
  Avendo che:
  \begin{itemize}
    \item la prima run è composta da un solo simbolo $\sigma=1$
    \item la seconda run è composta da quattro $\sigma=1$
  \end{itemize}
\end{esempio}
Le restanti informazioni, ovvero, per la colonna $k$, il valore $c[k]$, il
booleano $start^k$ e l'LCP array $l_k$ sono le medesime della variante
\textit{naive} della \textit{RLPBWT} (motivo per quale solo a breve si
tratterà l'algoritmo di match).\\
Lo pseudocodice relativo alla costruzione della colonna $k$-esima della
\textbf{RLPBWT con bitvector} è disponile all'algoritmo \ref{algo:cosbv} (dove
sono presenti anche le istruzioni per le varianti che verranno trattate in
seguito).\\
\textbf{CAPIRE SE METTERE UNO PSEUDO A PARTE}

% % sezione mapping RLPBWT
% %\input{include/map_rlpbwt}

% % sezione match massimali di RLPBWT naive/bitvectors
% \section{Algoritmo per match massimali}

% % sezione RLPBWT ms
% \section{RLPBWT con matching statistics}
Le precedenti versioni della \textit{RLPBWT}, come anticipato, hanno permesso di
poter ideare una variante \textit{run-length encoded} della \textit{PBWT}.
In realtà anche in questo caso c'è stata una fase transitoria di sviluppo,
avendo due varianti della struttura, che verranno introdotte a breve.\\
Il fine era quello di ottenere quanto visto per la \textbf{RLBWT} anche per la
variante posizionale, ovvero i concetti di:
\begin{itemize}
  \item \textit{matching statistics}
  \item \textit{threshold}
  \item \textit{LCE query}
\end{itemize}
A tal fine, come per la \textit{RLBWT}, si necessita di \textit{random access}
al pannello. A causa di questo si sono avute due varianti in fase di sviluppo:
\begin{itemize}
  \item una prima, ancora in ottica di ``studio introduttivo'', dove il pannello
  viene memorizzato come \textit{vettore di bitvector classici}
  \item una seconda, definitiva, dove il pannello è memorizzato come
  \textit{SLP}, nelle modalità introdotte nella sottosezione \ref{subslp}
\end{itemize}
Queste versioni, a loro volta, hanno permesso, in primis, l'ideazione di un
algoritmo che sfruttasse l'idea delle threshold, come visto per la \textit{BWT}
classica con \textit{MONI} \cite{moni}, e poi, per quella basata
sull'\textit{SLP}, di uno basato sulle \textit{LCE query}, come per
\textit{PHONI} \cite{phoni}. Quest'ultima, con l'aggiunta della
\textit{struttura per la funzione $\varphi$}, sarà l'implementazione definitiva,
per questa tesi, della \textit{RLPBWT}.\\
Avendo in memoria il pannello si può quindi fare a meno dell'array \textit{LCP}
della colonna $k$-esima, avendo quindi che, per ogni colonna $k$, si ha in
memoria:
\begin{itemize}
  \item un booleano $start^k$ per specificare se la colonna presenta la prima
  run costruita su simboli $\sigma =0$
  \item un bitvector sparso $h_k$ per indicare l'inizio delle run
  \item il valore $c[k]$ per sapere quanti simboli $\sigma=0$ si hanno nella
  colonna 
  \item i valori $u_k$ e $v_k$ per il mapping
  \item i cosiddetti \textbf{prefix array samples}, ovvero i valori del
  \textbf{prefix array} di inizio e fine di ogni run. Si noti quindi che, anche
  in questo caso, si ha un'informazione in memoria proporzionale al numero di
  run $r$ 
\end{itemize}
In pratica si hanno in memoria le stesse informazioni della \textit{RLPBWT con
  bitvector} al più di $l_k$, ovvero l'\textit{array LCP} della colonna
$k$-esima, e dei \textit{prefix array samples}. 
\subsection{Matching statistics per la RLPBWT}
La definizione formale per il concetto di \textbf{matching statistics}, nonché
il calcolo dell'array stesso, vista per
la \textit{RLBWT} deve essere ovviamente riadattata allo studio di match tra un
aplotipo e un pannello di aplotipi.
\begin{definizione}
  Dato un pannello $X$, di dimensioni $M\times N$, con $M$ individui e $N$ siti,
  e un aplotipo esterno/pattern $z$, tale che $|z|=N$, si definisce matching
  statistics di $z$ su $X$ un array $MS$ di coppie $(row,len)$, di lunghezza
  $N$, tale che (avendo che $x_i$ indica l'$i$-esima riga del pannello $X$): 
  \begin{itemize}
    \item $x_{MS[i].row}[i-MS[i].len+1,i]=z[i-MS[i].len+1,i]$, ovvero si ha che
    l'aplotipo query ha un match, terminante in colonna $i$, con la riga
    $MS[i].row$  
    \item $z[i-MS[i].len,i]$ non è un suffisso terminante in colonna $i$ di un
    qualsiasi sottoinsieme di righe di $X$. In altri termini il match non deve
    essere ulteriormente estendibile a sinistra
  \end{itemize}
\end{definizione}
Inoltre, analogamente al caso della variante classica, si ha il seguente lemma.
\begin{lemma}
  Dato un pannello $X$, di dimensioni $M\times N$, con $M$ individui e $N$
  siti, un aplotipo esterno/pattern $z$, tale che $|z|=N$, e il corrispondente
  array di matching statistics $MS$ si ha che:
  \[z[i-l+1:i]\]
  è un \textbf{MEM} di lunghezza $l$ in con la riga $MS[i].row$ del pannello
  $X$ sse: 
  \[MS[i].len=l\land(i=N\lor MS[i].len\geq MS[i+1].len)\]
\end{lemma}
\textit{Si vedrà in sezione \ref{secphi} come calcolare, a partire da tali
  \emph{MEM}, tutte le righe del panello per le quali si ha lo stesso
  \textup{MEM}}.\\
Il calcolo dell'array $MS$ di $z$ rispetto al pannello $X$ si basa su due fasi:
\begin{enumerate}
  \item la fase di \textbf{extension}
  \item la fase di \textbf{bootstrap}
\end{enumerate}
Si assuma di avere due indici $i$ e $j$, $0\leq i\leq j\leq N$, tali per cui
$z[i,j]$ è un suffisso di uno tra $x_1[1,j]$, \ldots, $x_M[1,j]$. \\
La \textbf{fase di extension} estende il match di $z[i,j]$ a $z[i,j+1]$ sse:
\begin{itemize}
  \item $j<M$
  \item $z[i,j+1]$ è un suffisso di uno tra $x_1[1,j+1]$, $\ldots$, $x_M[1,j+1]$
\end{itemize}
D'altro canto la \textbf{fase di bootstrap} cerca il più piccolo indice $i'$,
avendo $i\leq i'\leq j$, tale per cui $z[i',j]$ è un suffisso di uno tra
$x_1[1,j+1]$, $\ldots$, $x_M[1,j+1]$.\\
Si ha quindi il computo di ogni valore $MS[i]$, $\forall i\in[0,N)$, dell'array
delle \textit{matching statistics}:
\begin{itemize}
  \item si assume inizialmente che $MS[0].len=0$
  \item si applica la \textit{fase di bootstrap} per cercare il minimo indice
  $i'$, avendo $i\leq i'$, tale che $z[i',i'+MS[i].len]$ è un suffisso di uno
  tra $x_1[1,i'+MS[i].len],$ \ldots$, x_M[1,i'+MS[i].len]$. Inoltre, per
  minimalità di $i'$ si ha che, $\forall i<j<i'$, $MS[j].len=MS[j-1].len+1$
  \item a questo punto si itera la \textit{fase di estensione} per trovare il
  più lungo prefisso $z[i',k]$ che è anche un suffisso di uno tra $x_1[1,k]$,
  $\ldots$, $x_M[1,k]$, avendo che $MS[i'].len=k-i'+1$
  \item avendo che $i'>i$ si può procedere induttivamente al calcolo dell'array
  $MS$ 
\end{itemize}
\textbf{PARTE PRESA DAL PAPER: RIVEDERE PROFONDAMENTE}.\\
In altri termini, più ``pratici'', il calcolo dell'array $MS$ avviene nel
seguente modo:
\begin{itemize}
  \item si parte da una riga arbitraria $i$ della prima colonna
  \item se $x_i[0]=z[0]$ si procede salvando $MS[0].row=i$
  \item qualora si abbia $x_i[0]\neq z[0]$ si seleziona o l'ultima riga della
  run precedente o la prima riga della run successiva a quella a cui appartiene
  la riga $i$. Tale riga, $j$, verrà salvata in $MS$, avendo $MS[0].row=j$
  \item a questo punto si effettua il mapping verso la colonna successiva, $k$,
  e, a seconda di avere un match con $z[k]$ si procede come nei casi visti sopra
\end{itemize}
Si noti che non si è parlato di come calcolare i vari $MS[i].len$, questo in
quanto si hanno due soluzioni (che verranno poi approfondite), che riprendono
appunto \textit{MONI} e \textit{PHONI} per la \textit{RLBWT};
\begin{enumerate}
  \item si possono usare le threshold per capire che nuova riga selezionare in
  caso di mismatch. In tal caso i vari $MS[i].len$ devono essere calcolati dopo
  il calcolo di $MS[i].row$ tramite \textit{random access} al panel
  \item si possono usare le \textit{LCE query} per capire che nuova riga
  selezionare in caso di mismatch e in tal caso il calcolo delle $MS[i].len$
  avviene in contemporanea 
\end{enumerate}
Prima di procedere con i dettagli dei due metodi è bene proporre un veloce
esempio di array $MS$.
\begin{esempio}
  \label{es:ms}
  Si riprenda nuovamente l'esempio \ref{es:algo5}, con un pannello e i match con
  la query $z$:
  \begin{figure}[H]
    \centering
    \includegraphics[scale = 0.365]{img/pbwtmatch.pdf}
  \end{figure}
  In tal caso l'array $MS$ sarebbe, avendo scelto come riga iniziale la 19:
  \begin{table}[H]
    \footnotesize{}
    \centering
    \begin{tabular}{c|ccccccccccccccc}
      $k$ & 00 & 01 & 02 & 03 & 04 &  {\color{nordgreen}05} & 06 & 07 & 08
      &  {\color{nordgreen}09} & 10 &  {\color{nordgreen}11} & 12 & 13
      &  {\color{nordgreen}14} \\
      \hline
      \hline
      $z$ & 0 & 1 & 0 & 0 & 1 &  {\color{nordgreen}0} & 1 & 0 & 0
      &  {\color{nordgreen}0} & 1 &  {\color{nordgreen}1} & 1 & 0
      &  {\color{nordgreen}1} \\
      \hline
      $row$ & 19 & 19 & 16 & 15 & 13 &  {\color{nordgreen}13} & 19 & 19 & 19
      &  {\color{nordgreen}19} & 11 &  {\color{nordgreen}11} & 17 & 17
      &  {\color{nordgreen}17} \\
      $len$ & 1 & 2 & 3 & 4 & 5 & {\color{nordgreen}6} & 4 & 5 & 6
      & {\color{nordgreen}7} & 4 & {\color{nordgreen}5} & 2 & 3
      & {\color{nordgreen}4}\\
    \end{tabular}
  \end{table}
  Dove si possono riconoscere i vari \textit{MEM}, la cui colonna di fine è
  segnalata in verde, secondo la definizione data
  sopra (anche in questo caso i dettagli del calcolo
  verranno esplicitati successivamente). 
\end{esempio}
\subsection{Threshold per la RLPBWT}
Come anticipato una prima strategia per la scelta di una nuova riga $j$, qualora
la precedente riga $i$ comporti un mismatch in colonna $k$ con l'aplotipo query,
avendo $x_i[k]\neq z[k]$ è l'utilizzo delle \textbf{threshold}.
\begin{definizione}
  Data la colonna $k$-esima della \textbf{matrice PBWT}, $y^k$, memorizzata
  tramite compressione \textbf{run-length} e data la run $j$-esima, indicizzata
  da $i$ a $i'$, si definisce \textbf{threshold} come l'indice del minimo valore
  \textit{LCP}, che ricordiamo essere calcolato sull'ordinamento inverso,
  compreso negli indici della run, compreso l'eventuale 
  $LCP_k[i'+1]$, qualora $i'\neq M-1$. Si noti che quest'ultimo valore, se
  esistente, deve essere considerato in quanto per il suo calcolo, come
  specificato nei preliminari alla sezione \ref{secpbwt}, si prende in
  considerazione $y^k_{i'}$ e $y^k_{i'+1}$.
\end{definizione}
Con tale informazione, unita ai \textit{prefix array sample}, si può quindi
ottenere un comportamento analogo a quanto si ottiene con l'\textbf{R-index} per
la \textit{RLBWT}.\\
Sia infatti data $t$ la posizione della \textit{threshold} nella run corrente,
in colonna $k$, e
si supponga che tale run, con testa all'indice $h$, non sia associata al simbolo
desiderato, ovvero $z[k]$. Si supponga che, con il mapping, si sia arrivati
all'indice $i$ della colonna $k$. Si supponga inoltre che la run successiva
abbia testa in indice $e$. Si hanno due casi possibili, denotando con
$LCS(x,y)$ il \textit{longest common suffix} tra le stringhe $X$ e $Y$ e con
$a_k$ il \textit{prefix array} in colonna $K$:
\begin{enumerate}
  \item $i<t$ allora, per definizione di \textit{threshold}:
  \[LCS(z[0,k], x_{a_{k}[h-1]}[0,k])\geq LCS(z[0,k], x_{a_{k}[e]}[0,k])\]
  Quindi si ha che $MS[k].row=a_{k}[h-1]$ e il mapping potrà proseguire
  dall'indice $h-1$
  \item  $i\geq t$ allora, per definizione di \textit{threshold}:
  \[LCS(z[0,k], x_{a_{k}[h-1]}[0,k])\leq LCS(z[0,k], x_{a_{k}[e]}[0,k])\]
  Quindi si ha che $MS[k].row=a_{k}[e]$ e il mapping potrà proseguire
  dall'indice $e$
\end{enumerate}
Qualora una colonna presenti solo simboli $\sigma\neq z[k]$, per convenzione, si
imposta che $MS[k].row = M$ e si ricomincia, in colonna $k+1$, dall'ultima
posizione, indicizzata nel pannello originale dal valore finale del
\textit{prefix array sample} dell'ultima run.\\
In termini implementativi anche le posizioni delle \textit{threshold} vengono
memorizzate tramite un \textit{bitvector sparso} per ogni colonna $k$, avendo
che, qualora il minimo \textit{LCP} si ritrovi nell'indice della testa della run
successiva, la posizione della threshold verrà comunque memorizzata all'indice
della coda della run corrente. Purtroppo questa è una situazione di ambiguità,
avendo che, seguendo la definizione sopra, avendo la \textit{threshold} a fine
run, bisognerebbe scegliere la testa della run successiva, qualora l'indice $i$
si trovi esattamente a fine run. Invece, qualora la
\textit{threshold} sia a fine run a causa del fatto che il minimo \textit{LCP}
si trovi nella testa della run successiva, bisogna scegliere la coda della run
precedente. L'unico modo per disambiguare è quindi effettuare \textit{random
  access} al pannello per vedere quale sia la
soluzione migliore, ovvero quale tra la coda della run precedente e la testa
della run successiva siano relative alla riga del pannello originale con un
suffisso comune alla query più lungo.\\
Purtroppo non è possibile salvare la threshold direttamente nella testa della
run successiva in quanto questa potrebbe essere anche la posizione della
threshold della run successiva e avere due threshold sovrapposte impedirebbe di
capire a quale run appartiene una certa threshold, tramite la funzione
\textit{rank}. \\
Tale bitvector deve essere quindi aggiunto alle informazioni memorizzate per
ogni singola colonna. Lo pseudocodice per la costruzione di una colonna con
anche il bitvector per le \textit{threshold} è consultabile all'algoritmo
\ref{algo:cosbv}. 
\begin{esempio}
  \label{es:thr}
  Si vede quindi un esempio di funzionamento delle threshold.\\
  Si prenda pannello visto all'esempio \ref{es:pbwt1} e si effettui la
  permutazione secondo $a_2$:
  \begin{table}[H]
    \centering
    \footnotesize
    \begin{tabular}{c|cc|c|cccccccccccc}
      X & 00 & 01 & 02 & 03 & 04 & 05 & 06 & 07 & 08 & 09 & 10 & 11 & 12 & 13
      & 14 \\
      \hline
      00 & 1 & 0 & 0 & 1 & 0 & 0 & 0 & 0 & 0 & 0 & 0 & 1 & 1 & 0 & 1 \\
      01 & 1 & 0 & 0 & 1 & 1 & 0 & 0 & 1 & 0 & 0 & 0 & 0 & 0 & 1 & 1 \\
      02 & 1 & 0 & 0 & 1 & 1 & 0 & 0 & 1 & 0 & 0 & 0 & 1 & 0 & 0 & 1 \\
      03 & 1 & 0 & 0 & 1 & 1 & 0 & 0 & 1 & 0 & 0 & 0 & 1 & 0 & 0 & 1 \\
      04 & 0 & 1 & 0 & 1 & 0 & 1 & 0 & 0 & 0 & 0 & 0 & 1 & 0 & 0 & 1 \\
      05 & 0 & 1 & 0 & 1 & 0 & 1 & 0 & 0 & 0 & 0 & 0 & 1 & 0 & 0 & 1 \\
      06 & 0 & 1 & 0 & 1 & 0 & 1 & 0 & 0 & 0 & 0 & 0 & 1 & 0 & 0 & 1 \\
      07 & 0 & 1 & 0 & 1 & 0 & 1 & 0 & 0 & 0 & 0 & 0 & 0 & 1 & 0 & 1 \\
      08 & 0 & 1 & 0 & 0 & 1 & 0 & 0 & 0 & 0 & 1 & 1 & 1 & 0 & 0 & 1 \\
      09 & 0 & 1 & 0 & 1 & 0 & 0 & 0 & 0 & 1 & 0 & 0 & 0 & 0 & 1 & 1 \\
      10 & 0 & 1 & 0 & 1 & 0 & 0 & 0 & 0 & 1 & 0 & 0 & 0 & 0 & 1 & 1 \\
      11 & 0 & 1 & 0 & 0 & 1 & 0 & 0 & 0 & 0 & 0 & 1 & 1 & 0 & 0 & 0 \\
      12 & 0 & 1 & 0 & 0 & 1 & 0 & 0 & 0 & 1 & 0 & 1 & 1 & 0 & 0 & 1 \\
      13 & 0 & 1 & 0 & 0 & 1 & 0 & 0 & 0 & 1 & 0 & 1 & 1 & 0 & 0 & 1 \\
      14 & 0 & 1 & 0 & 0 & 0 & 0 & 0 & 0 & 1 & 0 & 0 & 0 & 1 & 0 & 1 \\
      15 & 0 & 1 & 0 & 0 & 0 & 0 & 0 & 0 & 1 & 0 & 0 & 0 & 1 & 0 & 1 \\
      16 & 0 & 1 & 0 & 1 & 0 & 0 & 0 & 0 & 0 & 0 & 0 & 1 & 1 & 0 & 1 \\
      17 & 0 & 1 & 1 & 0 & 1 & 0 & 0 & 0 & 0 & 0 & 0 & 1 & 0 & 0 & 1 \\
      18 & 0 & 1 & 1 & 0 & 1 & 0 & 1 & 0 & 0 & 0 & 0 & 0 & 1 & 0 & 1 \\
      19 & 1 & 1 & 0 & 0 & 0 & 1 & 0 & 0 & 0 & 0 & 0 & 1 & 1 & 0 & 1 \\
    \end{tabular}
  \end{table}
  Si prenda la seconda run, di simboli $\sigma=1$, indicizzata tra 17 e 18. \\
  Si supponga che, tramite il mapping, si sia arrivati alla riga 17 ma che si
  abbia $z[2]=0$. la scelta è quindi tra la coda della run precedente, avendo
  che $a_2[16]=16$ o la testa della run successiva, avendo che $a_2[19]=17$. Si
  può notare come il minimo \textit{LCP} si trovi, per la 
  run, all'indice 18 (a causa del fatto che il minimo \textit{LCP} è all'indice
  19, quello della testa della run successiva). Si può quindi proseguire o con
  la riga. Questo significa che il suffisso comune più lungo con la query si ha
  con la riga 16 del pannello, per definizione di threshold, avendo che questa
  sarà memorizzata nell'array $MS$:
  \[MS[2].row=16\]
  Successivamente, tramite \textit{random access} al testo, confrontando la riga
  $x_{16}$ e la query $z$, fino alla colonna $k=2$, si potrà calcolare che
  $MS[2].len=3$. 
\end{esempio}
\textbf{SISTEMARE ESEMPIO}\\
Una volta computato tutti i valori $MS[i].row$ per calcolare i vari $MS[i].len$
si scorre da sinistra a destra calcolando la lunghezza del match facendo random
access al pannello e confrontando la query $z$ con la riga $MS[i].row$. Si
assuma infatti di aver calcolato $MS[i-1].len$ e di voler calcolare $MS[i].len$.
Si hanno tre casi possibili:
\begin{enumerate}
  \item $MS[i].row=M$ e in tal caso, avendo segnalata l'inesistenza di alcun
  match, si ha che $MS[i].len=0$
  \item $MS[i].row=MS[i-1].row$, avendo $i\neq 0$ e $MS[i-1].len\neq 0$, allora
  si sta seguendo la stessa riga che si seguiva in colonna $i-1$ e quindi,
  banalmente, $MS[i].len=MS[i-1].len+1$
  \item in qualsiasi altro caso bisogna confrontare, a partire dalla colonna
  $i$, la query 
  $z$ con la riga $MS[i].row$ del pannello da destra a sinistra, fino a che non
  si trova un mismatch, calcolando la lunghezza $l$ del suffisso comune tra esse
  e memorizzando tale valore: $MS[i].len=l$
\end{enumerate}
In fase di costruzione delle lunghezze è possibile anche riportare i
\textit{MEM}, terminanti in colonna $i$, qualora:
\begin{itemize}
  \item $MS[i].len\geq MS[i+1].len \land MS[i].len\neq 0$
  \item si è arrivati a fine array, avendo $i=N-1\land MS[i].len\neq 0$
\end{itemize}
Come si può verificare nell'esempio \ref{es:ms}.\\
L'algoritmo per il match tramite \textit{threshold} è visualizzabile
all'algoritmo \ref{algo:matchthr}.
\textbf{CAPIRE SE DESCRIVERE ALGORITMO}
\subsection{LCE query per la RLPBWT}
La memorizzazione di un \textit{bitvector sparso} atto a rappresentare le
posizioni delle threshold ha però un costo in memoria. Inoltre, per ora, si è
parlato di tenere in memoria il pannello sotto forma di vettore di
\textit{bitvector} e anche questo ha un costo non indifferente in termini di
spazio necessario.\\
Per arrivare all'implementazione conclusiva della \textit{RLPBWT} si è quindi
optato per la memorizzazione del pannello sotto forma di \textit{SLP}, struttura
che, come anticipato, permette anche di eseguire efficientemente le \textit{LCE
  query}.
\begin{definizione}
  Dato un pannello $X$, $M\times N$, e due righe $x_i$ e $x_j$ tali che $0\leq
  i <m$ e $0\leq j <M$, con $i\neq j$, si definisce \textbf{LCE query} il
  suffisso comune più lungo tra le due stringhe. Per comodità nella sezione si
  definisce la funzione: 
  \[LCE_k(x_i,x_j)=l\]
  dove $l$ è la lunghezza del suffisso comune più lungo tra le due stringhe
  terminante in colonna $k-1$.
\end{definizione}
\subsubsection{Compressione del panel}
Bisogna quindi capire in primis come costruire l'\textit{SLP}. In primis, le
librerie per la costruzione di tale struttura assumono un input
``monodimensionale'', ovvero una singola sequenze lineare. Inoltre, anche per
permettere la costruzione efficiente della \textit{PBWT}, e conseguentemente
della \textit{RLPBWT}, il pannello in input risulta essere trasposto, avendo che
le righe nel file in input rappresentano i siti e non gli individui. Bisogna
quindi in primis trasporre tale pannello. Per procedere ulteriormente bisogna
però ricordare che sull'\textit{SLP} si avrà 
necessità di effettuare \textit{LCE query} che però, si anticipa, nel nostro
pannello, devono essere fatte tra due righe da destra a sinistra (a differenza
di quanto visto nel caso standard dove si confrontavano prefissi comuni). Per
rendere possibile questa operazione quindi il pannello deve essere sia 
salvato come un'unica riga, per ottenerne l'\textit{SLP}, che ``da destra a
sinistra'', per permettere le \textit{LCE query}. Si procede quindi concatenando
ogni riga, selezionandole consecutivamente e leggendone i singoli elementi da
destra a sinistra.
\begin{esempio}
  Si vede quindi un breve esempio.\\
  Si assuma di avere il seguente pannello nel file in input.
  \[
    X=\left[
      \begin{matrix}
        0 & 0 & 1 & 0 & 0\\
        1 & 1 & 1 & 0 & 1\\
        0 & 1 & 1 & 1 & 1\\
        0 & 0 & 0 & 1 & 0
      \end{matrix}
    \right]
  \]
  Dove però come detto le righe sono i siti e le colonne i sample. Per ottenere
  l'\textit{SLP} biosgna quindi, in primis, trasporre la matrice:
  \[
    X^T=\left[
      \begin{matrix}
        0 & 1 & 0 & 0\\
        0 & 1 & 1 & 0\\
        1 & 1 & 1 & 0\\
        0 & 0 & 1 & 1\\
        0 & 1 & 1 & 0
      \end{matrix}
    \right]
  \]
  A questo punto bisogna considerare l'ordine in cui si vorranno effettuare le
  \textit{LCE query}.
  Ad esempio, prendendo la seconda e la terza riga, facendo partire il confronto
  dall'ultima colonna, avremmo una \textit{LCE
    query} lunga 3, terminante nella prima colonna esclusa:
  \begin{figure}[H]
    \centering
    \includegraphics[scale = 0.38]{img/slppanel.pdf}
  \end{figure}
  Si procede quindi salvano la sequenza lineare relativa al pannelli come
  descritto sopra,
  ottenendo, con colorate gli stessi risultati della query fatta
  sopra:
  \[0010\,\,{\color{nordgreen}011}{\color{nordred}0}\,\,
    {\color{nordgreen}011}{\color{nordred}1} \,\,1100\,\,0110\]
  \textit{Si noti che qui si sono segnalate le varie righe con uno spazio ma
    solo per praticità ``visiva''.}
\end{esempio}
\textbf{ESEMPIO MAGARI DA SCRIVERE MEGLIO}
\subsubsection{LCE query}
Grazie all'uso delle \textit{LCE query} è quindi possibile calcolare l'array
delle \textit{matching statistics} in un solo scorrimento da sinistra a
destra. Infatti è possibile usare tali query per calcolare non solo quale nuova
sequenza scegliere in caso di mismatch con l'aplotipo query in colonna $i$, come
si faceva con l'uso delle \textit{threshold}, ma anche di computare la lunghezza
del suffisso comune tra essa e l'aplotipo query, calcolando nello stesso momento
sia $MS[i].row$ che $MS[i].len$.\\
Anche in questo caso, per convenzione, si inizia la computazione dell'ultima
riga della prima colonna.\\
Si illustra ora come computare l'array delle \textit{matching statistics}.
Si assuma di avere calcolato l'array $MS$ di una query $z$ rispetto al pannello
$X$. le cui righe si identificano tramite $x_i, \forall i\in\{0,M\}$, fino alla
colonna $k-1$. Sia $i$ 
l'indice di riga sulla \textit{matrice PBWT} al quale si è arrivati mediante il
mapping, avendo che tale riga è quella che ha il più lungo suffisso comune con
$z[1,k-1]$. Si assuma che l'indice $i$ appartenga alla run $r$, di simboli
$\sigma$, testa di indice $h$ e coda di indice $e-1$. Si hanno diversi casi:
\begin{enumerate}
  \item $z[k]=\sigma$, quindi la riga $i$ può essere usata per estendere il
  match, avendo che $MS[k].row=MS[k-1].row$ e $MS[k].len=MS[k-1].len+1$, e per
  proseguire col mapping in colonna $k+1$
  \item $z[k]\neq\sigma$ e si una sola run in colonna $k$, avendo quindi che non
  si possono avere match. Per convenzione, si
  imposta che $MS[k].row = M$ e $MS[k].len=0$. Infine si ricomincia, in colonna
  $k+1$, dall'ultima posizione, indicizzata nel pannello originale dal valore
  finale del \textit{prefix array sample} dell'ultima run
  \item $z[k]\neq\sigma$ ma si hanno anche altre run, dovendo quindi scegliere
  la nuova riga da seguire. Si ha che il più lungo suffisso di $z[1,k]$ che è
  anche suffisso di $x_1[1,k],\ldots, x_m[1,k]$ è uno tra:
  \begin{itemize}
    \item $x_{a_k[h-1]}$, se $h\neq 0$, ovvero la riga del pannello
    corrispondente alla fine della run precedente a $r$ nella \textit{matrice
      PBWT}, se esistente
    \item $x_{a_k[e+1]}$, se $e\neq M-1$, ovvero la riga del pannello
    corrispondente all'inizio della run successiva a $r$ nella \textit{matrice
      PBWT}, se esistente
  \end{itemize}
  Avendo quindi i \textit{prefix array sample} che ci dicono a quale riga nel
  pannello corrispondano tali valori e conoscendo $MS[k-1].row$ è possibile
  calcolare $LCE(MS[k-1].row, a_k[h-1])$ e $LCE(MS[k-1].row, a_k[e+1])$. A
  questo punto si sceglie il suffisso comune più lungo tra le due, ovvero il
  maggiore tra i valori ritornati dalla \textit{LCE query} e si sceglie la riga
  corrispondente per proseguire. Si ha quindi o $MS[k].row=a_k[h-1]$ o
  $MS[k].row=a_k[e+1]$. In merito alla lunghezza, assumendo che il miglior
  valore ritornato dalle due \textit{LCE query} sia $l$, si ha che:
  \[MS[k].len=\min(MS[k-1].len, l)+1\]
  In quanto la LCE query potrebbe restituire un valore più lungo dell'effettivo
  match con al query $z$ quindi si sceglie il minimo tra le due lunghezze,
  ottenendo l'effettiva lunghezza del suffisso comune tra $z$ e la nuova riga
  scelta fino a $k-1$, e lo si 
  incrementa di uno, contando il match ottenuto in colonna $k$
\end{enumerate}
\textbf{SISTEMARE}
\begin{esempio}
  Riprendiamo l'esempio \ref{es:thr}, visto per il calcolo
  tramite \textit{threshold}. \\
  Senza usare le \textit{threshold}, nella medesima situazione si dovrebbe
  calcolare, avendo che $MS[1].row=19$ e $MS[1].len =2$:
  \[LCE_1(x_{19}, x_{16})=2\]
  \[LCE_1(x_{19}, x_{17})=1\]
  Come verificabile dal pannello presente all'esempio \ref{es:pbwt1}.\\
  Si ha quindi che $MS[2].row=16$. Inoltre, sempre per quanto detto sopra:
  \[MS[2].len=\min(MS[1].len, 2)+1=2+1=3\]
\end{esempio}
Con questa variante quindi:
\begin{itemize}
  \item non si necessita di tenere in memoria le informazioni per le
  \textit{threshold}
  \item si tiene in memoria il pannello sotto forma di \textit{SLP}, soluzione
  vantaggiosa dal punto di vista della memoria anche se svantaggiosa da quello
  temporale (come descritto nella sezione \ref{secslp})
  \item si permette il calcolo dell'array $MS$ in una singola scansione del
  pattern 
\end{itemize}
\textbf{Essendo lo scopo principale della tesi la riduzione dello spazio
  occupato dalla struttura dati questa è la soluzione migliore, avendo in
  memoria una struttura run-length encoded per la PBWT in grado di permettere
  pattern matching con un aplotipo esterno}.


% % sezione per struttura phi
% \section{Funzione Phi}
\label{secphi}
L'ottenimento dell'array \textit{matching statistics} permette di sapere solo
l'indice di una della righe del pannello per le quali si ha un match con
l'aplotipo query. Analogamente a quanto discusso in \textit{PHONI} \cite{phoni},
anche per la \textit{RLPBWT} si è pensato a due funzioni, $\boldsymbol\varphi$ e
$\boldsymbol\varphi\mathbf{^{-1}}$, per il riconoscimento di tutte le
righe del pannello per cui si ha il match.\\
L'intuizione alla base del ragionamento è molto semplice. Nell'ordinamento alla
colonna $k$-esima, dato da $a_k$, tutte le righe per le quali si ha un match
sono poste consecutivamente, questo a causa del fatto che l'ordinamento è
lessicografico.
\begin{definizione}
  Dati:
  \begin{itemize}
    \item un pannello $X$, di dimensioni $N\times M$
    \item una colonna $k$, il \textit{prefix array} $a_k$ e la sua permutazione
    inversa $\alpha_k$
  \end{itemize}
  Si definiscono formalmente:
  \[\varphi_k(p)=
    \begin{cases}
      null&\mbox{se }\alpha_k[p]=0\\
      a_k[\alpha_k[p]-1]&\mbox{altrimenti}
    \end{cases},\forall p\in\{0,M-1\}
  \]
  \[\varphi^{-1}_k(p)=
    \begin{cases}
      null&\mbox{se }\alpha_k[p]=M-1\\
      a_k[\alpha_k[p]+1]&\mbox{altrimenti}
    \end{cases},\forall p\in\{0,M-1\}
  \]
  In altri termini, avendo $a_k[j]=p$ si ha che:
  \[\varphi_k(p)=
    \begin{cases}
      null&\mbox{se }j=0\\
      a_k[j-1]&\mbox{altrimenti}
    \end{cases},\forall p\in\{0,M-1\}
  \]
  \[\varphi^{-1}_k(p)=
    \begin{cases}
      null&\mbox{se }j=M-1\\
      a_k[j+1]&\mbox{altrimenti}
    \end{cases},\forall p\in\{0,M-1\}
  \]
\end{definizione}
\textbf{VERIFICARE DEFINIZIONE IN QUANTO ``NUOVA''}
\begin{esempio}
  Per praticità si riporta un breve esempio.\\
  Si ipotizzi di avere, come per l'esempio \ref{es:pbwt1}:
  \[a_6=[14,15,0,9,10,16,8,11,12,13,18,19,1,2,3,17,4,5,6,7]\]
  \[\alpha_6=[2,12,13,14,16,17,18,19,6,3,4,7,8,9,0,1,5,15,10,11]\]
  Si fissa quindi $p=3$ e si ottengono:
  \[\varphi(3)=a_6[\alpha_6[3]-1]=a_6[14-1]=a_6[13]=2\]
  \[\varphi^{-1}(3)=a_6[\alpha_6[3]+1]=a_6[14+1]=a_6[15]=17\]
\end{esempio}
Avendo quindi $MS[i].row=p$ e $MS[i].len=l$ basta iterare le righe a partire a
$p$ in $a_i$, che denotiamo con l'indice $q$, fino a che si ha si ha 
$LCE_k(x_p, x_q)\geq l$. Ovviamente bisogna iterare in entrambe le
direzioni. Tutte le righe $x_q$ che soddisfano un match di lunghezza $l$ con
l'aplotipo query. L'algoritmo \ref{algo:phiext} rappresenta esattamente quanto
detto, avendo che con la funzione $lce\_bounded$ si limita il calcolo della
\textit{LCE} alla lunghezza $l$.\\
\begin{algorithm}
  \begin{algorithmic}[1]
    \Function{extend\_matches}{$k, row, len$}
    \State $haplos\gets []$
    \State $check_{down}\gets \top,\,\,check_{up}\gets \top$
    \While {$check_{down}$}
    \State $down_{row}\gets \varphi^{-1}(row, k)$
    \If{$lce\_bounded(k, row, down_{row}, len)$}
    \State $push(haplos, down_{row})$
    \State $row \gets down_{row}$
    \Else
    \State $check_{down}\gets \bot$
    \EndIf
    \EndWhile
    \While {$up_{down}$}
    \State $up_{row}\gets \varphi(row, k)$
    \If{$lce\_bounded(k, row, up_{row}, len)$}
    \State $push(haplos, up_{row})$
    \State $row \gets up_{row}$
    \Else
    \State $check_{up}\gets \bot$
    \EndIf
    \EndWhile
    \State \textbf{return} $haplos$
    \EndFunction
  \end{algorithmic}
  \caption{Algoritmo per estendere un match in colonna $k$ usando $\varphi$,
  $\varphi^{-1}$.}
  \label{algo:phiext}
\end{algorithm}
Questa è la definizione formale delle due funzioni ma, all'atto pratico, in
memoria si hanno solo i \textit{prefix array sample}, ad inizio e fine di ogni
run, e nessuna informazione in merito alla \textit{permutazione inversa} del
\textit{prefix array}. Si è quindi pensato ad una struttura dati, basata
anch'essa su \textit{bitvector sparsi}, che permettesse il calcolo delle due
funzioni. 
\subsection{Costruzione della struttura di supporto}
L'idea di base per la costruzione della struttura a supporto delle
\textbf{funzioni} $\boldsymbol\varphi$ e $\boldsymbol\varphi\mathbf{^{-1}}$ si
basa sul fatto che, data una colonna $k$ e dati due valori consecutivi $p$ e $q$
in $a_k$ (avendo $a_k[i]=p$ e $a_k[i+1]=q$), essi rimarranno consecutivi anche
in $a_{k+o}$, \textit{prefix array} dell'arbitraria colonna $k+o$, fino a che
che $x_{p}[k+o]\neq x_{q}[k+o]$, ovvero fino a che, in colonna $k+o$, tali righe
non corrisponderanno a due simboli diversi. Cruciale è che, in quella colonna,
$p$ sarà memorizzato come \textit{prefix array sample} della fine della run $r$
mentre $q$ come \textit{prefix array sample} dell'inizio della run $r+1$. Grazie
a questa informazione si può costruire una struttura che, data una colonna
arbitraria e un arbitrario valore di \textit{prefix array}, permetta di
computare $\varphi$ e $\varphi^{-1}$.\\
Tale struttura dati è composta da:
\begin{itemize}
  \item un vettore di \textit{sparse bitvector} per $\varphi$, che denotiamo con
  $\varPhi$, tale che $\varPhi[i][j]=1$ sse la riga $i$ indicizza una testa di
  run alla colonna $j$ nella \textit{matrice PBWT}. Si ha quindi che $\varPhi$
  ha dimensione $M\times N$
  \item un vettore di \textit{sparse bitvector} per $\varphi^{-1}$, che
  denotiamo con $\varPhi^{-1}$, tale che $\varPhi[i][j]=1$ sse la riga $i$
  indicizza una coda di run alla colonna $j$ nella \textit{matrice PBWT}. Si ha
  quindi che $\varPhi^{-1}$ ha dimensione $M\times N$
  \item un vettore di interi a supporto, denotato $\varPhi_{supp}$, del vettore
  di \textit{sparse bitvector} 
  per $\varphi$ che memorizza, per ogni 1 di tale vettore, il
  \textit{prefix array sample} della coda della run precedente o l'altezza
  del pannello, $M$, qualora non si abbia alcuna run precedente
  \item un vettore di interi a supporto, denotato $\varPhi^{-1}_{supp}$, del
  vettore di \textit{sparse bitvector} 
  per $\varphi^{-1}$ che memorizza, per ogni 1 di tale vettore,
  il \textit{prefix array sample} della testa della run successiva o l'altezza
  del pannello, $M$, qualora non si abbia alcuna run successiva. 
\end{itemize}
Si ha quindi che la lunghezza della riga $i$-esima di $\varPhi_{supp}$ è
uguale al numero di uni presenti nella riga $i$-esima di $\varPhi$. Analogamente
si ha per $\varPhi^{-1}_{supp}$. In entrambi i casi, inoltre, si hanno $M$
righe.\\ 
Al fine della costruzione bisogna, inoltre, sfruttare $a_{N-1}$ per poter
identificare quelle coppie di valori consecutivi non presenti nei vari
\textit{prefix array samples}, in modo che sia possibile effettuare le query per
qualsiasi valore di \textit{prefix array} in input.\\
L'algoritmo \ref{algo:phicos} riporta quindi la costruzione della struttura,
iterando in primis i vari \textit{prefix array samples} e completando i
risultati con $a_{N-1}$.\\
\textbf{CAPIRE SE COMMENTARE ULTERIORMENTE LA COSTRUZIONE}
\begin{algorithm}
  \footnotesize
  \begin{algorithmic}[1]
    \Function{Build\_phi}{$cols, panel, prefix$}
    \Comment  $prefix$ is the last prefix array
    \State $\varPhi\gets [[0..0]..[0..0]],\,\,\varPhi^{-1}\gets
    [[0..0]..[0..0]]$ 
    \Comment sparse bit vector panels for $\varphi$ and $\varphi^{-1}$
    \State $\varPhi_{supp} = [],\,\,\varpPhi_{supp}^{-1} = []$
    \Comment vectors for $\varphi$ and $\varphi^{-1}$ row values
    \For {\textit{every} $k\in [0,|cols|)$}
    \For {\textit{every} $i\in [0,|samples_{beg}|)$}
    \State $\varPhi[sample_{beg}^{k}[i]][k]\gets 1$
    \If{$i=0$}
    \State $push(\varPhi_{supp}[sample_{beg}^{k}[i]], panel_{height})$
    \Else
    \State $push(\varPhi_{supp}[sample_{beg}^{k}[i]],sample_{end}^{k}[i-1])$
    \EndIf

    \State $\varPhi^{-1}[sample_{end}^{k}[i]][k]\gets 1$
    \If{$i=|sample_{beg}^k|-1$}
    \State $push(\varPhi_{supp}^{-1}[sample_{end}^{k}[i]], panel_{height})$
    \Else
    \State $push(\varPhi_{supp}^{-1}[sample_{end}^{k}[i]],sample_{beg}^{k}[i+1])$
    \EndIf
    \EndFor
    \EndFor
    \For {\textit{every} $k\in [0,|prefix|)$}
    \If{$\varPhi[k][|\varPhi[k]|-1] = 0$}
    \State $\varPhi[k][|\varPhi[k]|-1]\gets 1$
    \If{$k=0$}
    \State $push(\varPhi_{supp}[prefix[k]], panel_{height})$
    \Else
    \State $push(\varPhi_{supp}[prefix[k]] ,prefix^k[i-1])$
    \EndIf
    \EndIf
    \If{$\varPhi^{-1}[k][|\varPhi[k]|-1] = 0$}
    \State $\varPhi^{-1}[k][|\varPhi[k]|-1]\gets 1$
    \If{$k=|prefix|-1$}
    \State $push(\varPhi^{-1}_{supp}[prefix[k]], panel_{height})$
    \Else
    \State $push(\varPhi^{-1}_{supp}[prefix[k]],prefix^k[i+1])$
    \EndIf
    \EndIf
    \EndFor
    \State \textit{build rank for every sparse bitvector in} $\varPhi$
    and $\varPhi^{-1}$
    \EndFunction
  \end{algorithmic}
  \caption{Algoritmo per la costruzione della struttura per $\varphi$ e
  $\varphi^{-1}$}
  \label{algo:phicos}
\end{algorithm}
Dal punto di vista delle query, data una colonna $k$ e un valore di
\textit{prefix array} $p$, si procede quindi nel seguente modo:
\begin{itemize}
  \item per la funzione $\varphi$ si effettua la $rank^\varphi(k)$ sulla riga
  $p$ di $\varPhi$, avendo che:
  \[\varphi_k(p)=
    \begin{cases}
      null&\mbox{se }\varPhi_{supp}^p[rank^\varphi_p(k)]=M\\
      \varPhi_{supp}^p[rank^\varphi_p(k)]&\mbox{altrimenti }
    \end{cases}
  \]
  \item per la funzione $\varphi^{-1}$ si effettua la $rank^{\varphi^{-1}}(k)$
  sulla riga $p$ di $\varPhi^{-1}$, avendo che:
  \[\varphi_k^{-1}(p)=
    \begin{cases}
      null&\mbox{se }\varPhi^{-1\,\,p}_{supp}[rank^{\varphi^{-1}}_p(k)]=M\\
      \varPhi^{-1\,\,p}_{supp}[rank^{\varphi^{-1}}_p(k)]&\mbox{altrimenti }
    \end{cases}
  \]
\end{itemize}
Si riporta un esempio chiarificatore.
\begin{esempio}
  Si ha la seguente situazione nella \textit{matrice PBWT}:
  \begin{figure}[H]
    \centering
    \includegraphics[scale = 0.8]{img/phi.pdf}   
  \end{figure}
  Dove si noti che, a parità di colore, si ha lo stesso simbolo tra due indici
  consecutivi. \\
  In colonna $k$, che per praticità assumiamo essere $k=0$, si vorrebbe avere
  informazione in merito a $\varphi_k(j)$ e $\varphi^{-1}_k(m)$. \\
  Si nota che, per definizione della struttura dati, si ha che (limitandoci alle
  colonne della figura):
  \[\varPhi_j=[0,0,0,1,0, \ldots]\]
  \[\varPhi^{-1}_m=[0,0,0,1,0,\ldots]\]
  In quanto, in entrambi i casi, rispettivamente per la riga $j$ e la riga $m$,
  in colonna $k+3$, si che $j$ è il \textit{prefix array} di una testa di run
  mentre $m$ di una coda di run. In colonna $k+3$ si conoscono anche,
  rispettivamente, $i$, \textit{prefix array} della coda della run precedente a
  quella di $j$, e $n$, \textit{prefix array} della testa della run successiva
  quella di $m$. Si ottengono quindi:
  \[\varPhi_{supp}=[\ldots,i,\ldots]\]
  \[\varPhi^{-1}_{supp}=[n,\ldots]\]
  Si vogliono quindi calcolare $\varphi_0(j)$ e  $\varphi^{-1}_0(m)$. Si ha:
  \[\varPhi_{supp}^j[rank^\varphi_j(0)]=\varPhi_{supp}^j[0]=i\]
  \[\varPhi^{-1\,\,m}_{supp}[rank^{\varphi^{-1}}_m(0)]=\varPhi^{-1\,\,m}_{supp}[0]=n\]
  Si noti che uguali risultati si avrebbero per $k+1$, $k+2$ e $k+3$.
\end{esempio}
\textbf{SISTEMARE}\\
 


\chapter{Risultati sperimentali}
\label{reschap}
Verranno ora riportati alcuni risultati sperimentali
relativi all'implementazione della \textbf{RLPBWT}. In primis,
verranno discusse le modalità di sperimentazione per poi trattare i
risultati ottenuti su alcuni pannelli della \textit{phase 3} del \textbf{1000
  Genome Project (\textit{1KGP})} \cite{1kgp}, progetto, che ha avuto inizio nel
2008, il quale 
ha visto lo sforzo della comunità 
scientifica internazionale per catalogazione delle variazioni geniche
umane. Il \textit{1KGP} risulta essere uno dei più importanti progetti del
settore. \\ 
Verranno quindi confrontate le implementazioni degli algoritmi di calcolo degli
\textit{SMEM} della \textit{PBWT} di Durbin e delle varianti per la
\textit{RLPBWT}.  
\dc{Magari servono dettagli per il 1kgp}
% sezione strumenti
\paragraph{RLPBWT.}
In merito alle varianti della $\RLPBWT$, sono state testate le otto
strutture dati composte discusse nel capitolo \ref{metchap}:
\begin{enumerate}
  \item la struttura dati composta \texttt{MAP-INT + RLCP} e la struttura dati
  composta \texttt{MAP-BV + RLCP}. Si ricorda che tali soluzioni non supportano
  il riconoscimento delle righe del pannello, per cui si ha uno $\SMEM$, ma
  solo la cardinalità dell'insieme delle stesse
  \item le strutture dati composte basate sull'uso delle threshold per
  il calcolo dell'array delle matching statistics, 
  ovvero: \texttt{MAP-INT + THR-INT + RA-BV + PERM + PHI},  \texttt{MAP-INT +
    THR-INT + RA-SLP + PERM + PHI}, \texttt{MAP-BV + THR-BV + RA-BV + PERM +
    PHI} e \texttt{MAP-BV + THR-BV + RA-SLP + PERM + PHI} 
  \item le strutture dati composte basate sull'uso delle $\LCE$ query
  per
  il calcolo dell'array delle matching statistics,
  ovvero: \texttt{MAP-INT + LCE + PERM + PHI} e \texttt{MAP-BV + LCE + PERM +
    PHI}  
\end{enumerate}
L'implementazione è stata fatta in linguaggio \Cplusplus, usando
librerie esterne:
\begin{itemize}
  \item SDSL per intvector compressi,
  bitvector, bitvector sparsi, serializzazione e varie utility
  per il calcolo della memoria delle strutture dati
  \item BigRePair e ShapedSlp per la costruzione e l'uso degli $\SLP$ 
\end{itemize}
L'implementazione delle strutture composte per la
$\RLPBWT$ supporta lo studio parallelo di più query tramite
openMP \cite{openmp}. Al fine di un più 
corretto confronto con l'implementazione della $\PBWT$,
l'intera sperimentazione è stata
svolta sfruttando un solo thread per volta, tramite la variabile
d'ambiente \texttt{OMP\_NUM\_THREADS=1}.
\paragraph{PBWT.}
Per validare più correttamente i confronti tra le varie strutture dati
per la $\RLPBWT$ e la $\PBWT$ di Durbin, si è scelto di utilizzare
l'implementazione originale di
quest'ultima\footnote{\url{https://github.com/richarddurbin/pbwt}}. Tale
implementazione è scritta in 
linguaggio C e
fornisce tre algoritmi per il calcolo degli $\SMEM$, avendo $N$ siti, $M$ sample
e $Q$ query, per i quali l'autore ha riportato le complessità asintotiche 
nei commenti del codice: 
\begin{enumerate}
  \item \texttt{matchNaive}, ovvero un'implementazione na\"{i}ve del calcolo
  degli $\SMEM$ che non sfrutta la $\PBWT$. Questo algoritmo non è
  utilizzabile in casi reali. La complessità in tempo di tale 
  soluzione è stimata essere $\mathcal{O}(\mathit{NMQ})$ mentre quella in spazio
  è 
  $\mathcal{O}(\mathit{NM})$
  \item \texttt{matchIndexed}, ovvero l'algoritmo 5 del paper originale
  \cite{pbwt}. La complessità in tempo di tale 
  soluzione è stimata essere $\mathcal{O}(\mathit{NQ})$, dopo una fase di
  preprocessing 
  con complessità $\mathcal{O}(\mathit{NM})$. La complessità in spazio è stimata
  essere 
  $\mathcal{O}(\mathit{NM})$, ricordando che essa corrisponde a $13\mathit{NM}$
  byte in memoria
  \item \texttt{matchDynamic}, ovvero un algoritmo non approfondito nel paper,
  ma
  solo citato nei risultati sotto il nome di ``batch''.
  Si è dedotto che il suo funzionamento
  si basa sulla creazione della $\PBWT$ anche delle query, viste 
  sotto forma di pannello, e sull'applicazione dell'algoritmo per il calcolo
  degli $\SMEM$  
  interni alla sua ``fusione virtuale'' con la $\PBWT$ del pannello di aplotipi.
  Inoltre, il calcolo dei vari indici viene
  fatto di colonna in colonna, avendo una sola scansione
  della struttura dati per tutte le query, a differenza dell'algoritmo
  \texttt{matchIndexed} e degli algoritmi per la $\RLPBWT$.
  La complessità in tempo di tale 
  soluzione è stimata essere $\mathcal{O}(N(M+Q))$, mentre quella in spazio è
  $\mathcal{O}(N+M)$. 
\end{enumerate}
Si intuisce fin da subito come l'ultima soluzione, della quale si è avuta
conoscenza solo in fase di
sperimentazione, risulti essere la migliore a disposizione, sia in spazio che
in tempo. Si hanno solo due
limitazioni. La prima è dovuta al fatto che, dovendo computare
la trasformata anche per il pannello di query ed essendo l'algoritmo studiato
per lavorare sulla trasformata stessa, i tempi di calcolo per poche query sono
alti rispetto all'algoritmo \texttt{matchIndexed} e rispetto alle varie
soluzioni per la $\RLPBWT$. Il secondo limite è che i risultati non sono
ordinati, infatti l'algoritmo \texttt{matchDynamic}, studiando la trasformata
anche delle query,
presenta tutti i risultati permutati secondo la stessa, mentre tutti gli altri
algoritmi presentano i risultati query per query. Si rileva
come tali limiti possano essere per lo più trascurabili, nonostante il problema
su
cui si concentrano gli studi di questa tesi sia la ricerca degli $\SMEM$ tra
una 
singola query e un pannello di aplotipi.
\section{Pannelli del 1000 Genome Project}
Come anticipato, al fine di valorizzare i risultati teorici ottenuti in
questo progetto, si 
è deciso di procedere con lo studio di dati reali, relativi alla phase
  3 del 1KGP
\footnote{\url{https://ftp.1000genomes.ebi.ac.uk/vol1/ftp/release/20130502/}}
\cite{1kgp}.\\ 
Tali pannelli, disponibili in formato VCF (Variant Call Format) \cite{vcf},
presentano un numero 
costante di sample, ovvero 5.008, mentre varia il numero di siti. Essendo
dati reali, si ha anche la presenza di siti multiallelici. Si è quindi proceduto
alla selezione dei soli siti biallelici, ottenendo pannelli costruiti su
un alfabeto binario $\Sigma=\{0,1\}$, tramite l'uso di bcftools
\cite{bcftools}, con il comando \texttt{bcftools view -m2 -M2
  -v snps}.\\
Si sono scelti i pannelli relativi ai cromosomi 22 (\texttt{chr22}), 20
(\texttt{chr20}), 18 (\texttt{chr18}), 16 (\texttt{chr16}) e 1 (\texttt{chr1}).
Si noti che  
l'ordine è dato dal numero crescente di siti e che la scelta di includere il
cromosoma 1 è dettata dal fatto che è il più grande cromosoma
umano, quindi  anche il relativo pannello delle varianti geniche
è tra quelli col maggior numero di siti, mentre gli altri sono stati scelti
per praticità, in quanto pannelli non troppo estesi.\\
Trattandosi di pannelli reali, è risultata interessante una preliminare
indagine esplorativa sulla natura di tali pannelli in termini di
sparsità degli alleli e di conseguente numerosità attesa delle run. Si è
quindi calcolato, per i cinque pannelli, il numero di simboli $\sigma=0$ e
$\sigma=1$, notando come il numero di simboli $\sigma=1$ fosse molto ridotto
rispetto al totale ($\sim 0.03\%$ del totale in tutti e
cinque i casi). Una tale
sparsità del dato ha diretta conseguenza sul numero di run di ogni
colonna. Infatti, avendo 
pochi simboli $\sigma=1$ in ogni colonna, che possono anche
essere nella medesima run dopo la permutazione data dalla
$\PBWT$, si producono, nel complesso, poche run. Si ricordi, inoltre, che tale
permutazione, come la 
$\BWT$, è studiata per essere 
maggiormente efficiente nel caso del dato biologico, comportando un'alta
probabilità di produrre run del medesimo carattere. In tabella \ref{tab:panel}
si riportano il numero di siti di ogni cromosoma, il 
numero medio di run per colonna, il numero 
massimo di run in una colonna e il totale delle run. Si segnala, inoltre,
come la
mediana del numero di run per colonna abbia valore 3 per tutti e tre i pannelli.
I valori quantitativi sono 
stati calcolati a partire dai pannelli con un numero di sample pari a 4.908,
poiché 100 sample/righe sono state utilizzate come query nelle successive
fasi della sperimentazione.
\begin{table}
  \centering
  \caption{Informazioni quantitative relative ai cinque pannelli in analisi.}
  \label{tab:panel}
  \vspace{-2mm}
  \begin{tabular}{c||c|c|c|c}
    \textbf{Chr} & \textbf{\#Siti} & \textbf{\#Run totale}
    & \textbf{Max run} & \textbf{Media run} \\ 
    \hline
    \texttt{chr22} & 1.055.454 & 14.772.105 & 2.450 & 14\\
    \texttt{chr20} & 1.739.315 & 19.966.504 & 2.176 & 11\\
    \texttt{chr18} & 2.171.378 & 24.288.263 & 2.365 & 11\\
    \texttt{chr16} & 2.596.072 & 31.187.856 & 2.330 & 12\\
    \texttt{chr1} & 6.196.151 & 69.671.952 & 2.721 & 11\\
  \end{tabular}
\end{table}
In merito alla sparsità del dato e al
conseguente basso numero medio di run per colonna, si conferma il risultato
atteso che è a favore, in termini di 
complessità in spazio/tempo, della $\RLPBWT$, in quanto tutte le componenti sono
proporzionali, al numero di run (a eccezione della componente \texttt{RLCP}). In
figura 
\ref{fig:boxplot} si riportano i risultati statistici, sotto forma di
boxplot, relativi alla distribuzione delle run nei cinque pannelli
studiati. Il forte numero di outlier si ha poiché media e 
mediana del numero di run per colonna risultano essere molto piccole rispetto al
numero massimo di run riscontrabili in una colonna.
\begin{figure}
  \centering
  \includegraphics[width = \linewidth]{img/boxplotbi.png}
  \vspace{-5mm}
  \caption{Boxplot della distribuzione delle run per i pannelli dei cinque
    cromosomo studiati. Il grafico (a) presenta uno zoom che esclude la maggior
    parte degli outlier mentre il grafico (b) presenta, in scala logaritmica, il
    boxplot completo con tutti gli outlier.}
  \label{fig:boxplot}
\end{figure}
\subsection{Riproducibilità degli esperimenti}
Al fine di rendere riproducibili gli esperimenti, si è costruita una pipeline
per l'esecuzione dei vari algoritmi e l'estrazione dei dati quantitativi
relativi alle
performance\footnote{\url{https://github.com/dlcgold/rlpbwt-test}}.\\
L'intera pipeline è stata gestita tramite Snakemake \cite{snakemake}
(un workflow management system), uno strumento molto usato in
bioinformatica per creare analisi dati scalabili e riproducibili. Nel
dettaglio la pipeline comprende, come visualizzabile in figura \ref{fig:snake},
avendo in input una lista di pannelli con associato il numero di query:
\begin{itemize}
  \item lo scaricamento dei tool e delle dipendenze per la $\PBWT$ di
  Durbin e la $\RLPBWT$ proposta in questa tesi
  \item la produzione dell'input per la $\PBWT$ e per le varianti della
  $\RLPBWT$, in base alla quantità di query richiesta
  \item la produzione delle strutture dati
  \item l'esecuzione degli algoritmi per il calcolo degli $\SMEM$
  % \item produzione di vari grafici relativi sia ai tempi di esecuzione che alla
  % memoria richiesta
\end{itemize}
Si è deciso di estrarre dai pannelli un numero di righe pari al numero di 
query richieste, che, a loro volta, andranno a formare il pannello di
query, in modo che il calcolo non sia banale.\\
\begin{figure}
  \centering
  \includegraphics[width=\textwidth]{img/final_dag_r.pdf}
  \vspace{-5mm}
  \caption{Regole usate in Snakemake per la sperimentazione. Si hanno
  il download  dei vari software, la compilazione degli stessi, la produzione
  delle strutture dati,
  il calcolo degli $\SMEM$, l'estrazione dei risultati
  quantitativi e la produzione dei file CSV finali.}
  \label{fig:snake}
\end{figure}
\newline
\textit{La sperimentazione è stata effettuata su una macchina con processore
  Intel Xeon E5-2640 V4 ($2,40$GHz), $756$GB di RAM, $768$GB di swap e
  sistema operativo Ubuntu 20.04.4 LTS. Tale macchina è stata gentilmente messa
  a disposizione dalla \emph{University of Florida}.}
% sezione benchmark
%\section{Ambiente di benchmark}
\subsection{Descrizione input}

% sezione complessità
%\section{Confronto tra PBWT e RLPBWT}
Al fine di analizzare i risultati ottenuti si sono confrontate 5 varianti della
\textit{RLPBWT}:
\begin{itemize}
  \item \textit{RLPBWT naive}
  \item \textit{RLPBWT con bitvector}
  \item \textit{RLPBWT con pannello completo e threshold}
  \item \textit{RLPBWT con pannello compresso (SLP) e threshold}
  \item \textit{RLPBWT con pannello compresso (SLP) e LCE query}  
\end{itemize}
Confrontandole con l'implementazione originale dell'algoritmo 5 di Durbin,
nominato \textit{MatchIndexed}. Studiando la repository di Durbin inoltre si è
scoperto l'esistenza di un ulteriore algoritmo, non descritto
formalmente nel paper del 2014 \cite{pbwt} ma solo citato in una tabella, che
considera in un unico panello sia il pannello che l'insieme delle query ed
effettua il matching interno al pannello stesso, calcolando in modo dinamico
l'indicizzazione ad ogni colonna. Nonostante l'algoritmo presenti limiti dal
punto di vista dell'estendibilità ad altre problematiche, avendo che le varianti
della \textit{PBWT} citate in sezione \ref{secpbwt} si basano, nel caso di match
con aplotipi esterni, sulle idee dell'algoritmo 5, esso risulta essere davvero
molto performante sia dal punto di vista del tempo macchina che della memoria
occupata. A causa di ciò, per completezza, tale algoritmo, chiamato
\textit{MatchDynamic}, è stato incluso nei risultati sperimentali, pur
mancandone una trattazione teorica approfondita.
\subsection{Analisi spaziale}
Lo scopo principale di questa tesi era la riduzione delle informazioni in
memoria necessarie a permette il mapping, quindi in primis si sono valutati i
vari risultati dal punto di vista della memoria.
\subsubsection{Dimensioni dell'SLP}
Prima ancora di affrontare i requisiti
in memoria dell'intera struttura è interessante analizzare le capacità di
compressione che si ha con l'uso degli \textit{SLP}, grazie ai due tool sopra
citati. In figura \ref{fig:slpres1} si può iniziare ad apprezzare l'efficacia di
tale grammatica. Si nota infatti come, per quanto i pannelli siano di dimensione
modesta, hanno un peso che varia in un range di un centinaio di megabytes mentre
gli \textit{SLP} relativi nel centinaio di kilobytes. Si ha infatti:
\begin{table}[H]
  \centering
  \begin{tabular}{c|c|c|c|c}
    \textbf{altezza} & \textbf{larghezza} & \textbf{SLP (\textit{kb})}
    & \textbf{MACs (\textit{kb})} & \textbf{\%}\\
    \hline
    20000 & 4294 & 228.13 & 84050.48 & 0.2714\\
    21000 & 4294 & 238.83 & 88243.83 & 0.2707\\
    22000 & 4294 & 243.04 & 92437.18 & 0.2629\\
    23000 & 4294 & 250.37 & 96630.53 & 0.2591\\
    24000 & 4294 & 272.72 & 100823.87 & 0.2705\\
    25000 & 4294 & 278.22 & 105017.22 & 0.2649\\
    26000 & 4294 & 283.57 & 109210.57 & 0.2597\\
    27000 & 4294 & 288.62 & 113403.92 & 0.2545\\
    28000 & 4294 & 293.85 & 117597.27 & 0.2499\\
    29000 & 4294 & 298.76 & 121790.62 & 0.2453\\
  \end{tabular}
\end{table}
Notando come, per pannelli di grandezza simile, pare si abbia una compressione
proporzionale alla dimensione del pannello.\\
\begin{figure}
  \centering
  \includegraphics[scale = 0.6]{img/slp_vs_macs.png}
  \caption{Confronto delle dimensioni, espresse in kilobytes, dei pannelli in
    formato \texttt{macs} e dei rispettivi \textit{SLP}. Il grafico è in scala
    logaritmica.}
  \label{fig:slpres1}
\end{figure}
Andando a vedere pannelli molto più grossi si nota come il rateo di
compressione continui essere proporzionale alla dimensione del pannello e,
nonostante il esso cresca di 
dimensione, la grandezza dell'\textit{SLP} resta molto piccola:
\begin{table}[H]
  \centering
  \begin{tabular}{c|c|c|c|c}
    \textbf{altezza} & \textbf{larghezza} & \textbf{SLP (\textit{kb})}
    & \textbf{MACs (\textit{kb})} & \textbf{\%}\\
    \hline
    100000 & 358653 & 14771.0 & 35042963.54 & 0.0422\\
    100000 & 100000 & 9077.88 & 9075120.49 & 0.1\\
    100000 & 46538 & 8017.09 & 4448994.19 & 0.1802\\
  \end{tabular}
\end{table}
Tale risultato è anche apprezzabile in figura \ref{fig:slpres2}.
\begin{figure}
  \centering
  \includegraphics[scale = 0.6]{img/slp_vs_macs2.png}
  \caption{Confronto delle dimensioni, espresse in kilobytes, dei pannelli in
    formato \texttt{macs} e dei rispettivi \textit{SLP}. Il grafico è in scala
    logaritmica.}
  \label{fig:slpres2}
\end{figure}
Il caso estremo, un pannello $100000\times 358653$, occupante in memoria
circa 35gb in formato \texttt{.macs}, viene compresso in circa 15mb. Questo
accade soprattutto in quanto un pannello di soli simboli $\Sigma=\{0,1\}$
contiene molte ripetizioni, permettendo la costruzione di una grammatica,
tramite l'\textit{SLP}, particolarmente ``compatta''.
\subsubsection{Strutture dati}
Si analizzano ora le due strutture dati, confrontando lo spazio richiesto dalle
varie sotto-strutture per effettuare il match con una query esterna, descritte
alle sezioni \ref{secpbwt}, \ref{secrlpbwtnaive}, \ref{secrlpbwtbv} e
\ref{secrlpbwtms}.\\
Si precisa che i dati ora descritti sono stati calcolati nel seguente modo:
\begin{itemize}
  \item per quanto riguarda la \textbf{PBWT}, sfruttando le stime fatte da
  Durbin stesso
  \item per quanto riguarda la \textbf{RLPBWT}, sfruttando le serializzazioni
  ottenute tramite \textit{SDSL}
\end{itemize}
Con un studio al leggero variare del pannello si nota, graficamente in figura
\ref{memcomp1}, come quanto descritto precedentemente venga confermato. Le
informazioni richieste dall'algoritmo 5 di Durbin sono quelle che richiedono
maggior memoria mentre la variante della \textit{RLPBWT} basata su \textit{SLP}
e \textit{LCE query} risulta essere la soluzione migliore tra le varianti della
\textit{RLPBWT}. Bisogna però notare come la soluzione \textit{matchDynamic}
ritrovabile nella repository della \textit{PBWT} risulti essere incredibilmente
più efficace, avendo, secondo Durbin stesso, una richiesta in spazio
proporzionale a $\mathcal{O}(M+N)$. \\
Limitiamo però ora il confronto all'algoritmo 5 di Durbin, in quanto obbiettivo
della tesi. Da un punto di vista di guadagno percentuale in memoria i risultati
sembrano essere interessanti, confrontando tale soluzione con la migliore per la
\textit{RLPBWT}:
\begin{table}[H]
  \centering
  \footnotesize
  \begin{tabular}{c|c|c|c|c}
    \textbf{altezza} & \textbf{larghezza}
    & \textbf{RLPBWT SLP-LCE (\textit{kb})}
    & \textbf{PBWT Indexed (\textit{kb})} & \textbf{\%}\\
    \hline
    20000 & 4294 & 12118.62 & 1090270.65 & 1.1115\\
    21000 & 4294 & 12583.13 & 1144784.18 & 1.0992\\
    22000 & 4294 & 13033.78 & 1199297.71 & 1.0868\\
    23000 & 4294 & 13487.57 & 1253811.24 & 1.0757\\
    24000 & 4294 & 13954.44 & 1308324.78 & 1.0666\\
    25000 & 4294 & 14419.27 & 1362838.31 & 1.058\\
    26000 & 4294 & 14867.82 & 1417351.84 & 1.049\\
    27000 & 4294 & 15316.41 & 1471865.37 & 1.0406\\
    28000 & 4294 & 15765.41 & 1526378.9 & 1.0329\\
    29000 & 4294 & 16214.09 & 1580892.44 & 1.0256\\
  \end{tabular}
\end{table}
Provando in modo quantitativo l'efficacia in memoria della soluzione ultima
proposta in questa tesi.

% \begin{table}[H]
%   \centering
%   \tiny
%   \begin{tabular}{c|c|c|c|c|c|c|c|c}
%     \textbf{Altezza} & \textbf{Larghezza} & \textbf{Naive}
%     & \textbf{Bitvector} & \textbf{Pannello}
%     & \textbf{SLP-Thr} & \textbf{SLP-LCE} & \textbf{Indexed}&
%                                                               \textbf{Dynamic}\\
%     \hline
%     20000 & 4294 & 116872.34 & 118842.8 & 22956.56 & 12422.86 & 12118.62
%                                                      & 1090270.65 & 20004.19\\
%     21000 & 4294 & 122717.92 & 124689.87 & 23954.83 & 12884.39 & 12583.13
%                                                        & 1144784.18 & 21004.19\\
%     22000 & 4294 & 128541.43 & 130509.84 & 24911.04 & 13337.4 & 13033.78
%                                                        & 1199297.71 & 22004.19\\
%     23000 & 4294 & 134383.15 & 136347.63 & 25900.01 & 13789.62 & 13487.57
%                                                        & 1253811.24 & 23004.19\\
%     24000 & 4294 & 140197.73 & 142157.96 & 26853.09 & 14239.5 & 13954.44
%                                                        & 1308324.78 & 24004.19\\
%     25000 & 4294 & 146047.34 & 148014.12 & 27855.43 & 14705.09 & 14419.27
%                                                        & 1362838.31 & 25004.19\\
%     26000 & 4294 & 151872.56 & 153834.89 & 28841.15 & 15154.06 & 14867.82
%                                                        & 1417351.84 & 26004.19\\
%     27000 & 4294 & 157711.32 & 159670.04 & 29793.47 & 15603.17 & 15316.41
%                                                        & 1471865.37 & 27004.19\\
%     28000 & 4294 & 163536.4 & 165489.58 & 30779.6 & 16052.56 & 15765.41
%                                                        & 1526378.9 & 28004.19\\
%     29000 & 4294 & 169381.28 & 171331.14 & 31765.37 & 16501.58 & 16214.09
%                                                        & 1580892.44 & 29004.19\\
    
%   \end{tabular}
% \end{table}
\begin{figure}
  \centering
  \includegraphics[scale = 0.6]{img/pbwt_vs_rlpbwt_dyn.png}
  \caption{Confronto dello spazio in memoria, in kilobytes, richiesto dalle
    varie strutture dati.} 
  \label{memcomp1}
\end{figure}
\subsection{Analisi temporale}
Bisogna infine considerare i tempi di esecuzione per il pattern matching con un
pannello di query. Dal punto di vista della \textit{RLPBWT} bisogna considerare
in primis due aspetti:
\begin{itemize}
  \item avere meno informazione in memoria comporta molto probabilmente, a
  parità di risultati, tempi maggiori
  \item l'uso di strutture dati succinte ed eventualmente dell'\textit{SLP}
  comporta costi dal punto di vista temporale. Come anticipato in sezione
  \ref{bvsec}, le operazioni sugli sparse bitvector non sono tutte i tempo
  costante e, come invece anticipato in sezione \ref{slpsec}, gli \textit{SLP}
  non garantiscono \textit{random access} in tempo costante e questo, per quanto
  poi l'algoritmo di estensione sia efficiente, si ripercuote anche sul calcolo
  delle \textit{LCE query}
\end{itemize}
Questa premessa fa capire come ci si aspettasse che i tempi fossero maggiori con
la \textit{RLPBWT}, in ogni sua variante, rispetto all'algoritmo 5 di Durbin.
Parlando invece dell'algoritmo \textit{matchDynamic} si ha che, per quanto
asintoticamente presenti la stessa complessità dell'algoritmo 5, ovvero
$\mathcal{O}(N(M+Q))$, con $Q$ numero di query, esso risulta incredibilmente più
performante. \\
Alcuni risultati sono visualizzabili in figura \ref{fig:1000} e \ref{fig:10000},
dove si possono osservare sia i tempi che lo spazio richiesto. Anche la completa
esecuzione quindi conferma come l'algoritmo 5 sia incredibilmente esoso dal
punto di vista dello spazio richiesto, pur avendo ottime performance
temporali. Dal punto di vista invece delle varianti della \textit{RLPBWT} si
nota come:
\begin{itemize}
  \item la \textit{RLPBWT naive}, priva dell'uso dei bitvector e
  dell'\textit{SLP}, risulti essere la più performante, anche se, si ricordi,
  non permette di identificare quali righe stiano effettivamente matchando ma
  solo quante
  \item la \textit{RLPBWT con bitvector}, avente lo stesso limite della variante
  naive, presenta anche maggiori costi in termini di memoria di quest'ultima,
  avendo anch'essa ancora l'\textit{LCP array} completo ma anche tutte le
  informazioni memorizzate in bitvector, che aumentano, come anticipato, i tempi
  di calcolo. La chiave delle varianti che sfruttano
  le \textit{matching statistics} è infatti quella di non avere l'\textit{array
    LCP} in memoria, una delle cause principali dell'aumento di spazio richiesto
  \item le tre varianti basate sulle \textit{matching statistics} hanno spazio
  occupato pressoché uguale, anche se si può percepire, nei due casi studiati,
  come il tenere l'intero pannello in forma di bitvector, all'aumentare della
  grandezza dello stesso, comporti molta più memoria degli \textit{SLP}. Dal
  punto di vista temporale, inoltre, anche se si ha \textit{random access} in
  tempo costante, all'aumentare del pannello, il numero di accessi allo stesso
  comporta forti costi in termini di tempo macchina. Questi
  ultimi, infatti, come già visto occupano pochissima memoria anche con pannelli
  molto estesi. Dal punto di vista temporale si rileva come la variante basata
  su \textit{SLP} e \textit{threshold} richieda molto più tempo. Si nota che ciò
  accade a causa di due fattori:
  \begin{itemize}
    \item il continuo accesso all'\textit{SLP} per calcolare $MS[i].len$
    \item l'eventuale accesso all'\textit{SLP} per disambiguare le threshold a
    fine run
  \end{itemize}
  Tra le tre quindi la variante con \textit{SLP} e \textit{LCE query},
  all'aumentare della grandezza del pannello, risulta essere la soluzione
  migliore
\end{itemize}

\begin{figure}
  \centering
  \includegraphics[scale = 0.35]{img/time_vs_mem_1000.pdf}
  \includegraphics[scale = 0.35]{img/time_vs_mem-loglog_1000.pdf}
  \caption{Esecuzione dei vari algoritmi di match su un pannello
    $29000\times 4294$ e 1000 query.  Il grafico a destra è in scala
    logaritmica. }
  \label{fig:1000}
\end{figure}

\begin{figure}
  \centering
  \includegraphics[scale = 0.35]{img/time_vs_mem_10000.pdf}
  \includegraphics[scale = 0.35]{img/time_vs_mem-loglog_10000.pdf}
  \caption{Esecuzione dei vari algoritmi di match su un pannello
    $20000\times 4294$ e 10000 query. Il grafico a destra è in scala
    logaritmica. }
  \label{fig:10000}
\end{figure}
Per completezza, in figura \ref{fig:bigres}, si riportano anche i risultati in
tempo 
e spazio di una sperimentazione su un pannello di grandi dimensioni:
$70000\times 46538$ con $30000$ query. Sono riportati anche i risultati delle
tre varianti basate su \textit{matching statistics} senza l'estensione dei match
tramite le \textbf{funzioni} $\boldsymbol \phi$ e
$\boldsymbol\phi^{\mathbf{-1}}$. Si può notare come la struttura dati aggiuntiva
non comporti praticamente alcuna differenza sostanziale sia in termini di
memoria che di tempo di calcolo. In merito agli altri risultati si ha che
seguono tutti il trend già descritto negli esempi precedenti. In particolare si
nota che:
\begin{itemize}
  \item l'algoritmo 5 di Durbin ha una richiesta di memoria davvero molto
  grande, parlando di circa 40.75 gb di memoria richiesta
  \item l'algoritmo \textit{matchDynamic} di Durbin risulta essere migliore sia
  dal punto di vista dello spazio richiesto che del tempo d'esecuzione
  \item la variante \textit{RLPBWT} con \textit{SLP} e \textit{threshold}, per
  le problematiche già descritte richiede un tempo d'esecuzione importante,
  parlando di circa 2 ore di esecuzione
\end{itemize}
\begin{figure}
  \centering
  \includegraphics[scale = 0.4]{img/mem.png}
  \includegraphics[scale = 0.4]{img/time.png}
  \caption{Risultati, in scala logaritmica, dell'esecuzione dei vari algoritmi
    su un pannello di grosse dimensioni. Si noti che, quantitativamente, la
    variante \textit{matchIndexed} richieda $42736132$ kb di memoria mentre la
    \textit{RLPBWT} con \textit{SLP} e \textit{LCE query} solo $3286424$ kb,
    richiedendo quindi appena il 7\% di memoria richiesta dall'algoritmo 5 di
    Durbin.} 
  \label{fig:bigres}
\end{figure}
\section{Sperimentazione sui pannelli reali}
Come anticipato, al fine di valorizzare i risultati teorici ottenuti in
questo progetto, si 
è deciso di procedere con lo studio di dati reali, relativi alla \textit{phase
  3} del \textbf{1000 Genome Project}
\footnote{\url{https://ftp.1000genomes.ebi.ac.uk/vol1/ftp/release/20130502/}}
\cite{1kgp}.\\ 
Tali pannelli, disponibili in formato \textit{VCF}, presentano un numero
costante di sample, ovvero 5.008, mentre a variare è il numero di siti. Essendo
dati reali, si ha anche la presenza di siti multiallelici. Si è quindi proceduto
alla selezione dei soli siti biallelici, ottenendo quindi pannelli costruiti su
un alfabeto binario $\Sigma=\{0,1\}$, tramite l'uso di \textbf{bcftools}
\cite{bcftools}, tramite il comando \texttt{bcftools view -m2 -M2
  -v snps}.\\
La prima selezione dei pannelli è stata dettata dalla volontà di studiare, per
praticità, matrici non troppo estese. In aggiunta, si è studiato uno dei
pannelli più grossi disponibili, ovvero quello relativo al cromosoma 1. Si sono,
quindi, scelti i pannelli relativi al:
\begin{itemize}
  \item cromosoma 22, con 1.055.454 siti
  \item cromosoma 20, con 1.739.315 siti
  \item cromosoma 18, con 2.171.378 siti
  \item cromosoma 16, con 2.596.072 siti
  \item cromosoma 1, con 6.196.151 siti
\end{itemize}
Al fine del computo degli SMEM, avendo un numero ridotto di sample a
disposizione, si è scelto di estrarre da ognuno 100 sample da usare come
query.\\
Si segnala che, per quanto l'implementazione delle strutture per la
\textit{RLPBWT} supportino lo studio parallelo di più query tramite
\textbf{openMP}\cite{openmp}, al fine di un più 
corretto confronto con l'implementazione della \textit{PBWT}, specialmente
parlando dell'algoritmo \texttt{matchIndexed}, l'intera sperimentazione è stata
svolta sfruttando un solo \textit{thread} per volta, tramite la variabile
d'ambiente \texttt{OMP\_NUM\_THREADS=1}.
\subsection{Studio dei pannelli}
Prima di procedere con l'effettivo studio delle performance di calcolo degli
SMEM, trattandosi di pannelli reali, è risultata interessante una preliminare
indagine esplorativa sulla natura di tali pannelli in termini di
\textit{sparsità} degli alleli e di conseguente numerosità delle run. Si è
quindi calcolato, per i cinque pannelli, il numero di simboli $\sigma=0$ e
$\sigma=1$, notando come il numero di simboli $\sigma=1$ fosse molto ridotto
rispetto al totale, essendo il $\sim 0.03\%$ del totale dei simboli. Una tale
sparsità del dato ha diretta conseguenza sul numero di run di ogni colonna, avendo
pochi simboli $\sigma=1$ in ogni colonna, simboli che possono anche
essere nella medesima run dopo la permutazione data dalla
\textit{PBWT}. Si ricordi, inoltre, che tale permutazione, come la
\textit{BWT}, è studiata per essere 
maggiormente efficiente nel caso del dato biologico, comportando un'alta
probabilità di produrre run del medesimo carattere. Studiando, quindi, il
numero medio di run per colonna e il numero 
massimo di run in una colonna si è confermato tale risultato, infatti:
\begin{itemize}
  \item per il cromosoma 22 si ha un numero medio di 14 run, con un massimo di
  2.450 run. Il numero totale di run è 14.772.105
  \item per il cromosoma 20 si ha un numero medio di 11 run, con un massimo di
  2.176 run. Il numero totale di run è 19.966.504
  \item per il cromosoma 18 si ha un numero medio di 11 run, con un massimo di
  2.365 run. Il numero totale di run è 24.288.263
  \item per il cromosoma 16 si ha un numero medio di 12 run, con un massimo di
  2.330 run. Il numero totale di run è 31.187.856
  \item per il cromosoma 1 si ha un numero medio di 11 run, con un massimo di
  2.721 run. Il numero totale di run è 69.671.952
\end{itemize}
Si conferma quindi il risultato atteso, risultato che è a favore, in termini di
complessità in spazio, della \textit{RLPBWT} in quanto tutte le componenti sono
proporzionali al numero, sperimentalmente basso, di run. In figura
\ref{fig:chrrun} si riportano graficamente tali risultati.
\begin{figure}
  \centering
  \begin{subfigure}{.45\textwidth}
    \centering
    \includegraphics[width=\linewidth]{img/22_runs.png}
  \end{subfigure}%
  \begin{subfigure}{.45\textwidth}
    \centering
    \includegraphics[width=\linewidth]{img/20_runs.png}
  \end{subfigure}
  \begin{subfigure}{.45\textwidth}
    \centering
    \includegraphics[width=\linewidth]{img/18_runs.png}
  \end{subfigure}%
  \begin{subfigure}{.45\textwidth}
    \centering
    \includegraphics[width=\linewidth]{img/16_runs.png}
  \end{subfigure}
  \begin{subfigure}{.45\textwidth}
    \centering
    \includegraphics[width=\linewidth]{img/1_runs.png}
  \end{subfigure}
  \caption{Distribuzione delle run per colonna con il numero
    minimo/medio/massimo delle stesse. Nel titolo si hanno anche il numero
    complessivo di run, il numero di simboli $\sigma=0$ e $\sigma=1$, nonché la
    percentuale della quantità di questi ultimi sul totale dei simboli.}
  \label{fig:chrrun}
\end{figure}
\dc{Sistema numero grossi nei plot}
\subsection{Costruzione delle strutture e calcolo degli SMEM}
Viste le dimensioni di tali pannelli si ritiene necessario studiare, dal punto
di vista del tempo macchina e dei picchi di memoria necessaria, le varie fasi
della sperimentazione, ovvero:
\begin{itemize}
  \item la fase di \textit{preprocessing}, necessaria per la preparazione dei
  vari input della\textit{RLPBWT}, comprendente: 
  \begin{itemize}
    \item la conversione dei file in formato VCF nei file in formato MACs
    \item l'estrazione del pannello delle query e la creazione del nuovo
    pannello di aplotipi reference
    \item la produzione dell\textit{SLP} del pannello di reference, comprendente
    sia la produzione della stringa unica monodimensionale che l'esecuzione di
    \textit{BigRepair} e \textit{ShapedSlp}
  \end{itemize}
  \item la costruzione e serializzazione delle varie strutture dati composte per
  la \textit{RLPBWT} e dei file ad hoc per la \textit{PBWT}
  \item il calcolo degli \textit{SMEM}
\end{itemize}
\paragraph{Preprocessing}
In figura \ref{fig:prechr} si possono analizzare le prestazioni delle tre
fasi di preprocessing. La separazione del pannello con le query risulta essere
assolutamente 
ininfluente e, di fatto, anche la conversione tra i due formati non
necessita particolari considerazioni. Tale conversione 
diventerebbe non necessaria implementando l'input direttamente da file VCF per
le varie strutture dati relative alla \textit{RLPBWT}. Inoltre, in un contesto
reale, la costruzione del pannello di query non sarebbe un'operazione
necessaria.\\
Bisogna, però, analizzare la costruzione 
dell'\textit{SLP}. Per quanto quest'operazione sia da svolgersi \textit{una
  tantum}, le richieste in termini di memoria sono nell'ordine delle centinaia
di gigabytes di RAM mentre i tempi di calcolo sono nell'ordine delle
ore. D'altro canto, bisogna considerare che tutti gli algoritmi per la produzione
dell'\textit{SLP} sono studiati per partire da una singola stringa e non da una
matrice. Nuovi sviluppi in questa direzione potrebbero lasciar spazio a diverse
ottimizzazioni. Inoltre,
questa fase è necessaria solo per due delle tre soluzioni studiate per la
\textit{RLPBWT} e, come già detto, il fatto che sia necessaria solo una volta
deve essere preso in considerazione nell'ottica di un confronto con, ad esempio,
lo spazio richiesto dall'\textit{algoritmo 5} di Durbin, che richiede $13NM$
bytes ad ogni esecuzione di calcolo degli SMEM.
\begin{figure}
  \centering
  \includegraphics[width=\linewidth]{img/prep_mem_time.png}
  \caption{Picchi di memoria (a) e tempo richiesto (b) per le tre fasi di
    preprocessing dell'input per la \textit{RLPBWT}, in scala logaritmica.}
  \label{fig:prechr}
\end{figure}
In figura \ref{fig:slpmacschr} si può osservare il vantaggio in termini di
memoria che si ha con l'uso degli \textit{SLP}, confrontando il peso dei file
MACs con il peso delle grammatiche compresse. Si segnala, in ogni caso, che il
peso dei vari file MACs include anche i diversi header. In tabella
\ref{tab:slpmacs} si possono
confrontare quantitativamente tali risultati.
\begin{table}
  \centering
  \caption{Dimensioni, in gigabytes, degli \textit{SLP} e dei file \textit{MACs}
  per i vari pannelli del 1000 Genome Project.}
  \begin{tabular}{c|c|c|c|c}
    \textbf{\#Siti} & \textbf{\#Sample} & \textbf{SLP (\textit{GB})}
    & \textbf{MACs (\textit{GB})} & \textbf{\%}\\
    \hline
    1.055.454 & 4.908 & 0,04 & 4,84 & 0,9\\
    1.739.315 & 4.908 & 0,06 & 7,98 & 0,76\\
    2.171.378 & 4.908 & 0,08 & 9,97 & 0,79\\
    2.596.072 & 4.908 & 0,1 & 11,91 & 0,81\\
    6.196.151 & 4.908 & 0,22 & 28,44 & 0,78\\
  \end{tabular}
  \label{tab:slpmacs}
\end{table}
È possibile, grazie a tali risultati, apprezzare la compressione di tali
grammatiche.  
Si noti inoltre che, essendo la capacità di compressione di un \textit{SLP}
direttamente correlata alle ripetizioni presenti nella stringa da comprimere, la
percentuale decrescente risulti essere un risultato atteso, avendo maggior
probabilità di avere ripetizioni all'aumentare delle dimensioni del pannello,
parlando di dati reali.
\begin{figure}
  \centering
  \includegraphics[width=0.7\linewidth]{img/slp_vs_macs_log.png}
  \caption{Confronto tra la memoria richiesta dai file MACs e dagli SLP per i
    pannelli del 1000 Genome Project, in scala
    logaritmica.} 
  \label{fig:slpmacschr}
\end{figure}
\paragraph{Costruzione della struttura}
Passiamo ora ad analizzare tempi e picchi di memoria per la costruzione delle
strutture dati. Bisogna ricordare che:
\begin{itemize}
  \item nel caso della \textit{RLPBWT}, per ognuna delle strutture dati
  composte, questa fase prevede la costruzione e la 
  serializzazione dell'intera struttura dati
  \item nel caso della \textit{PBWT} questa fase crea unicamente un file
  compresso ad hoc, contenente le strutture base delle \textit{PBWT}. A partire
  da tale file, in fase di calcolo degli \textit{SMEM}, verranno calcolati anche
  tutti gli altri indici al calcolo degli stessi, a seconda dell'algoritmo usato
\end{itemize}
Fatte queste doverose premesse, passiamo ad analizzare i risultati.
In figura \ref{fig:maketimememchr} (a) vengono visualizzati i picchi di
memoria richiesti mentre in figura \ref{fig:maketimememchr} (b) i tempi di
calcolo delle strutture.\\ 
Come anticipato, l'implementazione della \textit{PBWT} non calcola e memorizza
tutti gli indici 
necessari al calcolo degli \textit{SMEM} in fase di costruzione, avendo quindi
una bassissima richiesta di memoria in questa fase. Discorso diverso si ha
parlando delle strutture dati per la \textit{RLPBWT}. Le strutture dati composte
\texttt{MAP-INT + LCP} e \texttt{MAP-BV + LCP}, dovendo 
memorizzare l'intero insieme degli \textit{array LCP}, hanno un elevato
consumo di memoria. Pur utilizzando degli \textit{int vector compressi}, in modo
analogo a quanto visto, ad esempio, per la componente \texttt{PERM}, si ha
necessità di salvare $NM$ valori interi, valori che, nel caso peggiore, sono
pari al numero di siti del pannello. Interessante notare, inoltre, come in
termini di costruzione, siano le due strutture che non scalano sul numero di run
ad aver maggiori richieste di memoria. L'utilizzo della componente
\texttt{MAP-BV}, come atteso, richiede maggior memoria della componente
\texttt{MP-INT}. In merito invece ai tempi di costruzione delle due strutture si
segnala come i tempi di calcolo della componente \texttt{MAP-BV} siano superiori
rispetto a quelli della \texttt{MAP-INT}, dovendo, ad esempio, calcolare
anche le strutture per le funzioni \textit{rank/select} e lavorare su strutture
dati succinte quali i \textit{bitvector sparsi}. Nel caso della componente
\texttt{MAP-INT}, invece, l'unica operazione aggiuntiva rispetto a quelle attese
per una classica popolazione di array di interi e la fase di compressione degli
stessi. Le analisi effettuate sulle componenti di \textit{mapping} sono da
ritenersi analoghe per le componenti dedicate alle \textit{threshold}, in
termini di uso di \textit{bitvector sparsi} e \textit{int vector
  compressi}. Tali considerazioni sono valide anche per le altre strutture
composte per la \textit{RLPBWT}. Infine, confrontando le strutture dati in grado
di computare l'array $MS$, si aggiungono le considerazioni sulle possibili
strutture per il \textit{random access} e, nel caso dell'\textit{SLP}, per l'uso
delle \textit{LCE query}. Si nota come l'uso della componente \texttt{RA-BV}
comporta, come atteso, una maggior richiesta di memoria, pur limitata dall'uso
dei \textit{bitvector}. In termini di tempi di calcolo, inoltre, si ha che la
componente \texttt{RA-BV} deve essere computata in fase di costruzione delle
strutture, comportando un aumento dei tempi di calcolo. Confrontando i tempi di
tutte le varianti si ha che tutti gli algoritmi di costruzione sono in tempo
proporzionale a $\mathcal{O}(NM)$ ma, come detto, le varianti della
\textit{RLPBWT} includono  in questa fase anche il calcolo delle strutture utili
al calcolo degli SMEM. Si noti come, parlando di \textit{RLPBWT}, la struttura
composta \texttt{MAP-INT + LCE + PERM + PHI} risulti essere la meno costosa in
termini di memoria, usando la struttura di mapping tramite \textit{int vector
  compressi} e l'\textit{SLP} (per le \textit{LCE query}). Anche in termini di
tempo, per i discorsi fatti sui tempi di calcolo delle singole componenti,
risulta essere la soluzione più vantaggiosa.\\
In generale, si può 
concludere che questa frase conferma quanto discusso nel capitolo \ref{metchap}.
\begin{figure}
  \centering
  \includegraphics[width=\linewidth]{img/make_time_mem_paper.png}
  \caption{Picchi di memoria (a) e tempi di calcolo (b) per la
    costruzione delle varianti della RLPWBT e per 
    la PBWT.}
  \label{fig:maketimememchr}
\end{figure}
Sfruttando i metodi offerti da \textit{SDSL-LITE}, è possibile studiare
l'occupazione di memoria delle singole componenti trattate nel capitolo
\ref{metchap}. 
In tabella \ref{tab:comp}, in tabella \ref{tab:comp2} e in figura \ref{fig:comp}
si riportano, in megabytes, 
le dimensioni di tali componenti. Si può,
innanzitutto, apprezzare il vantaggio dell'uso della componente
\texttt{RA-SLP/LCE} rispetto alla 
componente \texttt{RA-BV}. Numericamente tale vantaggio è riportato in tabella
\ref{tab:slppanel}.
\begin{table}
  \centering
  \caption{Vantaggio percentuale dell'uso delle componenti \texttt{RA-SLP/LCE}
    rispetto alla componente \texttt{RA-BV}.}
  \begin{tabular}{c|c|c|c|c}
    \textbf{\#Siti} & \textbf{\#Sample} & \textbf{\texttt{RA-SLP/LCE}
                                          (\textit{MB})}
    & \textbf{\texttt{RA-BV} (\textit{MB})} & \textbf{\%}\\
    \hline
    1.055.454 & 4.908 & 44,7915 & 628,093 & 7,13\\
    1.739.315 & 4.908 & 61,8933 & 1.035,05 & 5,98\\
    2.171.378 & 4.908 & 80,1648 & 1.292,17 & 6,2\\
    2.596.072 & 4.908 & 98,7267 & 1.544,9 & 6,39\\
    6.196.151 & 4.908 & 226,92 & 3.687,28 & 6,15\\
  \end{tabular}
  \label{tab:slppanel}
\end{table}
Confermando nuovamente quando detto diverse volte nella tesi, si segnala il
forte vantaggio in memoria nell'utilizzo delle componenti basate su \textit{int
  vector compressi}, rispetto che a quelle basate su \textit{sparse
  bitvector}.\\
Si nota, infine, come le componenti \texttt{PERM} e \texttt{PHI}, non
presentino particolari criticità dal punto di vista della memoria
richiesta. Infine, terminando l'analisi di tali risultati, senza trattare
nuovamente le componenti per il \textit{random access}, si ha conferma della
richiesta eccessiva in memoria della componente \texttt{LCP}.
\dc{Serve altro?}
\begin{figure}
  \centering
    \includegraphics[width=\linewidth]{img/comp_mem.png}
  \caption{Memoria occupata dalle singole componenti, avendo sulle ascisse in
    (a) il numero di siti e in (b) il numero di run. }
  \label{fig:comp}
\end{figure}
\begin{table}
  \centering
  \caption{Dimensioni, in megabytes, delle componenti di
    \textit{mapping} e \textit{threshold} usate nelle
    strutture dati per la \textit{RLPBWT}.} 
  \label{tab:comp}
  \begin{tabular}{c||c|c|c|c}
    Chr & \texttt{MAP-INT} & \texttt{MAP-BV} & \texttt{THR-INT}
    & \texttt{THR-BV}  \\ \hline
    22 & 74,275 & 542,531 & 36,077 & 198,964\\ \hline
    20 & 109,08 & 881,529 & 52,624 & 322,491\\ \hline
    18 & 137,49 & 1.099,598 & 61,352 & 401,965\\ \hline
    16 & 167,046 & 1.320,331 & 80,7 & 483,269\\ \hline
    1 & 384,12 & 3.133,008 & 185,228 & 1.145,67
  \end{tabular}
\end{table}
\begin{table}
  \centering
  \caption{Dimensioni, in megabytes, delle restanti componenti usate nelle
    strutture dati per la \textit{RLPBWT}.} 
  \label{tab:comp2}
  \begin{tabular}{c||c|c|c|c|c}
    Chr & \texttt{RA-BV} & \texttt{RA-SLP/LCE} & \texttt{PERM} & \texttt{PHI}
    & LCP\\
    \hline
    22 & 628,093 & 44,7915 & 71,3182 & 89,0774 & 9.094,841\\ \hline
    20 & 1.035,05 & 61,8933 & 103,882 & 117,856 & 15.467,57 \\ \hline
    18 & 1.292,17 & 80,1648 & 127,769 & 147,416 & 19.222,808\\ \hline
    16 & 1.544,9 & 98,7267 & 159,307 & 186,033 & 22.887,82\\ \hline
    1 & 3.687,28 & 226,92 & 365,615 & 410,983 & 54.588,092
  \end{tabular}
\end{table}
\paragraph{Calcolo degli SMEM}
Passando ora a discutere dei risultati ottenuti per il calcolo degli
\textit{SMEM}.\\ 
In figura \ref{fig:smemtimememchr} (a) si riportano i risultati i termini di
picchi di 
memoria durante la computazione degli \textit{SMEM}. Come previsto, l'algoritmo
\texttt{matchDynamic} della \textit{PBWT} ha le migliori prestazioni
in spazio, calcolando dinamicamente i vari indici necessari al calcolo degli
\textit{SMEM interni}. D'altro canto per quanto riguarda l'\textit{algoritmo 5}
di Durbin, ovvero l'algoritmo \texttt{matchIndexed}, si confermano le previsioni
fatte dall'autore stesso, avendo che la memoria utilizzata è circa $13NM$
bytes. Escludendo le strutture \texttt{MAP-INT + LCP} e \texttt{MAP-BV + LCP},
si parla di circa un'intero ordine di grandezza in più di memoria rispetto alle
strutture dati composte per la \textit{RLPBWT}. Parlando di queste ultime, la
differenza tra le varie strutture dati che supportano il calcolo 
dell'array $MS$ è dovuta, a parità di componente per il mapping (e
conseguentemente della componente per le threshold), dall'uso della componente
\texttt{RA-BV} o della componente \texttt{RA-SLP} (o della componente
\texttt{LCE}). All'aumentare delle dimensioni del pannello di aplotipi ci si
aspetta che tale vantaggio sia sempre maggiore, a causa della probabilità di
avere ripetizioni all'interno della stringa rappresentante la matrice.\\
% Interessante è notare il rapporto tra la memoria richiesta dalla \textit{RLPBWT}
% con \textit{SLP} e \textit{LCE} e la \textit{PBWT Indexed}:
% \begin{table}[H]
%   \centering
%   \footnotesize
%   \begin{tabular}{c|c|c|c|c}
%     \textbf{\#Samples} & \textbf{\#Siti}
%     & \textbf{RLPBWT SLP-LCE (\textit{kb})}
%     & \textbf{PBWT Indexed (\textit{kb})} & \textbf{\%}\\
%     \hline
%     4908 & 1055454 & 3058088 & 65975520 & 4.64\\
%     4908 & 1739315 & 4961664 & 108713424 & 4.56\\
%     4908 & 2171378 & 6190684 & 135726084 & 4.56\\
%     4908 & 2596072 & 7430300 & 162257008 & 4.58\\
%     4908 & 6196151 & 17635700 & 387252160 & 4.55
%   \end{tabular}
% \end{table}
% Anche in questo caso le percentuali risultano leggermente peggiori rispetto ai
% pannelli simulati, pur restando risultati molto interessanti.
In figura \ref{fig:smemtimememchr} (b) si riportano, invece, i risultati i
termini di tempo di calcolo. Anche in questo caso l'algoritmo
\texttt{matchDynamic} risulta essere il più performante, in quanto studia
contemporaneamente l'intero pannello di query. Parlando di \textit{RLPBWT},  la
struttura \texttt{MAP-INT + THR-INT + RA-SLP + PERM + PHI} e la 
struttura 
\texttt{MAP-BV + THR-BV + RA-SLP + PERM + PHI}, a causa delle
frequenti operazioni di \textit{random access} con la componente
\texttt{RA-SLP}, sia per il calcolo delle lunghezze delle 
matching statistics che per la fase di ``disambiguazione'', richiede più tempo
di tutte le altre varianti, soprattutto se si pensa alla corrispondente
variante con componente \texttt{RA-BV}. La struttura \texttt{MAP-INT + LCE +
  PERM + PHI} e la struttura 
\texttt{MAP-BV + LCE + PERM + PHI} risultano essere, al massimo,
il doppio più lenta rispetto all'algoritmo \texttt{matchIndexed}. Questo è un
risultato molto interessante se si tiene in considerazione la memoria necessaria
per il calcolo degli \textit{SMEM}.\\
Concludendo, si può  notare come la struttura composta
\texttt{MAP-INT + THR-INT + RA-BV + PERM + PHI}, tra quelle per la
\textit{RLPBWT}, risulti essere la migliore in termini di tempi di calcolo
mentre la struttura composta \texttt{MAP-INT + LCE + PERM + PHI} sia la
migliore in termini di memoria richiesta. Notando come quest'ultima sia circa 10
volte più lenta si può inferire come la scelta della miglior soluzione per la
\textit{RLPBWT} debba ricadere sulla \texttt{MAP-INT + THR-INT + RA-BV + PERM +
  PHI}, salvo situazioni in il risparmio di memoria sia un limite fondamentale
in fase di analisi dati. 
\begin{figure}
  \centering
  \includegraphics[width=\linewidth]{img/exe_time_mem_paper}
  \caption{Picchi di memoria (a) e tempi di esecuzione (b) per il calcolo degli
    SMEM.}
  \label{fig:smemtimememchr}
\end{figure}
\subsection{Tempo di una singola query}
Infine, per completare lo studio delle prestazioni temporali, si è deciso di
isolare il calcolo degli SMEM con ogni singola query, valutando media e
deviazione standard delle 100 query. A tal fine la misurazione è stata
effettuata sfruttando la libreria \texttt{time.h} presente nello standard del
linguaggio C, al fine di avere le medesime misurazioni sia con la \textit{PBWT}
che con la \textit{RLPBWT}. La misurazione è stata estremamente circoscritta, al
fine di misurare unicamente le istruzioni atte a cercare gli \textit{SMEM},
escludendo quelle per il computo degli indici o del caricamento delle strutture
dati.
Si segnala che nel caso dell'algoritmo \texttt{mathcDynamic} non si è
potuto, per natura stessa dell'algoritmo, isolare il computo degli indici
all'avanzamento alla colonna successiva. Resta esclusa, in ogni caso, la
costruzione della struttura base della \textit{PBWT}.
\dc{Non se dire questa cosa}
\\
Tale risultato è visualizzabile in figura
\ref{fig:smemsinglechr}, dove si è deciso di escludere le strutture
\texttt{MAP-INT + LCP} e \texttt{MAP-BV + LCP} in quanto non in grado di
computare quali righe presentino un certo \textit{SMEM}. Anche in questo caso,
si conferma molto di quanto ipotizzato e discusso precedentemente. Caso a parte
è dato dall'algoritmo \texttt{matchDynamic}, che risulta avere le performance
peggiori. Per natura stessa dell'algoritmo, 
le operazioni sono studiate al fine di essere ottimizzate per pannelli di query
e non per una query singola. Nel caso in analisi, infatti, si hanno molte
operazioni che potrebbero essere ottimizzate per il caso della singola
query, caso invece dell'algoritmo \texttt{matchIndexed}. Sperimentalmente si è
quindi notato che una query o un centinaio di query hanno quindi all'incirca lo
tempo di calcolo. Per quanto riguarda la \textit{RLPBWT}, con l'uso della
componente \texttt{RA-SLP}, si rilevano gli stessi problemi relativi all'random
access, precedentemente descritti. Questi problemi sono risolti con l'uso della
componente \texttt{RA-BV}. Inoltre, a parità di componenti per il mapping (e
conseguenti componenti per le threshold), l'uso della componente \texttt{LCE}
risulta più lenta dell'uso della componente \texttt{RA-BV}, a causa dei costi di
calcolo delle \textit{LCE query} stesse. Tutti questi sono risultati coerenti
con quanto visto nel caso di 100 query, anche in termini di migliori strutture
composte parlando di \textit{RLPBWT}.
\begin{figure}
  \centering
  \includegraphics[width=\textwidth]{img/exe_time_single_paper.png}
  \caption{Tempo medio di esecuzione del calcolo degli SMEM per una singola
    query. Il grafico di destra è in scala logaritmica e, in entrambi, le
    barre d'errore rappresentano la deviazione standard.}
  \label{fig:smemsinglechr}
\end{figure}
% LocalWords:  sottostrutture


\chapter{Conclusioni}
\label{conchap}
Fissato l'iniziale obbiettivo di risolvere le problematiche relative alla
memoria richiesta dall'\textit{algoritmo 5} di Durbin, l'implementazione della
\textbf{RLPBWT}, principalmente nelle strutture dati composte segnalate nel
Capitolo \ref{reschap}, ha riportato risultati molto incoraggianti. Come
descritto nel medesimo capitolo, la quantità di memoria richiesta risulta essere 
incredibilmente inferiore rispetto a quella richiesta dall'algoritmo
\texttt{matchIndexed}. D'altro canto, l'algoritmo \texttt{matchDynamic} di 
Durbin, per quanto non approfondito nell'articolo del 2014 \cite{pbwt}, risulta
essere ancor meno esoso di risorse, nonché incredibilmente più veloce dal punto
di vista dei tempi di calcolo, ad eccezion del caso limite di avere un numero
esiguo di query. Lo svantaggio si ritrova anche nell'ordinamento dei risultati
e nella creazione di un'ulteriore \textit{PBWT}
che, giudicando la letteratura degli ultimi anni le cui trattazioni si basano
sempre sull'algoritmo 5, sembrano implicare un non facile 
riadattamento per la risoluzione di altri task.\\
Si possono comunque rilevare alcune possibili migliorie in merito
alle varie implementazioni della \textit{RLPBWT} presentate in questa tesi,
riferendosi essenzialmente alle soluzioni che prevedono il calcolo
dell'array delle \textit{matching statistics}: 
\begin{itemize}
  \item si potrebbe pensare ad un metodo per gestire in modo efficiente lo
  studio di più query contemporaneamente, migliorando i tempi di calcolo
  complessivi. Studiando contemporaneamente tutte le query si potrebbe, alla
  stregua di quanto visto con l'algoritmo \texttt{matchDynamic}, caricare in
  memoria, di volta in volta, solo la 
  colonna necessaria ad un dato passo di computazione o comunque un sottoinsieme
  di colonne. In tal modo, si ridurrebbe l'uso massimo di memoria 
  \item studiare eventuali ottimizzazioni per la componente \texttt{MAP-BV} al
  fine di comprendere se sia possibile tenere in
  memoria un solo bitvector $uv_k$ che funzioni in modo similare a quanto si ha
  con la componente \texttt{MAP-INT}
\end{itemize}
Nonostante queste possibili migliorie, la qualità dei risultati è sufficiente
per stabilire che una variante \textit{run-length encoded} della \textit{PBWT},
alla 
stregua di quanto analizzato negli ultimi anni sulla \textit{RLBWT} con
\textit{MONI} \cite{moni} e \textit{PHONI} \cite{phoni}, sia possibile e possa
permettere, nel prossimo futuro, la memorizzazione compatta delle informazioni
necessarie allo studio di grandi pannelli di aplotipi. In un futuro in cui le
tecnologie di sequencing produrranno sempre più dati, provenienti da sempre più
individui, avere a disposizione strutture dati efficienti dal punto di vista
della memorizzazione permetterà uno studio sempre più approfondito dei dati
stessi, nei campi dei \textit{Genome-Wide Association Studies (GWAS)}, della
\textit{medicina personalizzata} etc\ldots 
% sezione sviluppi
\section{Sviluppi futuri}
Ovviamente questa prima implementazione completa della \textit{RLPBWT},
declinata nelle possibili strutture composte, non è da
considerarsi come un punto di arrivo. Come accaduto per la \textit{PBWT},
infatti, si potranno sviluppare nuove strutture dati basate su di essa per la
gestione di pannelli di varia natura. Principalmente si può pensare a due casi,
già anticipati nella sezione \ref{secpbwt}:
\begin{itemize}
  \item \textbf{pannelli multi-allelici}, ovvero costruiti su un alfabeto
  $\Sigma$ arbitrario e non limitato ai simboli $\sigma=0$ e $\sigma=1$
  \item \textbf{pannelli con dati mancanti}, ovvero pannelli costruiti
  direttamente da \textit{dati reali} che possono contenere siti, per certi
  individui, per i quali non si ha certezza in merito all'allele
\end{itemize}
Inoltre, allo stato attuale, la struttura dati è stata sviluppata per permettere
unicamente il calcolo dei match massimali con un aplotipo esterno. Anche in
questo caso, quindi, si potrebbe avere lo sviluppo di nuovi algoritmi che
rispondano a task diversi, come il calcolo dei match interni al panel, i
cosiddetti \textit{blocchi}, o anche il calcolo di tutti i match con un aplotipo
esterno di lunghezza maggiore ad una fissata o che includano un numero stabilito
di sequenze di aplotipi nel panello.
\paragraph{RLPBWT multi-allelica}
Per quanto i pannelli di aplotipi prodotti dal sequencing del genoma umano
raramente presentino siti multi-allelici si ha una presenza stimata, al momento,
di circa il 2\% di siti tri-allelici \cite{tri}. Inoltre, all'aumentare della
disponibilità di dati genomici, si ha in letteratura la propensione a credere
che tale percentuale di siti sia non solo sottostimata (stimando che sia stimato
circa un terzo dei reali siti tri-allelici) ma anche destinata a
cresce in modo non lineare rispetto al numero di individui sequenziati
\cite{tri2}. Inoltre, molte specie, soprattutto vegetali, sono già riconosciute
essere poliploidi, quindi una struttura dati efficiente in memoria in grado di
gestire pannelli costruiti su un alfabeto arbitrario risulterà necessaria nel
breve futuro.\\
Ipotizzando un possibile funzionamento della \textbf{RLPBWT
  multi-allelica (\textit{m-RLPBWT})} si può pensare ad una soluzione molto
simile a quanto visto per la \textit{RLPBWT}. Infatti, per ogni colonna, si
potrebbero memorizzare:
\begin{itemize}
  \item una stringa che memorizzi quale simbolo corrisponda ad una certa run,
  non potendo sfruttare l'alternanza di simboli vista nel caso binario
  \item una rivisitazione delle strutture necessarie al mapping, tenendo in
  memoria vettori di \textit{bitvector sparsi} o valori interi alla stregua
  della componente \texttt{MAP-INT}. Si segnala che si attende un inversione di
  tendenza in termini di memoria, avendo che, in tal caso, l'uso di \textit{int
    vector compressi} potrebbe rivelarsi meno efficiente dei \textit{bitvector
    sparsi} 
  \item riadattamento del calcolo dell'array delle \textit{matching statistics}
\end{itemize}
In merito allo spazio richiesto e ai tempi di calcolo bisognerà considerare la
grandezza dell'alfabeto su cui è costruito il pannello, che ci si aspetta comune
inferiore a $10$ nella maggioranza dei casi di studio biologico.\\
Nonostante, allo stato dell'arte, ci siano pochissimi studi in merito si ritiene
possibile generalizzare, in modo computazionalmente efficiente, la
\textit{RLPBWT} anche a questa casistica. 
\paragraph{RLPBWT con dati mancanti}
La maggior parte delle soluzioni attualmente sviluppate sono basate su una forte
assunzione: i dati in input sono corretti e senza dati mancanti. Ovviamente,
limitandosi a studiare pannelli simulati o comunque ``riempiti'' in una fase di
preprocessing, si rischia di non poter comprendere a fondo l'efficacia dei
metodi su dati reali, oltre che a limitare l'inferenza dai pannelli stessi.\\
Come anticipato alla sezione \ref{secpbwt}, si sono iniziate a sviluppare
estensioni della \textit{PBWT} che ammettano wildcard, ovvero simboli nel
pannello che possono assumere qualsiasi valore dell'alfabeto $\Sigma$, su cui è
costruito il pannello stesso.\\
Uno degli sviluppi futuri sarebbe quindi quello di generalizzare la
\textit{RLPBWT}, ma anche l'eventuale \textit{m-RLPBWT}, per la gestione di dati
mancanti nel pannello. Inoltre si potrebbero sviluppare algoritmi in grado di
gestire le wildcard anche all'interno delle query stesse.\\
Sempre in via ipotetica, l'uso di \textit{algoritmi parametrici} (ma anche
di \textit{algoritmi approssimati}) adattati al
funzionamento della \textit{RLPBWT} potrebbero portare a soluzioni interessanti
per la gestione di pannelli reali.
\paragraph{K-SMEM}
Come anticipato, oltre che variare le caratteristiche del pannello in analisi,
si possono studiare anche algoritmi per risolvere nuovi task con la
\textit{RLPBWT}.\\ 
Di recente, Gagie \cite{kmems} ha proposto un articolo in cui dimostra come
la struttura implementata in \textit{MONI} \cite{moni} sia già predisposta al
calcolo dei \textbf{k-MEM}, ovvero match massimali tra sotto-stringhe di un
pattern e un testo che occorrono esattamente $k$ volte nel testo stesso.\\
In merito alla \textit{RLPBWT} si potrebbe adattare l'idea di Gagie al calcolo
di \textbf{K-SMEM} tra sotto-stringhe dell'aplotipo query e il pannello che
comportino il match con esattamente $k$ righe del pannello stesso. L'ormai
empiricamente dimostrata correlazione tra la \textit{RLBWT} e la \textit{RLPBWT}
porta a pensare che tale problema sia risolvibile anche con la nuova definizione
di \textit{matching statistics} per la \textit{RLPBWT}.\\
Ovviamente nulla è stato sviluppato al momento ma si ritiene questo
un'interessante sviluppo futuro in quanto permetterebbe studi statistici, molto
comuni nei \textit{GWAS}, in merito alla presenza si sotto-sequenze di un
aplotipo esterno all'interno di un pannello di aplotipi.\\
\\
\\
\textit{La tematica della \emph{pangenomica} è praticamente nuova e il numero di
problemi aperti è incredibilmente grande. I dati aumentano sempre di più e gli
studi informatici devono evolversi per ``stare al passo'' con questa mole
d'informazioni. Gli \emph{sviluppi futuri} sono, da diversi punti di vista,
anche imprevedibili. Risulta quindi difficile elencare in modo completo le
possibilità future dietro questa branca della bioinformatica e dell'algoritmica
sperimentale.}
\dc{Frase conclusiva da modificare fortemente}
% LocalWords:  preprocessing

%% BIBLIOGRAFIA
% \addcontentsline{toc}{chapter}{Bibliografia e sitografia}
% \printbibliography[title={Bibliografia e sitografia}]
%\addcontentsline{toc}{chapter}{Bibliografia}
\addcontentsline{toc}{chapter}{Riferimenti}
\bibliographystyle{unsrt}
\bibliography{thesis}
%\dc{Sistemare tutte le citazione coi DOI}
%\appendix
% \chapter{Tabelle}
% % tabella dello spazio occupato dalle varianti dei bit vector
\begin{table}[H]
  \small
  \centering
  \caption{Stime dello spazio occupato per la memorizzazione di alcune varianti
    di \textit{bit vector}. Si 
    assume un bit vector di lunghezza $n$ con un numero di bit posti pari a
    1 (o $\top$) pari a $m$. $K$ indica un valore costante.} 
  \begin{tabular}{c|c}
    \textbf{Variante} & \textbf{Spazio occupato}\\
    \hline\xrowht{15pt}
    \textit{Plain bitvector} & $64\big\lceil\frac{n}{64}+1\big\rceil$\\
    \hline\xrowht{15pt}
    \textit{Interleaved bitvector} & $\approx n\left(1+\frac{64}{K}\right)$\\
    \hline\xrowht{15pt}
    \textit{$H_0$-compressed bitvector} & $\approx\big\lceil\log\binom{n}{m}\big\rceil$\\
    \hline\xrowht{15pt}
    \textit{Sparse bitvector} & $\approx m\left(2+\log\frac{n}{m}\right)$\\
  \end{tabular}
  \label{tab:bvspace}
\end{table}

% tabella relativa ai costi della funzione rank dei bitvector
\begin{table}[H]
  \small
  \centering
  \caption{Complessità temporali stimate della \textit{funzione rank} per alcune
    varianti di \textit{bit 
      vector}, con la quantità di bit aggiuntivi richiesta. Si assume un bit
    vector di lunghezza $n$, con un numero di bit 
    posti pari a 1 (o $\top$) pari a $m$, e un numero $k$ di valori prima della
    posizione richiesta.} 
  \begin{tabular}{c|c|c}
    \textbf{Variante} & \textbf{Bit aggiuntivi} & \textbf{Complessità
                                                  temporale}\\ 
    \hline\xrowht{15pt}
    \textit{Plain bitvector} & $0.0625\cdot n$ & $\mathcal{O}(1)$\\
    \hline\xrowht{15pt}
    \textit{Interleaved bitvector} & $128$ & $\mathcal{O}(1)$\\
    \hline\xrowht{15pt}
    \textit{$H_0$-compressed bitvector} & $80$ & $\mathcal{O}(k)$\\
    \hline\xrowht{15pt}
    \textit{Sparse bitvector} & $64$ & $\mathcal{O}\left(\log\frac{n}{m}\right)$\\ 
  \end{tabular}
  \label{tab:rank}
\end{table}

% tabella relativa ai costi della funzione select dei bitvector
\begin{table}[H]
  \small
  \centering
  \caption{Complessità temporali stimate della \textit{funzione select} per
    alcune varianti di \textit{bit 
      vector}, con la quantità di bit aggiuntivi richiesta. Si assume un bit
    vector di lunghezza $n$, con un numero di bit 
    posti pari a 1 (o $\top$) pari a $m$.} 
  \begin{tabular}{c|c|c}
    \textbf{Variante} & \textbf{Bit aggiuntivi} & \textbf{Complessità
                                                  temporale}\\ 
    \hline\xrowht{15pt}
    \textit{Plain bitvector} & $\leq 0.2\cdot n$ & $\mathcal{O}(1)$\\
    \hline\xrowht{15pt}
    \textit{Interleaved bitvector} & $64$ & $\mathcal{O}(\log n)$\\
    \hline\xrowht{15pt}
    \textit{$H_0$-compressed bitvector} & $64$ & $\mathcal{O}(\log n)$\\
    \hline\xrowht{15pt}
    \textit{Sparse bitvector} & $64$ & $\mathcal{O}(1)$\\ 
  \end{tabular}
  \label{tab:rank}
\end{table}
%\chapter{Algoritmi}
%


% \begin{algorithm}
%   \scriptsize
%   \begin{algorithmic}[1]
%     \Function{build\_ms}{$col,\,\, pref,\,\, div$}
%     \State $c\gets 0,\,\,u\gets 0,\,\,v\gets 0,\,\,u'\gets 0,\,\, v'\gets
%     0,\,\,curr_{lcs}\gets 0,\,\,tmp_{thr}\gets 0,\,\,tmp_{beg}\gets 0$
%     \State $start \gets \top,\,\,beg_{run}\gets \top,\,\,push_{zero}\gets
%     \bot,\,\,push_{one}\gets \bot$
%     \For {\textit{every} $k\in\left[0,\,\, height\right)$}
%     \If{$k=0\land col[pref[k]]=1$}
%     \State $start \gets \bot$
%     \EndIf
%     \If{$col[k]=0$}
%     \State $c\gets c+1$
%     \EndIf
%     \EndFor
%     \State $runs\gets[0..0]$
%     \Comment sparse bitvector for runs of length $height+1$
%     \State $thrs\gets[0..0]$
%     \Comment sparse bitvector for thresholds of length $height$
%     \State $zeros\gets[0..0]$
%     \Comment sparse bitvector for zeros of length $c$
%     \State $ones\gets[0..0]$
%     \Comment sparse bitvector for ones of length $height-c$
%     \State $samples_{beg} \gets [],\,\,samples_{end}\gets []$
%     \Comment couple of vectors for samples of length $r$
%     \If{$start$}
%     \State $push_{one}\gets \top$
%     \Else
%     \State $push_{zero}\gets \top$
%     \EndIf
%     \For {\textit{every} $k\in\left[0,\,\, height\right)$}
%     \If{$beg_{run}$}
%     \State $u\gets u',\,\,v\gets v',\,\,tmp_{beg}\gets pref[k]$
%     \State $beg_{run}\gets \bot$
%     \EndIf
%     \If{$col[pref[k]]=1$}
%     \State $v'\gets v'+1$
%     \Else
%     \State $u'\gets u'+1$
%     \EndIf
%     \If{$k=0\lor col[pref[k]]\neq col[pref[k-1]]$}
%     \State $curr_{lcs}\gets div[k],\,\,tmp_{thr}\gets k$
%     \EndIf
%     \If{$div[k]<curr_{lcs}$}
%     \State $curr_{lcs}\gets div[k],\,\,tmp_{thr}\gets k$
%     \EndIf
%     \If{$k=height-1\lor col[pref[k]]\neq col[pref[k+1]]$}
%     \State $runs[k]\gets 1$
%     \If{$k\neq height-1\land div[k+1]<div[tmp_{thr}]$}
%     \State $thrs[k]\gets 1$
%     \Else
%     \State $thrs[tmp_{thr}]\gets 1$
%     \EndIf
%     \State $push(samples_{beg}, tmp_{beg})$
%     \State $push(samples_{end}, pref[k])$
%     \If{$push_{one}$}
%     \If{$v\neq 0$}
%     \State $ones[k-1\gets 1$
%     \EndIf
%     \State $swap(push_{zero},\,\,push_{one})$
%     \Else
%     \If{$u\neq 0$}
%     \State $zeros[k-1]\gets 1$
%     \EndIf
%     \State $swap(push_{zero},\,\,push_{one})$
%     \EndIf
%     \State $beg_{run}\gets \top$
%     \EndIf
%     \EndFor
%     \If{$|zeros|\neq 0$}
%     \State $zeros[|zeros|-1]\gets 1$
%     \EndIf
%     \If{$|ones|\neq 0$}
%     \State $ones[|ones|-1]\gets 1$
%     \EndIf
%     \State \textit{build rank/select for the four bitvectors}
%     \State \textbf{return}
%     $(start,\,\,c,\,\,runs,\,\,zeros,\,\,ones,\,\,,\,\,thr,\,\,samples_{beg},\,\,
%     samples_{end},\,\,div)$  
%     \EndFunction
%   \end{algorithmic}
%   \caption{{\footnotesize{Algoritmo per la costruzione di una colonna della
%   \textit{RLPBWT} con matching statistics}}}
%   \label{algo:cosms}
% \end{algorithm}




% \begin{algorithm}
%   \footnotesize
%   \begin{algorithmic}[1]
%     \Function{external\_matches}{$z$}
%     \Comment assuming $|z|=rlpbwt.width$
%     \State $f\gets 0,\,\,f_{run}\gets 0,\,\,f'\gets 0,\,\,e\gets 0,\,\,l\gets 0$
%     \State $g\gets 0,\,\,g_{run}\gets 0,\,\,g'\gets 0,\,\,f_{off}\gets
%     0,\,\,g_{off}\gets 0$ 
%     \For {\textit{every} $k\in\left[0,\,\, |z|\right)$}
%     \State $f_{run}\gets index\_to\_run(f,k),\,\,g_{run}\gets
%     index\_to\_run(g,k)$
%     \State $f_{off}\gets f-p_k[f_{run}],\,\,g_{off}\gets g-p_k[g_{run}]$
%     \If{$z[k]=0$}
%     \State \textbf{if} $get\_symbol(start^k, f_{run})=1$ \textbf{then}
%     $f_{off}\gets 0$
%     \State \textbf{if} $get\_symbol(start^k, g_{run})=1$ \textbf{then}
%     $g_{off}\gets 0$ 
%     \Else
%     \State \textbf{if} $get\_symbol(start^k, f_{run})=0$ \textbf{then}
%     $f_{off}\gets 0$
%     \State \textbf{if} $get\_symbol(start^k, g_{run})=0$ \textbf{then}
%     $g_{off}\gets 0$ 
%     \EndIf
%     \State $f'\gets lf\_off(k,\,\, f,\,\, z[k], f_{off}),\,\,g'\gets lf(k,\,\,
%     g,\,\, z[k], g_{off}),\,\,l\gets g-f$
%     \State \textbf{if} $f'>height$ \textbf{then} $f'\gets f'-f_{off}$
%     \State \textbf{if} $g'>height$ \textbf{then} $g'\gets g'-g_{off}$
%     \If{$f'<g'$}
%     \State $f\gets f',\,\,g\gets g'$
%     \Else
%     \If{$k\neq 0$}
%     \State \textbf{report} \textit{matches in} $[e,\,\, k-1]$ \textit{with} $l$
%     haplotypes   
%     \EndIf
%     \State \textbf{if} $f'=|lcp^{k+1}|$ \textbf{then} $e\gets k+1$ \textbf{else}
%     $e\gets k-lcp^{k+1}[f']$ 
    
%     \If{$(z[e]=0\land f'>0)\lor f'=height$}
%     \State $f'\gets g'-1$
%     \If{$e\geq 1$}
%     \State $f_{rev}\gets f',\,\,k'\gets k+1$
%     \State \textbf{while} $k'\neq e-1$ \textbf{do} $f_{rev}\gets reverse\_map(k',
%     \,\,f_{rev}),\,\,k'\gets k'-1$ 
%     \State $run\gets index\_to\_run(f_{rev},k'),\,\,symb\gets
%     get\_symbol(start^{k'}, 
%     run)$ 
%     \While {$e>0\land z[e-1]=symb$}
%     \State $f_{rev}\gets reverse\_map(e, \,\,f_{rev})$
%     \State $run\gets index\_to\_run(f_{rev}, e-1),\,\,symb\gets
%     get\_symbol(start^{e-1}, run)$ 
%     \EndWhile
%     \EndIf
%     \State \textbf{while} $f'>0\land (k+1)-lcp^{k+1}[f]\leq e$ \textbf{do}
%     $e\gets e-1$ 
%     \State $f\gets f',\,\,g\gets g'$
%     \Else
%     \State $g'\gets f'-1$
%     \If{$e\geq 1$}
%     \State $f_{rev}\gets f',\,\,k'\gets k+1$
%     \State \textbf{while} $k'\neq e-1$ \textbf{do} $f_{rev}\gets reverse\_map(k',
%     \,\,f_{rev}),\,\,k'\gets k'-1$ 
%     \State $run\gets index\_to\_run(f_{rev},k'),\,\,symb\gets
%     get\_symbol(start^{k'}, 
%     run)$ 
%     \While {$e>0\land z[e-1]=symb$}
%     \State $f_{rev}\gets reverse\_map(e, \,\,f_{rev})$
%     \State $run\gets index\_to\_run(f_{rev},e-1),\,\,symb\gets
%     get\_symbol(start^{e-1}, run)$ 
%     \EndWhile
%     \EndIf
%     \State \textbf{while} $e<height\land (k+1)-lcp^{k+1}[e]\leq e$ \textbf{do}
%     $e\gets e+1$  
%     \State $f\gets f',\,\,g\gets g'$
%     \EndIf
%     \EndIf
%     \EndFor
%     \If{$f<g$}
%     \State \textbf{report} \textit{matches in} $[e,\,\, |z|-1]$ \textit{with}
%     $l\gets g-f$ haplotypes   
%     \EndIf
%     \EndFunction
%   \end{algorithmic}
%   \caption{Algoritmo per match con aplotipo esterno con panel $width\times
%   height$ naive}
%   \label{algo:matchpanel}
% \end{algorithm}


% \begin{algorithm}
%   \footnotesize
%   \begin{algorithmic}[1]
%     \Function{external\_matches}{$z$}
%     \Comment assuming $|z|=rlpbwt.width$
%     \State $f\gets 0,\,\,f_{run}\gets 0,\,\,f'\gets 0$
%     \State $g\gets 0,\,\,g_{run}\gets 0,\,\,g'\gets 0$
%     \State $e\gets 0,\,\,l\gets 0$
%     \For {\textit{every} $k\in\left[0,\,\, |z|\right)$}
%     \State $f_{run}\gets rank_h^k(f),\,\,g_{run}\gets rank_h^k(g)$
%     \State $f'\gets lf(k,\,\, f,\,\, z[k]),\,\,g'\gets lf(k,\,\, g,\,\, z[k])$
%     \State $l\gets g-f$
%     \If{$f'<g'$}
%     \State $f\gets f',\,\,g\gets g'$
%     \Else
%     \If{$k\neq 0$}
%     \State \textbf{report} \textit{matches in} $[e,\,\, k-1]$ \textit{with} $l$
%     haplotypes   
%     \EndIf
%     \If{$f'=|lcp^{k+1}|$}
%     \State $e\gets k+1$
%     \Else
%     \State $e\gets k-lcp^{k+1}[f']$
%     \EndIf
    
%     \If{$(z[e]=0\land f'>0)\lor f'=height$}
%     \State $f'\gets g'-1$
%     \If{$e\geq 1$}
%     \State $f_{rev}\gets f',\,\,k'\gets k+1$
%     \While {$k'\neq e-1$}
%     \State $f_{rev}\gets reverse\_map(k', \,\,f_{rev}),\,\,k'\gets k'-1$
%     \EndWhile
%     \State $run\gets rank_h^{k'}(f_{rev}),\,\,symb\gets get\_symbol(start^{k'},
%     run)$ 
%     \While {$e>0\land z[e-1]=symb$}
%     \State $f_{rev}\gets reverse\_map(e, \,\,f_{rev})$
%     \State $run\gets rank_h^{e-1}(f_{rev})$
%     \State $symb\gets get\_symbol(start^{e-1}, run)$
%     \EndWhile
%     \EndIf
%     \State \textbf{while} $f'>0\land (k+1)-lcp^{k+1}[f]\leq e$ \textbf{do}
%     $e\gets e-1$ 
%     \State $f\gets f',\,\,g\gets g'$
%     \Else
%     \State $g'\gets f'-1$
%     \If{$e\geq 1$}
%     \State $f_{rev}\gets f',\,\,k'\gets k+1$
%     \While {$k'\neq e-1$}
%     \State $f_{rev}\gets reverse\_map(k', \,\,f_{rev}),\,\,k'\gets k'-1$
%     \EndWhile
%     \State $run\gets rank_h^{k'}(f_{rev}),\,\,symb\gets get\_symbol(start^{k'},
%     run)$ 
%     \While {$e>0\land z[e-1]=symb$}
%     \State $f_{rev}\gets reverse\_map(e, \,\,f_{rev})$
%     \State $run\gets rank_h^{e-1}(f_{rev})$
%     \State $symb\gets get\_symbol(start^{e-1}, run)$
%     \EndWhile
%     \EndIf
%     \State \textbf{while} $e<height\land (k+1)-lcp^{k+1}[e]\leq e$ \textbf{do}
%     $e\gets e+1$  
%     \State $f\gets f',\,\,g\gets g'$
%     \EndIf
%     \EndIf
%     \EndFor
%     \If{$f<g$}
%     \State \textbf{report} \textit{matches in} $[e,\,\, |z|-1]$ \textit{with}
%     $l\gets g-f$ haplotypes   
%     \EndIf
%     \EndFunction
%   \end{algorithmic}
%   \caption{Algoritmo per match con aplotipo esterno con panel $width\times
%   height$ con bitvectors}
%   \label{algo:matchpanelbv}
% \end{algorithm}



% \begin{algorithm}
%   \begin{algorithmic}
%     \Function {down}{$pos, prev, next$}
%     \State \textit{using LCE queries or random access check the longest common
%     prefix between $pos$ and $prev$ and between $pos$ and $next$}
%     \State \textit{if the latter is greater or equal return $\top$, else $\bot$}
%     \EndFunction
%   \end{algorithmic}
% \end{algorithm}




% \begin{algorithm}
%   \begin{algorithmic}[1]
%     \Function {$\varphi$}{$prefix_{value}, col$}
%     \State $res\gets
%     \varphi_{supp}^{prefix_{value}}[rank_{\varphi}^{prefix_{value}}(col)]$ 
%     \If{$res = panel_{height}$}
%     \State \textbf{return} $null$
%     \Else
%     \State \textbf{return} $res$
%     \EndIf
%     \EndFunction
%   \end{algorithmic}
%   \begin{algorithmic}[1]
%     \Function {$\varphi^{-1}$}{$prefix_{value}, col$}
%     \State $res\gets \varphi^{-1}_{supp}\,^{prefix_{value}}
%     [rank_{\varphi^{-1}}^{prefix_{value}}(col)]$
%     \If{$res = panel_{height}$ }
%     \State \textbf{return} $null$
%     \Else
%     \State \textbf{return} $res$
%     \EndIf
%     \EndFunction
%   \end{algorithmic}
%   \caption{Algoritmi per le query a $\varphi$ e $\varphi^{-1}$}
%   \label{algo:phiquery}
% \end{algorithm}


% \chapter{Esempi di File}
% \begin{listing}[H]
  \inputminted[obeytabs]{text}{code/pbwtres.txt}
  \caption{Esempio del formato file in output dopo il calcolo dei match con
    l'implementazione della \textbf{PBWT} di Durbin.}
  \label{lst:pbwtres}
\end{listing}
\newpage
\begin{listing}[H]
  \inputminted[obeytabs]{text}{code/example.macs}
  \caption{Esempio di output di \textbf{MaCS}.}
  \label{lst:pbwtres}
\end{listing}

% ringraziamenti
\begin{comment}
  \newpage
  \thispagestyle{empty}
  \begin{flushleft}
    \textit{}.%\\
    \[\sim\cdot\sim\]
    % \vspace{6mm}
    \textit{}.%\\ 
    \[\sim\cdot\sim\]
    % \vspace{6mm}
    \textit{}.\\
    \vspace{2mm}
    \textit{}.%\\
    
    \[\sim\cdot\sim\]
    % \vspace{6mm}
    \textit{}.\\
    \vspace{2mm}
    \textit{}.\\
    \vspace{2mm}
    \textit{}.
    \[\sim\cdot\sim\]
    % \vspace{6mm}
    \textit{}.
  \end{flushleft}
\end{comment}
\end{document}

% LocalWords:  divergence prefix primis