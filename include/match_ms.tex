\section{Calcolo degli SMEM con matching statistics}
L'obbiettivo di questa tesi era quello di applicare i metodi e gli algoritmi già
studiati per la \textit{BWT}, riferendosi al calcolo dei \textit{MEM} a partire
dall'array delle \textit{matching statistics}, alla \textit{PBWT}.\\
Nelle sei strutture dati dedicate al calcolo degli SMEM tramite \textit{matching
  statistics} si riconoscono le due modalità già descritte con \textit{MONI} e
\textit{PHONI}:
\begin{enumerate}
  \item calcolare l'array MS in due passaggi sfruttando il \textit{random
    access} al pannello per calcolare i vari $MS[i].len$
  \item calcolare l'array MS in un passaggio sfruttando le \textit{LCE query}
  sia per scegliere i vari $MS[i].row$ che per calcolare, in contemporanea, i
  vari $MS[i].len$ 
\end{enumerate}
\subsection{Calcolo dell'array MS con threshold}
Questa prima soluzione, necessitando sia della componente
\texttt{THR-INT}/\texttt{THR-BV} che della componente
\texttt{RA-BV}/\texttt{RA-SLP}, è relativa alle seguenti strutture dati
composte: 
\begin{itemize}
  \item \texttt{MAP-INT + THR-INT + RA-BV + PERM + PHI}
  \item \texttt{MAP-INT + THR-INT + RA-SLP + PERM + PHI}
  \item \texttt{MAP-BV + THR-BV + RA-BV + PERM + PHI}
  \item \texttt{MAP-BV + THR-BV + RA-SLP + PERM + PHI}
\end{itemize}
Tra le quali le uniche differenze si riscontrano nei tempi d'esecuzione e nella
memoria richiesta.\\
\dc{Cercare di capire come legare a R-index}
% Si noti che unendo la componente \texttt{PERM} alla componente
% \texttt{THR-INT}/\texttt{THR-BV} si ottiene una variante dell'\textbf{R-index}
% visto per la \textit{RLBWT}.\\
Vediamo il funzionamento dell'algoritmo.
Sia data $t$ la posizione della \textit{threshold} nella run corrente,
in colonna $k$, e
si supponga che tale run, con testa all'indice $h$, non sia associata al simbolo
desiderato, ovvero $z[k]$. Si supponga che, con il mapping, si sia arrivati
all'indice $i$ della colonna $k$. Si supponga inoltre che la run successiva
abbia testa in indice $e$. Si hanno due casi possibili, denotando con
$LCS(x,y)$ il \textit{longest common suffix} tra le stringhe $X$ e $Y$ e con
$a_k$ il \textit{prefix array} in colonna $k$:
\begin{enumerate}
  \item $i<t$ allora, per definizione di \textit{threshold}:
  \[LCS(z[0,k], x_{a_{k}[h-1]}[0,k])\geq LCS(z[0,k], x_{a_{k}[e]}[0,k])\]
  Quindi si ha che $MS[k].row=a_{k}[h-1]$ e il mapping potrà proseguire
  dall'indice $h-1$
  \item  $i\geq t$ allora, per definizione di \textit{threshold}:
  \[LCS(z[0,k], x_{a_{k}[s-1]}[0,k])\leq LCS(z[0,k], x_{a_{k}[e]}[0,k])\]
  Quindi si ha che $MS[k].row=a_{k}[e]$ e il mapping potrà proseguire
  dall'indice $e$
\end{enumerate}
Qualora una colonna presenti solo simboli $\sigma\neq z[k]$, per convenzione, si
imposta che $MS[k].row = M$ e si ricomincia, in colonna $k+1$, dall'ultima
posizione, indicizzata nel pannello originale dal valore finale del
\textit{prefix array sample} dell'ultima run.
\begin{esempio}
  \label{es:thr}
  Si vede, quindi, un esempio di funzionamento delle threshold per la scelta
  della riga da memorizzare in $MS[i].row$ dopo un mismatch.\\
  Si prenda pannello visto all'esempio \ref{es:pbwt1} e si effettui la
  permutazione secondo $a_2$:
  \begin{table}[H]
    \centering
    \footnotesize
    \begin{tabular}{c|cc|c|cccccccccccc}
      X & 00 & 01 & 02 & 03 & 04 & 05 & 06 & 07 & 08 & 09 & 10 & 11 & 12 & 13
      & 14 \\
      \hline
      00 & 1 & 0 & 0 & 1 & 0 & 0 & 0 & 0 & 0 & 0 & 0 & 1 & 1 & 0 & 1 \\
      01 & 1 & 0 & 0 & 1 & 1 & 0 & 0 & 1 & 0 & 0 & 0 & 0 & 0 & 1 & 1 \\
      02 & 1 & 0 & 0 & 1 & 1 & 0 & 0 & 1 & 0 & 0 & 0 & 1 & 0 & 0 & 1 \\
      03 & 1 & 0 & 0 & 1 & 1 & 0 & 0 & 1 & 0 & 0 & 0 & 1 & 0 & 0 & 1 \\
      04 & 0 & 1 & 0 & 1 & 0 & 1 & 0 & 0 & 0 & 0 & 0 & 1 & 0 & 0 & 1 \\
      05 & 0 & 1 & 0 & 1 & 0 & 1 & 0 & 0 & 0 & 0 & 0 & 1 & 0 & 0 & 1 \\
      06 & 0 & 1 & 0 & 1 & 0 & 1 & 0 & 0 & 0 & 0 & 0 & 1 & 0 & 0 & 1 \\
      07 & 0 & 1 & 0 & 1 & 0 & 1 & 0 & 0 & 0 & 0 & 0 & 0 & 1 & 0 & 1 \\
      08 & 0 & 1 & 0 & 0 & 1 & 0 & 0 & 0 & 0 & 1 & 1 & 1 & 0 & 0 & 1 \\
      09 & 0 & 1 & 0 & 1 & 0 & 0 & 0 & 0 & 1 & 0 & 0 & 0 & 0 & 1 & 1 \\
      10 & 0 & 1 & 0 & 1 & 0 & 0 & 0 & 0 & 1 & 0 & 0 & 0 & 0 & 1 & 1 \\
      11 & 0 & 1 & 0 & 0 & 1 & 0 & 0 & 0 & 0 & 0 & 1 & 1 & 0 & 0 & 0 \\
      12 & 0 & 1 & 0 & 0 & 1 & 0 & 0 & 0 & 1 & 0 & 1 & 1 & 0 & 0 & 1 \\
      13 & 0 & 1 & 0 & 0 & 1 & 0 & 0 & 0 & 1 & 0 & 1 & 1 & 0 & 0 & 1 \\
      14 & 0 & 1 & 0 & 0 & 0 & 0 & 0 & 0 & 1 & 0 & 0 & 0 & 1 & 0 & 1 \\
      15 & 0 & 1 & 0 & 0 & 0 & 0 & 0 & 0 & 1 & 0 & 0 & 0 & 1 & 0 & 1 \\
      16 & 0 & 1 & 0 & 1 & 0 & 0 & 0 & 0 & 0 & 0 & 0 & 1 & 1 & 0 & 1 \\
      17 & 0 & 1 & 1 & 0 & 1 & 0 & 0 & 0 & 0 & 0 & 0 & 1 & 0 & 0 & 1 \\
      18 & 0 & 1 & 1 & 0 & 1 & 0 & 1 & 0 & 0 & 0 & 0 & 0 & 1 & 0 & 1 \\
      19 & 1 & 1 & 0 & 0 & 0 & 1 & 0 & 0 & 0 & 0 & 0 & 1 & 1 & 0 & 1 \\
    \end{tabular}
  \end{table}
  Si prenda la seconda run, di simboli $\sigma=1$, indicizzata tra 17 e 18. \\
  Si supponga che, tramite il mapping, si sia arrivati alla riga 17 ma che si
  abbia $z[2]=0$. la scelta è quindi tra la coda della run precedente, avendo
  che $a_2[16]=16$ o la testa della run successiva, avendo che $a_2[19]=17$. Si
  può notare come il minimo \textit{LCP} si trovi, per la 
  run, all'indice 18 (a causa del fatto che il minimo \textit{LCP} è all'indice
  19, quello della testa della run successiva). Si può quindi proseguire o con
  la riga. Questo significa che il suffisso comune più lungo con la query si ha
  con la riga 16 del pannello, per definizione di threshold, avendo che questa
  sarà memorizzata nell'array $MS$:
  \[MS[2].row=16\]
  Successivamente, tramite \textit{random access} al testo, confrontando la riga
  $x_{16}$ e la query $z$, fino alla colonna $k=2$, si potrà calcolare che
  $MS[2].len=3$. 
\end{esempio}
\dc{SISTEMARE ESEMPIO}\\
Una volta computato tutti i valori $MS[i].row$ per calcolare i vari $MS[i].len$
si scorre da sinistra a destra calcolando la lunghezza dello SMEM facendo random
access al pannello e confrontando la query $z$ con la riga $MS[i].row$. Si
assuma infatti di aver calcolato $MS[i-1].len$ e di voler calcolare $MS[i].len$.
Si hanno tre casi possibili:
\begin{enumerate}
  \item $MS[i].row=M$ e in tal caso, avendo segnalata l'inesistenza di alcuno
  SMEM, si ha che $MS[i].len=0$
  \item $MS[i].row=MS[i-1].row$, avendo $i\neq 0$ e $MS[i-1].len\neq 0$, allora
  si sta seguendo la stessa riga che si seguiva in colonna $i-1$ e quindi,
  banalmente, $MS[i].len=MS[i-1].len+1$
  \item in qualsiasi altro caso bisogna confrontare, a partire dalla colonna
  $i$, la query 
  $z$ con la riga $MS[i].row$ del pannello da destra a sinistra, fino a che non
  si trova un mismatch, calcolando la lunghezza $l$ del suffisso comune tra esse
  e memorizzando tale valore, tramite $MS[i].len=l$
\end{enumerate}
In fase di costruzione delle lunghezze è possibile anche riportare gli
\textit{SMEM}, terminanti in colonna $i$, qualora:
\begin{itemize}
  \item $MS[i].len\geq MS[i+1].len \land MS[i].len\neq 0$
  \item si è arrivati a fine array, avendo $i=N-1\land MS[i].len\neq 0$
\end{itemize}
Come si può verificare nell'esempio \ref{es:ms}.\\
L'algoritmo per il match tramite \textit{threshold} è visualizzabile
all'algoritmo \ref{algo:matchthr}. Anche in questo caso la stima delle
complessità non è di facile ottenimento. Dividendo nelle varie parti l'algoritmo
si ha che:
\begin{itemize}
  \item il calcolo dei valori $row$ dell'array $MS$ varia a seconda dell'uso
  della componente \texttt{MAP-INT} o \texttt{MAP-BV}. Il costo della funzione
  \textit{down}, variabile a seconda della componente \texttt{RA} e
  dell'eventuale componente \texttt{LCE}, risulta trascurabile vista la bassa
  frequenza d'uso, in termini probabilistici. Si ha quindi che, con $\rho$
  numero medio di run per colonna, usando \texttt{MAP-INT} si ha tempo
  proporzionale a:
  \begin{equation}
    \label{eq:msthr1int}
    \mathcal{O}(N\log\rho)
  \end{equation}
  Mentre con \texttt{MAP-BV} è proporzionale a:
  \begin{equation}
    \label{eq:msthr1bv}
    \mathcal{O}\left(N\log\frac{M}{\rho}\right)
  \end{equation}
  
  \item il calcolo dei valori $len$ dell'array $MS$ è il più complesso da
  stimare in termini di complessità asintotica. Questa difficoltà è dovuta dal
  fatto che gli accessi al pannello vengono fatti solo quando $MS.row[i]\neq
  MS.row[i-1]$. Per semplicità denotiamo con $\gamma$ il numero di accessi al
  pannello e, nel caso della componente \texttt{RA-BV}, si ha che il calcolo
  complessivo delle lunghezza è proporzionale a:
  \begin{equation}
    \label{eq:msthr2bv}
    \mathcal{O}(N\gamma)
  \end{equation}
  Mentre con \texttt{RA-SLP} è proporzionale, con $s$ lunghezza della stringa
  unica prodotta dall'\textit{SLP}, a:
  \begin{equation}
    \label{eq:msthr2slp}
    \mathcal{O}\left(N\gamma\log s\right)
  \end{equation}
  \item si ha infine il calcolo effettivo degli \textit{SMEM} e di tutte le
  righe del pannello per le quali si hanno tali match. Denotando con $\delta$ il
  numero complessivo di accessi alla componente \texttt{RA-BV} o di usi della
  componente \texttt{LCE}, qualora disponibile, e con $\mu$ il numero di
  \textit{SMEM}, si ha che, nel primo caso, la complessità di tale operazione è
  proporzionale a:
  \begin{equation}
    \label{eq:msthr3bv}
    \mathcal{O}(\mu\delta)
  \end{equation}
  mentre nel caso di uso dell'\textit{SLP} e della componente \texttt{LCE},
  avendo sempre $s$ lunghezza della stringa unica prodotta dall'\textit{SLP}:
  \begin{equation}
    \label{eq:msthr3slp}
    \mathcal{O}(\mu\delta\log s)
  \end{equation}
\end{itemize}
Facendo una stima complessiva si può ipotizzare come la struttura \texttt{MAP-BV
+ THR-BV + RA-SLP + PERM + PHI}, a causa della maggior lentezza in fase di
mapping e di accesso al pannello per il calcolo delle lunghezze, sia quella con
prestazioni peggiori mentre, per il ragionamento inverso, la struttura
\texttt{MAP-INT + THR-INT + RA-BV + PERM + PHI} sia quella con le migliori
performance dal punto di vista del tempo macchina.\\
In termini di memoria, invece,  la struttura
\texttt{MAP-INT + THR-INT + RA-SLP + PERM + PHI} risulta essere la più
vantaggiosa mentre la struttura
\texttt{MAP-BV + THR-BV + RA-BV + PERM + PHI} la peggiore, per le stime viste
nelle sezioni precedenti dedicate alle singole componenti. 
\begin{algorithm}
  \scriptsize
  \begin{algorithmic}[1]
    \Function{external\_matches}{$z$}
    \State $ms_{row}\gets [0..0],\,\,ms_{len}\gets [0..0]$
    \Comment vettore $MS$ di lunghezza $|z|$
    \State $curr_{row}\gets
    rlpbwt[0].samples_{end}[|rlpbwt[0].samples_{end}|-1]$
    \State $curr_{index}\gets curr_{row}$
    \State  $curr_{run}\gets index\_to\_run(curr_{index},0)$ \textbf{oppure}
    $curr_{run}\gets rank_h^0(curr_{index})$    
    \State $symb\gets get\_symbol(start_0, curr_{run})$
    \Comment \textbf{Costruzione righe dell'array $MS$}
    \For {\textit{every} $k\in[0, |z|)$}

    \If{$z[i]=symb$}
    \State $ms_{row}[k]\gets curr_{row}$
    \State \hspace{-1.1mm}\textbf{if} $k\neq |z|-1$ \textbf{then}
    $(curr_{index},\,\,curr_{run},\,\,symb)\gets UPDATE(k, curr_{index},z)$ 
    \Else
    \State  $curr_{thr}\gets t_k[curr_{run}]$ \textbf{oppure}
    $curr_{thr}\gets rank_t^k(curr_{index})$ 
    \State $force_{down} \gets \top$\textit{ sse l'indice è sovrapposto ad una
    threshold non in coda di run}
     \State $force_{down} \gets \top$\textit{ sse l'indice è sovrapposto ad una
    threshold in coda di run e $DOWN(\ldots)=\top$}
    \If{$|samples_{beg}^k|=1$}
    \State $ms_{row}[k]\gets M$
    \If{$k\neq |z|-1$}
    \State $curr_{row}\gets
    rlpbwt[k+1].samples_{end}[|rlpbwt[k+1].samples_{end}|-1]$
    \State $curr_{index}\gets M-1$
    \State $curr_{run}\gets index\_to\_run(curr_{index},k+1)$ \textbf{oppure}
    $curr_{run}\gets rank_h^{k+1}(curr_{index})$
    \State $symb\gets get\_symbol(start_{k+1}, curr_{run})$
    \EndIf
    \ElsIf{$(curr_{run}\neq 0 \land curr_{run}=curr_{thr}\land \neg down)\lor
    curr_{run}=|samples_{beg}^k|-1$} 
    \State $curr_{index}\gets p_k[curr_{run}]-1$ \textbf{oppure}
    $curr_{index}\gets select_h^{k}(curr_{run})$
    \State $curr_{row}\gets samples_{end}^k[curr_{run}-1]$
    \State $ms_{row}[k]\gets curr_{row}$
    \State \textbf{if} $k\neq |z|-1$ \textbf{then}
    $(curr_{index},\,\,curr_{run},\,\,symb)\gets UPDATE(k, curr_{index},z)$ 
    % \If{$k\neq |z|-1$}
    % \State $(curr_{index},\,\,curr_{run},\,\,symb)\gets UPDATE(k, curr_{index},
    % z)$ 
    % \EndIf
    \Else
    \State $curr_{index}\gets  p_k[curr_{run}+1]$ \textbf{oppure}
    $curr_{index}\gets select_h^{k}(curr_{run}+1)+1$
    \State $curr_{row}\gets samples_{beg}^k[curr_{run}+1]$
    \State $ms_{row}[k]\gets curr_{row}$
    \State \textbf{if} $k\neq |z|-1$ \textbf{then} $(curr_{index},\,\,curr_{run},
    \,\,symb)\gets UPDATE(k, curr_{index}, z)$ 
    % \If{$k\neq |z|-1$}
    %  \State $(curr_{index},\,\,curr_{run},\,\,symb)\gets UPDATE(k, curr_{index},
    % z)$ 
    % \EndIf
    \EndIf
    \EndIf
    \EndFor
    \For {\textit{every} $k\in[0,|z|)$}
    \Comment \textbf{Costruzione lunghezze dell'array $MS$}
    \If{$ms_{row}[k] = M$}
    \State $ms_{len}[k]\gets 0$
    \ElsIf{$k\neq 0\land ms_{row}[i]=ms_{row}[i-1]\land
    ms_{len}[i-1]\neq 0$}
    \State $ms_{len}[i]\gets ms_{len}[i-1]+1$
    \Else
    \Comment $ra$ effettua il random access con la componente \texttt{RA-BV} o
    \texttt{RA-SLP} 
    \State $tmp_{index}\gets i,\,\,tmp_{len}\gets 0$
    \While {$tmp_{index}\geq 0 \land z[tmp_{index}]=ra(ms_{row}[k],
    tmp_{index})$}
    \State $tmp_{index}\gets tmp_{index}-1,\,\,tmp_{len}\gets tmp_{len}+1$
    \EndWhile
    \State $ms_{len}[k]\gets tmp_{len}$
    \EndIf
    \EndFor
    \For {\textit{every} $k\in[0,|z|)$}
    \Comment \textbf{Calcolo dei match da $MS$}
    \If{$(ms_{len}[k]>1 \land ms_{len}[k]\geq ms_{len}[k+1])\lor(k = |z|-1 \land
    ms_{len}[k]\neq 0$}
    \State \textit{report dello SMEM terminante in colonna $k$}
    \State \textit{SMEM di lunghezza $ms_{len}[k]$ con la riga $ms_{row}[k]$ e
    quelle estese da essa tramite} \texttt{PHI}
    \EndIf
    \EndFor
    \EndFunction

    
    \Function {down}{$pos, prev, next$}
    \State \textit{si usano le LCE queries o il random access per calcolare il
    suffisso comune più lungo tra quelli delle righe}
    \State \textit{$pos$/$prev$ e
    $pos$/$next$ fino alla colonna precedente a quella corrente} 
    \State \textit{se il secondo è maggiore o uguale al primo ritorna $\top$,
    altrimenti $\bot$} 
    \EndFunction
  \end{algorithmic}
  \caption{\footnotesize{Calcolo degli SMEM con aplotipo esterno con componenti
  \texttt{MAP-INT/BV},
  \texttt{THR-INT/BV} (i cui usi diversificati di entrambe le componenti sono
  segnalati con ``oppure''), \texttt{RA-BV/SLP}, \texttt{PERM} e \texttt{PHI}.}}    
  \label{algo:matchthr}
\end{algorithm}
\begin{algorithm}
  \footnotesize
  \begin{algorithmic}[1]
    \Function{update}{$k, curr_{index}, z$}
    \State $curr_{index}\gets lf(k, curr_{index}, z[k])$
    \State $curr_{run}\gets index\_to\_run(curr_{index},k+1)$ \textbf{oppure}
    $curr_{run}\gets rank_h^{k+1}(curr_{index})$
    \State $symb\gets get\_symbol(start_{k+1}, curr_{run})$
    \State \textbf{return} $(curr_{index},\,\,curr_{run},\,\,symb)$
    \EndFunction
  \end{algorithmic}
  \caption{Algoritmo per l'update con componenti \texttt{MAP-INT} e
  \texttt{MAP-BV}.} 
  \label{algo:updatems}
\end{algorithm}
\subsection{Calcolo dell'array MS con LCE query}
Come anticipato, grazie all'uso delle \textit{LCE query} è possibile calcolare
l'array 
delle \textit{matching statistics} in un solo scorrimento da sinistra a
destra. Infatti è possibile usare tali query per calcolare non solo quale nuova
sequenza scegliere in caso di mismatch con l'aplotipo query in colonna $i$, come
si faceva con l'uso delle \textit{threshold}, ma anche di computare la lunghezza
del suffisso comune tra essa e l'aplotipo query, calcolando nello stesso momento
sia $MS[i].row$ che $MS[i].len$.\\
Tale soluzione è quindi relativa alle seguenti strutture dati composte:
\begin{itemize}
  \item \texttt{MAP-INT + LCE + PERM + PHI}
  \item \texttt{MAP-BV + LCE + PERM + PHI}
\end{itemize}
Con la notazione:
\begin{equation}
  \label{eq:lcecol}
  LCE(k, x, y)
\end{equation}
Si indica il calcolo della \textit{LCE query} tra le righe di indice $x$ e
indice $y$ terminante in colonna $k-1$ (quindi escludendo la colonna
$k$-esima).\\ 
Si illustra ora come computare l'array delle \textit{matching statistics}. Anche
in questo caso, per convenzione, si inizia la computazione dell'ultima 
riga della prima colonna.
Si assuma di avere calcolato l'array $MS$ di una query $z$ rispetto al pannello
$X$. le cui righe si identificano tramite $x_i, \forall i\in\{0,M\}$, fino alla
colonna $k-1$. Sia $i$ 
l'indice di riga sulla \textit{matrice PBWT} al quale si è arrivati mediante il
mapping, avendo che tale riga è quella che ha il più lungo suffisso comune con
$z[1,k-1]$. Si assuma che l'indice $i$ appartenga alla run $r$, di simboli
$\sigma$, testa di indice $h$ e coda di indice $e-1$. Si hanno diversi casi:
\begin{enumerate}
  \item $z[k]=\sigma$, quindi la riga $i$ può essere usata per estendere il
  match, avendo che $MS[k].row=MS[k-1].row$ e $MS[k].len=MS[k-1].len+1$, e per
  proseguire col mapping in colonna $k+1$
  \item $z[k]\neq\sigma$ e si ha una sola run in colonna $k$, avendo quindi che
  non si possono avere match. Per convenzione, si
  imposta che $MS[k].row = M$ e $MS[k].len=0$. Infine si ricomincia, in colonna
  $k+1$, dall'ultima posizione, indicizzata nel pannello originale dal valore
  finale del \textit{prefix array sample} dell'ultima run
  \item $z[k]\neq\sigma$ ma si hanno anche altre run, dovendo quindi scegliere
  la nuova riga da seguire. Si ha che il più lungo suffisso di $z[1,k]$ che è
  anche suffisso di $x_1[1,k],\ldots, x_m[1,k]$ è uno tra:
  \begin{itemize}
    \item $x_{a_k[h-1]}$, se $h\neq 0$, ovvero la riga del pannello
    corrispondente alla fine della run precedente a $r$ nella \textit{matrice
      PBWT}, se esistente
    \item $x_{a_k[e+1]}$, se $e\neq M-1$, ovvero la riga del pannello
    corrispondente all'inizio della run successiva a $r$ nella \textit{matrice
      PBWT}, se esistente
  \end{itemize}
  Avendo quindi i \textit{prefix array sample} che ci dicono a quale riga nel
  pannello corrispondano tali valori e conoscendo $MS[k-1].row$ è possibile
  calcolare $LCE(k,MS[k-1].row, a_k[h-1])$ e $LCE(k,MS[k-1].row, a_k[e+1])$. A
  questo punto si sceglie il suffisso comune più lungo tra le due, ovvero il
  maggiore tra i valori ritornati dalla \textit{LCE query} e si sceglie la riga
  corrispondente per proseguire. Si ha quindi o $MS[k].row=a_k[h-1]$ o
  $MS[k].row=a_k[e+1]$. In merito alla lunghezza, assumendo che la lunghezza
  maggiore delle due \textit{LCE query} sia $l$, si ha che:
  \[MS[k].len=\min(MS[k-1].len, l)+1\]
  In quanto la LCE query potrebbe restituire un valore più lungo dell'effettivo
  match con al query $z$ quindi si sceglie il minimo tra le due lunghezze,
  ottenendo l'effettiva lunghezza del suffisso comune tra $z$ e la nuova riga
  scelta fino a $k-1$, e lo si 
  incrementa di uno, contando il match ottenuto in colonna $k$
\end{enumerate}
\dc{Sistemare esempio.}
\begin{esempio}
  Riprendiamo l'esempio \ref{es:thr}, visto per il calcolo di $MS[i].row$, dopo
  un mismatch,
  tramite \textit{threshold}. \\
  Senza usare le \textit{threshold}, nella medesima situazione si dovrebbe
  calcolare, avendo che $MS[1].row=19$ e $MS[1].len =2$:
  \[LCE(2, x_{19}, x_{16}) = \mbox{"01"} \implies|LCE(2, x_{19}, x_{16})|=2\]
  \[LCE(2, x_{19}, x_{17}) = \mbox{"1"} \implies|LCE(2, x_{19}, x_{17})|=1\]
  Come verificabile dal pannello presente all'esempio \ref{es:pbwt1}.\\
  Si ha quindi che $MS[2].row=16$. Inoltre, sempre per quanto detto sopra:
  \[MS[2].len=\min(MS[1].len, 2)+1=2+1=3\]
\end{esempio}
Con questa soluzione, il cui pseudocodice è consultabile all'algoritmo
\ref{algo:matchlce}, quindi: 
\begin{itemize}
  \item non si necessita di tenere in memoria le informazioni per le
  \textit{threshold}
  \item si permette il calcolo dell'array $MS$ in una singola scansione del
  pattern 
  \item non si necessita di memorizzare l'intero array $MS$ ma solamente quattro
  variabili relative alla coppia
  $(row,len)$ corrente e quella precedente. \textit{Per facilità di lettura si è
  lasciato, nello pseudocodice, l'uso dell'intero array $MS$}
\end{itemize}
Dal punto di vista della complessità temporale, per il calcolo dell'array $MS$,
si hanno solo due casistiche 
possibili, al variare della componente di mapping. 
Nel caso della componente \texttt{MAP-INT}, avendo $s$ lunghezza della singola
parola prodotta dall'\textit{SLP} e $\rho$ numero medio di run per colonna, si
ha un tempo proporzionale, dovendo iterare la query, fare il mapping e usare la
componente \texttt{LCE}, a: 
\begin{equation}
  \label{eq:mslce1}
  \mathcal{O}(N(\log \rho+\log s))
\end{equation}
Mentre nel caso dell'uso della componente \texttt{MAP-BV} si ha tempo
proporzionale a:
\begin{equation}
  \label{eq:mslce2}
  \mathcal{O}\left(N\left(\log \frac{M}{\rho}+\log s\right)\right)
\end{equation}
Infine, per il calcolo di tutte le righe del pannello per cui si ha uno
\textit{SMEM} si può fare riferimento all'equazione \ref{eq:msthr3slp}, avendo
la medesima situazione.\\
Si deduce quindi come la struttura composta \texttt{MAP-INT + LCE + PERM + PHI}
sia, a 
livello di tempo macchina, la soluzione più vantaggiosa usando la componente
\texttt{LCE}. Tale soluzione risulta, sempre nel contesto delle strutture basate
sulla componente \texttt{LCE}, essere anche la soluzione più vantaggiosa in
termini di memoria.\\
Si vedrà, sperimentalmente, nel capitolo \ref{reschap}, il
confronto con le altre strutture dati.
\begin{algorithm}
  \scriptsize
  \begin{algorithmic}[1]
    \Function{matches\_ms\_lce}{$z$}
    \State $ms_{row}\gets [0..0],\,\,ms_{len}\gets [0..0]$
    \Comment array $MS$ di lunghezza $|z|$
    \State $curr_{row}\gets
    rlpbwt[0].samples_{end}[|rlpbwt[0].samples_{end}|-1],\,\,curr_{index}\gets
    curr_{row}$ 
    \State $curr_{run}\gets index\_to\_run(curr_{index},0)$ \textbf{oppure}
    $curr_{run}\gets rank_h^0(curr_{index})$  
    \State $symb\gets get\_symbol(start_0, curr_{run})$
    \Comment \textbf{Costruzione dell'array $MS$}
    \For {\textit{every} $k\in[0, |z|)$}
    \If{$z[i]=symb$}
    \State $ms_{row}[k]\gets curr_{row}$
    \State \textbf{if} $k=0$ \textbf{then} $ms_{len}[k] \gets 1$ \textbf{else}
    $ms_{len}[k] \gets ms_{len}[k-1]+1$
    \State \hspace{-1.1mm}\textbf{if} $k\neq |z|-1$ \textbf{then}
    $(curr_{index},\,\,curr_{run},\,\,symb)\gets UPDATE(k, curr_{index},z)$ 
    % \If{$k\neq |z|-1$}
    % \State $(curr_{index},\,\,curr_{run},\,\,symb)\gets UPDATE(k, curr_{index},
    % z)$ 
    % \EndIf
    \Else
    \If{$|samples_{beg}^k|=1$}
    \State $ms_{row}[k]\gets M$
    \State $ms_{len}[k]\gets 0$
    \If{$k\neq |z|-1$}
    \State $curr_{row}\gets
    rlpbwt[k+1].samples_{end}[|rlpbwt[k+1].samples_{end}|-1]$
    \State $curr_{index}\gets M-1$
    \State $curr_{run}\gets index\_to\_run(curr_{index},k+1)$ \textbf{oppure}
    $curr_{run}\gets rank_h^{k+1}(curr_{index})$
    \State $symb\gets get\_symbol(start_{k+1}, curr_{run})$
    \EndIf
    \Else
    \If{$curr_{run}=|samples_{beg}^k|-1$}
    \State $curr_{index}\gets p_k[curr_{run}-1]$ \textbf{oppure}
    $curr_{index}\gets select_h^k(curr_{run})$
    \State $prev_{row}\gets
    samples_{end}^k[curr_{run}-1]$ 
    \State $lce\gets LCE(k, curr_{row}, prev_{row})$
    \State $ms_{row}[k]\gets prev_{row},\,\,curr_{row}\gets prev_{row}$
    \State \textbf{if} $k=0$ \textbf{then} $ms_{len}[k] \gets 1$ \textbf{else}
    $ms_{len}[k] \gets min(ms_{len}[k-1], |lce|)+1$ 
    % \If{$k=0$}
    % \State $ms_{len}[k] \gets 1$
    % \Else
    % \State $ms_{len}[k] \gets min(ms_{len}[k-1], |lce|)+1$
    % \EndIf
    \State \textbf{if} $k\neq |z|-1$ \textbf{then}
    $(curr_{index},\,\,curr_{run},\,\,symb)\gets UPDATE(k, curr_{index},z)$  
    % \If{$k\neq |z|-1$}
    % \State $(curr_{index},\,\,curr_{run},\,\,symb)\gets UPDATE(k, curr_{index},
    % z)$  
    % \EndIf    
    \ElsIf{$curr_{run}=0$}
    \State $curr_{index}\gets p_k[curr_{run}+1]$ \textbf{oppure}
    $curr_{index}\gets select_h^k(curr_{run}+1)+1$
    \State $next_{row}\gets samples_{beg}^k[curr_{run}+1]$ 
    \State $lce\gets LCE(k, curr_{row}, next_{row})$
    \State $ms_{row}[k]\gets next_{row},\,\,curr_{row}\gets next_{row}$
    \State \textbf{if} $k=0$ \textbf{then} $ms_{len}[k] \gets 1$ \textbf{else}
    $ms_{len}[k] \gets min(ms_{len}[k-1], |lce|)+1$
    % \If{$k=0$}
    % \State $ms_{len}[k] \gets 1$
    % \Else
    % \State $ms_{len}[k] \gets min(ms_{len}[k-1], |lce|)+1$
    % \EndIf
    \State \textbf{if} $k\neq |z|-1$ \textbf{then} $(curr_{index},\,\,
    curr_{run},\,\,symb)\gets UPDATE(k, curr_{index},z)$  
    % \If{$k\neq |z|-1$}
    % \State $(curr_{index},\,\,curr_{run},\,\,symb)\gets UPDATE(k, curr_{index},
    % z)$ 
    % \EndIf
    \Else
    \State $prev_{row}\gets samples_{end}^k[curr_{run}-1],\,\,next_{row}\gets
    samples_{beg}^k[curr_{run}+1]$ 
    \State $lce\gets \max (|LCE(k, curr_{row}, prev_{row})|, |LCE(k,
    curr_{row}, next_{row})|)$
    \State $curr_{row}\gets lce_{row}$
    \Comment $lce_{row}$ segnala l'indice della riga con \textit{LCE query} più
    lunga 
    \State $ms_{row}[k]\gets curr_{row}$
    \State \textbf{if} $k=0$ \textbf{then} $ms_{len}[k] \gets 1$ \textbf{else}
    $ms_{len}[k] \gets min(ms_{len}[k-1], |lce|)+1$
    % \If{$k=0$}
    % \State $ms_{len}[k] \gets 1$
    % \Else
    % \State $ms_{len}[k] \gets min(ms_{len}[k-1], |lce|)+1$
    % \EndIf
    \State \textbf{if} $k\neq |z|-1$ \textbf{then}
    $(curr_{index},\,\,curr_{run},\,\,symb)\gets UPDATE(k, curr_{index},z)$ 
    % \If{$k\neq |z|-1$}
    % \State $(curr_{index},\,\,curr_{run},\,\,symb)\gets UPDATE(k,curr_{index},
    % z)$ 
    % \EndIf
    \EndIf
    \EndIf
    \EndIf
    \EndFor
    
    \For {\textit{every} $k\in[0,|z|)$}
    \Comment \textbf{Calcolo dei match da $MS$}
    \If{$(ms_{len}[k]>1 \land ms_{len}[k]\geq ms_{len}[k+1])\lor(k = |z|-1 \land
    ms_{len}[k]\neq 0$}
    \State \textit{report degli SMEM di lunghezza $ms_{len}[k]$, terminanti in
    colonna $k$}
    \State \textit{con la riga $ms_{row}[k]$ e quelle estese da essa tramite
    la componente \texttt{PHI}} 
    \EndIf
    \EndFor
    \EndFunction
    
  \end{algorithmic}
  \caption{\footnotesize{Calcolo degli SMEM con aplotipo esterno con componenti
  \texttt{MAP-INT/BV} (i cui usi diversificati sono segnalati con ``oppure''),
  \texttt{LCE}, \texttt{PERM} e \texttt{PHI}.}}
  \label{algo:matchlce}
\end{algorithm}
\dc{Sistemare pseudocodice per non avere salvato intero $MS$}

% LocalWords:  pseudocodice
