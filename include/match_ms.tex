\subsection{Calcolo degli SMEM con matching statistics}
L'obbiettivo maggiore di questa tesi era quello di applicare i metodi e gli
algoritmi già 
studiati per la $\RLBWT$, riferendosi al calcolo dei $\MEM$ a partire
dall'array delle matching statistics, alla $\RLPBWT$.\\
Nelle sei strutture dati dedicate al calcolo degli $\SMEM$ tramite matching
statistics, si riconoscono le due modalità già descritte con MONI \cite{moni} e
PHONI \cite{phoni}:
\begin{enumerate}
  \item calcolare l'array $\MS$, in due passaggi, sfruttando le threshold per
  computare i valori $\MS[i].\row$ e il
  random access al pannello per calcolare i valori $\MS[i].\len$
  \item calcolare l'array $\MS$, in un passaggio, sfruttando le $\LCE$ query
  sia per scegliere le righe da memorizzare in $\MS[i].\row$ che per calcolare,
  in contemporanea, i valori $\MS[i].\len$ 
\end{enumerate}
\subsubsection{Calcolo dell'array delle matching statistics tramite threshold}
Questa prima soluzione, necessitando sia della componente
\texttt{THR-INT}/\texttt{THR-BV} che della componente
\texttt{RA-BV}/\texttt{RA-SLP}, è relativa alle seguenti strutture dati
composte: 
\begin{itemize}
  \item \texttt{MAP-INT + THR-INT + RA-BV + PERM + PHI}
  \item \texttt{MAP-INT + THR-INT + RA-SLP + PERM + PHI}
  \item \texttt{MAP-BV + THR-BV + RA-BV + PERM + PHI}
  \item \texttt{MAP-BV + THR-BV + RA-SLP + PERM + PHI}
\end{itemize}
Tra queste soluzioni, le uniche differenze si riscontrano nei tempi d'esecuzione
e nella memoria richiesta.\\
\dc{Cercare di capire se e come legare a R-index}
% Si noti che unendo la componente \texttt{PERM} alla componente
% \texttt{THR-INT}/\texttt{THR-BV} si ottiene una variante dell'\textbf{R-index}
% visto per la \textit{RLBWT}.\\
Si presenta ora il funzionamento dell'algoritmo.
Si supponga di aver computato l'array delle matching statistics fino alla
colonna $k-1$, avendo che si sta processando la colonna corrente $k$.
Sia $i$ la riga della matrice $\PBWT$ che ha un match con il più lungo
suffisso di $z[0,k-1]$ che è suffisso di almeno uno tra $x_0[0,k-1]$, $\ldots$,
$x_{M-1}[0,k-1]$. Sia $p$ la corrispondente riga, sul pannello in input $X$,
della riga
$i$ sulla matrice $\PBWT$. Tale valore è ottenibile tramite il prefix array,
avendo $p=a_k[i]$. Si ha, 
denotando con
$\lcs(A,B)$ il più lungo suffisso comune tra le stringhe $A$ e $B$, in
termini formali, che:
\begin{equation}
  \label{eq:thr1}
  |\lcs(z[0,k-1], x_{p}[0,k-1])|\geq |\lcs(z[0,k-1],
  x_{a_k[j]}[0,k-1])|,\,\,\forall\, 
  j\in[0,M-1] 
\end{equation}
Fatta questa premessa, qualora si avesse $y_i^k[k]=z[k]$, ovvero un match tra il
$k$-esimo carattere della query e il carattere in riga $i$ della matrice
$\PBWT$, si 
avrebbe semplicemente che $\MS[k].\row=a_k[i]=p$ e $\MS[k].\len =
\MS[k-1].\len+1$, in 
quanto si starebbe seguendo la medesima riga dello step precedente del calcolo
delle matching statistics.\\
In caso contrario, si avrebbe 
$y_i^k[k]\neq z[k]$, ovvero un mismatch tra il $k$-esimo carattere
della query è il carattere in riga $i$ della matrice $\PBWT$. Bisogna,
quindi, scegliere un nuovo indice $i'$ nella matrice $\PBWT$,
corrispondente all'indice $p'$ nella pannello in input, che, a sua volta,
corrisponde alla riga 
con il più lungo suffisso possibile con la query, che sia estendibile anche in
colonna $k$. Sia $y_{[s,e]}^k[k]$ l'intervallo corrispondente alla run che
contiene l'indice $i$, nella matrice $\PBWT$. Il più lungo suffisso
di $z[0,k]$, che è suffisso di almeno uno tra $x_0[0,k]$, $\ldots$,
$x_{M-1}[0,k]$, corrisponde o alla fine della run precedente di simboli
$\sigma=z[k]$ o all'inizio della run successiva di simboli
$\sigma=z[k]$, nella colonna $k$ della matrice $\PBWT$. Formalmente,
tale suffisso corrisponde ad uno tra $X_{a_k[s-1]}[0,k]$, se $s>0$, o
$X_{a_k[e+1]}[0,k]$, se 
$e<M-1$. L'uso delle threshold permette di capire quale tra le righe $a_k[s-1]$
e $a_k[e+1]$ del pannello in input, se esistenti, abbia il più lungo suffisso
comune con $z[0,k]$. \\
Sia $t$ l'indice della threshold nella run corrente. Si hanno due casi
possibili:
\begin{enumerate}
  \item $i<t$, allora, per definizione di threshold:
  \begin{equation}
    \label{eq:thrra2}
    |\lcs(z[0,k], x_{a_{k}[s-1]}[0,k])|\geq |\lcs(z[0,k], x_{a_{k}[e+1]}[0,k])|
  \end{equation}
  Quindi si ha che $\MS[k].\row=a_{k}[s-1]=p$ e il mapping potrà proseguire
  dall'indice $s-1$
  \item  $i\geq t$ allora, per definizione di threshold:
  \begin{equation}
    \label{eq:thrra1}
    |\lcs(z[0,k], x_{a_{k}[s-1]}[0,k])|\leq |\lcs(z[0,k], x_{a_{k}[e+1]}[0,k])|
  \end{equation}
  Quindi si ha che $\MS[k].\row=a_{k}[e+1]$ e il mapping potrà proseguire
  dall'indice $e+1$
\end{enumerate}
Si ricorda il caso in cui la threshold sia posta a fine run, nel caso
della componente \texttt{THR-BV}. In tale situazione, bisognerebbe scegliere la
testa della run successiva, qualora l'indice $i$ 
si trovi esattamente a fine run. Invece, qualora la
threshold sia a fine run a causa del fatto che il minimo $\RLCP$
si trovi nella testa della run successiva, bisognerebbe scegliere la coda della
run 
precedente. L'unico modo per disambiguare è effettuare random
access al pannello o calcolare una $\LCE$ query, per vedere quale sia la
soluzione migliore, ovvero quale tra la coda della run precedente e la testa
della run successiva sia relativa alla riga del pannello originale con un
suffisso comune alla query più lungo.\\
Una volta computato tutti i valori $\MS[i].\row$, per calcolare i valori
$\MS[i].\len$,
si scorre da sinistra a destra calcolando la lunghezza del match fino alla
colonna $i$, facendo random
access al pannello e confrontando la query $z$ con la riga $\MS[i].\row$. Si
assuma di aver calcolato $\MS[i-1].\len$ e di voler calcolare
$\MS[i].\len$. 
Si hanno tre casi possibili:
\begin{enumerate}
  \item $\MS[i].\row=M$. In tal caso, avendo segnalata l'inesistenza di un
  match terminante in colonna $i$, si ha che $\MS[i].\len=0$
  \item $\MS[i].\row=\MS[i-1].\row$, avendo $i\neq 0$ e $\MS[i-1].\len\neq 0$.
  In tal caso
  si sta seguendo la medesima riga seguita in colonna $i-1$ e quindi,
  banalmente, $\MS[i].\len=\MS[i-1].\len+1$, dovendo conteggiare il carattere
  della colonna corrente
  \item in qualsiasi altro caso bisogna confrontare, a partire dalla colonna
  $i$, la query 
  $z$ con la riga $\MS[i].\row$ del pannello, da destra a sinistra, fino a che
  non si trova un mismatch, calcolando la lunghezza $l$ del suffisso comune tra
  esse e memorizzando tale valore come $\MS[i].\len=l$
\end{enumerate}
\begin{esempio}
  \label{es:thr}
  Si vede, quindi, un esempio di funzionamento delle threshold per la scelta
  della riga da memorizzare in $\MS[i].\row$, dopo un mismatch.\\
  Si prenda pannello visto all'esempio \ref{es:pbwt1} e si effettui la
  permutazione secondo $a_2$:
  \begin{table}[H]
    \centering
    \footnotesize
    \begin{tabular}{c|cc|c|cccccccccccc}
      X & 00 & 01 & 02 & 03 & 04 & 05 & 06 & 07 & 08 & 09 & 10 & 11 & 12 & 13
      & 14 \\
      \hline
      00 & 1 & 0 & 0 & 1 & 0 & 0 & 0 & 0 & 0 & 0 & 0 & 1 & 1 & 0 & 1 \\
      01 & 1 & 0 & 0 & 1 & 1 & 0 & 0 & 1 & 0 & 0 & 0 & 0 & 0 & 1 & 1 \\
      02 & 1 & 0 & 0 & 1 & 1 & 0 & 0 & 1 & 0 & 0 & 0 & 1 & 0 & 0 & 1 \\
      03 & 1 & 0 & 0 & 1 & 1 & 0 & 0 & 1 & 0 & 0 & 0 & 1 & 0 & 0 & 1 \\
      04 & 0 & 1 & 0 & 1 & 0 & 1 & 0 & 0 & 0 & 0 & 0 & 1 & 0 & 0 & 1 \\
      05 & 0 & 1 & 0 & 1 & 0 & 1 & 0 & 0 & 0 & 0 & 0 & 1 & 0 & 0 & 1 \\
      06 & 0 & 1 & 0 & 1 & 0 & 1 & 0 & 0 & 0 & 0 & 0 & 1 & 0 & 0 & 1 \\
      07 & 0 & 1 & 0 & 1 & 0 & 1 & 0 & 0 & 0 & 0 & 0 & 0 & 1 & 0 & 1 \\
      08 & 0 & 1 & 0 & 0 & 1 & 0 & 0 & 0 & 0 & 1 & 1 & 1 & 0 & 0 & 1 \\
      09 & 0 & 1 & 0 & 1 & 0 & 0 & 0 & 0 & 1 & 0 & 0 & 0 & 0 & 1 & 1 \\
      10 & 0 & 1 & 0 & 1 & 0 & 0 & 0 & 0 & 1 & 0 & 0 & 0 & 0 & 1 & 1 \\
      11 & 0 & 1 & 0 & 0 & 1 & 0 & 0 & 0 & 0 & 0 & 1 & 1 & 0 & 0 & 0 \\
      12 & 0 & 1 & 0 & 0 & 1 & 0 & 0 & 0 & 1 & 0 & 1 & 1 & 0 & 0 & 1 \\
      13 & 0 & 1 & 0 & 0 & 1 & 0 & 0 & 0 & 1 & 0 & 1 & 1 & 0 & 0 & 1 \\
      14 & 0 & 1 & 0 & 0 & 0 & 0 & 0 & 0 & 1 & 0 & 0 & 0 & 1 & 0 & 1 \\
      15 & 0 & 1 & 0 & 0 & 0 & 0 & 0 & 0 & 1 & 0 & 0 & 0 & 1 & 0 & 1 \\
      16 & 0 & 1 & 0 & 1 & 0 & 0 & 0 & 0 & 0 & 0 & 0 & 1 & 1 & 0 & 1 \\
      17 & 0 & 1 & 1 & 0 & 1 & 0 & 0 & 0 & 0 & 0 & 0 & 1 & 0 & 0 & 1 \\
      18 & 0 & 1 & 1 & 0 & 1 & 0 & 1 & 0 & 0 & 0 & 0 & 0 & 1 & 0 & 1 \\
      19 & 1 & 1 & 0 & 0 & 0 & 1 & 0 & 0 & 0 & 0 & 0 & 1 & 1 & 0 & 1 \\
    \end{tabular}
  \end{table}
  \noindent
  Si prenda la seconda run, di simboli $\sigma=1$, indicizzata tra 17 e 18. \\
  Si supponga che, tramite il mapping, si sia arrivati alla riga 17 ma che si
  abbia $z[2]=0$. La scelta è, quindi, tra la coda della run precedente, avendo
  che $a_2[16]=16$ o la testa della run successiva, avendo che $a_2[19]=17$. Si
  può notare come il minimo $\RLCP$ si trovi, per la 
  run, all'indice 18 (a causa del fatto che il minimo $\RLCP$ è all'indice
  19, quello della testa della run successiva). L'indice in cui ci si trova, il
  17, è quindi sopra la threshold.
  Questo significa che il suffisso comune più lungo
  con la query si ha 
  con la riga 16 del pannello, per definizione di threshold, avendo che questa
  sarà memorizzata nell'array $\MS$, avendo $\MS[2].\row=16$.\\
  Successivamente, o sfruttando $\MS[1].\len$ o tramite random access al testo,
  confrontando la riga 
  $x_{16}$ e la query $z$, fino alla colonna $k=2$, si potrà calcolare che
  $\MS[2].\len=3$. 
\end{esempio}
\dc{SISTEMARE ESEMPIO}
In fase di costruzione delle lunghezze, è possibile anche riportare gli
$\SMEM$, terminanti in colonna $i$, qualora:
\begin{itemize}
  \item $\MS[i].\len\geq \MS[i+1].\len \land \MS[i].\len\neq 0$
  \item si è arrivati a fine query, avendo $i=N-1\land \MS[i].\len\neq 0$
\end{itemize}
Queste condizioni segnalano che non è possibile estendere a destra il più lungo
suffisso comune, terminante in colonna $i$, tra la query e una qualsiasi riga del
pannello di input. 
Questo si può verificare anche nell'esempio \ref{es:ms}.\\
L'algoritmo per il il calcolo degli $\SMEM$ tramite threshold è visualizzabile
all'algoritmo \ref{algo:matchthr}. All'algoritmo \ref{algo:updatems}
si riporta il metodo di $\UP$ delle informazioni, nel passaggio dalla
colonna 
$k$ alla colonna $k+1$, di complessità pari a quelle per effettuare il
mapping. Si noti che è 
possibile usare la funzione $\W$, 
spiegata in precedenza, in quanto, avendo per costruzione
$y^k_{curr\_index}[k]=z[k]$, nel momento dell'applicazione della funzione, si ha
che la funzione segue esattamente una certa riga da una colonna alla
successiva, nella matrice $\PBWT$. \\
Anche in questo caso, la stima delle
complessità non è di facile ottenimento. Dividendo nelle varie parti
l'algoritmo, si ha che:
\begin{itemize}
  \item il calcolo dei valori $MS[i].\row$ varia a
  seconda dell'uso 
  della componente \texttt{MAP-INT} o \texttt{MAP-BV}. Il costo della funzione
  \texttt{down}, che risolve l'eventuale ambiguità della threshold a fine run,
  è variabile a seconda della componente usata per il random access (e 
  dell'eventuale componente \texttt{LCE}) e risulta trascurabile vista la bassa
  frequenza d'uso, in termini probabilistici. Si ha, quindi, che, con $\rho$
  numero medio di run per colonna, usando \texttt{MAP-INT}, si ha tempo
  proporzionale a:
  \begin{equation}
    \label{eq:msthr1int}
    \mathcal{O}(N\log\rho)
  \end{equation}
  Mentre, usando la componente \texttt{MAP-BV}, è proporzionale a:
  \begin{equation}
    \label{eq:msthr1bv}
    \mathcal{O}\left(N\log\frac{M}{\rho}\right)
  \end{equation}
  \item il calcolo dei valori $\MS[i].\len$ (e degli $\SMEM$ direttamente
  calcolabili da essi) è il
  più complesso da 
  stimare, in termini di complessità asintotica. Questa difficoltà è dovuta dal
  fatto che gli accessi al pannello vengono fatti solo quando $\MS.\row[i]\neq
  \MS.\row[i-1]$.
  Ipotizzando un caso peggiore dove si necessità di accedere al pannello in ogni
  colonna, si ha, nel caso della componente \texttt{RA-BV}, che il calcolo
  complessivo delle lunghezza è proporzionale a:
  \begin{equation}
    \label{eq:msthr2bv}
    \mathcal{O}(N^2)
  \end{equation}
  Mentre, con l'uso della componente \texttt{RA-SLP}, la complessità in tempo
  sarebbe proporzionale a: 
  \begin{equation}
    \label{eq:msthr2slp}
    \mathcal{O}\left(N^2\log (NM)\right)
  \end{equation}
  In aggiunta bisogna considerare i costi della componente \texttt{PHI} per il
  computo di tutti gli $\SMEM$.
  % Per semplicità denotiamo con $\gamma$ il numero di accessi al
  % pannello e, nel caso della componente \texttt{RA-BV}, si ha che il calcolo
  % complessivo delle lunghezza è proporzionale a:
  % \begin{equation}
  %   \label{eq:msthr2bv}
  %   \mathcal{O}(N\gamma)
  % \end{equation}
  % Mentre, con l'uso della componente \texttt{RA-SLP}, è proporzionale a:
  % \begin{equation}
  %   \label{eq:msthr2slp}
  %   \mathcal{O}\left(N\gamma\log (NM)\right)
  % \end{equation}
\end{itemize}
Tali stime teoriche sono, in ogni caso, fortemente approssimative.
\dc{Tutti questi tempi sono abbastanza a caso per ora.}
Facendo una stima complessiva si può ipotizzare come la struttura \texttt{MAP-BV
+ THR-BV + RA-SLP + PERM + PHI}, a causa della maggior lentezza in fase di
mapping e di accesso al pannello per il calcolo delle lunghezze, sia quella con
prestazioni peggiori mentre, per il ragionamento inverso, la struttura
\texttt{MAP-INT + THR-INT + RA-BV + PERM + PHI} sia quella con le migliori
performance, dal punto di vista del tempo macchina.\\
In termini di memoria, invece,  la struttura
\texttt{MAP-INT + THR-INT + RA-SLP + PERM + PHI} risulta essere la più
vantaggiosa mentre la struttura
\texttt{MAP-BV + THR-BV + RA-BV + PERM + PHI} la peggiore, per le stime sulle
singole componenti, viste
nelle sezioni precedenti. 
\begin{algorithm}
  \scriptsize
  \begin{algorithmic}[1]
    \Function{external\_matches}{$z$}
    \State $ms_{row}\gets [0..0],\,\,ms_{len}\gets [0..0]$
    \Comment vettore $MS$ di lunghezza $|z|$
    \State $curr_{row}\gets
    rlpbwt[0].samples_{end}[|rlpbwt[0].samples_{end}|-1]$
    \State $curr_{index}\gets curr_{row}$
    \State  $curr_{run}\gets \ITR(curr_{index},0)$ \textbf{oppure}
    $curr_{run}\gets \rank_h^0(curr_{index})$    
    \State $symb\gets \GS(start_0, curr_{run})$
    \Comment \textbf{Costruzione righe dell'array $MS$}
    \For {\textit{every} $k\in[0, |z|)$}

    \If{$z[i]=symb$}
    \State $ms_{row}[k]\gets curr_{row}$
    \State \hspace{-1.1mm}\textbf{if} $k\neq |z|-1$ \textbf{then}
    $(curr_{index},\,\,curr_{run},\,\,symb)\gets \UP(k, curr_{index},z)$ 
    \Else
    \State  $curr_{thr}\gets t_k[curr_{run}]$ \textbf{oppure}
    $curr_{thr}\gets \rank_t^k(curr_{index})$ 
    \State $force_{down} \gets \top$\textit{ sse l'indice è sovrapposto ad una
    threshold non in coda di run}
    \State $force_{down} \gets \top$\textit{ sse l'indice è sovrapposto ad una
    threshold in coda di run e $\mathtt{DOWN}(\ldots)=\top$}
    \If{$|samples_{beg}^k|=1$}
    \State $ms_{row}[k]\gets M$
    \If{$k\neq |z|-1$}
    \State $curr_{row}\gets
    rlpbwt[k+1].samples_{end}[|rlpbwt[k+1].samples_{end}|-1]$
    \State $curr_{index}\gets M-1$
    \State $curr_{run}\gets \ITR(curr_{index},k+1)$ \textbf{oppure}
    $curr_{run}\gets \rank_h^{k+1}(curr_{index})$
    \State $symb\gets \GS(start_{k+1}, curr_{run})$
    \EndIf
    \ElsIf{$(curr_{run}\neq 0 \land curr_{run}=curr_{thr}\land \neg down)\lor
    curr_{run}=|samples_{beg}^k|-1$} 
    \State $curr_{index}\gets p_k[curr_{run}]-1$ \textbf{oppure}
    $curr_{index}\gets \select_h^{k}(curr_{run})$
    \State $curr_{row}\gets samples_{end}^k[curr_{run}-1]$
    \State $ms_{row}[k]\gets curr_{row}$
    \State \textbf{if} $k\neq |z|-1$ \textbf{then}
    $(curr_{index},\,\,curr_{run},\,\,symb)\gets UPDATE(k, curr_{index},z)$ 
    % \If{$k\neq |z|-1$}
    % \State $(curr_{index},\,\,curr_{run},\,\,symb)\gets UPDATE(k, curr_{index},
    % z)$ 
    % \EndIf
    \Else
    \State $curr_{index}\gets  p_k[curr_{run}+1]$ \textbf{oppure}
    $curr_{index}\gets \select_h^{k}(curr_{run}+1)+1$
    \State $curr_{row}\gets samples_{beg}^k[curr_{run}+1]$
    \State $ms_{row}[k]\gets curr_{row}$
    \State \textbf{if} $k\neq |z|-1$ \textbf{then} $(curr_{index},\,\,curr_{run},
    \,\,symb)\gets \UP(k, curr_{index}, z)$ 
    % \If{$k\neq |z|-1$}
    %  \State $(curr_{index},\,\,curr_{run},\,\,symb)\gets \UP(k, curr_{index},
    % z)$ 
    % \EndIf
    \EndIf
    \EndIf
    \EndFor
    \For {\textit{every} $k\in[0,|z|)$}
    \Comment \textbf{Costruzione lunghezze dell'array $MS$}
    \If{$ms_{row}[k] = M$}
    \State $ms_{len}[k]\gets 0$
    \ElsIf{$k\neq 0\land ms_{row}[i]=ms_{row}[i-1]\land
    ms_{len}[i-1]\neq 0$}
    \State $ms_{len}[i]\gets ms_{len}[i-1]+1$
    \Else
    \Comment $ra$ effettua il random access con la componente \texttt{RA-BV} o
    \texttt{RA-SLP} 
    \State $tmp_{index}\gets i,\,\,tmp_{len}\gets 0$
    \While {$tmp_{index}\geq 0 \land z[tmp_{index}]=ra(ms_{row}[k],
    tmp_{index})$}
    \State $tmp_{index}\gets tmp_{index}-1,\,\,tmp_{len}\gets tmp_{len}+1$
    \EndWhile
    \State $ms_{len}[k]\gets tmp_{len}$
    \EndIf
    \EndFor
    \For {\textit{every} $k\in[0,|z|)$}
    \Comment \textbf{Calcolo dei match da $MS$}
    \If{$(ms_{len}[k]>1 \land ms_{len}[k]\geq ms_{len}[k+1])\lor(k = |z|-1 \land
    ms_{len}[k]\neq 0$}
    \State \textit{report dello SMEM terminante in colonna $k$}
    \State \textit{SMEM di lunghezza $ms_{len}[k]$ con la riga $ms_{row}[k]$ e
    quelle estese da essa tramite} \texttt{PHI}
    \EndIf
    \EndFor
    \EndFunction

    
    \Function {down}{$pos, prev, next$}
    \State \textit{si usano le LCE queries o il random access per calcolare il
    suffisso comune più lungo tra quelli delle righe}
    \State \textit{$pos$/$prev$ e
    $pos$/$next$ fino alla colonna precedente a quella corrente} 
    \State \textit{se il secondo è maggiore o uguale al primo ritorna $\top$,
    altrimenti $\bot$} 
    \EndFunction
  \end{algorithmic}
  \caption{\footnotesize{Calcolo degli SMEM con aplotipo esterno con componenti
  \texttt{MAP-INT/BV},
  \texttt{THR-INT/BV} (i cui usi diversificati di entrambe le componenti sono
  segnalati con ``oppure''), \texttt{RA-BV/SLP}, \texttt{PERM} e \texttt{PHI}.}}    
  \label{algo:matchthr}
\end{algorithm}
\begin{algorithm}
  \footnotesize
  \begin{algorithmic}[1]
    \Function{update}{$k, curr_{index}, z$}
    \State $curr_{index}\gets \W(k, curr_{index}, z[k])$
    \State $curr_{run}\gets \ITR(curr_{index},k+1)$ \textbf{oppure}
    $curr_{run}\gets \rank_h^{k+1}(curr_{index})$
    \State $symb\gets \GS(start_{k+1}, curr_{run})$
    \State \textbf{return} $(curr_{index},\,\,curr_{run},\,\,symb)$
    \EndFunction
  \end{algorithmic}
  \caption{Algoritmo per l'update con componenti \texttt{MAP-INT} e
  \texttt{MAP-BV}.} 
  \label{algo:updatems}
\end{algorithm}
\dc{Magari mettere complessità finali}
\subsubsection{Calcolo dell'array delle matching statistics tramite LCE query}
Come anticipato, grazie all'uso delle $\LCE$ query, è possibile calcolare
l'array 
delle matching statistics in un solo scorrimento, da sinistra a
destra, sull'aplotipo query. Infatti, è possibile usare le $\LCE$ query per
calcolare non solo quale 
nuova 
sequenza scegliere in caso di mismatch, in una certa colonna, con l'aplotipo
query, come 
si faceva con l'uso delle threshold, ma anche di computare la lunghezza
del suffisso comune tra essa e l'aplotipo query. In tal modo, si calcolano nello
stesso momento sia i valori di $\MS[i].\row$ che di $\MS[i].\len$ (e di
conseguenza anche gli $\SMEM$).\\
Tale soluzione è quindi relativa alle seguenti strutture dati composte:
\begin{itemize}
  \item \texttt{MAP-INT + LCE + PERM + PHI}
  \item \texttt{MAP-BV + LCE + PERM + PHI}
\end{itemize}
Con la notazione $\lce(k, x, y)$,
si indica il calcolo della $\LCE$ query, terminante in colonna $k-1$ (quindi
escludendo la colonna 
$k$-esima), tra le righe di indice $x$ e
indice $y$.\\ 
Si illustra ora come computare l'array delle matching statistics tramite le
$\LCE$ query. Anche
in questo caso, per convenzione, si inizia la computazione dall'ultima 
riga della prima colonna.
Si assuma di avere calcolato l'array $\MS$, per una query $z$ rispetto al
pannello $X$, fino alla
colonna $k-1$. Sia $i$ 
l'indice di riga sulla matrice $\PBWT$ al quale si è arrivati mediante il
mapping, avendo che tale riga corrisponde a quella, in $X$, che ha il più lungo
suffisso 
comune con 
$z[0,k-1]$. Si assuma che l'indice $i$ appartenga alla run $r$, di simboli
$\sigma$, con testa di indice $s$ e coda di indice $e$. Si hanno diversi casi:
\begin{enumerate}
  \item $z[k]=y_i^k[k]=\sigma$, quindi la riga $i$ può essere usata per
  estendere il 
  match, avendo che $\MS[k].\row=\MS[k-1].\row$ e $\MS[k].\len=\MS[k-1].\len+1$,
  e per 
  proseguire col mapping in colonna $k+1$
  \item $z[k]\neq y_i^k[k]=\sigma$ e si ha una sola run in colonna $k$, avendo
  che 
  non si possono avere match in quella colonna. Per convenzione, si
  ha che $\MS[k].\row = M$ e $\MS[k].\len=0$. Infine, si ricomincia, in
  colonna 
  $k+1$, dall'ultima posizione. Tale indice corrisponde, nel pannello originale,
  alla riga specificata dal valore del prefix array sample della coda
  dell'ultima run 
  \item $z[k]\neq y_i^k[k]=\sigma$ ma si hanno anche altre run, dovendo quindi
  scegliere 
  la nuova riga da seguire. Si ha che il più lungo suffisso di $z[0,k]$, che è
  anche suffisso di $x_0[0,k],\ldots, x_{M-1}[0,k]$, è uno tra:
  \begin{itemize}
    \item $x_{a_k[s-1]}$, se $s\neq 0$, ovvero la riga del pannello
    corrispondente alla coda della run precedente alla run corrente, nella
    matrice $\PBWT$, se esistente
    \item $x_{a_k[e+1]}$, se $e\neq M-1$, ovvero la riga del pannello
    corrispondente alla testa della run successiva alla run corrente, nella
    matrice $\PBWT$, se esistente
  \end{itemize}
  Anche in questo caso, questo fatto è dovuto all'ordinamento lessicografico
  inverso, che si ha per la costruzione della $\PBWT$.
  Avendo quindi i prefix array sample, che ci dicono a quale riga nel
  pannello corrispondano tali valori, e conoscendo $\MS[k-1].\row$, è possibile
  calcolare $\lce(k,\MS[k-1].\row, a_k[s-1])$ e $\lce(k,\MS[k-1].\row,
  a_k[e+1])$. Si sceglie il suffisso comune più lungo tra le due, ovvero il più
  lungo risultato tra le due funzioni $\lce$, e si sceglie la riga
  corrispondente per proseguire. Si ha quindi o $\MS[k].\row=a_k[s-1]$ o
  $\MS[k].\row=a_k[e+1]$. In merito al campo $\len$, assumendo che la lunghezza
  maggiore delle due $\LCE$ query sia $l$, si ha che:
  \begin{equation}
    \label{eq:mslce1}
    \MS[k].\len=\min(\MS[k-1].\len, l)+1
  \end{equation}
  Questa assegnazione si ha in quanto la $\LCE$ query potrebbe restituire un
  valore più lungo dell'effettivo 
  match con la query $z$. Si sceglie, di conseguenza, il minimo tra le due
  lunghezze, per considerare l'overlap, 
  ottenendo l'effettiva lunghezza del suffisso comune tra $z$ e la nuova riga
  scelta, fino alla colonna $k-1$, incrementandolo di uno per conteggiare il
  match ottenuto in colonna $k$ 
\end{enumerate}
\dc{Sistemare esempio.}
\begin{esempio}
  Si riprende l'esempio \ref{es:thr}, visto per il calcolo di $\MS[i].\row$,
  dopo un mismatch, tramite threshold. \\
  Senza usare le threshold, nella medesima situazione si dovrebbero
  calcolare, avendo che $\MS[1].\row=19$ e $\MS[1].\len =2$:
  \[\lce(2, x_{19}, x_{16}) = \mbox{"01"} \implies|\lce(2, x_{19}, x_{16})|=2\]
  \[\lce(2, x_{19}, x_{17}) = \mbox{"1"} \implies|\lce(2, x_{19}, x_{17})|=1\]
  Come verificabile dal pannello presente all'esempio \ref{es:pbwt1}.\\
  Si ha quindi che $\MS[2].\row=16$. Inoltre, sempre per quanto detto sopra:
  \[\MS[2].\len=\min(\MS[1].\len, 2)+1=2+1=3\]
\end{esempio}
\noindent
Con questa soluzione, il cui pseudocodice è consultabile all'algoritmo
\ref{algo:matchlce}: 
\begin{itemize}
  \item non si necessita di tenere in memoria le informazioni per le
  threshold
  \item si permette il calcolo dell'array $\MS$ in una singola scansione della
  query
  \item non si necessita di memorizzare l'intero array $\MS$ ma solamente
  quattro variabili relative alla coppia
  $(\row,\len)$ corrente e a quella precedente. Infatti, per computare i valori
  in 
  colonna $k+1$ dell'array $\MS$ e gli $\SMEM$ terminanti in colonna $k+1$, si
  necessita solo delle informazioni in colonna $k$. \\
  \textit{Per facilità di
    lettura si è lasciato, nello pseudocodice, l'uso dell'intero array $\MS$}
\end{itemize}
Dal punto di vista della complessità temporale, per il calcolo dell'array $\MS$
tramite $\LCE$ query,
si hanno solo due casistiche 
possibili, al variare della componente di mapping. 
Nel caso della componente \texttt{MAP-INT}, avendo $\rho$ numero medio di run
per colonna, si 
ha un tempo proporzionale, dovendo iterare la query, fare il mapping e usare la
componente \texttt{LCE}, a: 
\begin{equation}
  \label{eq:mslce22}
  \mathcal{O}(N(\log \rho+\log (NM)))
\end{equation}
Mentre, nel caso dell'uso della componente \texttt{MAP-BV}, si ha tempo
proporzionale a:
\begin{equation}
  \label{eq:mslce3}
  \mathcal{O}\left(N\left(\log \frac{M}{\rho}+\log (NM)\right)\right)
\end{equation}
Infine, per il calcolo di tutte le righe del pannello per cui si ha uno
$\SMEM$, bisogna considerare quando analizzato per la componente
\texttt{PHI}.\\ 
Si deduce come la struttura composta \texttt{MAP-INT + LCE + PERM + PHI}
sia, a 
livello di tempo macchina, la soluzione più vantaggiosa usando la componente
\texttt{LCE}. Tale soluzione risulta essere, sempre nel contesto delle strutture
basate 
sulla componente \texttt{LCE}, anche la miglior soluzione in
termini di memoria.\\
Si vedrà, sperimentalmente, nel capitolo \ref{reschap}, il
confronto con le altre strutture dati. Una prima intuizione in merito è quella
che, usando le $\LCE$ query, si avranno sicuramente, a parità di componenti per
il mapping, tempi peggiori rispetto
all'uso della componente \texttt{RA-BV}, come mostrato dalle complessità
temporali. Un confronto con le strutture basate su \texttt{RA-SLP} risulta.
invece, più complesso da analizzare, limitandosi alle stime asintotiche, avendo
quindi forte necessità di un'analisi più sperimentale.
\begin{algorithm}
  \scriptsize
  \begin{algorithmic}[1]
    \Function{matches\_ms\_lce}{$z$}
    \State $ms_{row}\gets [0..0],\,\,ms_{len}\gets [0..0]$
    \Comment array $MS$ di lunghezza $|z|$
    \State $curr_{row}\gets
    rlpbwt[0].samples_{end}[|rlpbwt[0].samples_{end}|-1],\,\,curr_{index}\gets
    curr_{row}$ 
    \State $curr_{run}\gets \ITR(curr_{index},0)$ \textbf{oppure}
    $curr_{run}\gets rank_h^0(curr_{index})$  
    \State $symb\gets \GS(start_0, curr_{run})$
    \Comment \textbf{Costruzione dell'array $MS$}
    \For {\textit{every} $k\in[0, |z|)$}
    \If{$z[i]=symb$}
    \State $ms_{row}[k]\gets curr_{row}$
    \State \textbf{if} $k=0$ \textbf{then} $ms_{len}[k] \gets 1$ \textbf{else}
    $ms_{len}[k] \gets ms_{len}[k-1]+1$
    \State \hspace{-1.1mm}\textbf{if} $k\neq |z|-1$ \textbf{then}
    $(curr_{index},\,\,curr_{run},\,\,symb)\gets \UP(k, curr_{index},z)$ 
    % \If{$k\neq |z|-1$}
    % \State $(curr_{index},\,\,curr_{run},\,\,symb)\gets \UP(k, curr_{index},
    % z)$ 
    % \EndIf
    \Else
    \If{$|samples_{beg}^k|=1$}
    \State $ms_{row}[k]\gets M$
    \State $ms_{len}[k]\gets 0$
    \If{$k\neq |z|-1$}
    \State $curr_{row}\gets
    rlpbwt[k+1].samples_{end}[|rlpbwt[k+1].samples_{end}|-1]$
    \State $curr_{index}\gets M-1$
    \State $curr_{run}\gets \ITR(curr_{index},k+1)$ \textbf{oppure}
    $curr_{run}\gets rank_h^{k+1}(curr_{index})$
    \State $symb\gets \GS(start_{k+1}, curr_{run})$
    \EndIf
    \Else
    \If{$curr_{run}=|samples_{beg}^k|-1$}
    \State $curr_{index}\gets p_k[curr_{run}-1]$ \textbf{oppure}
    $curr_{index}\gets select_h^k(curr_{run})$
    \State $prev_{row}\gets
    samples_{end}^k[curr_{run}-1]$ 
    \State $lce\gets \lce(k, curr_{row}, prev_{row})$
    \State $ms_{row}[k]\gets prev_{row},\,\,curr_{row}\gets prev_{row}$
    \State \textbf{if} $k=0$ \textbf{then} $ms_{len}[k] \gets 1$ \textbf{else}
    $ms_{len}[k] \gets min(ms_{len}[k-1], |lce|)+1$ 
    % \If{$k=0$}
    % \State $ms_{len}[k] \gets 1$
    % \Else
    % \State $ms_{len}[k] \gets min(ms_{len}[k-1], |lce|)+1$
    % \EndIf
    \State \textbf{if} $k\neq |z|-1$ \textbf{then}
    $(curr_{index},\,\,curr_{run},\,\,symb)\gets \UP(k, curr_{index},z)$  
    % \If{$k\neq |z|-1$}
    % \State $(curr_{index},\,\,curr_{run},\,\,symb)\gets \UP(k, curr_{index},
    % z)$  
    % \EndIf    
    \ElsIf{$curr_{run}=0$}
    \State $curr_{index}\gets p_k[curr_{run}+1]$ \textbf{oppure}
    $curr_{index}\gets select_h^k(curr_{run}+1)+1$
    \State $next_{row}\gets samples_{beg}^k[curr_{run}+1]$ 
    \State $lce\gets \lce(k, curr_{row}, next_{row})$
    \State $ms_{row}[k]\gets next_{row},\,\,curr_{row}\gets next_{row}$
    \State \textbf{if} $k=0$ \textbf{then} $ms_{len}[k] \gets 1$ \textbf{else}
    $ms_{len}[k] \gets min(ms_{len}[k-1], |lce|)+1$
    % \If{$k=0$}
    % \State $ms_{len}[k] \gets 1$
    % \Else
    % \State $ms_{len}[k] \gets min(ms_{len}[k-1], |lce|)+1$
    % \EndIf
    \State \textbf{if} $k\neq |z|-1$ \textbf{then} $(curr_{index},\,\,
    curr_{run},\,\,symb)\gets \UP(k, curr_{index},z)$  
    % \If{$k\neq |z|-1$}
    % \State $(curr_{index},\,\,curr_{run},\,\,symb)\gets \UP(k, curr_{index},
    % z)$ 
    % \EndIf
    \Else
    \State $prev_{row}\gets samples_{end}^k[curr_{run}-1],\,\,next_{row}\gets
    samples_{beg}^k[curr_{run}+1]$ 
    \State $lce\gets \max (|\lce(k, curr_{row}, prev_{row})|, |\lce(k,
    curr_{row}, next_{row})|)$
    \State $curr_{row}\gets lce_{row}$
    \Comment $lce_{row}$ segnala l'indice della riga con \textit{LCE query} più
    lunga 
    \State $ms_{row}[k]\gets curr_{row}$
    \State \textbf{if} $k=0$ \textbf{then} $ms_{len}[k] \gets 1$ \textbf{else}
    $ms_{len}[k] \gets min(ms_{len}[k-1], |lce|)+1$
    % \If{$k=0$}
    % \State $ms_{len}[k] \gets 1$
    % \Else
    % \State $ms_{len}[k] \gets min(ms_{len}[k-1], |lce|)+1$
    % \EndIf
    \State \textbf{if} $k\neq |z|-1$ \textbf{then}
    $(curr_{index},\,\,curr_{run},\,\,symb)\gets \UP(k, curr_{index},z)$ 
    % \If{$k\neq |z|-1$}
    % \State $(curr_{index},\,\,curr_{run},\,\,symb)\gets \UP(k,curr_{index},
    % z)$ 
    % \EndIf
    \EndIf
    \EndIf
    \EndIf
    \EndFor
    
    \For {\textit{every} $k\in[0,|z|)$}
    \Comment \textbf{Calcolo dei match da $MS$}
    \If{$(ms_{len}[k]>1 \land ms_{len}[k]\geq ms_{len}[k+1])\lor(k = |z|-1 \land
    ms_{len}[k]\neq 0$}
    \State \textit{report degli SMEM di lunghezza $ms_{len}[k]$, terminanti in
    colonna $k$}
    \State \textit{con la riga $ms_{row}[k]$ e quelle estese da essa tramite
    la componente \texttt{PHI}} 
    \EndIf
    \EndFor
    \EndFunction
    
  \end{algorithmic}
  \caption{\footnotesize{Calcolo degli SMEM con aplotipo esterno con componenti
  \texttt{MAP-INT/BV} (i cui usi diversificati sono segnalati con ``oppure''),
  \texttt{LCE}, \texttt{PERM} e \texttt{PHI}.}}
  \label{algo:matchlce}
\end{algorithm}
\dc{Sistemare pseudocodice per non avere salvato intero $MS$}
\dc{Uniformare gli pseudocodici}
% LocalWords:  pseudocodice
