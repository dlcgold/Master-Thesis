\section{Bitvector}
\label{bvsec}
\textbf{TUTTE LE TABELLE VANNO VERIFICATE!!!}\\
Nell'ambito delle \textit{strutture dati succinte}, una delle strutture dati
principali, ormai sviluppatasi in molteplici varianti, è quella denominata
\textbf{bit vector}.
\begin{definizione}
  Si definisce un \textbf{bit vector} $B$ come un array di lunghezza $n$,
  popolato da elementi binari. Formalmente si ha quindi:
  \[B[i]=\{0,1\},\,\,\forall i \mbox{ t.c. } 0\leq i < n\]
  In alternativa si potrebbe avere come formalismo:
  \[B[i]=\{\bot,\top\},\,\,\forall i \mbox{ t.c. } 0\leq i < n\]
\end{definizione}
Nel corso degli ultimi anni si sono sviluppate diverse varianti dei \textit{bit
  vector}, finalizzate ad offrire diversi costi di complessità spaziale e
diversi tempi computazionali per le principali funzioni offerte.\\
Il primo vantaggio di questa struttura dati, nelle varianti che si andranno poi
a nominare, è quella di garantire \textit{random access} in tempo costante pur
sfruttando varie tecniche per la memorizzazione efficiente della stessa in
memoria. Lo spazio necessario per l'implementazione, presente nella
\textbf{Succinct Data Structure Library (\textit{SDSL})} \cite{sdsl}, una delle
  principali librerie (scritta in \Cplusplus 11) per strutture dati succinte,
  delle principali varianti è visualizzabile in tabella 
\ref{tab:bvspace}. Il secondo vantaggio consiste nel fatto che i \textit{bit
  vector} permettono l'implementazione efficiente di due funzioni:
\begin{enumerate}
  \item la \textbf{funzione rank}
  \item la \textbf{funzione select}
\end{enumerate}
Tali funzioni, al costo di $\mathcal{O}(n)$ bit aggiuntivi, possono essere
supportate in tempo costante. Questo è però un discorso prettamente teorico,
infatti si vedrà come, nelle implementazioni in \textit{SDSL}, le complessità
temporali delle due funzioni non siano mai entrambe costanti.
% tabella dello spazio occupato dalle varianti dei bit vector
\begin{table}[H]
  \small
  \centering
  \caption{Stime dello spazio occupato per la memorizzazione di alcune varianti
    di \textit{bit vector}. Si 
    assume un bit vector di lunghezza $n$ con un numero di bit posti pari a
    1 (o $\top$) pari a $m$. $K$ indica un valore costante.} 
  \begin{tabular}{c|c}
    \textbf{Variante} & \textbf{Spazio occupato}\\
    \hline\xrowht{15pt}
    \textit{Plain bitvector} & $64\big\lceil\frac{n}{64}+1\big\rceil$\\
    \hline\xrowht{15pt}
    \textit{Interleaved bitvector} & $\approx n\left(1+\frac{64}{K}\right)$\\
    \hline\xrowht{15pt}
    \textit{$H_0$-compressed bitvector} & $\approx\big\lceil\log\binom{n}{m}\big\rceil$\\
    \hline\xrowht{15pt}
    \textit{Sparse bitvector} & $\approx m\left(2+\log\frac{n}{m}\right)$\\
  \end{tabular}
  \label{tab:bvspace}
\end{table}


\subsection{Funzione rank}
La prima funzione che si approfondisce è la \textbf{funzione rank}. Tale
funzione permette di calcolare il \textit{rango} di un dato elemento del
bit vector $B$, $|B|=n$. In altri termini, data una certa posizione $i$ del
\textit{bit vector}, la funzione restituisce il numero di 1 presenti fino a
quella data posizione, escluda. Più formalmente si ha:
\[rank_B(i)=\sum_{k=0}^{k<i}B[k],\,\,\forall i \mbox{ t.c. } 0\leq i < n\]
Come detto, da un punto di vista teorico, al costo di $\mathcal{O}(n)$ bit
aggiuntivi in memoria tale funzione sarebbe supportata in tempo
$\mathcal{O}(1)$. Questo però non risulta vero nelle principali
implementazioni. La complessità temporale varia infatti a seconda
dell'implementazione, anche in conseguenza al fatto che si ha una quantità
diversa di bit aggiuntivi salvati in memoria.
La tabella con le complessità temporali stimate della \textit{funzione rank},
per le varianti di \textit{bit vector} implementate in \textit{SDSL}, è
visualizzabile in tabella \ref{tab:rank}.
% tabella relativa ai costi della funzione rank dei bitvector
\begin{table}[H]
  \small
  \centering
  \caption{Complessità temporali stimate della \textit{funzione rank} per alcune
    varianti di \textit{bit 
      vector}, con la quantità di bit aggiuntivi richiesta. Si assume un bit
    vector di lunghezza $n$, con un numero di bit 
    posti pari a 1 (o $\top$) pari a $m$, e un numero $k$ di valori prima della
    posizione richiesta.} 
  \begin{tabular}{c|c|c}
    \textbf{Variante} & \textbf{Bit aggiuntivi} & \textbf{Complessità
                                                  temporale}\\ 
    \hline\xrowht{15pt}
    \textit{Plain bitvector} & $0.0625\cdot n$ & $\mathcal{O}(1)$\\
    \hline\xrowht{15pt}
    \textit{Interleaved bitvector} & $128$ & $\mathcal{O}(1)$\\
    \hline\xrowht{15pt}
    \textit{$H_0$-compressed bitvector} & $80$ & $\mathcal{O}(k)$\\
    \hline\xrowht{15pt}
    \textit{Sparse bitvector} & $64$ & $\mathcal{O}\left(\log\frac{n}{m}\right)$\\ 
  \end{tabular}
  \label{tab:rank}
\end{table}
\subsection{Funzione select}
La seconda funzione fondamentale è la \textbf{funzione select}. Tale funzione
permette, dato un valore intero $i$, di calcolare l'indice dell'$i$-esimo valore
pari a 1 nel \textit{bit vector} $B$, tale che $|B|=n$. Più formalmente si ha
che:
\[select_B(i)=\min\{j < n\,|\,\,rank_B(j+1)=1\},\,\,\forall i \mbox{ t.c. }
  0<i\leq rank_B(n)\]
Anche in questo caso vale lo stesso discorso fatto per la \textit{funzione rank}
in merito alla complessità temporale e ai bit aggiuntivi. La tabella con le
complessità temporali stimate della \textit{funzione select}, 
per le varianti di \textit{bit vector} implementate in \textit{SDSL}, è
visualizzabile in tabella \ref{tab:select}.\\
% tabella relativa ai costi della funzione select dei bitvector
\begin{table}[H]
  \small
  \centering
  \caption{Complessità temporali stimate della \textit{funzione select} per
    alcune varianti di \textit{bit 
      vector}, con la quantità di bit aggiuntivi richiesta. Si assume un bit
    vector di lunghezza $n$, con un numero di bit 
    posti pari a 1 (o $\top$) pari a $m$.} 
  \begin{tabular}{c|c|c}
    \textbf{Variante} & \textbf{Bit aggiuntivi} & \textbf{Complessità
                                                  temporale}\\ 
    \hline\xrowht{15pt}
    \textit{Plain bitvector} & $\leq 0.2\cdot n$ & $\mathcal{O}(1)$\\
    \hline\xrowht{15pt}
    \textit{Interleaved bitvector} & $64$ & $\mathcal{O}(\log n)$\\
    \hline\xrowht{15pt}
    \textit{$H_0$-compressed bitvector} & $64$ & $\mathcal{O}(\log n)$\\
    \hline\xrowht{15pt}
    \textit{Sparse bitvector} & $64$ & $\mathcal{O}(1)$\\ 
  \end{tabular}
  \label{tab:select}
\end{table}
Si può quindi vedere un semplice esempio esplicativo.
\begin{esempio}
  Ipotizziamo di avere il seguente bit vector $B$, di lunghezza $n=14$:
  \begin{center}
    \begin{tikzpicture} [nodes in empty cells,
      nodes={minimum width=0.6cm, minimum height=0.6cm},
      row sep=-\pgflinewidth, column sep=-\pgflinewidth]
      border/.style={draw}
      
      \matrix(vector)[matrix of nodes,
      row 1/.style={nodes={draw=none, minimum width=0.3cm}},
      nodes={draw}]
      {
        \tiny{0} & \tiny{1} & \tiny{2} & \tiny{3} & \tiny{4} & \tiny{5}&\tiny{6}
        & \tiny{7} & \tiny{8} & \tiny{9} & \tiny{10} & \tiny{11} & \tiny{12} &
        \tiny{13}\\  
        $\mathbf{1}$ & $0$ & $0$ & $\mathbf{1}$ & $0$ & $\mathbf{1}$ & $0$ &
        $\mathbf{1}$ & $0$ & $\mathbf{1}$ & $0$ & $0$ & $\mathbf{1}$ & $0$\\ 
      };
    \end{tikzpicture}
  \end{center}
  Si ha che, per esempio:
  \[\mathtt{rank}(6)=3\]
  \[\mathtt{select}(5) =9\]
\end{esempio}
Da un punto di vista pratico si vedrà nel corso di questa tesi come l'uso di
tali strutture, nel dettaglio l'uso dei \textit{bit vector plain} e dei
\textit{bit vector sparsi}, sia fondamentale sia nella costruzione delle
\textit{strutture run-length encoded} che nelle interrogazioni alle stesse.


