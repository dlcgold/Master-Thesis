% tabella dello spazio occupato dalle varianti dei bit vector
\begin{table}[H]
  \small
  \centering
  \caption{Stime dello spazio occupato per la memorizzazione di alcune varianti
    di \textit{bit vector}. Si 
    assume un bit vector di lunghezza $n$ con un numero di bit posti pari a
    1 (o $\top$) pari a $m$. $K$ indica un valore costante.} 
  \begin{tabular}{c|c}
    \textbf{Variante} & \textbf{Spazio occupato}\\
    \hline\xrowht{15pt}
    \textit{Plain bitvector} & $64\big\lceil\frac{n}{64}+1\big\rceil$\\
    \hline\xrowht{15pt}
    \textit{Interleaved bitvector} & $\approx n\left(1+\frac{64}{K}\right)$\\
    \hline\xrowht{15pt}
    \textit{$H_0$-compressed bitvector} & $\approx\big\lceil\log\binom{n}{m}\big\rceil$\\
    \hline\xrowht{15pt}
    \textit{Sparse bitvector} & $\approx m\left(2+\log\frac{n}{m}\right)$\\
  \end{tabular}
  \label{tab:bvspace}
\end{table}

% tabella relativa ai costi della funzione rank dei bitvector
\begin{table}[H]
  \small
  \centering
  \caption{Complessità temporali stimate della \textit{funzione rank} per alcune
    varianti di \textit{bit 
      vector}, con la quantità di bit aggiuntivi richiesta. Si assume un bit
    vector di lunghezza $n$, con un numero di bit 
    posti pari a 1 (o $\top$) pari a $m$, e un numero $k$ di valori prima della
    posizione richiesta.} 
  \begin{tabular}{c|c|c}
    \textbf{Variante} & \textbf{Bit aggiuntivi} & \textbf{Complessità
                                                  temporale}\\ 
    \hline\xrowht{15pt}
    \textit{Plain bitvector} & $0.0625\cdot n$ & $\mathcal{O}(1)$\\
    \hline\xrowht{15pt}
    \textit{Interleaved bitvector} & $128$ & $\mathcal{O}(1)$\\
    \hline\xrowht{15pt}
    \textit{$H_0$-compressed bitvector} & $80$ & $\mathcal{O}(k)$\\
    \hline\xrowht{15pt}
    \textit{Sparse bitvector} & $64$ & $\mathcal{O}\left(\log\frac{n}{m}\right)$\\ 
  \end{tabular}
  \label{tab:rank}
\end{table}

% tabella relativa ai costi della funzione select dei bitvector
\begin{table}[H]
  \small
  \centering
  \caption{Complessità temporali stimate della \textit{funzione select} per
    alcune varianti di \textit{bit 
      vector}, con la quantità di bit aggiuntivi richiesta. Si assume un bit
    vector di lunghezza $n$, con un numero di bit 
    posti pari a 1 (o $\top$) pari a $m$.} 
  \begin{tabular}{c|c|c}
    \textbf{Variante} & \textbf{Bit aggiuntivi} & \textbf{Complessità
                                                  temporale}\\ 
    \hline\xrowht{15pt}
    \textit{Plain bitvector} & $\leq 0.2\cdot n$ & $\mathcal{O}(1)$\\
    \hline\xrowht{15pt}
    \textit{Interleaved bitvector} & $64$ & $\mathcal{O}(\log n)$\\
    \hline\xrowht{15pt}
    \textit{$H_0$-compressed bitvector} & $64$ & $\mathcal{O}(\log n)$\\
    \hline\xrowht{15pt}
    \textit{Sparse bitvector} & $64$ & $\mathcal{O}(1)$\\ 
  \end{tabular}
  \label{tab:select}
\end{table}