\section{Sviluppi futuri}
Questa prima implementazione completa della $\RLPBWT$,
declinata nelle possibili strutture composte, non è da
considerarsi come un punto di arrivo. Come accaduto per la $\PBWT$,
infatti, si potranno sviluppare nuove strutture dati basate su di essa per la
gestione di pannelli di varia natura. Principalmente si può pensare a due casi,
già anticipati nella sezione \ref{secpbwt}:
\begin{itemize}
  \item lo studio di pannelli multiallelici, ovvero costruiti su un alfabeto
  $\Sigma$ arbitrario e non limitato ai simboli $\sigma=0$ e $\sigma=1$
  \item lo studio di pannelli con dati mancanti, ovvero pannelli costruiti
  da dati reali che possono contenere siti per cui non si ha certezza
  dell'allele  
\end{itemize}
Inoltre la struttura dati è stata sviluppata per permettere
unicamente il calcolo degli $\SMEM$ con un aplotipo esterno, ma
si potrebbero sviluppare nuovi algoritmi che
rispondano a task diversi, come il calcolo degli $\SMEM$ interni al
panello, il calcolo di tutti i match con
un aplotipo esterno di lunghezza maggiore di una fissata o il calcolo degli
$\SMEM$ con numero stabilito di aplotipi nel panello.
\paragraph{RLPBWT multiallelica.}
Per quanto i pannelli di aplotipi prodotti dal sequencing del genoma umano
raramente presentino siti multiallelici, si ha una presenza stimata
di circa il 2\% di siti triallelici \cite{tri}. Inoltre, all'aumentare della
disponibilità di dati genomici, si ha in letteratura la propensione a credere
che tale percentuale di siti sia non solo sottostimata (si suppone sia
stimato circa un terzo dei reali siti triallelici), ma anche destinata a
crescere in modo non lineare rispetto al numero di individui sequenziati
\cite{tri2}. Quindi, una struttura dati efficiente in memoria, in grado di
gestire pannelli costruiti su un alfabeto arbitrario, risulterà necessaria nel
breve futuro.\\
Ipotizzando un possibile funzionamento della $\RLPBWT$
multiallelica ($\MRLPBWT$), si può pensare ad una soluzione molto
simile a quanto visto per la $\RLBWT$. Infatti, per ogni colonna, si
potrebbero memorizzare:
\begin{itemize}
  \item una stringa che tenga traccia di quale simbolo corrisponda a una certa
  run, 
  non potendo sfruttare l'alternanza di simboli vista nel caso binario
  \item una rivisitazione delle strutture necessarie al mapping, tenendo in
  memoria vettori di bitvector sparsi o di intvector compressi, per poter
  computare la funzione $w(i,\sigma)$ anche nel caso multiallelico. Si ipotizza
  un'inversione di 
  tendenza in termini di memoria, dato che l'uso di intvector
  compressi potrebbe rivelarsi meno efficiente dell'uso dei bitvector sparsi,
  anche con pochi sample
  \item un riadattamento del calcolo dell'array delle matching statistics 
\end{itemize}
In merito allo spazio richiesto e ai tempi di calcolo, bisognerà considerare la
grandezza dell'alfabeto su cui è costruito il pannello, che ci si aspetta,
fortunatamente, 
inferiore a $10$ nella maggioranza dei casi di studio biologico.\\
Quindi, nonostante, allo stato dell'arte, ci siano pochissimi studi in merito,
si ritiene 
possibile generalizzare, in modo computazionalmente efficiente, la
$\RLPBWT$ anche a questa casistica. 
\paragraph{RLPBWT con dati mancanti.}
Limitandosi a studiare pannelli corretti in una fase di
preprocessing, si rischia di limitare la capacità di inferenza dai pannelli
stessi, poiché si assumono esatti e senza dati mancanti.\\ 
Come anticipato alla sezione \ref{secpbwt}, si sono iniziate a sviluppare
estensioni della $\PBWT$ che ammettono wildcard, ovvero simboli nel
pannello che possono assumere qualsiasi valore dell'alfabeto $\Sigma$ del
pannello.\\
Uno degli sviluppi futuri potrebbe essere quello di generalizzare la
$\RLPBWT$, ma anche l'eventuale $\MRLPBWT$, per la gestione di dati
mancanti nel pannello. Inoltre, si potrebbero sviluppare algoritmi in grado di
gestire le wildcard anche all'interno delle query stesse.\\
L'uso di algoritmi parametrici (ma anche
di algoritmi approssimati), adattati al
funzionamento della $\RLPBWT$, potrebbe portare a soluzioni interessanti
per la gestione di pannelli reali non preprocessati.
\paragraph{K-SMEM.}
Oltre a variare le caratteristiche del pannello in analisi,
si possono studiare anche algoritmi per risolvere nuovi task con la
$\RLPBWT$.\\ 
Di recente, Gagie \cite{kmems} ha proposto uno studio in cui dimostra come
la $\RLBWT$, implementata in MONI \cite{moni}, sia già predisposta al
calcolo dei $\KMEM$, ovvero $\MEM$, tra sottostringhe di un
pattern e un testo che occorrono esattamente $k$ volte nel testo stesso.\\
Si potrebbe adattare l'idea di Gagie al calcolo
dei $\KSMEM$, ovvero $\SMEM$ che occorrono tra l'aplotipo query e esattamente
$k$ righe del pannello. La 
correlazione tra la $\RLBWT$ e la $\RLPBWT$ 
porta a pensare che tale problema sia risolvibile anche con la nuova definizione
di matching statistics, presentata in questa tesi.
Si ritiene la risoluzione di questo task
un interessante sviluppo futuro in quanto potrebbe permettere studi statistici
(caratteristici dei GWAS),
riguardanti la presenza di specifici $\SMEM$ tra un aplotipo query e un pannello
di aplotipi.\\
\\
\\
\textit{La tematica della pangenomica è innovativa, il numero di
problemi aperti è incredibilmente grande, i dati aumentano sempre di più e gli
studi informatici/bioinformatici devono evolversi per ``stare al passo'' con
questa mole 
d'informazioni. Gli sviluppi futuri sono, da diversi punti di vista,
imprevedibili e quindi risulta difficile elencare, in modo completo, le
prospettive future di questa branca della bioinformatica e dell'algoritmica
sperimentale.}
% LocalWords:  preprocessing
