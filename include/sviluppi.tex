\section{Sviluppi futuri}
Ovviamente questa prima implementazione completa della \textit{RLPBWT} non è da
considerarsi come un punto di arrivo. Come accaduto per la \textit{PBWT},
infatti, si potranno sviluppare nuove strutture dati basate su di essa per la
gestione di pannelli di varia natura. Principalmente si può pensare a due casi,
già anticipati nella sezione \ref{secpbwt}:
\begin{itemize}
  \item \textbf{pannelli multi-allelici}, ovvero costruiti su un alfabeto
  $\Sigma$ arbitrario e non limitato ai simboli $\sigma=0$ e $\sigma=1$
  \item \textbf{pannelli con dati mancanti}, ovvero pannelli costruiti
  direttamente da \textit{dati reali} che possono contenere siti, per certi
  individui, per i quali non si ha certezza in merito all'allele
\end{itemize}
Inoltre, allo stato attuale, la struttura dati è stata sviluppata per permettere
unicamente il calcolo dei match massimali con un aplotipo esterno. Anche in
questo caso, quindi, si potrebbe avere lo sviluppo di nuovi algoritmi che
rispondano a task diversi, come il calcolo dei match interni al panel, i
cosiddetti \textit{blocchi}, o anche il calcolo di tutti i match con un aplotipo
esterno di lunghezza maggiore ad una fissata o che includano un numero stabilito
di sequenze di aplotipi nel panello.
\paragraph{RLPBWT multi-allelica}
Per quanto i pannelli di aplotipi prodotti dal sequencing del genoma umano
raramente presentino siti multi-allelici si ha una presenza stimata, al momento,
di circa il 2\% di siti tri-allelici \cite{tri}. Inoltre, all'aumentare della
disponibilità di dati genomici, si ha in letteratura la propensione a credere
che tale percentuale di siti sia non solo sottostimata (stimando che sia stimato
circa un terzo dei reali siti tri-allelici) ma anche destinata a
cresce in modo non lineare rispetto al numero di individui sequenziati
\cite{tri2}. Inoltre, molte specie, soprattutto vegetali, sono già riconosciute
essere poliploidi, quindi una struttura dati efficiente in memoria in grado di
gestire pannelli costruiti su un alfabeto arbitrario risulterà necessaria nel
breve futuro.\\
Ipotizzando un possibile funzionamento della \textbf{RLPBWT
  multi-allelica (\textit{m-RLPBWT})} si può pensare ad una soluzione molto
simile a quanto visto per la \textit{RLPBWT}. Infatti, per ogni colonna, si
potrebbero memorizzare:
\begin{itemize}
  \item una stringa che memorizzi quale simbolo corrisponda ad una certa run,
  non potendo sfruttare l'alternanza di simboli vista nel caso binario
  \item una rivisitazione delle strutture necessarie al mapping, tenendo in
  memoria vettori di \textit{bitvector sparsi}
  \item riadattamento del calcolo dell'array delle \textit{matching statistics}
\end{itemize}
In merito allo spazio richiesto e ai tempi di calcolo bisognerà considerare la
grandezza dell'alfabeto su cui è costruito il pannello, che ci si aspetta comune
inferiore a $10$ nella maggioranza dei casi di studio biologico.\\
Nonostante, allo stato dell'arte, ci siano pochissimi studi in merito si ritiene
possibile generalizzare, in modo computazionalmente efficiente, la
\textit{RLPBWT} anche a questa casistica. 
\paragraph{RLPBWT con dati mancanti}
La maggior parte delle soluzioni attualmente sviluppate sono basate su una forte
assunzione: i dati in input sono corretti e senza dati mancanti. Ovviamente,
limitandosi a studiare pannelli simulati o comunque ``riempiti'' in una fase di
preprocessing, si rischia di non poter comprendere a fondo l'efficacia dei
metodi su dati reali, oltre che a limitare l'inferenza dai pannelli stessi.\\
Come anticipato alla sezione \ref{secpbwt}, si sono iniziate a sviluppare
estensioni della \textit{PBWT} che ammettano wildcard, ovvero simboli nel
pannello che possono assumere qualsiasi valore dell'alfabeto $\Sigma$, su cui è
costruito il pannello stesso.\\
Uno degli sviluppi futuri sarebbe quindi quello di generalizzare la
\textit{RLPBWT}, ma anche l'eventuale \textit{m-RLPBWT}, per la gestione di dati
mancanti nel pannello. Inoltre si potrebbero sviluppare algoritmi in grado di
gestire le wildcard anche all'interno delle query stesse.\\
Sempre in via ipotetica, l'uso di \textit{algoritmi parametrici} (manche anche
di \textit{algoritmi approssimati}) adattati al
funzionamento della \textit{RLPBWT} potrebbero portare a soluzioni interessanti
per la gestione di pannelli reali.
\paragraph{K-MEM}
Come anticipato, oltre che variare le caratteristiche del pannello in analisi,
si possono studiare anche algoritmi per risolvere nuovi task con la
\textit{RLPBWT}.\\ 
Di recente, Gagie \cite{kmems} ha proposto un articolo in cui dimostra come
la struttura implementata in \textit{MONI} \cite{moni} sia già predisposta al
calcolo dei \textbf{k-MEM}, ovvero match massimali tra sotto-stringhe di un
pattern e un testo che occorrono esattamente $k$ volte nel testo stesso.\\
In merito alla \textit{RLPBWT} si potrebbe adattare l'idea di Gagie al calcolo
di match massimali tra sotto-stringhe dell'aplotipo query e il pannello che
comportino il match con esattamente $k$ righe del pannello stesso. L'ormai
empiricamente dimostrata correlazione tra la \textit{RLBWT} e la \textit{RLPBWT}
porta a pensare che tale problema sia risolvibile anche con la nuova definizione
di \textit{matching statistics} per la \textit{RLPBWT}.\\
Ovviamente nulla è stato sviluppato al momento ma si ritiene questo
un'interessante sviluppo futuro in quanto permetterebbe studi statistici, molto
comuni nei \textit{GWAS}, in merito alla presenza si sotto-sequenze di un
aplotipo esterno all'interno di un pannello di aplotipi.\\
\textbf{SISTEMARE}
% LocalWords:  preprocessing
