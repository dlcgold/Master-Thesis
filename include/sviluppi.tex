\section{Sviluppi futuri}
Ovviamente, questa prima implementazione completa della $\RLPBWT$,
declinata nelle possibili strutture composte, non è da
considerarsi come un punto di arrivo. Come accaduto per la $\PBWT$,
infatti, si potranno sviluppare nuove strutture dati basate su di essa per la
gestione di pannelli di varia natura. Principalmente si può pensare a due casi,
già anticipati nella sezione \ref{secpbwt}:
\begin{itemize}
  \item lo studio di pannelli multiallelici, ovvero costruiti su un alfabeto
  $\Sigma$ arbitrario e non limitato ai simboli $\sigma=0$ e $\sigma=1$
  \item lo studio di pannelli con dati mancanti, ovvero pannelli costruiti
  da dati reali che possono contenere siti, per certi
  individui, per i quali non si ha certezza in merito all'allele
\end{itemize}
Inoltre, allo stato attuale, la struttura dati è stata sviluppata per permettere
unicamente il calcolo degli $\SMEM$ con un aplotipo esterno. Anche in
questo caso, quindi, si potrebbe avere lo sviluppo di nuovi algoritmi che
rispondano a task diversi, come, ad esempio, il calcolo degli $\SMEM$ interni al
panello, il 
calcolo dei cosiddetti \textit{blocchi}, o anche il calcolo di tutti i match con
un aplotipo 
esterno di lunghezza maggiore a una fissata o che includano un numero stabilito
di sequenze di aplotipi nel panello.
\dc{Serve definire i blocchi?}
\paragraph{RLPBWT multiallelica.}
Per quanto i pannelli di aplotipi, prodotti dal sequencing del genoma umano.
raramente presentino siti multiallelici, si ha una presenza stimata, al momento,
di circa il 2\% di siti triallelici \cite{tri}. Inoltre, all'aumentare della
disponibilità di dati genomici, si ha in letteratura la propensione a credere
che tale percentuale di siti sia non solo sottostimata (evidenziando che sia
stimato 
circa un terzo dei reali siti triallelici) ma anche destinata a
crescere in modo non lineare rispetto al numero di individui sequenziati
\cite{tri2}. Inoltre, molte specie, soprattutto vegetali, sono già riconosciute
essere poliploidi. Una struttura dati efficiente in memoria, in grado di
gestire pannelli costruiti su un alfabeto arbitrario, risulterà necessaria nel
breve futuro.\\
Ipotizzando un possibile funzionamento della $\RLPBWT$
multiallelica ($\MRLPBWT$), si può pensare ad una soluzione molto
simile a quanto visto per la $\RLBWT$. Infatti, per ogni colonna, si
potrebbero memorizzare:
\begin{itemize}
  \item una stringa che memorizzi quale simbolo corrisponda ad una certa run,
  non potendo sfruttare l'alternanza di simboli vista nel caso binario
  \item una rivisitazione delle strutture necessarie al mapping, tenendo in
  memoria vettori di bitvector sparsi o di intvector compressi, al fine di poter
  computare la funzione $w(i,\sigma)$ anche nel caso multiallelico. Si segnala
  che si attende un inversione di 
  tendenza in termini di memoria, avendo che, in tal caso, l'uso di intvector
  compressi potrebbe rivelarsi meno efficiente dell'uso dei bitvector sparsi,
  anche con pochi sample
  \item un riadattamento del calcolo dell'array delle matching statistics 
\end{itemize}
In merito allo spazio richiesto e ai tempi di calcolo bisognerà considerare la
grandezza dell'alfabeto su cui è costruito il pannello, che ci si aspetta,
fortunatamente, 
inferiore a $10$ nella maggioranza dei casi di studio biologico.\\
Quindi, nonostante, allo stato dell'arte, ci siano pochissimi studi in merito,
si ritiene 
possibile generalizzare, in modo computazionalmente efficiente, la
$\RLPBWT$ anche a questa casistica. 
\paragraph{RLPBWT con dati mancanti.}
La maggior parte delle soluzioni attualmente sviluppate sono basate su una forte
assunzione: i dati in input sono corretti e senza dati mancanti. Ovviamente,
limitandosi a studiare pannelli simulati corretti in una fase di
preprocessing, si rischia di limitare la capacità di inferenza dai pannelli
stessi.\\ 
Come anticipato alla sezione \ref{secpbwt}, si sono iniziate a sviluppare
estensioni della $\PBWT$ che ammettano wildcard, ovvero simboli nel
pannello che possono assumere qualsiasi valore dell'alfabeto $\Sigma$, su cui è
costruito il pannello stesso.\\
Uno degli sviluppi futuri sarebbe quindi quello di generalizzare la
$\RLPBWT$, ma anche l'eventuale $\MRLPBWT$, per la gestione di dati
mancanti nel pannello. Inoltre, si potrebbero sviluppare algoritmi in grado di
gestire le wildcard anche all'interno delle query stesse.\\
Sempre in via ipotetica, l'uso di algoritmi parametrici (ma anche
di algoritmi approssimati), adattati al
funzionamento della $\RLPBWT$, potrebbero portare a soluzioni interessanti
per la gestione di pannelli reali non preprocessati.
\paragraph{K-SMEM.}
Come anticipato, oltre che variare le caratteristiche del pannello in analisi,
si possono studiare anche algoritmi per risolvere nuovi task con la
$\RLPBWT$.\\ 
Di recente, Gagie \cite{kmems} ha proposto un articolo in cui dimostra come
la $\RLBWT$, implementata in MONI \cite{moni}, sia già predisposta al
calcolo dei $\KMEM$, ovvero $\MEM$, tra sottostringhe di un
pattern e un testo, che occorrono esattamente $k$ volte nel testo stesso.\\
In merito alla $\RLPBWT$, si potrebbe adattare l'idea di Gagie al calcolo
dei $\KSMEM$, tra sottostringhe dell'aplotipo query e il pannello, che
comportino $\SMEM$ con esattamente $k$ righe del pannello stesso. La
correlazione tra la $\RLBWT$ e la $\RLPBWT$ 
porta a pensare che tale problema sia risolvibile anche con la nuova definizione
di matching statistics presentata in questa tesi.\\
Nulla è stato sviluppato al momento ma si ritiene questo
un'interessante sviluppo futuro in quanto permetterebbe studi statistici, molto
comuni nei GWAS, in merito alla presenza di sottostringhe di un
aplotipo esterno all'interno di un pannello di aplotipi.\\
\\
\\
\textit{La tematica della \emph{pangenomica} è innovativa e il numero di
problemi aperti è incredibilmente grande. I dati aumentano sempre di più e gli
studi informatici devono evolversi per ``stare al passo'' con questa mole
d'informazioni. Gli \emph{sviluppi futuri} sono, da diversi punti di vista,
anche imprevedibili. Risulta quindi difficile elencare, in modo completo, le
possibilità future dietro questa branca della bioinformatica e dell'algoritmica
sperimentale.}
\dc{Frase conclusiva da modificare fortemente}
% LocalWords:  preprocessing
