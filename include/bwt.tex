\section{Trasformata di Burrows-Wheeler}
Introdotta nel 1994 da Burrows e Wheeler con lo scopo di comprimere testi, la
\textbf{Burrows-Wheeler Transform} \cite{bwt} è divenuta ormai uno standard nel
campo dell'\textit{algoritmica su stringhe} e della \textit{bioinformatica},
grazie ai suoi molteplici vantaggi sia dal punto di vista della complessità
temporale che da quello della complessità spaziale.\\
Nel dettaglio la \textit{BWT} è una \textit{trasformata reversibile} che
permette una \textit{compressione lossless}, quindi senza perdita
d'informazione. Tale trasformazione vien costruita a partire dal riordinamento
dei caratteri del testo in input, fattore che ha 
portato all'evidenza per cui caratteri uguali tendono ad essere posti
consecutivamente all'interno della stringa prodotta dalla trasformata.
\begin{definizione}
  Dato un testo $T$ \$-terminato, tale che $|T|=n$, si definisce la
  \textbf{Burrows-Wheeler Transform (\textit{BWT})} di $T$, denotata con
  $BWT_T$, come un array di caratteri lungo $n$ dove l'elemento $i$-esimo è il
  carattere che precede l'$i$-esimo suffisso $T$ nel riordinamento
  lessicografico. Più formalmente si ha che, con $0\leq i<n$:
  \[BWT_T[i]=
    \begin{cases}
      T[SA_T[i]-1]&\mbox{ se } SA_T[i]\neq 1\\
      \$&\mbox{ altrimenti}
    \end{cases}
  \]
\end{definizione}
In termini più pratici, la \textit{BWT} di un testo è calcolabile riordinando
lessicograficamente tutte le possibili \textbf{rotazioni} del testo $T$.
\begin{definizione}
  Si definisce \textbf{rotazione $\mathbf{i}$-esima}, denotata con $rot_T(i)$ di
  un testo $T$, tale che $|T|=n$, come la stringa ottenuta dalla concatenazione
  del suffisso $i$-esimo con la restante porzione del testo. Più formalmente si
  ha che, avendo $0\leq i<n$:
  \[rot_T(i)=T[i:n-1]\cdot T[0:i-1]\]
\end{definizione}
Data questa definizione quindi la \textit{BWT} del testo $T$ risulta essere
l'ultima colonna della matrice che si ottiene riordinando tutte le
\textit{rotazioni} di $T$, che altro non sono che i suffissi già riordinati per
il calcolo del \textit{SA} a cui viene concatenata la parte restante del
testo.\\
Un altro array spesso utilizzato insieme alla \textit{BWT} è il cosiddetto
\textbf{array $\mathbf{F}$}, lungo $|T|$, che altro non è che l'array formato
dalla prima colonna della matrice delle rotazioni. In termini ancora più
semplicistici l'array $F$ è banalmente l'array formato dal riordinamento
lessicografico dei caratteri del testo $T$.\\
Per chiarezza si vede un esempio. 
\begin{esempio}
   Si prenda la stringa:
  \[s=\mbox{mississippi\$},\,\,|s|=12\]
  Si produce la seguente matrice delle rotazioni riordinate:
  \begin{table}[H]
    \centering
    \footnotesize
    \begin{tabular}{c|c|c|c|c} 
      \textbf{Indice} & $\mathbf{SA_T}$ & $\mathbf{F_T}$ & \textbf{Rotazione}
      & $\mathbf{BWT_T}$\\ 
      \hline
      0 & 11 & \$ & \$mississippi & i\\
      1 & 10 & i & i\$mississipp & p\\
      2 & 7 & i & ippi\$mississ & s\\
      3 & 4 & i & issippi\$miss & s\\
      4 & 1 & i & ississippi\$m & m\\
      5 & 0 & m & mississippi\$ & \$\\
      6 & 9 & p & pi\$mississip & p\\
      7 & 8 & p & ppi\$mississi & i\\
      8 & 6 & s & sippi\$missis & s\\
      9 & 3 & s & sissippi\$mis & s\\
      10 & 5 & s & ssippi\$missi & i\\
      11 & 2 & s & ssissippi\$mi & i\\
    \end{tabular}
  \end{table}
  Avendo quindi:
  \[F_T=\mbox{\$iiiimppssss}\mbox{ e }BWT_T=\mbox{ipssm\$pissii}\]
\end{esempio}
L'importanza di questa trasformata è dovuta soprattutto al fatto che sia
\textit{reversibile}, implicando quindi che a partire da $BWT_T$ è possibile
ricostruire $T$. Questo è possibile grazie ad una proprietà intrinseca della
trasformata che viene riassunta nel cosiddetto \textbf{LF-mapping}.
\begin{definizione}
  Dato un testo $T$, tale che $|T|=n$, data la sua $BWT_T$ e il suo array $F_T$
  si definisce \textbf{LF-mapping} come la proprietà per la quale l'$i$-esima
  occorrenza di un carattere $\sigma$ in $BWT_T$ corrisponde all'$i$-esima
  occorrenza dello stesso carattere in $F_T$.
\end{definizione}
Grazie a questa definizione è possibile partire dall'ultimo carattere del testo,
\$, e ricostruire l'intero testo a ritroso. Si vede quindi un breve esempio.
\begin{esempio}
  Si riprende l'esempio precedente, avendo:
  \[BWT_T=\mbox{ipssm\$pissii}\mbox{ e }F_T=\mbox{\$iiiimppssss}\]
  % ricostruisci testo
  
\end{esempio}

\subsection{Trasformata di Burrows-Wheeler run-length}
\subsection{Matching Statistics}
\subsection{R-index}
\subsection{PHONI}