\section{RLPBWT con matching statistics}
Le precedenti versioni della \textit{RLPBWT}, come anticipato, hanno permesso di
poter ideare una variante \textit{run-length encoded} della \textit{PBWT}.
In realtà anche in questo caso c'è stata una fase transitoria di sviluppo,
avendo due varianti della struttura, che verranno introdotte a breve.\\
Il fine era quello di ottenere quanto visto per la \textbf{RLBWT} anche per la
variante posizionale, ovvero i concetti di:
\begin{itemize}
  \item \textit{matching statistics}
  \item \textit{threshold}
  \item \textit{LCE query}
\end{itemize}
A tal fine, come per la \textit{RLBWT}, si necessita di \textit{random access}
al testo. A causa di questo si sono avute due varianti in fase di sviluppo:
\begin{itemize}
  \item una prima, ancora in ottica di ``studio introduttivo'', dove il pannello
  viene memorizzato come \textit{vettore di bitvector classici}
  \item una seconda, definitiva, dove il pannello è memorizzato come
  \textit{SLP}, nelle modalità introdotte nella sottosezione \ref{subslp}
\end{itemize}
Questee versioni, a loro volta, hanno permesso, in primis, l'ideazione di un
algoritmo che sfruttasse l'idea delle threshold, come visto per la \textit{BWT}
classica con \textit{MONI} \cite{moni}, e poi, per quella basata
sull'\textit{SLP}, di uno basato sulle \textit{LCE query}, come per
\textit{PHONI} \cite{phoni}. Quest'ultima, con l'aggiunta della
\textit{struttura per la funzione $\varphi$}, sarà l'implementazione definitiva,
per questa tesi, della \textit{RLPBWT}.
\subsection{Matching statistics per la RLPBWT}
\subsection{Match con threshold per la RLPBWT}
\subsection{Match con LCE query per la RLPBWT}
