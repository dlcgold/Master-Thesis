\section{Suffix Array}
Nel 1976 Manber e Myers proposero una struttura dati per la memorizzazione di
stringhe e la loro interrogazione, efficiente sia per
l'aspetto temporale che spaziale, chiamata \textbf{Suffix Array (\textit{SA})}
\cite{sa}.
\begin{definizione}
  Dato un testo $T$, \$-terminato, tale che $|T|=n$, si definisce \textbf{suffix
    array} di $T$, denotato con $SA_T$, un array lungo $n$ di interi, tale che
  $SA_T[i]=j$ sse il suffisso di ordine $j$, ovvero $T[SA_T[i]:n-1]$, è
  l'$i$-esimo suffisso nell’ordinamento lessicografico dei suffissi di $T$. In
  altri termini quindi il \textbf{suffix array} altro non è che una permutazione
  dell'intervallo di numeri interi $[0,n-1]$.
\end{definizione}
Grazie a questa definizione si può quindi dire che, presi $i,i'\in \mathbb{N}$
tali che $0\leq i < i' < n$ allora vale che, indicando con $<$ anche
l'ordinamento lessicografico:
\[T[SA_T[i]:n-1] < T[SA_T[i']:n-1]\]
Si vede quindi esempio chiarificatore.
\begin{esempio}
  Si prenda la stringa:
  \[s=\mbox{mississippi\$},\,\,|s|=12\]
  Si producono quindi i seguenti suffissi e il loro riordinamento:
  \begin{table}[H]
    \footnotesize
    \centering
    \begin{tabular}{c|l}
      \textbf{Indice del suffisso} & \textbf{Suffisso}\\
      \hline
      0 & mississippi\$\\
      1 & ississippi\$\\
      2 & ssissippi\$\\
      3 & sissippi\$\\
      4 & issippi\$\\
      5 & ssippi\$\\
      6 & sippi\$\\
      7 & ippi\$\\
      8 & ppi\$\\
      9 & pi\$\\
      10 & i\$\\
      11 & \$\\
    \end{tabular}
    \quad
    \begin{tabular}{c|l} 
      \textbf{Indice del suffisso} & \textbf{Suffisso}\\ 
      \hline
      11 & \$\\
      10 & i\$\\
      7 & ippi\$\\
      4 & issippi\$\\
      1 & ississippi\$\\
      0 & mississippi\$\\
      9 & pi\$\\
      8 & ppi\$\\
      6 & sippi\$\\
      3 & sissippi\$\\
      5 & ssippi\$\\
      2 & ssissippi\$\\
    \end{tabular}
  \end{table}
  Ottenendo quindi che:
  \[SA_T=[11,10,7,4,1,0,9,8,6,3,5,2]\]
\end{esempio}
L'uso del \textit{suffix array} è spesso accompagnato dal \textbf{Longest Common
  Prefix}.
\begin{definizione}
  Si definisce il \textbf{Longest Common Prefix (\emph{LCP})} di un testo $T$,
  tale che $|T|=n$,
  denotato con $LCP_T$, come un array lungo $n+1$, contenente la
  lunghezza del prefisso comune tra ogni coppia di suffissi consecutivi
  nell'ordinamento lessicografico dei suffissi, quindi in $SA_T$. Più formalmente
  $LCP_T$ è un array tale che, avendo $0\leq i\leq n$ e indicando con $lcp(x,y)$
  il più lungo prefisso comune tra le stringhe $x$ e $y$:
  \[LCP_T[i]=
    \begin{cases}
      -1&\mbox{ se } i=0 \lor i=n\\
      lcp(T[SA_T[i-1]: n],T[SA_T [i]: n])&\mbox{ altrimenti}
    \end{cases}
  \]
\end{definizione}
\begin{esempio}
  Riprendendo l'esempio precedente si avrebbe quindi:
  \begin{table}[H]
    \centering
    \footnotesize
    \begin{tabular}{c|c|c|l} 
      \textbf{Indice} & $\mathbf{SA_T}$ & $\mathbf{LCP_T}$ & \textbf{Suffisso}\\ 
      \hline
      0 & 11 & -1 & \$\\
      1 & 10 & 0 & i\$\\
      2 & 7 & 1 & ippi\$\\
      3 & 4 & 1 & issippi\$\\
      4 & 1 & 4 & ississippi\$\\
      5 & 0 & 0 & mississippi\$\\
      6 & 9 & 0 & pi\$\\
      7 & 8 & 1 & ppi\$\\
      8 & 6 & 0 & sippi\$\\
      9 & 3 & 2 & sissippi\$\\
      10 & 5 & 1 & ssippi\$\\
      11 & 2 & 3 & ssissippi\$\\
      12 & - & -1 & -
    \end{tabular}
  \end{table}
\end{esempio}
Senza entrare in ulteriori dettagli relativi all'algoritmo di pattern matching
tramite \textit{SA} e \textit{LCP}, in quanto non centrali per il resto della
trattazione, risulta comunque interessante riportare le complessità
temporali. Si ha quindi che per l'algoritmo di query su \textit{SA} senza l'uso
dell'\textit{LCP} si ha, per un testo lungo $n$ e un pattern lungo $m$:
\[\mathcal{O}(m\log n)\]
Con l'uso dell'\textit{LCP} questo si riduce a:
\[\mathcal{O}(m+\log n)\]
Per ulteriori approfondimenti si rimanda al testo di Gusfield
\cite{gusfield_1997}. 