\subsection{Componente per i prefix array sample}
Come introdotto parlando delle matching statistics, qualora si abbia un
cambio di riga da memorizzare, si seleziona sempre o quella relativa alla coda
della run precedente o quella relativa alla testa della run successiva. Risulta
quindi necessario, in colonna $k$, memorizzare i valori di $a_k$ all'inizio e
alla fine di ogni run. Anche in questo caso si sono scelti i vettori di interi
bit-compressed. Tali valori sono un sample dei valori che permettono
le permutazioni che costruiscono la matrice $\PBWT$ e quindi tale
componente prende il nome di \texttt{PERM}.\\
All'algoritmo \ref{algo:buildperm} è possibile analizzare lo pseudocodice del
metodo usato per calcolare tale componente per la colonna
$k$-esima. L'algoritmo, dovendo iterare l'intera colonna della \textit{matrice
  PBWT} ha costo, in tempo:
\begin{equation}
  \label{eq:timeperm}
  \mathcal{O}(M)
\end{equation}
La costruzione può essere fatta in contemporanea a quelle delle componenti gia
descritte, ovvero: 
\texttt{MAP-INT}/\texttt{MAP-BV} e \texttt{THR-INT}/\texttt{THR-BV}.
\begin{algorithm}
  \small
  \begin{algorithmic}[1]
    \Function{build\_perm}{$col,\,\, pref$}
    \Comment $pref = a_k$
    \State $tmp_{beg}\gets 0,\,\,beg_{run}\gets \top$
    \State $samples_{beg} \gets []$
    \Comment vettore per i prefix array sample ad inizio di ogni run
    \State $samples_{end}\gets []$
    \Comment vettore per i prefix array sample a fine di ogni run
    \For {\textit{every} $k\in\left[0,\,\, height\right)$}
    \If{$beg_{run}$}
    \State $tmp_{beg}\gets pref[k]$
    \State $beg_{run}\gets \bot$
    \EndIf
    \If{$k=height-1\lor col[pref[k]]\neq col[pref[k+1]]$}  
    \State $push(samples_{beg}, tmp_{beg})$
    \State $push(samples_{end}, pref[k])$
    \State $beg_{run}\gets \top$
    \EndIf
    \EndFor
    \State \textbf{return} $(samples_{beg},\,\, samples_{end})$  
    \EndFunction
  \end{algorithmic}
  \caption{{\footnotesize{Algoritmo per la costruzione della componente
  \texttt{PERM} per la colonna $k$.}}}
  \label{algo:buildperm}
\end{algorithm}
\noindent
In termini di memoria necessaria per questa componente si hanno le medesime
considerazioni fatte nel caso della componente \texttt{MAP-INT}, con l'uso degli
intvector compressi.