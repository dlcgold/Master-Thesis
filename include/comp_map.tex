\subsection{Componente per il mapping}
La prima componente, nelle sue due varianti, che si descrive è quella relativa
al mapping tra una colonna e la sua successiva. Bisogna quindi memorizzare per
ogni colonna $k$, in modo proporzionale al numero di run della stessa, le
informazioni atte a ottenere i medesimi risultati ottenibili con la funzione
$w(i,\sigma)$, secondo la notazione di Durbin \cite{pbwt}. 
\subsubsection{Mapping con intvector}
La prima variante che si descrive è quella denominata \texttt{MAP-INT}.\\
L'ispirazione iniziale per tale componente è stata data dall'articolo di Gagie
et al \cite{tricks}, nonostante si abbiano, di fatto, diverse modifiche
strutturali. Riprendendo quanto descritto al termine della sezione
\ref{secpbwt}, si è quindi deciso di
memorizzare gli indici delle \textit{teste di run}, ovvero gli indici iniziali
di ogni run. Ovviamente questa informazione non è sufficiente per poter sapere
se una run sia composta da simboli $\sigma=0$ o simboli
$\sigma=1$. Fortunatamente, essendo lo studio limitato, come per la
\textit{PBWT}, a pannelli costruiti su alfabeto binario $\Sigma=\{0,1\}$, si è
potuto sfruttare il fatto che le run si alternano tra un carattere e
l'altro. Basta quindi tenere in memoria un valore booleano nominato
$start_k$, che permetta di 
capire se, in colonna $k$, la prima run sia una run di simboli
$\sigma=0$. Infatti le run di 
indice pari presentano lo stesso simbolo della prima run e quindi, dato un
qualsiasi indice di run, è possibile sapere quale sia il simbolo corrispondente
a tale run. L'implementazione di questo concetto è
visualizzabile all'algoritmo \ref{algo:extrchar} e richiede tempo costante.
\begin{algorithm}
  \footnotesize
  \begin{algorithmic}[1]
    \Function{get\_symbol}{$s, \,\,r$}
    \Comment $s=\top$ sse la prima run ha simbolo $\sigma=0$, $r$ indice di run
    \If{$s$}
    \State \textbf{if} $r\bmod 2 = 0$ \textbf{then} \textbf{return} $0$
    \textbf{else} \textbf{return} $1$
    \Else
    \State \textbf{if} $r\bmod 2 = 0$ \textbf{then} \textbf{return} $1$
    \textbf{else} \textbf{return} $0$
    \EndIf
    \EndFunction
  \end{algorithmic}
  \caption{Algoritmo per estrazione simbolo da una run in una colonna}
  \label{algo:extrchar}
\end{algorithm}

Si memorizzano gli indici delle teste di run in un array $p_k$, di lunghezza
pari al numero di run in colonna $k$. In pratica si memorizza un indice $i$ sse:
\begin{equation}
  \label{eq:naive1}
  y_{i-1}^k[k]\neq y_i^k[k]
\end{equation}
Il passaggio successivo è stato quello di capire se le informazioni necessarie
al mapping fossero tutte necessarie. In altri termini se, data la colonna $k$
nella \textit{matrice PBWT}, fossero necessari $c[k]$, $u_k$ e $v_k$. In merito
al valore $c[k]$, per quanto calcolabile, ipotizzando di avere solo $p_k$, in
tempo $\mathcal{O}(r)$, dove $r$ è 
il numero di run della colonna $k$-esima, si è deciso che si potesse calcolarlo
in fase di costruzione delle \textit{RLPBWT} e memorizzarlo esattamente come per
la \textit{PBWT}. In merito invece ai vettori $u_k$ e $v_k$ si è cercato un modo
per ottenerne una rappresentazione che implicasse avere un solo valore per ogni
run della colonna. In altri termini si è cercato di capire se fosse possibile
tenere in memoria $r$ valori che permettessero di effettuare comunque il
mapping, a partire da un indice arbitrario $i\in\{0,\ldots,M-1\}$. Anche in
questo caso l'alternanza data dal caso binario ha permesso di trovare una
semplice soluzione. I valori di $u_k$ e $v_k$ crescono infatti in modo
alternato. Infatti, a seconda del simbolo $\sigma$ rappresentato in una data
run, si ha che solo i valori dell'array relativo a tale simbolo, nel range di
indici di quella run, verranno incrementati, ad ogni passo, di una
unità. Facendo un semplice esempio, se siamo in una run di 0 e iteriamo
virtualmente all'interno di tale run, solo i valori di $u_k$, in quel
range di indici, cresceranno di volta in volta di uno mentre per $v_k$, nello
stesso range, si avrà sempre lo stesso valore.
\begin{esempio}
  Si vede un esempio per chiarire meglio quanto espresso in merito a $u_k$ e
  $v_k$.\\
  Sia data la seguente colonna:
  \[y^5=00101111000000000000\]
  Si hanno, oltre a $c[5]=15$:
  \begin{table}[H]
    \footnotesize
    \centering
    \begin{tabular}{c||cc|c|c|cccc|cccccccccccc}
      & 0 & 1 & 2 & 3 & 4 & 5 & 6 & 7 & 8 & 9 & 10 & 11 & 12 & 13 & 14 & 15 & 16
      & 17 & 18 & 19\\
      \hline
      \hline
      $y^5$ & 0 & 0 & 1 & 0 & 1 & 1 & 1 & 1 & 0 & 0 & 0 & 0 & 0 & 0 & 0 & 0 & 0
      & 0 & 0 & 0\\
      \hline
      \hline
      $u_5$ & 0 & 1 & 2 & 2 & 3 & 3 & 3 & 3 & 3 & 4 & 5 & 6 & 7 & 8 & 9 & 10
      & 11 & 12 & 13 & 14\\
      \hline
      $v_5$ & 0 & 0 & 0 & 1 & 1 & 2 & 3 & 4 & 5 & 5 & 5 & 5 & 5 & 5 & 5 & 5 & 5
      & 5 & 5 & 5
    \end{tabular}
  \end{table}
  Dove si nota l'alternanza di crescita dei valori sopra descritta.
\end{esempio}
Grazie a questo comportamento è possibile memorizzare, per ogni indice di
testa di run $i$, tale che $i\neq 0$, solo il valore di $u_k[i]$ o $v_k[i]$,
rispettivamente se sia una run su simboli $\sigma=1$ o $\sigma=0$. Questo in
quanto, se si analizza una run di zeri si avrà che solo i valori di $v_k$, nel
range della run, verranno incrementati ad ogni step. Per $i=0$ banalmente si ha
che $u_k[i]=v_k[i]=0$.\\
Memorizzando i valori di $u_k$ e $v_k$ in un array $uv_k$, tale che $|uv_k|=r$,
con $r$ numero di run, e dato $i\in\{0,\ldots, r-1\}$, a seconda che la colonna
presenti o meno la prima run  
con simboli $\sigma=0$, si possono estrarre, in tempo costante, i valori di
$u_k$ e $v_k$ per una data testa di run. Nel dettaglio, dato $i\in{0,\ldots,
  r-1}$:
\begin{itemize}
  \item se $i=0$ si ha che $u_k[p[i]]=v_k[p[i]]=uv_k[0]=0$
  \item se $i\mod 2 =0$ si hanno due casi:
  \begin{itemize}
    \item la prima run è di simboli $\sigma=0$ e quindi si ottiene
    $u_k[p[i]]=uv_k[i-1]$ e $v_k[p[i]]=uv_k[i]$
    \item la prima run è di simboli $\sigma=1$ e quindi si ottiene
    $u_k[p[i]]=uv_k[i]$ e $v_k[p[i]]=uv_k[i-1]$
  \end{itemize}
  \item se $i\mod 2 \neq 0$ si hanno due casi, che sono l'inverso della
  situazione descritta precedentemente:
  \begin{itemize}
    \item la prima run è di simboli $\sigma=0$ e quindi si ottiene
    $u_k[p[i]]=uv_k[i]$ e $v_k[p[i]]=uv_k[i-1]$
    \item la prima run è di simboli $\sigma=1$ e quindi si ottiene
    $u_k[p[i]]=uv_k[i-1]$ e $v_k[p[i]]=uv_k[i]$   
  \end{itemize}
\end{itemize}
Tale operazione è eseguibile in tempo costante lo pseudocodice relativo a
quanto appena detto è consultabile all'algoritmo \ref{algo:uvnaive}.
\begin{algorithm}
  \small
  \begin{algorithmic}[1]
    \Function{uvtrick}{$k,\,\, i$}
    \Comment $k$ indice di colonna, $i$ indice di run
    \If{$i = 0$}
    \State \textbf{return} $(0,\,\,0)$
    \ElsIf{$i\mod 2=0$}
    \State $u\gets uv_k[i-1],\,\,v\gets uv_k[i]$
    \If{$start_k$}
    \State \textbf{return} $(u,\,\,v)$
    \Else
    \State \textbf{return} $(v,\,\,u)$
    \EndIf
    \Else
    \State $u\gets uv_k[i],\,\,v\gets uv_k[i-1]$
    \If{$start_k$}
    \State \textbf{return} $(u,\,\,v)$
    \Else
    \State \textbf{return} $(v,\,\,u)$
    \EndIf
    \EndIf
    \EndFunction
  \end{algorithmic}
  \caption{Algoritmo per uvtrick con \texttt{MAP-INT}.}
  \label{algo:uvnaive}
\end{algorithm}

Da un punto di vista implementativo, gli array di interi $p_k$ e $uv_k$ vengono
memorizzati in \textit{vettori di interi bit-compressed}. Dato un array $v$,
tale che $x$ è il massimo indice, si calcola:
\begin{equation}
  \label{eq:bc1}
  b=\lceil\log(x-1)\rceil+1
\end{equation}
e si memorizzano le componenti di $v$ in un vettore di interi con componenti
memorizzate in di $b$ bit.\\
Ricapitolando, per la componente \texttt{MAP-INT}, si hanno in memoria, per
ogni colonna $k$:
\begin{itemize}
  \item $start_k$, con il booleano atto a capire il simbolo della prima run
  \item $p_k$, con gli indici delle teste di run
  \item $uv_k$, coi valori compatti di $u_k$ e $v_k$ per le teste di run
  \item $c[k]$, per sapere il numero totale di simboli $\sigma=0$ nella colonna
  $k$ della \textit{matrice PBWT}
\end{itemize}
\begin{algorithm}
  \small
  \begin{algorithmic}[1]
    \Function{build\_map\_int}{$col,\,\, pref$}
    \Comment $pref=a_k$
    \State $c\gets 0,\,\,u\gets 0,\,\,v\gets 0,\,\,u'\gets 0,\,\, v'\gets
    0,\,\,run\gets 0$
    \State $start \gets \top,\,\,beg_{run}\gets \top,\,\,push_{zero}\gets
    \bot,\,\,push_{one}\gets \bot$
    \State $p\gets [],\,\,uv\gets []$
    \For {\textit{every} $k\in\left[0,\,\, M\right)$}
    \If{$k=0\land col[pref[k]]=1$}
    \State $start \gets \bot$
    \EndIf
    \If{$col[k]=0$}
    \State $c\gets c+1$
    \EndIf
    \EndFor
    \If{$start$}
    \State $push_{one}\gets \top$
    \Else
    \State $push_{zero}\gets \top$
    \EndIf
    \For{\textit{every} $k\in[0,M)$}
    \If{$beg_{run}$}
    \State $u\gets u',\,\,v\gets v'$
    \State $beg_{run}\gets \bot$
    \EndIf
    \If{$col[pref[k]]=1$}
    \State $v'\gets v'+1$
    \Else
    \State $u'\gets u'+1$
    \EndIf
    \If{$k=0\lor col[pref[k]]\neq col[pref[k-1]]$}
    \State $run\gets k$
    \EndIf
    \If{$k=M-1\lor col[pref[k]]\neq col[pref[k+1]]$}
    \If {$push_{one}$}
    \State $push(p, run),\,\,push(uv, v)$
    \State $swap(push_{one}, push_{zero})$
    \Else
    \State $push(p, run),\,\,push(uv, u)$
    \State $swap(push_{one}, push_{zero})$
    \EndIf
    \State $beg_{run}\gets \top$
    \EndIf
    \EndFor
    \State \textbf{return} $(start,\,\, c,\,\, p,\,\, uv)$
    \EndFunction
  \end{algorithmic}
  \caption{\footnotesize{Algoritmo per la costruzione della componente
  \textit{MAP-INT} per la colonna $k$.}}
  \label{algo:buildnaive}
\end{algorithm}
\begin{esempio}
  Sia data la seguente colonna:
  \[y^5=00101111000000000000\]
  Per la componente \texttt{MAP-INT} della colonna 5, si hanno in memoria:
  \[p_5=[0,2,3,4,8]\]
  \[uv_5=[0,2,1,3,5]\]
  \[c[5]=15\]
\end{esempio}
La costruzione della componente \texttt{MAP-INT} per una certa colonna,
analizzabile nell'algoritmo 
\ref{algo:buildnaive}, ha costo $\mathcal{O}(M)$, avendo che la costruzione
avviene scorrendo la colonna $k$ permutata dal 
\textit{prefix array} $a_k$.\\
Si hanno quindi le informazioni relative alle teste di run. Per
l'implementazione dei vari algoritmi si ha però necessità di usare anche indici
con valori $\{0, M-1\}$. Una delle operazioni fondamentali è quindi
quella, dato un indice $i\in\{0,\ldots,M-1\}$, di computare a quale run esso
appartenga, in una certa colonna $k$. Tale operazione può essere svolta usando
una semplice variante della \textit{ricerca binaria}. Questa variante
dell'algoritmo tradizionale, anziché
ritornare l'indice di un elemento qualora esista nell'array, restituisce
l'ultimo indice iniziale del sottointevallo usato dalla ricerca binaria
calcolato prima dell'interruzione dell'esecuzione dell'algoritmo (che avviene
secondo l'algoritmo standard). Pur essendo, di fatto, una funzione \textit{rank}
si è deciso, per comodità di lettura, di chiamare tale funzione
\textit{index\_to\_run}, la quale, come visualizzabile 
all'algoritmo \ref{algo:itr}, ha complessità in tempo, con $r=|p_k|$ (ovvero è
il numero di run in colonna $k$): 
\begin{equation}
  \label{eq:itrcomp}
  \mathcal{O}(\log (r))
\end{equation}
\begin{algorithm}
  \footnotesize
  \begin{algorithmic}[1]
    \Function{index\_to\_run}{$k,\,\, i$}
    \Comment $k$ indice di colonna, $i$ indice di riga della colonna $k$
    \If{$i\geq p_k[|p_k|-1]$}
    \State \textbf{return} $|p_k|-1$
    \EndIf
    \State $b\gets 0,\,\, e \gets |p_k|$
    \State $run\gets \frac{e-b}{2}$
    \While {$run\neq e \land p_k[run] \neq i$ }
    \If {$i< p_k[run]$ }
    \State $e\gets run$
    \Else
    \If {$run+1=e \lor  p_k[run+1] > i$}
    \State \textbf{break}
    \EndIf
    \State $b\gets run+1$
    \EndIf
    \State $run\gets b+\frac{e-b}{2}$
    \EndWhile
    \State \textbf{return} $run$
    \EndFunction
  \end{algorithmic}
  \caption{Algoritmo per convertire un indice di colonna in indice di run, con
    \texttt{MAP-INT}.} 
  \label{algo:itr}
\end{algorithm}

Ipotizzando, inoltre, di avere
un indice $i\in\{0,\ldots,M-1\}$ è possibile risalire ai valori
$u_k[i]$ e $v_k[i]$, sfruttando l'\textit{offset} dell'indice rispetto alla
testa della run a cui appartiene. Banalmente, ipotizzando di essere in una run
di simboli $\sigma$ con testa di run all'indice $p$, si avranno, avendo
ottenuto $u_k[p]$ e $v_k[p]$ da $uv_k[p]$:
\begin{equation}
  \small
  \label{eq:naive2}
  \begin{cases}
    v_k[i]=v_k[p]\\
    u_k[i]=u_k[p]+(i-p)
  \end{cases},\mbox{sse } y^k_p[k]=0
  \quad
  \begin{cases}
    u_k[i]=u_k[p]\\
    v_k[i]=v_k[p]+(i-p)
  \end{cases},\mbox{sse } y^k_p[k]=1
\end{equation}
Tenendo eventualmente conto dell'offset $off$, qualora si abbia un simbolo
$\sigma$ uguale a quello della run in analisi, è quindi possibile
riadattare l'algoritmo per il 
mapping visto per la \textit{PBWT} di Durbin, da intendersi alla stregua del
\textit{backward step} visto nel caso 
della \textit{BWT}, dalla colonna $k$ alla colonna $k+1$ guidato da
$z[k+1]$. Tale soluzione è riportata all'algoritmo \ref{algo:lfoff}.
Ricordando che si può risalire ai valori $u[p]$ e $v[p]$ in tempo costante,
anche il mapping da una colonna alla successiva avviene in tempo costante.
\begin{algorithm}
  \begin{algorithmic}[1]
    \Function{w}{$k,\,\, i, \,\,\sigma,\,\,o$}
    \Comment $k$ indice di colonna, $i$ indice di riga, $\sigma$ simbolo
    \State $run\gets index\_to\_run(k,i)$
    \If{$\sigma=0\land get\_symbol(start_k, run)=1$}
    \State $off\gets 0$
    \ElsIf {$\sigma=1\land get\_symbol(start_k, run)=0$} 
    \State $off\gets 0$
    \Else
    \State $off\gets i-p_k[run]$
    \EndIf
    \State $(u,v)\gets uvtrick(k,\,\,i)$
    \If{$p_k[i]+off=M$}
    \If{$get\_symbol(start_k, i)=0$}
    \State $v\gets v-1$
    \Else
    \State $u\gets u-1$
    \EndIf
    \EndIf
    \If{$\sigma = 0$}
    \State \textbf{return} $u+off$
    \Else
    \State \textbf{return} $c[k]+v+off$
    \EndIf
    \EndFunction
  \end{algorithmic}
  \caption{Algoritmo per il mapping con \texttt{MAP-INT}.}
  \label{algo:lfoff}
\end{algorithm}
\subsubsection{Mapping con bitvector}
La seconda variante della componente di mapping sfrutta, al posto degli
\textit{intvector}, i \textit{bitvector sparsi}, da cui la nomenclatura
\texttt{MAP-BV}.\\
L'idea è quindi quella di sostituire, data una colonna $k$, quanto necessario a
rappresentare le run (ovvero il vettore $p_k$ della \texttt{MAP-INT}) e quanto
necessario a permettere il mapping (ovvero il vettore $uv_k$ della
\texttt{MAP-INT}).\\ 
In primis, per poter localizzare le run nella $k$-esima colonna, si è scelto di
usare un \textit{bitvector sparso}, che denominiamo per praticità $h_k$, tale
che $|h_k|=M$. Formalmente si ha che:
\begin{equation}
  \label{eq:bv1}
  h_k[i]=
  \begin{cases}
    1&\mbox{se } y^k_{i}[k]\neq y^k_{i+1}[k]\lor i=M-1\\
    0&\mbox{altrimenti}
  \end{cases},\forall i\in \{0,\ldots,M-1\}
\end{equation}
Informalmente, quindi, si ha che si ha $h_k[j]=1$ sse è l'indice di fine di una
run.\\
mpiricamente ci si aspettano ``poche'' run all'interno di una colonna della
\textit{matrice PBWT}, per quanto già discusso nella sezione
\ref{secpbwt}. Avendo poche run ci si aspetta anche ``pochi'' 1 all'interno di
$h_k$, di conseguenza si è optato per usare i \textit{bitvector sparsi} per la
memorizzazione in memoria di ogni $h_k$, ricordando che, secondo quanto
riportato per la libreria \textit{SDSL} \cite{sdsl}, tale variante richiede in
memoria, indicando con $r$ il numero di run:
\begin{equation}
  \label{eq:bv2}
  \approx r\left(2+\log\frac{M}{r}\right)\mbox{ bit}
\end{equation}
Pensando ad una correlazione tra \texttt{MAP-INT} e \texttt{MAP-BV}, si ha che
$rank_{h_k}$ fa le veci della funzione \texttt{index\_to\_run} mentre
$select_{h_k}$ equivale a $p_k$.\\
Più elaborata è la rappresentazione dei vettori $u_k$ e $v_k$. In questo caso si
è deciso, a differenza della rappresentazione unica vista con la
\texttt{MAP-INT}, di optare per due \textit{bitvector sparsi}. In particolare,
per il vettore $u_k$, tale che $|u_k|=c[k]$, si ha che, $\forall
i\in\{0,\ldots,|u_k|-1\}$: 
\begin{equation}
  \label{eq:bv3}
  u_k[i]=
  \begin{cases}
    1&\mbox{se }i \mbox{ è il numero di simboli che contiene la
    }rank_{u_k}(i)\mbox{-esima run di 0}\\
    0&\mbox{altrimenti}
  \end{cases}
\end{equation}
Analogamente si definisce $v_k$, avendo $|v_k|=M-c[k]$ e $\forall
i\in\{0,\ldots,|v_k|-1\}$, come: 
\begin{equation}
  \label{eq:bv4}
  v_k[i]=
  \begin{cases}
    1&\mbox{se }i \mbox{ è il numero di simboli che contiene la
    }rank_{v_k}(i)\mbox{-esima run di 1}\\
    0&\mbox{altrimenti}
  \end{cases}
\end{equation}
Si noti che:
\begin{equation}
  \label{eq:bv5}
  rank_{h_k}(|h_k|-1)+1=(rank_{u_k}(|u_k|-1)+1)+(rank_{v_k}(|v_k|-1)+1)
\end{equation}
Ovvero il numero di 1 presenti in $h_k$ è pari alla somma di quelli presenti in
$u_k$ e $v_k$. Si noti che i vari $+1$ sono
dovuti al fatto che la funzione 
$rank(i)$ esclude dal computo la posizione $i$ stessa e tutti e tre i bitvector,
per costruzione, presentano $\sigma=1$ in ultima posizione. Ne segue che, anche
per questi ultimi due bitvector, la scelta di 
usare \textit{bitvector sparsi} per la loro memorizzazione sia giustificata,
empiricamente, dalla poca quantità attesa di simboli $\sigma=1$.
\begin{esempio}
  \label{es:bv1}
  Sia data la seguente colonna:
  \[y^5=00101111000000000000\]
  Si ha quindi che:
  \[h_5=01110001000000000001\]
  Avendo appunto un numero di run pari a:
  \[rank_{h_5}(|h_5|-1)+1=4+1=5\]
  In merito alle run composte da simboli $\sigma=0$ si ha che:
  \[u_5=011000000000001\]
  Avendo infatti che si segnalano:
  \begin{itemize}
    \item la prima run composta da due simboli $\sigma=0$
    \item la seconda run composta da un solo simbolo $\sigma=0$
    \item la terza run composta da dodici simboli $\sigma=0$
  \end{itemize}
  Parlando invece di $v_5$ si ha:
  \[v_5=10001\]
  Avendo che:
  \begin{itemize}
    \item la prima run è composta da un solo simbolo $\sigma=1$
    \item la seconda run è composta da quattro $\sigma=1$
  \end{itemize}
  Si conferma, inoltre, quanto detto nell'equazione \ref{eq:bv5}, avendo:
  \[rank_{h_5}(|h_5|-1)+1=5 = (rank_{u_5}(13)+1)+(rank_{v_5}(4)+1)=(2+1)+(1+1)=5\]
\end{esempio}
Lo pseudocodice relativo alla costruzione della componente \texttt{MAP-BV} per
la colonna $k$-esima è disponile all'algoritmo \ref{algo:cosbv}. Anche in questo
caso la costruzione avviene scorrendo la colonna $k$ permutata dal
\textit{prefix array} $a_k$. \\ 
Assumendo che la complessità in tempo delle costruzioni delle
strutture a supporto per le funzioni \textit{rank} e \textit{select} dei tre
bitvector sparsi sia limitata superiormente dalla loro lunghezza massima, ovvero
$M$, si ha che la costruzione della componente \texttt{MAP-BV} per una singola
colonna avviene in tempo:
\begin{equation}
  \label{eq:bvcos}
  \mathcal{O}(M)
\end{equation}
\begin{algorithm}
  \small
  \begin{algorithmic}[1]
    \Function{build\_map\_bv}{$col,\,\, pref$}
    \Comment $pref=a_k$
    \State $c\gets 0,\,\,u\gets 0,\,\,v\gets 0,\,\,u'\gets 0,\,\, v'\gets
    0,\,\,curr_{lcs}\gets 0$
    \State $start \gets \top,\,\,beg_{run}\gets \top,\,\,push_{zero}\gets
    \bot,\,\,push_{one}\gets \bot$
    \For {\textit{every} $k\in\left[0,\,\, M\right)$}
    \If{$k=0\land col[pref[k]]=1$}
    \State $start \gets \bot$
    \EndIf
    \If{$col[k]=0$}
    \State $c\gets c+1$
    \EndIf
    \EndFor
    \State $runs\gets[0..0]$
    \Comment bitvector sparso per le run, di lunghezza $M+1$
    \State $zeros\gets[0..0]$
    \Comment bitvector sparso per $u_k$, di lunghezza $c[k]$
    \State $ones\gets[0..0]$
    \Comment bitvector sparso per $v_k$, di lunghezza $M-c$
    \If{$start$}
    \State $push_{one}\gets \top$
    \Else
    \State $push_{zero}\gets \top$
    \EndIf
    \For {\textit{every} $k\in\left[0,\,\, M\right)$}
    \If{$beg_{run}$}
    \State $u\gets u',\,\,v\gets v',\,\,beg_{run}\gets \bot$
    \EndIf
    \If{$col[pref[k]]=1$}
    \State $v'\gets v'+1$
    \Else
    \State $u'\gets u'+1$
    \EndIf
    \If{$k=M-1\lor col[pref[k]]\neq col[pref[k+1]]$}
    \State $runs[k]\gets 1$
    \If{$push_{one}$}
    \If{$v\neq 0$}
    \State $ones[k-1]=1$
    \EndIf
    \State $swap(push_{zero},\,\,push_{one})$
    \Else
    \If{$u\neq 0$}
    \State $zeros[k-1]=1$
    \EndIf
    \State $swap(push_{zero},\,\,push_{one})$
    \EndIf
    \State $beg_{run}\gets \top$
    \EndIf
    \EndFor
    \If{$|zeros|\neq 0$}
    \State $zeros[|zeros|-1]\gets 1$
    \EndIf
    \If{$|ones|\neq 0$}
    \State $ones[|ones|-1]\gets 1$
    \EndIf
    \State \textit{costruzione delle strutture per rank/select dei tre
    bitvector} 
    \State \textbf{return}
    $(start,\,\,c,\,\,runs,\,\,zeros,\,\,ones)$  
    \EndFunction
  \end{algorithmic}
  \caption{\footnotesize{Algoritmo per la costruzione della componente
  \texttt{MAP-BV} per la colonna $k$.}}
  \label{algo:cosbv}
\end{algorithm}
Bisogna spiegare come, dato un indice di aplotipo $i\in\{0,\ldots,M-1\}$ e una
colonna $k$, estrarre $u'_k[i]$ e $v'_k[i]$, ovvero come se si stesse usando la
\textit{PBWT} classica, a partire dagli attuali $u_k[i]$ e $v_k[i]$. Ovviamente,
se $i=0$, si ha che $u'_k[0]=v'_k[0]=0$. In caso contrario bisogna, in primis,
calcolare la run 
in cui si trova l'indice $i$. Questo si ottiene direttamente sfruttando $h_k$:
\begin{equation}
  \label{eq:bv7}
  run = rank_{h_k}(i)
\end{equation}
Una volta calcolato l'indice di run si hanno tre possibilità:
\begin{enumerate}
  \item si ha che $run=0$ e una run di simboli $\sigma=b$, con $b\in\{0,1\}$
  allora:
  \begin{equation}
    \label{eq:bv8}
    (u,v)=
    \begin{cases}
      (i,0)&\mbox{se } b=0\\
      (0,i)&\mbox{altrimenti}
    \end{cases}
  \end{equation}
  \item si ha che $run=1$ e una run di simboli $\sigma=b$, con $b\in\{0,1\}$. In
  tal caso bisogna per prima cosa individuare l'indice di inizio della seconda
  run, sfruttando $h_k$:
  \begin{equation}
    \label{eq:bv9}
    beg = select_{h_k}(1)+1
  \end{equation}
  A questo punto si ha il numero di simboli della prima run, indicizzata a 0, e,
  calcolando la distanza tra l'indice di riga e quello di inizio della prima
  run, avendo che:
  \begin{equation}
    \label{eq:bv10}
    (u,v)=
    \begin{cases}
      (beg,i-beg)&\mbox{se } b=0\\
      (i-beg,beg)&\mbox{altrimenti}
    \end{cases}
  \end{equation}
  \item si ha che $run=j$, con $j\in\{2,r-1\}$. Anche in questo caso  si procede
  calcolando l'indice di inizio della run:
  \begin{equation}
    \label{eq:bv11}
    beg = select_{h_k}(run)+1
  \end{equation}
  e l'offset rispetto all'indice $i$ dato:
  \begin{equation}
    \label{eq:bv12}
    offset = i-beg
  \end{equation}
  Poi, sfruttando la solita dicotomia fornita dal caso binario in studio, si
  hanno due casi: 
  \begin{enumerate}
    \item si è in una run di indice pari.
    Si sfruttano poi $u_k$ e $v_k$ per sapere l'indice della precedente run con
    simboli $\sigma=0$:
    \begin{equation}
      \label{eq:bv13}
      pre_u=select_{u_k}\left(\left\lfloor\frac{run}{2}\right\rfloor\right)+1
    \end{equation}
    e quello della run con simboli simboli $\sigma=1$:
    \begin{equation}
      \label{eq:bv14}
      pre_v=select_{v_k}\left(\left\lfloor\frac{run}{2}\right\rfloor\right)+1
    \end{equation}
    Si noti che si usa $\frac{run}{2}$ in quanto, essendo in una run di indice
    pari si hanno precedentemente lo stesso numero di run per $\sigma=0$ e per
    $\sigma=1$ e quindi si considera lo stesso numero di ``run'' nei due
    \textit{bitvector sparsi} $u_k$ e $v_k$.\\
    A questo punto, sempre per il ragionamento per cui solo uno tra $u$ e $v$
    non è costante all'interno di una run si ha che o $pre_u$ o $pre_v$ è tale
    costante mentre l'altro valore deve essere calcolato considerando l'offset:
    \begin{equation}
      \label{eq:bv15}
      (u,v)=
      \begin{cases}
        (pre_u+offset,pre_v)&\mbox{se } b=0\\
        (pre_u,pre_v+offset)&\mbox{altrimenti}
      \end{cases}
    \end{equation}
    \item ci si trova in una run di indice dispari, quindi non si hanno
    precedentemente lo stesso numero di run per i due simboli. Bisogna quindi
    calcolare quante siano tali run. Se la prima run è di zeri:
    \begin{equation}
      \label{eq:bv16}
      run_u=select_{u_k}\left(\left\lfloor\frac{run}{2}\right\rfloor\right)+1
    \end{equation}
    \begin{equation}
      \label{eq:bv17}
      run_v=select_{v_k}\left(\left\lfloor\frac{run}{2}\right\rfloor\right)
    \end{equation}
    mentre se la prima run non è di zeri si devono invertire i due valori. Si sa
    quindi quali ``run'' considerare sui due \textit{bitvector sparsi} $u_k$ e
    $v_k$.\\ 
    Posso quindi procedere come nel caso precedente, avendo:
    \begin{equation}
      \label{eq:bv18}
      pre_u=select_{u_k}(run_u)+1
    \end{equation}
    \begin{equation}
      \label{eq:bv19}
      pre_v=select_{v_k}(run_v)+1
    \end{equation}
    E potendo quindi restituire:
    \begin{equation}
      \label{eq:bv20}
      (u,v)=
      \begin{cases}
        (pre_u,pre_v+offset)&\mbox{se } b=0\\
        (pre_u+offset,pre_v)&\mbox{altrimenti}
      \end{cases}
    \end{equation}
  \end{enumerate}
\end{enumerate}
\dc{Sistemare}
\begin{esempio}
  Si prendano i dati e i risultati ottenuti all'esempio \ref{es:bv1}. Si
  vogliono calcolare $u$ e $v$ per $i=6$.\\
  In primis si ha quindi:
  \[run=rank_{h_5}(6)=3\]:
  \[beg = select_{h_5}(3)+1=3+1=4\]
  \[offset = i-beg=6-4=2\]
  Quindi ci si trova nel terzo caso e, nel dettaglio, avendo una run di
  indice dispari. Si calcolano quindi:
  \[run_u=select_{u_5}\left(\left\lfloor\frac{3}{2}\right\rfloor\right)+1=
    select_{u_5}(1)+1 =1+1=2\] 
  \[run_v=select_{v_5}\left(\left\lfloor\frac{3}{2}\right\rfloor\right)=
    select_{v_5}(1)=0\] 
  che non andranno invertiti avendo $start^5=\top$.\\
  Si calcolano quindi:
  \[pre_u=select_{u_5}(2)+1=2+1=3\]
  \[pre_v=select_{v_5}(0)+1=0+1=1\]
  Avendo infatti, in totale, tre simboli $\sigma=0$ e un simbolo $\sigma=1$
  prima dell'indice 6.\\ 
  Concludendo, avendo $start^5=\top$:
  \[(u,v)=(pre_u, pre_v + offset)=(3,1+2)=(3,3)\]
\end{esempio}
Lo pseudocodice per il calcolo di $u_k[i]$ e $v_k[i]$ è disponibile
all'algoritmo \ref{algo:uvbv}. In merito alla 
complessità in tempo del calcolo di $u_k[i]$ e $v_k[i]$, si ha che essa è
limitata superiormente dal costo della \textit{funzione rank} su
\textit{bitvector sparsi}, essendo la \textit{funzione select} disponibile in
tempo costante. Ne segue che, avendo $r$ run nella colonna $k$, si ha un tempo
proporzionale a:
\begin{equation}
  \label{eq:bvuvtime}
  \mathcal{O}\left(\log\frac{M}{r}\right)
\end{equation}
Non dovendo in tal
caso considerare esplicitamente l'offset, come nel caso della \texttt{MAP-INT},
il mapping dalla colonna $k$ alla 
colonna $k+1$, guidato da $z[k+1]$, viene fatto come nel caso della
\textit{PBWT}, come visualizzabile all'algoritmo \ref{algo:lfr}, che presenta
quindi la medesima complessità del calcolo di $u[i]$ e $v[i]$, ovvero quello
visto all'equazione \ref{eq:bvuvtime}.
\begin{algorithm}
  %\small
  \begin{algorithmic}[1]
    \Function{uvtrick}{$k,\,\, i$}
    \Comment $k$ indice di colonna, $i$ indice di riga
    \If{$i = 0$}
    \State \textbf{return} $(0,\,\,0)$
    \EndIf
    \State $run \gets rank_h^{k}(i)$
    \If{$run=0$}
    \If{$start_k$}
    \State \textbf{return} $(i,\,\, 0)$
    \Else
    \State \textbf{return} $(0, \,\,i)$
    \EndIf
    \ElsIf{$run=1$}
    \If{$start_k$}
    \State \textbf{return} $(select_h^{k}(run)+1,\,\, i-(select_h^{k}(run)+1))$
    \Else
    \State \textbf{return} $(i-(select_h^{k}(run)+1),\,\, select_h^{k}(run)+1)$
    \EndIf
    \Else
    \If{$run\bmod 2 = 0$}
    \State $pre_u\gets
    select_u^{k}\left(\left\lfloor\frac{run}{2}\right\rfloor\right)+1$ 
    \State $pre_v\gets
    select_v^{k}\left(\left\lfloor\frac{run}{2}\right\rfloor\right)+1$ 
    \State $offset \gets i -(select_h^{k}(run)+1)$
    \If{$start_k$}
    \State \textbf{return} $(pre_u+offset,\,\, pre_v)$
    \Else
    \State \textbf{return} $(pre_u, \,\,pre_v+offset)$
    \EndIf
    \Else
    \State $run_u\gets \left(\left\lfloor\frac{run}{2}\right\rfloor\right)+1$
    \State $run_v\gets \left\lfloor\frac{run}{2}\right\rfloor$
    \If{$\neg start_k$}
    \State $swap(run_u, run_v)$
    \EndIf
    \State $pre_u\gets select_u^{k}(run_u)+1$
    \State $pre_v\gets select_v^{k}(run_v)+1$
    \State $offset \gets i -(select_h^{k}(run)+1)$
    \If{$start_k$}
    \State \textbf{return} $(pre_u, \,\,pre_v+offset)$
    \Else
    \State \textbf{return} $(pre_u+offset, \,\,pre_v)$
    \EndIf
    \EndIf
    \EndIf
    \EndFunction
  \end{algorithmic}
  \caption{Algoritmo per uvtrick con \texttt{MAP-BV}.}
  \label{algo:uvbv}
\end{algorithm}
\begin{algorithm}
  \begin{algorithmic}[1]
    \Function{w}{$k,\,\, i, \,\,\sigma$}
    \Comment $k$ indice di colonna, $i$ indice di riga, $\sigma$ simbolo
    \State $c\gets c[k]$
    \State $(u, v) \gets uvtrick(k,\,\,i)$
    \If{$\sigma = 0$}
    \State \textbf{return} $u$
    \Else
    \State \textbf{return} $c+v$
    \EndIf
    \EndFunction
  \end{algorithmic}
  \caption{Algoritmo per il mapping con \texttt{MAP-BV}.}
  \label{algo:lfr}
\end{algorithm}
\dc{Serve altro?}