Negli ultimi anni, a partire dall'articolo di Durbin del 2014, la
\textbf{Trasformata di Burrows-Wheeler Posizionale (\textit{PBWT})} è stata al
centro delle ricerche riguardanti 
il disegno di algoritmi efficienti per il pattern matching su grandi collezioni
di aplotipi. Come indicato da Durbin stesso, una \textbf{rappresentazione
  run-length encoded della PBWT} risulta essere molto efficiente dal punto di
vista della memorizzazione della stessa.\\
In questa tesi, svolta in collaborazione con il
laboratorio di ricerca \textbf{BIAS} del \textbf{Dipartimento di Informatica
  Sistemistica e Comunicazione \textit{(DISCo})}, con professori e ricercatori
dell'\textbf{University of Florida (\textit{UFL})} e della \textbf{DalHousie
  University}, si è quindi implementata una variante della \textbf{RLPBWT}, 
ispirata ai risultati già ottenuti con la \textbf{variante run-length encoded
  della BWT} tradizionale, che permettesse di risolvere il problema del matching
tra un aplotipo esterno e un pannello di aplotipi.\\
A tal fine si sono selezionate le
informazioni minimali da memorizzare per ogni run, utilizzando strutture dati
succinte (come gli sparse bit-vectors) al fine di ottimizzare la complessità
spaziale della struttura dati, e costruendo un efficiente algoritmo per
effettuare query alla struttura stessa. ​ 