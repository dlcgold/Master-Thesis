% \documentclass[handout]{beamer}
\documentclass{beamer}
\mode<presentation>
{
  \usetheme{Warsaw}
  % \setbeamercovered{transparent}
  \useoutertheme{infolines}

}

\usepackage[italian]{babel}
\usepackage[utf8]{inputenc}
\usepackage{wrapfig}
\usepackage[T1]{fontenc}
\usepackage{float}
\usepackage{multicol}
\usepackage{algorithm}
\usepackage[noend]{algpseudocode}
\usepackage{blindtext}
\usepackage{mwe}

\definecolor{nord1}{RGB}{46, 52, 64}
\definecolor{nord2}{RGB}{76, 105, 141}
\definecolor{nord3}{RGB}{94, 129, 172}
\definecolor{nord4}{RGB}{129, 161, 193}
\definecolor{nord5}{RGB}{136, 192, 208}
\definecolor{nord6}{RGB}{163, 190, 140}
\definecolor{nord7}{RGB}{191, 97, 106}
\definecolor{nordred}{RGB}{191, 97, 106}
\definecolor{nordgreen}{RGB}{135, 157, 116}

\setbeamercolor{palette primary}{bg=nord2,fg=white}
\setbeamercolor{palette secondary}{bg=nord3,fg=white}
\setbeamercolor{palette tertiary}{bg=nord4,fg=white}


\setbeamercolor{block title}{bg=nord2,fg=white}

\setbeamercolor{itemize item}{fg=nord3}
\setbeamercolor{itemize subitem}{fg=nord4}
\setbeamercolor{itemize subsubitem}{fg=nord5}

\setbeamertemplate{itemize item}[square]
\setbeamertemplate{itemize subitem}[circle]
\setbeamertemplate{itemize subsubitem}[triangle]
\setbeamerfont{bibliography item}{size=\scriptsize}
\setbeamerfont{bibliography entry author}{size=\scriptsize}
\setbeamerfont{bibliography entry title}{size=\scriptsize}
\setbeamerfont{bibliography entry location}{size=\tiny}
\setbeamerfont{bibliography entry note}{size=\tiny}

\usecolortheme[named=nord2]{structure}
\setbeamertemplate{bibliography item}{\insertbiblabel}

\title[] {Algoritmi per la trasformata di Burrows-Wheeler posizionale con
  compressione run-length}

% \subtitle
% {Presentation Subtitle} % (optional)

\author[1]{\Large{Davide Cozzi}}


\institute[] {\large{\textbf{{\color{nord2}Relatore:}} \textit{Prof.~Raffaella
      Rizzi}\quad 
    \textbf{\color{nord2}{Correlatore:}} \textit{Dr.~Yuri Pirola}}\\
  \vspace{4mm}
  \small{\textit{Dipartimento di Informatica, Sistemistica e Comunicazione
    (DISCo)\\
    Università degli Studi di Milano Bicocca}}}

\date[] {25 Ottobre 2022}

\subject{Presentatazione}
\pgfdeclareimage[height=0.5cm]{university-logo}{img/logo_unimib.pdf}
\logo{\pgfuseimage{university-logo}}

\AtBeginSection[]
{
  \begin{frame}<beamer>{Outline}
    \tableofcontents[currentsection, currentsubsection]
  \end{frame}
}


% If you wish to uncover everything in a step-wise fashion, uncomment
% the following command: 

% \beamerdefaultoverlayspecification{<+->}


\begin{document}

\begin{frame}
  \titlepage
\end{frame}

\begin{frame}{Outline}
  \setcounter{tocdepth}{1}
  \tableofcontents
\end{frame}
\section{Introduzione}
\subsection{Pangenoma e aplotipi}
\begin{frame}{Un punto di vista per il pangenoma}
  \begin{block}{}
    Negli ultimi anni si è assistito a un cambio di paradigma nel campo della
    \textit{bioinformatica}, ovvero il passaggio dallo studio della sequenza
    lineare di 
    un singolo genoma a quello di un insieme di genomi, provenienti da un gran
    numero di individui, al fine di poter considerare anche le varianti geniche.
    Questo nuovo concetto è stato introdotto da Tettelin, nel 2005, con il
    termine di \textbf{pangenoma}. 
  \end{block}
  \pause
  \begin{block}{}
    Uno degli approcci più usati per rappresentare il \textbf{pangenoma}
    è attraverso un pannello di aplotipi, ovvero, da un punto di vista
    computazionale, una matrice di M righe, corrispondenti agli individui, e N
    colonne, corrispondenti ai siti con le varianti. Con il
    termine \textbf{aplotipo}, si intende l'insieme di alleli, ovvero di
    varianti che, a meno di mutazioni, un organismo eredita da ogni genitore. 
  \end{block}
\end{frame}
\section{Preliminari}
\subsection{Bitvector e straight-line program}
\subsection{Trasformata di Burrows-Wheeler, suffix array e maximal exact
  matches} 
\subsection{Run-length encoded BWT e matching statistics}
\subsection{Algoritmo di Bannai, MONI e PHONI}
\subsection{Trasformata di Burrows-Wheeler posizionale}
\subsection{Set-maximal exact match con la PBWT}
\section{Metodo}
\subsection{Perché una variante run-length della PBWT}
\subsection{Mapping, threshold e prefix array sample}
\subsection{Random access e longest common extension query}
\subsection{Struttura per le funzioni $\varphi$ e $\varphi^{-1}$}
\subsection{Algoritmo di calcolo degli SMEM con LCP}
\subsection{Algoritmo di calcolo degli SMEM con MS}
\section{Risultati sperimentali}
\subsection{1000 Genome Project, ambiente di test e implementazioni}
\subsection{Risultati costruzione strutture}
\subsection{Risultati calcolo degli SMEM}
\section{Conclusioni e sviluppi futuri}
\subsection{Ottimizzazioni, generalizzazioni ed estensioni}
% \section{Bibliografia}
% \begin{frame}[allowframebreaks]{Bibliografia} 
%   \bibliographystyle{unsrt}
%   \bibliography{slides}
% \end{frame}
\begin{frame}{}
  \setbeamercolor{palette primary}{bg=nord3,fg=white}
  
  \title[] {Grazie per l'attenzione}

  % \subtitle
  % {Presentation Subtitle} % (optional)

  \author[1]{\Large{Davide Cozzi}}


  \institute[] {\large{\textbf{{\color{nord2}Relatore:}} \textit{Prof.~Raffaella
        Rizzi}\quad 
      \textbf{\color{nord2}{Correlatore:}} \textit{Dr.~Yuri Pirola}}\\
    \vspace{4mm}
    \small{\textit{Dipartimento di Informatica, Sistemistica e Comunicazione
        (DISCo)\\
        Università degli Studi di Milano Bicocca}}}

  \maketitle
\end{frame}

\end{document}


